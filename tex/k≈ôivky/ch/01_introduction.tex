
\chapter{Úvod}

\section{Nevidomost na světě a význam braillova písma}

Nevidomost je fenomen rozvojových zemi a staři. 39 milionů nevídomých lidí žijou na světě. Devadesat procent z nich bydli v rozvojové zemi a 65\% jsou starší 50 let\citep{whodata}.

Gramotnost nevídomých je schopnost čist rukama, to je, čist braillská písma.

V vyspělých zemich gramotnost souvisly s zaměstnanost. Podle statistiky zveřejněné v spojených statů 90\% procent gramotných nevídomých lidí jsou zaměstnání ale jenom 30\% negramotných nevidomých jsou zaměstnání.

Nenašel jsem podobně čísla pro české republiky, ani rozojový svět.  Do jisté míru, statistiky jsou zkresleně.  V statech gramotnost značně poklesa.  V roce 1960, 50\% nevidomých amerických žáků se učili číst rukama, dneska jenom 12\% se učí číst.  Dříve, mnohou lidí ztratili zrak kvůli infekčích nemoci, dneska skoro nikdo. Dale, už umíme lečít bílý zakal, a ve vyspělých zemi ne-existuje nevidomost kvůli bilých zakalům.  To znamená, že daleko větší proporce nevidomých lidí ve vyspělem světě maji vedlejší nemoc, která překáží gramotnosti\citep{perkins,whozakal,whodata}.

Je další důvod proč gramotnost nevidomých je tak nízká.  Materiály v braillů jsou velmi omezený.  V Knihovně a tiskárna pro nevidomé K.E.Macana jsou 5693 titulů tisknutí braillské písmo\citep{biblio}. V Městské Knihovně v Praze jsou 387625 knih\citep{mlp}.

Složení nevidomost v rozvojových zemi je zvyrazně jiná. 80\% nevidomost je či lečitelná anebo nevyhnutenlá\citep{whodata}.  Však celý 50\% nevidomost na světě je kvůli bilému zakalu\citep{whozakal}.

Bilý zakal není jen lečitelný. Je to docela snadno a levně lečitelný. V Indie, chyrurgie pro bilý zakal soji mezi 1 000 a 1 700 USD\citep{cataractsindia}. Ztoho, víme, že většina nevidomých lidí na světě jsou velmi chudé.

\section{Co to je Braillovo písmo?}

Braillovo písmo je hmatní abeceda, které lze číst a psát bez zraku.


Každé písmeno má dvě sloupce zvýšených bodů. Podle toho, kolik bodů jsou zvýšený, a kde jsou mezery mezi body víme, které písmeno je napsáno.

Braillovo písmo je jiné v každém země.  V čechách, použiváme šesti bodové písmo pro tisk, a osmibodové písmo na počitače.  V německu, písmo je vždy osmibodové.

Body jsou čislované takhle:

% putStrLn $ utf8ToBrailleBox "1⠁2⠂3⠄4⠈5⠐6⠠7⡀8⢀"
\begin{tabular}{|c|c|}
\hline
1\braillebox{1       }&4\braillebox{   4    }\\
2\braillebox{ 2      }&5\braillebox{    5   }\\
3\braillebox{  3     }&6\braillebox{     6  }\\
7\braillebox{      7 }&8\braillebox{       8}\\
\hline
\end{tabular}

%%%%%%%%%%%



Tady je česká braillská abeceda:

% putStrLn $ utf8ToBrailleBox "a ⠁ b ⠃ c ⠉ d ⠙ e ⠑ f ⠋ g ⠛ h ⠓ i ⠊ j ⠚ k ⠅ l ⠇ m ⠍ n ⠝ o ⠕ p ⠏ q ⠟ r ⠗ s ⠎ t ⠞ u ⠥ v ⠧ x ⠭ y ⠽ z ⠵ ý ⠯ w ⠷ ž ⠮ ů ⠾ á ⠡ ě ⠣ č ⠩ ď ⠹ š ⠳ ň ⠫  ť ⠳ ó ⠪ ř ⠺ í ⠌ é ⠜ ú ⠬ "

a \braillebox{1       } b \braillebox{12      } c \braillebox{1  4    } d \braillebox{1  45   } e \braillebox{1   5   } f \braillebox{12 4    } g \braillebox{12 45   } h \braillebox{12  5   } i \braillebox{ 2 4    } j \braillebox{ 2 45   } k \braillebox{1 3     } l \braillebox{123     } m \braillebox{1 34    } n \braillebox{1 345   } o \braillebox{1 3 5   } p \braillebox{1234    } q \braillebox{12345   } r \braillebox{123 5   } s \braillebox{ 234    } t \braillebox{ 2345   } u \braillebox{1 3  6  } v \braillebox{123  6  } x \braillebox{1 34 6  } y \braillebox{1 3456  } z \braillebox{1 3 56  } ý \braillebox{1234 6  } w \braillebox{123 56  } ž \braillebox{ 234 6  } ů \braillebox{ 23456  } á \braillebox{1    6  } ě \braillebox{12   6  } č \braillebox{1  4 6  } ď \braillebox{1  456  } š \braillebox{12  56  } ň \braillebox{12 4 6  }  ť \braillebox{12  56  } ó \braillebox{ 2 4 6  } ř \braillebox{ 2 456  } í \braillebox{  34    } é \braillebox{  345   } ú \braillebox{  34 6  } 
% -- https://github.com/timthelion/utf8-to-braillebo://github.com/timthelion/utf8-to-braillebox

dejte pozor na to, že nejsou tam uvedené velké písmo ani čísla.  V šestibodové braill píšeme velký A jako
% putStrLn $ utf8ToBrailleBox "⠠⠁"
\braillebox{     6  }\braillebox{1       }
%%%%
malý a nasleduje samotný šesti bod. To je, že velké braillské písmena jsou escapování. Samotní šestí bod je escape sekvence pro velká písmena.  Čísla jsou taky escapování. Escape sekvence pro čísla je
% putStrLn $ utf8ToBrailleBox "⠼"
\braillebox{  3456  }
%%%%
tak, že čísla v šestibodové braillu jsou:

% putStrLn $ utf8ToBrailleBox "1⠼⠁2⠼⠃3⠼⠉4⠼⠙5⠼⠑6⠼⠋7⠼⠛8⠼⠓9⠼⠊0⠼⠚"
1\braillebox{  3456  }
\braillebox{1       }
2\braillebox{  3456  }\braillebox{12      }
3\braillebox{  3456  }\braillebox{1  4    }
5\braillebox{  3456  }\braillebox{1   5   }
6\braillebox{  3456  }\braillebox{12 4    }
7\braillebox{  3456  }\braillebox{12 45   }
8\braillebox{  3456  }\braillebox{12  5   }
9\braillebox{  3456  }\braillebox{ 2 4    }
0\braillebox{  3456  }\braillebox{ 2 45   }
%%%%

V osmibodové braill nepoužíva escape sekvence. Velký A je
% putStrLn $ utf8ToBrailleBox "⡁"
\braillebox{1     7 }
%%%%
. A čísla jsou podznačené osmím bodem: 1 je
% putStrLn $ utf8ToBrailleBox "⢁"
\braillebox{1      8}
%%%%
. Ne-existuje žádný standard pro osmibodové braillské písmo.\citep{6dotbraille}

\section{Jak funguje braillský řádek a jak funguje FCHAD?}

\subsection{Braillský Řádek}
Už jsem zminíl, že existuje malo braillských knih. Do velké míru braill má podstatu jako grafické rozehraní u počitačů.  Nevídomé lidí mohou použivat či braill nebo hlasový vystup když něco čtou na počitač, ale aby mohly pochopit a upravit složité, formatované texty, potřebujou braillský řádek.

Braillský řádek je elektromechanické zařízení, který "zobrazuje" jeden řádek braillského textu.  Braillský řádek stojí dva až 8 tisic dolarů a zobrazí mezi 40 a 80 znaků.  Obecně se platí za znak.  Braillský řádek který zobrazuje jen 12 znaků stoji "jen" tisic dolarů\citep{perkinsdisplays}.

Dnešní braillské řádky používaji piezoelektrickou technologie.  Piezoelektrické krystaly jsou krystaly, které změní velikost když jím projde elektrický proud. Piezoelektrické řádky maji 16 malých krystalů za každý znak.  Samotné krystaly jsou dost drahé, a to je vidět v ceně braillského řádků ale to není jediný problem.  Ještě větší cena než cena samotného řádku, je cena udřby.  Prah, který vleze do malé díry v řádce muže poškodit piezoelektrický mechanismu.  Když kupujete řádek od americké nezískovky Perkins například, byste měl taky zařídit smlouvu na udřbu, která stoji mezi 300 a 500 dolarů USD\citep{perkinsdisplays}.

\subsection{FCHAD}

Už par let pracují na nový vynalez, tzn. FCHAD. FCHAD je jednoduší varianta braillského řádku, který je odolnější a levněší.  FCHAD, který jsem použival v tehle studie stalo míň než sto dolarů vyrobit.

Není to žadná genialita, že jsem dokazal vyrobit řádek, který stoji o tolik nasobek míň.  FCHAD řádek je jednoznačně horší než peizoelektrický řádek.  Hlavně, že je to levný!

Najdete obrazky FCHADu v appendixu, kde jsou zobrazené dvě verzi FCHADu, které byly použity v studie.  FCHAD je zkratka pro "Fast Character Display" tj rychly zobrazovač písmena.  FCHADu umí zobrazit jen jendo písmeno na řaz.  Ale to neznamena, že nemužete přečist delší texty s FCHADem.  FCHAD má ještě další komponent a to je kurzor selektor.  Obrazek 1 B a C ukazujou dvě ruzné selektory.  Selektor může taky byt počitačový myš.

Už někomu napadlo, že by to bylo levný zobrazit jen jedno písmeno. Však mnoho FCHADu již existujou ale žádné na trhu nejsou.

\section{Význam pedagogiky v oblasti zařízení pro nevidomé}

Uznavam, že nevím proč tolik FCHADu byly navržené aniž by dostali na trhu.  Když jsem se zeptal různé lidí, které s braillem pracovali, jsem slyšel často cynický nazor, že je větší zisk vyrobit drahý piezo než levný FCHAD.

Dale, mnoho vynalezců pracujou v ramci akademie.  Sam nemaji možnost založet firmu.  Musím řict, že je třeba hodně trpělivost a stalost v praci přenašet nový vynalez na trhu.  Nejsem ani polovinu cestu, a už mě trvalo vic jako rok.

Velázquez, kdy píše o nositelných pomocníků nevidomých má jiný napad proč tolik zařízení se strateji na cestě.  V zavěru vlastní studie piše:

"Přes obrovské úsilí batatelů, a přes obrovské množství technologii na trhu, nevidomých lidí odmitaji novou technologie. Audoknihy, braiské řádky, bílý hůl, a vodicí pes jsou stale nejoblibenější technologii u nevidomých.

Přijatý portabilných a nositelných technologie je jeden z nejtěžších ukolů pro badatele.  Motivace, kooperace, optimismus, ochotnost/schopnost se učit anebo přizpusobit se není nikdy samolžejmost u nevidomých."\citep{velazquez2010wearable}%Despite efforts and the great variety of wearable assistive devices available, user acceptance is quite low. Audio books and Braille displays (for those who can read Braille) and the white cane and guide dog will continue to be the most popular reading/travel assistive devices for the blind.
% Acceptance of any other portable or wearable assistive device is always a challenge in blind population. Motivation, cooperation, optimism, willingness/ability to learn or adapt new skills is not a combination that can be taken for granted.

Sam jsem zažil neochota od nevidomých lidí. První nevidomý člověk, který jsem ukazoval stroj vůbec nepochopil jak to má fungovat ale byl jistý, že to není dobrý.

Učit, a učit i přes neochota žáka, je ukolem pedagoga.  Naštěstí, většina nevidomých lidí ochotní se učit jsou.  Trva ale, že nezanedbatelný čast role badatela je jako pedagog a nezanedbatelný čast ukolu pedagoga je pomahat žáka navigovat obrovské volny nových technologii první a dvacatého stoleti.

\section{Co vlástně výzkoumáme?}

Moje otazka je, jak lidí reagujou na neznamý nastroj během první 45 minut použity?  Jak oni přispusobujou k novému nastroje?  Jaké maji potom dojmy?  Lze z těchto setkání taky kreslit křívky učení.

Seděl jsem s šesti nevidomým lidmi vyzkoušet svoj FCHAD. Učastníki četli kratky text s FCHADem a pak řekli své dojmy.  První čtyřy krat jsem dorazil na technické potiže, ale mám platný vysledky z tři učastníků.

