
\chapter{Úvod}

\section{Nevidomost na světě a význam Braillova písma}

Nevidomost je fenomen rozvojových zemí a starších lidech. 39 milionů nevidomých lidí žijí na světě. Devadesát procent z nich bydli v rozvojových zemích a 65\% jsou starší 50 let\citep{whodata}.

Gramotnost nevidomých je schopnost čist Braillova písma.  Mezi sebou nevidomé říkají číst rukama.

V vyspělých zemí gramotnost souvislí s zaměstnaností. Podle statistiky zveřejněné v spojených státech 90\% procent gramotných nevidomých lidí jsou zaměstnání oproti jenom 30\% negramotných nevidomých lidí.

Nenašel jsem podobná čísla pro české republiky ani rozvojový svět.  Do jisté míru statistiky jsou zkreslené.  V státech gramotnost značně poklesá.  Padesát procent nevidomých amerických žáků se učili číst rukama v roce 1960. Dneska jenom dvanáct procent.  Dříve mnohou lidí ztratili zrak kvůli infekčním nemoci dneska skoro nikdo. Dále už umíme léčit bílý zákal, a ve vyspělých zemí ne-existuje nevidomost kvůli bílých zákalům.  Jsou miň nevidomé lidi ale ty nevidomé lidi, kteří jsou jsou vice nemocné.  Daleko větší proporce nevidomých lidí ve vyspělém světě mají vedlejší nemoc, která překáží gramotnosti.  Mnohou nevidomých trpěli předčasného porodu anebo jiné perinatalním závad\citep{perkins,whozakal,whodata}.

Je další důvod proč gramotnost nevidomých je tak nízká.  Materiály v braillském písmu jsou velmi omezený.  V Knihovně a tiskárna pro nevidomé K.E.Macana jsou 5693 titulů tisknutí braillském písmu\citep{biblio}. V Městské Knihovně v Praze jsou 387625 knih pro vidící \citep{mlp}.

Složení nevidomost v rozvojových zemí je výrazně jiná než ve vyspělém světě. Osmdesát procent nevidomost na světě je či léčitelná anebo nevyhnutelná\citep{whodata}.  Však celý 50\% nevidomosti na světě je kvůli bílému zákalu\citep{whozakal}.

Bílý zákal není jen léčitelný. Je to docela snadno a levně léčitelný. Chirurgie pro bílý zákal soji mezi 1 000 a 1 700 USD v Indii\citep{cataractsindia}. Z toho víme, že většina nevidomých lidí na světě jsou velmi chudé.

\section{Co to je Braillovo písmo?}

Braillovo písmo je hmatní abeceda, které lze číst a psát bez zraku.

Každé písmeno má dvě sloupce zvýšených bodů. Podle toho kolik bodů jsou zvýšené a kde jsou mezery mezi body víme, které písmeno je napsáno.

Braillovo písmo je jiné v každém země.  V Čechách používáme šestibodové písmo pro tisk a osmibodové písmo na počítače.  V Německu písmo je vždy osmibodové.

Abychom mohli mluvit o braillských písmenech existuje číselní označení pro body.

% putStrLn $ utf8ToBrailleBox "1⠁2⠂3⠄4⠈5⠐6⠠7⡀8⢀"
\begin{tabular}{|c|c|}
\hline
1\braillebox{1       }&4\braillebox{   4    }\\
2\braillebox{ 2      }&5\braillebox{    5   }\\
3\braillebox{  3     }&6\braillebox{     6  }\\
7\braillebox{      7 }&8\braillebox{       8}\\
\hline
\end{tabular}

%%%%%%%%%%%

A tady mate česká braillská abeceda:

% putStrLn $ utf8ToBrailleBox "a ⠁ b ⠃ c ⠉ d ⠙ e ⠑ f ⠋ g ⠛ h ⠓ i ⠊ j ⠚ k ⠅ l ⠇ m ⠍ n ⠝ o ⠕ p ⠏ q ⠟ r ⠗ s ⠎ t ⠞ u ⠥ v ⠧ x ⠭ y ⠽ z ⠵ ý ⠯ w ⠷ ž ⠮ ů ⠾ á ⠡ ě ⠣ č ⠩ ď ⠹ š ⠳ ň ⠫  ť ⠳ ó ⠪ ř ⠺ í ⠌ é ⠜ ú ⠬ "

a \braillebox{1       } b \braillebox{12      } c \braillebox{1  4    } d \braillebox{1  45   } e \braillebox{1   5   } f \braillebox{12 4    } g \braillebox{12 45   } h \braillebox{12  5   } i \braillebox{ 2 4    } j \braillebox{ 2 45   } k \braillebox{1 3     } l \braillebox{123     } m \braillebox{1 34    } n \braillebox{1 345   } o \braillebox{1 3 5   } p \braillebox{1234    } q \braillebox{12345   } r \braillebox{123 5   } s \braillebox{ 234    } t \braillebox{ 2345   } u \braillebox{1 3  6  } v \braillebox{123  6  } x \braillebox{1 34 6  } y \braillebox{1 3456  } z \braillebox{1 3 56  } ý \braillebox{1234 6  } w \braillebox{123 56  } ž \braillebox{ 234 6  } ů \braillebox{ 23456  } á \braillebox{1    6  } ě \braillebox{12   6  } č \braillebox{1  4 6  } ď \braillebox{1  456  } š \braillebox{12  56  } ň \braillebox{12 4 6  }  ť \braillebox{12  56  } ó \braillebox{ 2 4 6  } ř \braillebox{ 2 456  } í \braillebox{  34    } é \braillebox{  345   } ú \braillebox{  34 6  } 
% -- https://github.com/timthelion/utf8-to-braillebo://github.com/timthelion/utf8-to-braillebox

Dejte pozor na to, že nejsou tam uvedené velké písmo ani čísla. Velká braillská písmena jsou escapování. V šestibodové braillských písmenech píšeme velký A jako
% putStrLn $ utf8ToBrailleBox "⠠⠁"
\braillebox{     6  }\braillebox{1       }
%%%%
. Samotní šestí bod je escape sekvence pro velká písmena. Čísla jsou taky escapování. Escape sekvence pro čísla je
% putStrLn $ utf8ToBrailleBox "⠼"
\braillebox{  3456  }
%%%%
tak, že čísla v šestibodové braillských písmenech jsou:

% putStrLn $ utf8ToBrailleBox "1⠼⠁2⠼⠃3⠼⠉4⠼⠙5⠼⠑6⠼⠋7⠼⠛8⠼⠓9⠼⠊0⠼⠚"
1\braillebox{  3456  }\braillebox{1       }
2\braillebox{  3456  }\braillebox{12      }
3\braillebox{  3456  }\braillebox{1  4    }
5\braillebox{  3456  }\braillebox{1   5   }
6\braillebox{  3456  }\braillebox{12 4    }
7\braillebox{  3456  }\braillebox{12 45   }
8\braillebox{  3456  }\braillebox{12  5   }
9\braillebox{  3456  }\braillebox{ 2 4    }
0\braillebox{  3456  }\braillebox{ 2 45   }
%%%%

Osmibodové Braillovo písmo nemá escape sekvence.  Sedmi a osmi body se používají k číslům a kapitalizaci. Velký A je
% putStrLn $ utf8ToBrailleBox "⡁"
\braillebox{1     7 }
%%%%
. A čísla jsou podznačené osmým bodem: 1 je
% putStrLn $ utf8ToBrailleBox "⢁"
\braillebox{1      8}
%%%%
. Ne-existuje žádný standard pro osmibodové braillské písmo\citep{6dotbraille}.  Protože osmibodové braillské písmo je obecně zobrazené na počítače uživatele mohou nastavit vlastní kódování.

Existuje Braillovo kódování pro hudební noty i matematické rovnice.

\section{Jak funguje braillský řádek a jak funguje FCHAD?}

\subsection{Braillský Řádek}

Už jsem zmínil, že existuje málo braillských knih. Braillovo písmo má podstatu hlavně jako grafické rozehraní u počítačů.  Mnoho nevidomých lidí používají hlasový vystup ale aby mohly pochopit a upravit složité formátované texty potřebují braillský řádek.

Braillský řádek je elektromechanické zařízení, který \uv{zobrazuje} jeden řádek braillského textu.  Braillský řádek stojí 2 až 8 tisíc dolarů a zobrazí 40 až 80 znaků.  Obecně cena stoupá s výším počtem znaků.  Braillský řádek, který zobrazuje jen 12 znaků stoji jen tisíc dolarů\citep{perkinsdisplays}.

Dnešní braillské řádky používají piezoelektrickou technologie.  Piezoelektrické krystaly jsou krystaly, které změní velikost když jím projde elektrický proud. Piezoelektrické řádky mají 16 dlouhých krystalů za každý znak.  Samotné krystaly jsou dost drahé a to je vidět v ceně braillského řádku ale to není jediný problém.  Ještě větší cena než cena samotného řádku je cena údržby.  Prach, který vleze do malé díry v řádce muže poškodit piezoelektrický mechanismus.  Když kupujete řádek od americké nezískovky Perkins například byste měl taky zařídit smlouvu na údržbu, která stoji mezi 300 a 500 dolarů USD\citep{perkinsdisplays}.

\subsection{FCHAD}

Už par let pracují na nový vynález tzn. FCHAD. FCHAD je jednoduší varianta braillského řádku, který je odolnější a levnější.  FCHAD, který jsem používal v studie stalo miň než sto dolarů vyrobit.

Není to žádná genialita, že jsem dokázal vyrobit řádek, který stoji o tolik krát miň.  FCHAD řádek je jednoznačně horší než peizoelektrický řádek.  Čte se na to pomalejší a je nutný používat oba ruce k čtení.  Hlavně, že je to levný!

Najdete obrázky FCHADu v apendixu A2 kde jsou zobrazené dvě verzi FCHADu, které byly použity v studie.

FCHAD je zkratka pro \uv{Fast Character Display} tj rychlý zobrazovač písmen.  FCHAD umí zobrazit jenom jedno písmeno naráz ale to neznamená, že nemůžete přečíst delší texty s FCHADem.  FCHAD má ještě další komponent a to je selektor.  Obrázek 1 B a C jsou dvě různé selektory.  Selektor může taky byt počítačový myš.  Selektorem můžete změnit písmeno, které je zobrazené.

Už někomu napadlo, že by to bylo levný zobrazit jen jedno písmeno. Však mnoho FCHADu již existují ale žádné na trhu nejsou.  Můžete prohlídnout různé FCHADy v sekce \uv{Analýza a srovnání.}

\section{Význam pedagogiky v oblasti zařízení pro nevidomé}

Uznávám, že nevím proč tolik FCHADu byly navržené aniž by dostali na trhu.  Když jsem se zeptal různé lidí, které s Braillským písmenem pracovali jsem slyšel často cynický názor, že je větší zisk vyrobit drahý piezo než levný FCHAD.

Mnoho vynálezců pracují v rámci akademie. Sám nemají možnost založit firmu. Třeba vynalezli ty nastroje ale nenašli nikoho, který je chtěl vyrobit. Musím říct, že je třeba hodně trpělivost a stálost v práci přenášet nový vynález na trhu.  Nejsem ani polovinu cestu a už mě trvalo vice než rok.

Velázquez, kdy píše o přenosných pomocníků nevidomých má jiný názor proč tolik zařízení se ztratí na cestě.  V závěru vlastní studie píše:
\em
\uv{Přes obrovské úsilí badatelů, a přes obrovské množství technologii na trhu, nevidomých lidí odmítají novou technologie. Audio knihy, braillské řádky, bílý hůl, a vodicí pes jsou stále nejoblíbenější technologii u nevidomých.

Přijatý přenositelných technologie je těžký úkolů pro badatele.  Motivace, kooperace, optimismus, ochotnost/schopnost se učit anebo přizpůsobit se není nikdy samozřejmost u nevidomých.}\em \citep{velazquez2010wearable}%Despite efforts and the great variety of wearable assistive devices available, user acceptance is quite low. Audio books and Braille displays (for those who can read Braille) and the white cane and guide dog will continue to be the most popular reading/travel assistive devices for the blind.
% Acceptance of any other portable or wearable assistive device is always a challenge in blind population. Motivation, cooperation, optimism, willingness/ability to learn or adapt new skills is not a combination that can be taken for granted.

Sám jsem zažil neochota od nevidomých lidí. První nevidomý člověk, který jsem ukazoval stroj vůbec nepochopil jak to má fungovat ale byl jistý, že to není dobrý.

Spousta technologie musejí čekat až lidi jsou na ně zvykle aby mohly byt použité.  Pamatují si na mámu a jak ona stále hledá na mapě na papírové mapě cestu, kterou by našla několik krát rychlejší přes internetem.  A myslím o tom jak dlouho ještě hledala v telefonním seznamu a pak volala pro informace, kterou lze jednoduše přečíst na internetu.  Vidím jasně, že zásadnější část než všimneme, technologický rozvoj je řízený technologické učení a zvyklost světové obyvatelství.  Pedagog a pedagogické teorii sobě najdou uprostřed velké civilizační pokroky a i když pedagogové samotné jsou často nejhorší na tom, když to jde o adaptace na novou technologii, učit a učit i přes neochota žáka je úkolem pedagoga, a je úkolem pedagoga pomáhat žákovi navigovat světě, které je ve stavu trvalé změny.  Ještě štěstí většina nevidomých lidí ochotní se učit jsou.

\section{Co vlastně vyzkoušíme?}

Moje otázka je, jak lidí reagují na neznámý nastroj během první 45 minut použity?  Jak oni přizpůsobují k novému nastroje?  Jaké mají potom dojmy?  Lze z těchto setkání taky kreslit křivky učení.

Zajímavý vlastnost studie je v tom, že FCHAD je úplně neznámý pro účastnici.  Nikdy v životě neslyšeli zkratu FCHAD a nikdy nesetkali dříve s podobným nástroji.  Použity FCHADu je složitý motorický a konceptuálně.

Seděl jsem s šesti nevidomým lidmi vyzkoušet svůj FCHAD. Účastníci četli krátký text s FCHADem a pak řekli své dojmy.  První čtyři krát jsem dorazil na technické potíže, ale mám platný výsledky z tři účastníků.
