\subsubsection{Žák 5}
Setkání 12:00 20.3.2013

Paty žák je 20-25 let starý.  Jeho nejvyšší úroveň dosažené vzdělávání je maturita. Je úplně nevidomý.  Taký čte Braillova písma od základní školy.  Používá braillský řádek.

Kdy začal číst, zase selektor zle choval.  Musel jsem přenastavit prahy.  To mně trval asi deset minut.  To znamenalo, že potom co on četl o stroje, a potom co on teprve dal ruce na stroje, on seděl a čekal.

Kdy četl tisknutí text nejdřív se zeptal, jestli má číst z obou stran.  Kdy četl, otočil stranu velmi rychle, tak že celou dobu plynule četl bez jakékoliv přestání.  Přečetl jedno slovo nesprávně: četl \uv{nejlevnější} jako \uv{nejlepší}.

Četl tiskový text obě ruce.

% začatek 176.8 otočení 192.9 konec 212.6

Čas na čtení celého textu:35.8 vteřin

CPS: 11.37

Čas na čtení druhé strany tiskového textu:19.7 vteřin

CPS: 13.45

Nejdřív jsem ho ukazoval selektor.  Popisoval jsem to jako \uv{dvanáct kovových bodů, vypadají jako písmo[sic] `k'.}.  Pak jsem ho ukazoval zobrazovač.  Řekl jsem ho, že \uv{má to osm dír[sic] a, že z každé z nich skáče kovový tyč.}

\paragraph{První Řádek}
\begin{tabular}{|c|c|c|c|c|c|c|c|c|c|c|c|}
\hline
B&r&a&i&l&l&s&&&&&\\
\braillebox{1278}&\braillebox{1235}&\braillebox{1}&\braillebox{24}&\braillebox{123}&\braillebox{123}&\braillebox{234}&\braillebox{}&\braillebox{2358}&\braillebox{123}&\braillebox{}&\braillebox{}\\
\hline
\end{tabular}

Přečetl první písmeno správně ale potom začali skákat tyče na zobrazovače náhodně, jsem uvědomoval, že senzory byly přecitlivé.  Musel jsem snížit citlivost aby fungovaly a protože jsem to dělal poprvé, trvalo mě to asi 10 minut.

Vedl jsem jeho ruku nad první tří senzory(abych věděl, že fungují) a řekl jsem na hlas `b' `r' `a'.  Tak že druhý dvě písmena sám nečetl.  Potřeboval jsem namočit jeho prst, aby ty senzory fungovaly. Tak jsem se ho zeptal kde mohu najít vodu a on mě řekl o umyvadle.  Namočil jsem jeho pravou ukazovátku a začaly jsme číst.

Četl začátek řádku správně. Jedinou chybu byl, že četl `s' jako `i'. Řekl to správně potom co jsem ho říkal, že to není správný.

\paragraph{Druhý Řádek}
\begin{tabular}{|c|c|c|c|c|c|c|c|c|c|c|c|}
\hline
k&ý& &ř&á&d&e&k&,& &k&t\\
\braillebox{1378}&\braillebox{12346}&\braillebox{}&\braillebox{2456}&\braillebox{16}&\braillebox{145}&\braillebox{15}&\braillebox{13}&\braillebox{2}&\braillebox{}&\braillebox{13}&\braillebox{2345}\\
\hline
\end{tabular}

Četl správně až na `,'.  Choval se, jak že nerozumí.  Zeptal jsem se, který bod je nahoru.  On správně říkal, že druhý.  Říkal jsem ho, že druhý bod je čárka.  Myslel, že je u konce řádku kdy dorazil na `k'.

\paragraph{Třetí Řádek}
\begin{tabular}{|c|c|c|c|c|c|c|c|c|c|c|c|}
\hline
e&r&ý& &t&e&ď& &p&o&u&ž\\
\braillebox{1578}&\braillebox{1235}&\braillebox{12346}&\braillebox{}&\braillebox{2345}&\braillebox{15}&\braillebox{1456}&\braillebox{}&\braillebox{1234}&\braillebox{135}&\braillebox{136}&\braillebox{2346}\\
\hline
\end{tabular}

Četl správně až na druhý `e'.  Ho zřejmě zmátl, to že body nespadnou samo. Kvůli nekvalitě nahrávky nevím co odhadl ale říkal \uv{nějaké body} a jsem odpověděl \uv{ano, nepadnou když netlačíte na ně}.  Zřejmě už sledoval slova, protože potom co přečetl \uv{teď} po jednotlivých písmenech, říkal \uv{teď}.

\paragraph{Čtvrtý Řádek}
\begin{tabular}{|c|c|c|c|c|c|c|c|c|c|c|c|}
\hline
í&v&á&t&e&,& &n&e&n&í& \\
\braillebox{3478}&\braillebox{1236}&\braillebox{16}&\braillebox{2345}&\braillebox{15}&\braillebox{2}&\braillebox{}&\braillebox{1345}&\braillebox{15}&\braillebox{2345}&\braillebox{34}&\braillebox{}\\
\hline
\end{tabular}

Četl `t' nejdřív jako `j'.  Četl první `n' nejdřív jako `d'.  Přeskočil druhý `e' a byl zase na `n'.  Říkal jsem ho, ať se vrátí zpátky a on to dělal a četl zbytek řádku správně.

\paragraph{Pátý Řádek}
\begin{tabular}{|c|c|c|c|c|c|c|c|c|c|c|c|}
\hline
p&r&v&n&í& &ř&á&d&e&k&,\\
\braillebox{123478}&\braillebox{1235}&\braillebox{1236}&\braillebox{1345}&\braillebox{34}&\braillebox{}&\braillebox{1235}&\braillebox{16}&\braillebox{145}&\braillebox{15}&\braillebox{13}&\braillebox{2}\\
\hline
\end{tabular}

Četl všechno správně.

\paragraph{Šestý Řádek}
\begin{tabular}{|c|c|c|c|c|c|c|c|c|c|c|c|}
\hline
 &k&t&e&r&ý& &z&o&b&r&a\\
\braillebox{78}&\braillebox{13}&\braillebox{2345}&\braillebox{15}&\braillebox{1235}&\braillebox{12346}&\braillebox{}&\braillebox{1356}&\braillebox{135}&\braillebox{12}&\braillebox{1235}&\braillebox{1}\\
\hline
\end{tabular}

U začátek řádku byl nějak překvapený.  Vysvětlil jsem mu, že body 7 a 8 \uv{znamenají} začátek řádku.  Jinak četl správně.

\paragraph{Řádek 7}
\begin{tabular}{|c|c|c|c|c|c|c|c|c|c|c|c|}
\hline
z&u&j&e& &j&e&n&o&m& &j\\
\braillebox{135678}&\braillebox{136}&\braillebox{245}&\braillebox{15}&\braillebox{}&\braillebox{245}&\braillebox{15}&\braillebox{1345}&\braillebox{135}&\braillebox{134}&\braillebox{}&\braillebox{245}\\
\hline
\end{tabular}

Četl všechno správně.

\paragraph{Řádek 8}
\begin{tabular}{|c|c|c|c|c|c|c|c|c|c|c|c|}
\hline
e&d&n&o& &p&í&s&m&e&n&o\\
\braillebox{1578}&\braillebox{145}&\braillebox{1345}&\braillebox{135}&\braillebox{}&\braillebox{1234}&\braillebox{24}&\braillebox{234}&\braillebox{134}&\braillebox{15}&\braillebox{1345}&\braillebox{135}\\
\hline
\end{tabular}

Četl všechno správně a jsem přestal potvrdit jeho čtení. Pote, zřejmě zrychlil.

\paragraph{Řádek 9}
\begin{tabular}{|c|c|c|c|c|c|c|c|c|c|c|c|}
\hline
.& & &P&r&v&n&í& &b&y&l\\
\braillebox{378}&\braillebox{}&\braillebox{}&\braillebox{12347}&\braillebox{1235}&\braillebox{1236}&\braillebox{1345}&\braillebox{34}&\braillebox{}&\braillebox{12}&\braillebox{13456}&\braillebox{123}\\
\hline
\end{tabular}

Četl všechno správně. Zeptal se, jestli má říct, že `P' byl velký.

\paragraph{Řádek 10}
\begin{tabular}{|c|c|c|c|c|c|c|c|c|c|c|c|}
\hline
 &v&y&n&a&l&e&z&e&n& &v\\
\braillebox{78}&\braillebox{1236}&\braillebox{13456}&\braillebox{1345}&\braillebox{1}&\braillebox{123}&\braillebox{15}&\braillebox{1356}&\braillebox{15}&\braillebox{1345}&\braillebox{}&\braillebox{1236}\\
\hline
\end{tabular}

Přečetl všechno správně.

\paragraph{Řádek 11}
\begin{tabular}{|c|c|c|c|c|c|c|c|c|c|c|c|}
\hline
 &r&o&c&e& &1&9&1&3& &v\\
\braillebox{78}&\braillebox{1235}&\braillebox{135}&\braillebox{14}&\braillebox{15}&\braillebox{}&\braillebox{18}&\braillebox{248}&\braillebox{18}&\braillebox{148}&\braillebox{}&\braillebox{1236}\\
\hline
\end{tabular}

Kdy dorazil na datum, říkal \uv{předpokládám, že to asi bude číslo    jedna devět   to znamená devatenáct set třináct}. Četl všechno správně.

\paragraph{Řádek 12}
\begin{tabular}{|c|c|c|c|c|c|c|c|c|c|c|c|}
\hline
 &A&n&g&l&i&i&.& & &J&m\\
\braillebox{78}&\braillebox{17}&\braillebox{1345}&\braillebox{1245}&\braillebox{123}&\braillebox{24}&\braillebox{24}&\braillebox{3}&\braillebox{}&\braillebox{}&\braillebox{2457}&\braillebox{134}\\
\hline
\end{tabular}

Četl `n' nejdřív jako `d'.  Četl `.' nejdřív jako mezera.  Jinak četl správně.

\paragraph{Řádek 13}
\begin{tabular}{|c|c|c|c|c|c|c|c|c|c|c|c|}
\hline
e&n&o&v&a&l& &s&e& &O&p\\
\braillebox{1578}&\braillebox{1345}&\braillebox{135}&\braillebox{1236}&\braillebox{1}&\braillebox{123}&\braillebox{}&\braillebox{234}&\braillebox{15}&\braillebox{}&\braillebox{1357}&\braillebox{1234}\\
\hline
\end{tabular}

Četl všechno správně kromě `O' což původně četl jako `n'.

\paragraph{Řádek 14}
\begin{tabular}{|c|c|c|c|c|c|c|c|c|c|c|c|}
\hline
t&o&f&o&n& &a& &p&ř&e&v\\
\braillebox{234578}&\braillebox{135}&\braillebox{124}&\braillebox{135}&\braillebox{1345}&\braillebox{}&\braillebox{1}&\braillebox{}&\braillebox{1234}&\braillebox{2456}&\braillebox{15}&\braillebox{1236}\\
\hline
\end{tabular}

Četl všechno správně.

\paragraph{Řádek 15}
\begin{tabular}{|c|c|c|c|c|c|c|c|c|c|c|c|}
\hline
á&d&ě&l& &s&v&ě&t&l&o& \\
\braillebox{1678}&\braillebox{145}&\braillebox{126}&\braillebox{123}&\braillebox{}&\braillebox{234}&\braillebox{1236}&\braillebox{126}&\braillebox{2345}&\braillebox{123}&\braillebox{135}&\braillebox{}\\
\hline
\end{tabular}

Četl všechno správně.

\paragraph{Řádek 16}
\begin{tabular}{|c|c|c|c|c|c|c|c|c|c|c|c|}
\hline
n&a& &z&v&u&k&.& & &K&d\\
\braillebox{134578}&\braillebox{1}&\braillebox{}&\braillebox{1356}&\braillebox{1236}&\braillebox{136}&\braillebox{13}&\braillebox{3}&\braillebox{}&\braillebox{}&\braillebox{137}&\braillebox{145}\\
\hline
\end{tabular}

Četl `z' nejdřív jako `e'. Jinak četl správně.

\paragraph{Řádek 17}
\begin{tabular}{|c|c|c|c|c|c|c|c|c|c|c|c|}
\hline
y&ž& &č&t&e&n&á&ř& &p&o\\
\braillebox{1345678}&\braillebox{2346}&\braillebox{}&\braillebox{146}&\braillebox{2345}&\braillebox{15}&\braillebox{1345}&\braillebox{16}&\braillebox{2456}&\braillebox{}&\braillebox{1234}&\braillebox{135}\\
\hline
\end{tabular}

Četl všechno správně.

Dokončil jsem setkání a on mě říkal:
\uv{to se hrozně špatně jako totiž poznává když člověk je zvyklý číst Braillovo písmo ... Nevlastně jenom tím, že ten taktilní bod má určitý zvyklost vlastně, jo, že pro mě je to těžší poznat}

Tim: \uv{Vy jste moc nepoložil tu ruku rovně na ten článek musím říct, že Já čtu docela rychlejší než jste četl vy teď na ten stroj tak m I když určitě čtu rukama pomalejší než vy}

Žák : \uv{jo jako jo jenom Já si myslím, že prostě, nevím no ale prostě, jenom že někomu kdo už má zkušenosti s normálním Braillovo písmo od začátku trošku problém}

Tim: \uv{poznal jste, že jsou tam slova?}

Žák: \uv{ano}
Tim: \uv{Rozuměl jste?}
Žák: \uv{Ano}

Pozoroval jsem, že pro něj bylo těžký zvyknout na malý selektor.  Byl zvyklý používat dlouhý braillský řádek. Kdy se vrátil na začátek selektorů kdy dočetl řádek textu často pohyboval svou ruku příliš daleko do levá.
