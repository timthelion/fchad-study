\subsubsection{Žák 7}
Setkání 13:40 20.3.2013

Žák 7 je mezi 20-25 let starý.  Čte Braillovo písmo od začátku školní docházky.  Jeho nejvyšší dosažení úroveň vzdělávání je základní škola.

Četl tiskový text obě ruce.

% začátek 85.2 otočení 106.7 konec 134.6

Čas na čtení celého textu: 49.4 vteřin

CPS: 8.24

Čas na čtení druhé strany tiskového textu: 27.9 vteřin

CPS: 9.5

Namočil jsem vodou jeho ukazovací prst na pravou ruku aby se lepe fungovaly senzory. Ukázal jsem ho selektor a pak zobrazovač stručně bez delší popsání.

\paragraph{První řádek}
\begin{tabular}{|c|c|c|c|c|c|c|c|c|c|c|c|}
\hline
B&r&a&i&l&l&s&&&&&\\
\braillebox{1278}&\braillebox{1235}&\braillebox{1}&\braillebox{24}&\braillebox{123}&\braillebox{123}&\braillebox{234}&\braillebox{}&\braillebox{2358}&\braillebox{123}&\braillebox{}&\braillebox{}\\
\hline
\end{tabular}

Říkal jsem ať čte první \uv{písmo} a on říkal, že tam nic není.  Říkal jsem, že musí mít ruku na selektor a on tam ji dal ale ne na první senzor jen náhodně.  Začal hádat `a' ale jsem ho zastavil ať začíná na první písmeno.  Odhadl nejdřív jako `ě'\braillebox{126} ale podruhy odhadl správně.

Nehledal body jen položil ruku na selektor a říkal co si myslí.  Často odstranil ruku ze zobrazovače a pohyboval hlavu v stereotypickým pohybem tak aby tvař koukal nahoru a doleva. Hlavu pohyboval takhle po skoro každém písmenu ale pohyby nebyly formou rozptýlení. Samostatně se vrátil ke čtení potom co takhle pohyboval jeden až tři krát.

Druhý písmeno odhadl nejdřív jako `h' ale odhadl správně podruhy.  Četl `a' správně, a `i' četl na druhý pokus(první říkal čárka ale ani nedokončil slovo než samostatně se upravil).  Četl `l' správně, přeskočil další `l'. Myslím si, že používal zvuk cvaknutí od zobrazovače aby věděl o tom, kdy se změnil písmeno, ale protože nic se neměnil mezi `l' a `l' k žádnému cvaknutí nedošlo.  Neopravil jsem ho.  Četl `s' správně.  Říkal jsem mu, ať se vrátí na začátek selektoru abychom četli další řádek.

\paragraph{Druhý řádek}
\begin{tabular}{|c|c|c|c|c|c|c|c|c|c|c|c|}
\hline
k&ý& &ř&á&d&e&k&,& &k&t\\
\braillebox{1378}&\braillebox{12346}&\braillebox{}&\braillebox{2456}&\braillebox{16}&\braillebox{145}&\braillebox{15}&\braillebox{13}&\braillebox{2}&\braillebox{}&\braillebox{13}&\braillebox{2345}\\
\hline
\end{tabular}

Druhý řádek četl bez chyb kromě tomu, že jedno se vrátil na začátek selektorů.  Dělal jsem dost važnou chybu.  Upravil jsem ho nesprávně kvůli neznalost češtiny. Nerozuměl jsem ho kdy mě říkal `ř' jako `eř' a říkal jsem ho, že četl nesprávně.  Nikdy nevyslovoval mezery nahlas(ne, že bych to přikazoval).

\paragraph{Třetí řádek}
\begin{tabular}{|c|c|c|c|c|c|c|c|c|c|c|c|}
\hline
e&r&ý& &t&e&ď& &p&o&u&ž\\
\braillebox{1578}&\braillebox{1235}&\braillebox{12346}&\braillebox{}&\braillebox{2345}&\braillebox{15}&\braillebox{1456}&\braillebox{}&\braillebox{1234}&\braillebox{135}&\braillebox{136}&\braillebox{2346}\\
\hline
\end{tabular}

Četl `e' jako `o'.  To byl protože nespadnou body samou. Samostatně to pochopil a opravil se.  Potom co se přečetl `p', přeskočil několik písmen a četl správně `ž'. Kdy se začal číst znovu od `p', četl `o' nejdřív jako `k' ale opravil se než jsem ho říkal, že to není správný.

\paragraph{Čtvrtý řádek}
\begin{tabular}{|c|c|c|c|c|c|c|c|c|c|c|c|}
\hline
í&v&á&t&e&,& &n&e&n&í& \\
\braillebox{3478}&\braillebox{1236}&\braillebox{16}&\braillebox{2345}&\braillebox{15}&\braillebox{2}&\braillebox{}&\braillebox{1345}&\braillebox{15}&\braillebox{2345}&\braillebox{34}&\braillebox{}\\
\hline
\end{tabular}

První dvě písmena četl správně. Četl `á' jako `u'\braillebox{136}.  Nějak jsem se bal, že jeho chyba může byt kvůli vadný spojení v zobrazovače ale zjistil jsem, že ne.  Kdy se vrátil na začátek řádku ještě četl `v' jako `u'. Zase četl `á' jako `u' ale konečně správně říkal `á'.  Četl `e' jako `o'. Připomněl jsem ho, že body nespadnou samo.  Zbytek řádku četl správně ale četl `n' jako `d'(upravil se stejném dechem).

\paragraph{Pátý řádek}
\begin{tabular}{|c|c|c|c|c|c|c|c|c|c|c|c|}
\hline
p&r&v&n&í& &ř&á&d&e&k&,\\
\braillebox{123478}&\braillebox{1235}&\braillebox{1236}&\braillebox{1345}&\braillebox{34}&\braillebox{}&\braillebox{1235}&\braillebox{16}&\braillebox{145}&\braillebox{15}&\braillebox{13}&\braillebox{2}\\
\hline
\end{tabular}

Četl slovo \uv{první} bez chyb.  Audio je zaseknutí, a nevím co se stalo dal na řádce.

\paragraph{Šesti řádek}
\begin{tabular}{|c|c|c|c|c|c|c|c|c|c|c|c|}
\hline
 &k&t&e&r&ý& &z&o&b&r&a\\
\braillebox{78}&\braillebox{13}&\braillebox{2345}&\braillebox{15}&\braillebox{1235}&\braillebox{12346}&\braillebox{}&\braillebox{1356}&\braillebox{135}&\braillebox{12}&\braillebox{1235}&\braillebox{1}\\
\hline
\end{tabular}

První část četl správné ale myslel, že `z' je `ř'. Zbytek taky četl správně.

Přesouval ruku na selektor úplně samostatně a správně.

\paragraph{Sedmi řádek}
\begin{tabular}{|c|c|c|c|c|c|c|c|c|c|c|c|}
\hline
z&u&j&e& &j&e&n&o&m& &j\\
\braillebox{135678}&\braillebox{136}&\braillebox{245}&\braillebox{15}&\braillebox{}&\braillebox{245}&\braillebox{15}&\braillebox{1345}&\braillebox{135}&\braillebox{134}&\braillebox{}&\braillebox{245}\\
\hline
\end{tabular}

Četl správně.

\paragraph{Osmi řádek}
\begin{tabular}{|c|c|c|c|c|c|c|c|c|c|c|c|}
\hline
e&d&n&o& &p&í&s&m&e&n&o\\
\braillebox{1578}&\braillebox{145}&\braillebox{1345}&\braillebox{135}&\braillebox{}&\braillebox{1234}&\braillebox{24}&\braillebox{234}&\braillebox{134}&\braillebox{15}&\braillebox{1345}&\braillebox{135}\\
\hline
\end{tabular}

Četl správně kromě `s', nevím přesně co odhadl první kvůli špatné kvalitě zvuku.  `o' četl nejdřív jako  `a' ale opravil se stejným dechem.

\paragraph{Devátý řádek}
\begin{tabular}{|c|c|c|c|c|c|c|c|c|c|c|c|}
\hline
.& & &P&r&v&n&í& &b&y&l\\
\braillebox{378}&\braillebox{}&\braillebox{}&\braillebox{12347}&\braillebox{1235}&\braillebox{1236}&\braillebox{1345}&\braillebox{34}&\braillebox{}&\braillebox{12}&\braillebox{13456}&\braillebox{123}\\
\hline
\end{tabular}

Četl bez chyb.

\paragraph{Desátý řádek}
\begin{tabular}{|c|c|c|c|c|c|c|c|c|c|c|c|}
\hline
 &v&y&n&a&l&e&z&e&n& &v\\
\braillebox{78}&\braillebox{1236}&\braillebox{13456}&\braillebox{1345}&\braillebox{1}&\braillebox{123}&\braillebox{15}&\braillebox{1356}&\braillebox{15}&\braillebox{1345}&\braillebox{}&\braillebox{1236}\\
\hline
\end{tabular}

Četl `v' nejdřív nejsprávně ale kvalita zvuku mě nedovolí vám říct jak.  Četl až na `e' správně ale selektor začal zle fungovat. Správil jsem to a řekl jsem ho ať přečte od začatku.  Četl správně kromě `e' což četl nejdřív jako `o' a mezera, která chvílí myslel je tečka.  

\paragraph{Jedenacti řádek}
\begin{tabular}{|c|c|c|c|c|c|c|c|c|c|c|c|}
\hline
 &r&o&c&e& &1&9&1&3& &v\\
\braillebox{78}&\braillebox{1235}&\braillebox{135}&\braillebox{14}&\braillebox{15}&\braillebox{}&\braillebox{18}&\braillebox{248}&\braillebox{18}&\braillebox{148}&\braillebox{}&\braillebox{1236}\\
\hline
\end{tabular}

Četl `c' nejdřív jako `n'.  Četl `1' nejdřív jako `a'. Vysvětlil jsem, že osmi bod je pro čísla.  Nevím zda pochopil, protože `9' nejdřív četl jako `i'.  Když jsem ho říkal, že `i' je nesprávně, četl zbytek čísel správně.  Četl `v' jako `l' ale odvolal stejným dechem.  Pak odhadl `r' a konečně správně `v'.

\paragraph{Dvanácti řádek}
\begin{tabular}{|c|c|c|c|c|c|c|c|c|c|c|c|}
\hline
 &A&n&g&l&i&i&.& & &J&m\\
\braillebox{78}&\braillebox{17}&\braillebox{1345}&\braillebox{1245}&\braillebox{123}&\braillebox{24}&\braillebox{24}&\braillebox{3}&\braillebox{}&\braillebox{}&\braillebox{2457}&\braillebox{134}\\
\hline
\end{tabular}

Řešil jsem problém se selektorem u začátku řádku.  Musím poznamenat, že on poznal, že selektor nefungoval a čekal na mně. Četl `i' nejdřív jako `e'.  Vím, že už dobře rozuměl selektor, protože poznal, že jsou dvě `i' a dvě mezery po tečce.  Zbytek řádku četl správně.

\paragraph{Řádek 13}
\begin{tabular}{|c|c|c|c|c|c|c|c|c|c|c|c|}
\hline
e&n&o&v&a&l& &s&e& &O&p\\
\braillebox{1578}&\braillebox{1345}&\braillebox{135}&\braillebox{1236}&\braillebox{1}&\braillebox{123}&\braillebox{}&\braillebox{234}&\braillebox{15}&\braillebox{}&\braillebox{1357}&\braillebox{1234}\\
\hline
\end{tabular}

Četl `v' nejdřív jako `l' ale opravil se stejným dechem. Jinak četl správně.

\paragraph{Řádek 14}
\begin{tabular}{|c|c|c|c|c|c|c|c|c|c|c|c|}
\hline
t&o&f&o&n& &a& &p&ř&e&v\\
\braillebox{234578}&\braillebox{135}&\braillebox{124}&\braillebox{135}&\braillebox{1345}&\braillebox{}&\braillebox{1}&\braillebox{}&\braillebox{1234}&\braillebox{2456}&\braillebox{15}&\braillebox{1236}\\
\hline
\end{tabular}

Jedno jsem si myslel, že selektor nefunguje, protože nějak rychle cvakl zobrazovač. Ale kdy jsem zkusil, jsem zjistil, že zkutečně funguje správně.  Jinak četl bez chyb.

\paragraph{Řádek 15}
\begin{tabular}{|c|c|c|c|c|c|c|c|c|c|c|c|}
\hline
á&d&ě&l& &s&v&ě&t&l&o& \\
\braillebox{1678}&\braillebox{145}&\braillebox{126}&\braillebox{123}&\braillebox{}&\braillebox{234}&\braillebox{1236}&\braillebox{126}&\braillebox{2345}&\braillebox{123}&\braillebox{135}&\braillebox{}\\
\hline
\end{tabular}

Četl bez chyb.

\paragraph{Řádek 16}
\begin{tabular}{|c|c|c|c|c|c|c|c|c|c|c|c|}
\hline
n&a& &z&v&u&k&.& & &K&d\\
\braillebox{134578}&\braillebox{1}&\braillebox{}&\braillebox{1356}&\braillebox{1236}&\braillebox{136}&\braillebox{13}&\braillebox{3}&\braillebox{}&\braillebox{}&\braillebox{137}&\braillebox{145}\\
\hline
\end{tabular}

Četl bez chyb a poznal `K' jako velké.

\paragraph{Řádek 17}
\begin{tabular}{|c|c|c|c|c|c|c|c|c|c|c|c|}
\hline
y&ž& &č&t&e&n&á&ř& &p&o\\
\braillebox{1345678}&\braillebox{2346}&\braillebox{}&\braillebox{146}&\braillebox{2345}&\braillebox{15}&\braillebox{1345}&\braillebox{16}&\braillebox{2456}&\braillebox{}&\braillebox{1234}&\braillebox{135}\\
\hline
\end{tabular}

Četl správně až na `ř' a u toho jsem mu říkal, že budeme zkusit něco jiné a jsem se zkusil ho naučit nechat ruku položený na selektor.  Musel jsem zase resetovat stroj.

Audio je zase zpřeházený ale je rozumět kdy četl písmeno `y'. Odhadl `ď'\braillebox{1456} ale dal najevo, že nebyl jistý.  Teď, už nechal ruku na zobrazovač.  Taky četl `č' nesprávně ale není rozumět jak.

\paragraph{Řádek 18}
\begin{tabular}{|c|c|c|c|c|c|c|c|c|c|c|c|}
\hline
h&y&b&o&v&a&l& &s&p&e&c\\
\braillebox{12578}&\braillebox{13456}&\braillebox{12}&\braillebox{135}&\braillebox{1236}&\braillebox{1}&\braillebox{123}&\braillebox{}&\braillebox{234}&\braillebox{1234}&\braillebox{15}&\braillebox{14}\\
\hline
\end{tabular}
Četl `h' správně na druhý pokus, první odhad není rozumět.  Četl `y' jako `d' ale uvědomoval stejném dechem, že to není správně odhadl správně na druhý pokus.

\paragraph{Řádek 19}
\begin{tabular}{|c|c|c|c|c|c|c|c|c|c|c|c|}
\hline
i&á&l&n&í&m& &p&e&r&e&m\\
\braillebox{2478}&\braillebox{16}&\braillebox{123}&\braillebox{1345}&\braillebox{34}&\braillebox{134}&\braillebox{}&\braillebox{1234}&\braillebox{15}&\braillebox{1235}&\braillebox{15}&\braillebox{134}\\
\hline
\end{tabular}
Četl `i' nejdřív jako `s'.  Četl `á' nejdřív jako `e'. Četl druhý `e' nejdřív jako `o'. Četl `m' jako `p' ale upravil se stejným dechem.

U toho jsem končil setkání.  Zeptal jsem se ho \uv{myslíte, že byste se naučil to používat lip s praxi?} on odpovídal \uv{ne, to je těžký, já mám na to notebook, psací stroj.} Zeptal jsem se \uv{a používáte braillský řádek?} On: \uv{Ne, hlasový vystup.}

