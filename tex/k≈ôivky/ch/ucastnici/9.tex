\subsubsection{Žák 9}
Setkání 23.4.2013

Žák 9 je ve věku mezi 20 a 25 lety.  Jeho nejvyšší dosažené vzdělání je maturita.  Je úplně nevidomý a má problémy se sluchem.  Při našem setkání jsem na sobě měl mikrofon, který byl bezdrátově připojen k jeho sluchadlům.  Rozuměl mi bez znatelných potíží.

Četl tištěný text jednou (pravou) rukou.

%začatek 197.8 otočení 230.2 konec 279
Čas čtení tištěného textu: 81.2 vteřin

CPS: 5

Čas čtení tištěného textu po otočení listu: 48.8 vteřin

CPS: 5.4

Když jsem žákovi ukazoval zobrazovač, říkal jsem: \uv{Tady se zobrazí to písmeno. Tady je osm děr.} On vzal zobrazovač do obou rukou a začal dírky počítat. Nepočítal je ve standardním pořadí, jen je počítal.  Během celého setkání mluvil milým a překvapeným hlasem. Když si byl jistý, že je tam skutečně osm děr, vyzval jsem ho, aby položil levou ruku na zobrazovač.  Odpověděl se smíchem: \uv{Já jsem sice levák, ale čtu pravou}. Vysvětlil jsem mu, že současný stroj má jenom jedno nastavení.  Potom jsem mu ukázal selektor.  Řekl jsem, že zvuk, který vydává zobrazovač, je zvuk změny písmen, a že by měl taky cítit, jak se to písmeno změní.  On si začal se selektorem hrát, a tak jsem si myslel, že se samostatně pokusí o čtení.  Měl ruku správně položenou na zobrazovači a prst na prvním senzoru selektoru.  Občas říkal \uv{uf}.

Protože jsem se chtěl dozvědět, jestli bude číst sám, čekal jsem 146 vteřin.  Bohužel na audionahrávce slyším, jak na začátku této pauzy říkám \uv{um}, takže je docela možné, že žák čekal na mě.  Když jsem pochopil, že sám číst nebude, vrátil jsem se k výkladu o použití nástroje.

Vysvětlil jsem mu, že \uv{ty písmena `k'} jsou senzory.  Začal je počítat a když spočítal, že je jich 12, tak řekl \uv{aha}, jako že si je spojil s těmi dvanácti senzory zmíněnými v úvodním textu. Zase jsem ho nechal si je osahat a tentokrát jsem nedělal žádné matoucí \uv{um}.  Skutečně si s nástrojem hrál a já věřil, že se snaží číst. O 210 vteřin později však stále ještě nepokračoval.  Hmatal na zobrazovač, jako by se snažil z něj číst, ale nečetl. %881-671.9

\paragraph{První řádek}

\begin{tabular}{|c|c|c|c|c|c|c|c|c|c|c|c|}
\hline
B&r&a&i&l&l&s&&&&&\\
\braillebox{1278}&\braillebox{1235}&\braillebox{1}&\braillebox{24}&\braillebox{123}&\braillebox{123}&\braillebox{234}&\braillebox{}&\braillebox{2358}&\braillebox{123}&\braillebox{}&\braillebox{}\\
\hline
\end{tabular}


Zeptal jsem se ho, jaké je první písmeno, a on se zeptal \uv{Jak je to poznat?}.  Řekl jsem mu, ať počítá dírky.  Nejdřív mě nepochopil, zřejmě kvůli chybě mé češtiny.  Počítal, že jsou dva body nahoře, ale nepochopil ještě, o co jde.  Uchopil jsem jeho levou ruku a ukazoval mu \uv{To je jedna a dva, tak které písmeno to je?} Na to už bez přemýšlení odpověděl `b'.  

Bohužel žák 8 kymácel trupem tak, že jeho ruka na selektoru nebyla stabilní, a dále trpěl drobnými zrcadlovými pohyby, když se snažil číst na zobrazovači.  Myslím si, že už rozuměl, že by měl jít na druhé písmeno, i bez toho, že bych mu to řekl, ale nebyl toho fyzicky schopný.  Věděl, že jeho prsty se pohybují bez ohledu na jeho přání, a brzo ho to začalo frustrovat.  Držel jsem jeho prsty tak, aby se samovolně nepohybovaly, a vedl jsem je na selektoru.  Na `r' reagoval: \uv{první a pátý koukám}. Řekl jsem mu \uv{ještě}. Když objevil další body, řekl překvapeně: \uv{Druhý a třetí, aha, takže erko}.

`a' četl opět způsobem, že nejdřív jmenoval čísla bodů a písmeno až potom.  `l' už četl bez jmenování bodů. Další písmeno četl jako `p'. Když jsem ho řekl, že četl nesprávně, opravil se na `s': \uv{Aha, tohle zmizelo, tak pozor na to, co zmizí.} Body nespadnou, pokud na ně netlačíte, takže při změně z `l'\braillebox{123} na `s'\braillebox{234} se skutečně na chvíli zobrazilo `p'\braillebox{1234}.

\paragraph{Druhý řádek}
\begin{tabular}{|c|c|c|c|c|c|c|c|c|c|c|c|}
\hline
k&ý& &ř&á&d&e&k&,& &k&t\\
\braillebox{1378}&\braillebox{12346}&\braillebox{}&\braillebox{2456}&\braillebox{16}&\braillebox{145}&\braillebox{15}&\braillebox{13}&\braillebox{2}&\braillebox{}&\braillebox{13}&\braillebox{2345}\\
\hline
\end{tabular}

Přišli jsme na další řádek.  `k' četl bez problému. `ý' četl zase tím, že pojmenoval body.  Kdy jsme dorazili k mezeře, on říkal \uv{nic} a začal se smát. Písmena slova \uv{řádek} četl po jednotlivých písmenech ale bez chyb a bez pojmenování čísla bodů.  Četl tak dobře, že jsem chvíli pustil jeho pravou ruku aby použival selektor samostatně.  Zase ta ruka pochybovala rytmický bez jeho ovládání.  Našli jsme místo kde jsme skončili a pokračovali jsme.

`k' četl nejdřív jako `a' ale sám se opravil když jsem ho nazačil, že to není správný.

\paragraph{Třetí řádek}

\begin{tabular}{|c|c|c|c|c|c|c|c|c|c|c|c|}
\hline
e&r&ý& &t&e&ď& &p&o&u&ž\\
\braillebox{1578}&\braillebox{1235}&\braillebox{12346}&\braillebox{}&\braillebox{2345}&\braillebox{15}&\braillebox{1456}&\braillebox{}&\braillebox{1234}&\braillebox{135}&\braillebox{136}&\braillebox{2346}\\
\hline
\end{tabular}

Přešli jsme na třetí řádek.  Vždy, kdy dostal k mezeře a hmatal, že tam nic není zobrazený usmíval se.  Řekl \uv{mezera} s radosti tím jak je lehký poznat mezery.  Až na `ž'\braillebox{2346} četl správně obracil ji jako `z'\braillebox{1356}. Řekl jsem \uv{To je jako `z' ale obraceně}.  On pak četl(obraceně) \uv{jedna tři pět šest, to je `z'.} Ukazal jse ho, že pojmenoval čísla bodů obraceně a on odpovídal rozčileně \uv{Tak, že jsem to celou dobu říkal blbě!}  Uklidnil jsem ho, že četl správně do té doby a on se krasně usmíval a říkal, že je úplně zmatený.

\paragraph{Čtvrtý řádek}

\begin{tabular}{|c|c|c|c|c|c|c|c|c|c|c|c|}
\hline
í&v&á&t&e&,& &n&e&n&í& \\
\braillebox{3478}&\braillebox{1236}&\braillebox{16}&\braillebox{2345}&\braillebox{15}&\braillebox{2}&\braillebox{}&\braillebox{1345}&\braillebox{15}&\braillebox{1345}&\braillebox{34}&\braillebox{}\\
\hline
\end{tabular}

Kdy jsme pokračovali na další řádek, `í'\braillebox{34} četl jako `á'\braillebox{16}.  Tento krát upravil se sám když jsem ho označil, že to není správní.  Zase u toho usmíval.  Pak `v' a `á' četl bez chyb.  `e' četl jako `i' ale tento krát jsem Já dělal chybu a neopravil jsem ho. Čarka `,' četl správně.  Když jsme byli na `n'\braillebox{1345} přemyslel a pojmenoval body hodně dlouho(39 vteřin) než odhadl `ž', který hned sám odvolal. Čekal jsem další 18 vteřin a říkal jsem ho \uv{je to `ž' z horu nohama}.  I Já jsem byl zřejmě trochu zmatený, protože `ž' z horu nohama je `ň'.  Konečně jsem ho jen tak říkal, že je to `n'.  Nevěřil mě a řekl \uv{mate to, že to je zrcadlově obracený}. Konečně řekl \uv{aha, už to vidím} a zase se krasně usmíval.  %1838.4

Během toho, jsme přeskočili dvě písmena.  Byly jsme u písmena `í', což četl nejdřív jako `t', pak `á' a konečně `í'.

Zeptal jsem ho jestli není unavený, ale řekl že je \uv{v pohodě}.

\paragraph{Pátý řádek}

\begin{tabular}{|c|c|c|c|c|c|c|c|c|c|c|c|}
\hline
p&r&v&n&í& &ř&á&d&e&k&,\\
\braillebox{123478}&\braillebox{1235}&\braillebox{1236}&\braillebox{1345}&\braillebox{34}&\braillebox{}&\braillebox{1235}&\braillebox{16}&\braillebox{145}&\braillebox{15}&\braillebox{13}&\braillebox{2}\\
\hline
\end{tabular}

Na další řádek četl písmena \uv{první ř} bez chyb. Dal velký důraz na `n'.  `á' opět četl nejdříve jako `í'. `d' a `e' četl správně ale `k' četl nejdříve jako `a'. Čarku `,' četl správně.

\paragraph{Šesti řádek}
\begin{tabular}{|c|c|c|c|c|c|c|c|c|c|c|c|}
\hline
 &k&t&e&r&ý& &z&o&b&r&a\\
\braillebox{78}&\braillebox{13}&\braillebox{2345}&\braillebox{15}&\braillebox{1235}&\braillebox{12346}&\braillebox{}&\braillebox{1356}&\braillebox{135}&\braillebox{12}&\braillebox{1235}&\braillebox{1}\\
\hline
\end{tabular}

Šesti řádek četl správně až na `z', který nejdřív četl jako `n'.  `o' a `b' četl správně ale `r' četl nejdřív jako `h'.

Je možní, že protože jsem musel vest jeho pravou ruku, on ještě nerozuměl selektor. Kdy jsme byli u konce řádku myslel, že se zobrazuje mezera.  Nic se nezobrazoval ale to byl protože nebyl v kontaktu s žádným senzorem.

\paragraph{Sedmi řádek}
\begin{tabular}{|c|c|c|c|c|c|c|c|c|c|c|c|}
\hline
z&u&j&e& &j&e&n&o&m& &j\\
\braillebox{135678}&\braillebox{136}&\braillebox{245}&\braillebox{15}&\braillebox{}&\braillebox{245}&\braillebox{15}&\braillebox{1345}&\braillebox{135}&\braillebox{134}&\braillebox{}&\braillebox{245}\\
\hline
\end{tabular}

`z' četl jako `n' ale sám se hned upravil.  Říkal jsem ho ať zkusí posunut pravou ruku na selektor samostatně a přestal jsem držet jeho ruku.  Opět, tím že jsem nedržel jeho ruka pohybovala sama.  On si povzdechl frustraci.  Zase jsem tu ruku držel.   `u' četl nejdřív jako `a' ale když jsem ho řekl, že to není správný opravil se.  `j'\braillebox{245} četl jako `h'\braillebox{125}.  Opět upravil se sám když jsem ho řekl, že je obracený. `e' a mezera četl správně. Opět obrátil `j'.  `e' četl správně ale `n' četl jako `d'. Když jsem ho řekl, že to není správný, nejdřív si myslel, že četl obraceně. Řekl jsem ho, že to neobracel, a že je ještě nějaký bod dolu.  Rychle už našel, že je se zobrazuje `n'. `o' četl správně ale `m' četl nejdřív jako `k'.  `j' četl jako `,' a jsem ho připomněl, že je ještě další řádek bodu než svůj odhad opravil. Jinak četl správně.

\paragraph{Osmi řádek}
\begin{tabular}{|c|c|c|c|c|c|c|c|c|c|c|c|}
\hline
e&d&n&o& &p&í&s&m&e&n&o\\
\braillebox{1578}&\braillebox{145}&\braillebox{1345}&\braillebox{135}&\braillebox{}&\braillebox{1234}&\braillebox{34}&\braillebox{234}&\braillebox{134}&\braillebox{15}&\braillebox{1345}&\braillebox{135}\\
\hline
\end{tabular}

Na osmi řádek textu četl správně až na `í' což zase obrátil jako `á'. `s' a `m' četl správně.  Myslel si, že už jsme na další písmeno když jsme v zkutečnosti nebyli.  `m'\braillebox{134} četl podruhy jako `c'\braillebox{14}.  `e' četl jako `o' ale sám si objevil tu chybu.  `o' četl nejdřív jako `e'.

\paragraph{Devátý řádek}
\begin{tabular}{|c|c|c|c|c|c|c|c|c|c|c|c|}
\hline
.& & &P&r&v&n&í& &b&y&l\\
\braillebox{378}&\braillebox{}&\braillebox{}&\braillebox{12347}&\braillebox{1235}&\braillebox{1236}&\braillebox{1345}&\braillebox{34}&\braillebox{}&\braillebox{12}&\braillebox{13456}&\braillebox{123}\\
\hline
\end{tabular}

Devátý řádek začíná s tečkou `.' ale bylo zobrazený jako \braillebox{378} protože jsme byli na začátek řádku.  On to četl jako závorka což je \braillebox{236}.  Připomněl jsem ho, že body sedm a osm jsou nahoře protože je u začátku řádku. Zeptal jsem se, které body jsou ještě nahoře.  Jeho druhý odhad byl obracený, že to je znak pro velké písmeno\braillebox{6} a konečně odhadl správně když jsem ho řekl že to není.  Není důvod proč by věděl, že znak pro velké písmeno je bod šest a ne bod 3.  Tento znak je použitý jenom v tištěném Braillovu písmu a bez rámečku.  Není znatelný rozdíl.  Zbytek řádku četl bez problému.

\paragraph{Desátý řádek}
\begin{tabular}{|c|c|c|c|c|c|c|c|c|c|c|c|}
\hline
 &v&y&n&a&l&e&z&e&n& &v\\
\braillebox{78}&\braillebox{1236}&\braillebox{13456}&\braillebox{1345}&\braillebox{1}&\braillebox{123}&\braillebox{15}&\braillebox{1356}&\braillebox{15}&\braillebox{1345}&\braillebox{}&\braillebox{1236}\\
\hline
\end{tabular}

Desátý řádek četl skoro bez chyb, jedině byl částečně moje vína(protože jsem vedl jeho ruce na selektor). Tahl jsem jeho ruku příliš daleko od `n' na `l'. Museli jsme se vrátit. Kdy `l' četl podruhy myslel, že je to `b'. Opravil se hned než jsem stihl něco říct.

\paragraph{Jedenáctý řádek}
\begin{tabular}{|c|c|c|c|c|c|c|c|c|c|c|c|}
\hline
 &r&o&c&e& &1&9&1&3& &v\\
\braillebox{78}&\braillebox{1235}&\braillebox{135}&\braillebox{14}&\braillebox{15}&\braillebox{}&\braillebox{18}&\braillebox{248}&\braillebox{18}&\braillebox{148}&\braillebox{}&\braillebox{1236}\\
\hline
\end{tabular}

Obracel `r' jako `ř'.  `o' četl nejdřív jako `e'. Kdy dočetl \uv{roce} zeptal jsem ho jestli pozná, které slovo tedy četl.  Nevěděl.  Vrátili jsme se na začatku slova a on četl písmena s chybou, že `e' četl nejdříve jako `a'. Začal vykládat slovo jako \uv{ro} ale nepokračoval. Místo tomu, že bych ho zbytečně otrávil, jsem ho řekl, že to bylo slovo \uv{roce}.  Řekl jsem mu, že teď zřejmě bude číslo nebo datum.  Myslím, že jsem ho zmátl dost.  Nejdřív 1\braillebox{18} četl jako `a'.  Řekl jsem ho, že to má byt číslo a že bod osm je číselní bod. Stále nepochopil.  Přemyšlel o 4.5 vteřin a pak, řekl \uv{aha, jedna}. Devět četl nejdřív jako `i'.  Potom se zeptal jestli to je ještě číslo.  Správně doplnil, že to je devět. Četl `1' a `3' správně. Doplnil jsem, že to je \uv{devatenátset třináct}.  Mezera a `v' četl správně.  Předtím, než odhádal `v' řekl \uv{předpokládám, že tohle už není číslo}. Nevím jestli on k tomu závěru dostal protože žádný číslo je v tvaru\braillebox{1236} anebo protože pochopil, že jsme dočetli datum.

\paragraph{Dvanáctý řádek}
\begin{tabular}{|c|c|c|c|c|c|c|c|c|c|c|c|}
\hline
 &A&n&g&l&i&i&.& & &J&m\\
\braillebox{78}&\braillebox{17}&\braillebox{1345}&\braillebox{1245}&\braillebox{123}&\braillebox{24}&\braillebox{24}&\braillebox{3}&\braillebox{}&\braillebox{}&\braillebox{2457}&\braillebox{134}\\
\hline
\end{tabular}

Dvanáctý řádek.  Zase jsem ho nechal číst samostatně. Už zvládl nepohybovat pravou ruku.  Přizpůsoboval tak, že tlačil hodně silně prstem.  Jestli nějaký pohyb byl, pohyboval celý selektor. `n' četl nejdřív jako `d'. Měl velký problém s písmenem `g'\braillebox{1245}, což četl jako `x'\braillebox{1346}.  Musel jsem mu říct ať pohybuje pravou ruku na selektor.  `l' četl původně jako `b'.  `j' četl jako čarka `,' a `m' jako `c'.  Ještě mu pohybovala ruka sama na selektor ale snažil se, úspěšně to překonat.  U konce řádku myslel, že je mezera.

\paragraph{Řádek 13}
\begin{tabular}{|c|c|c|c|c|c|c|c|c|c|c|c|}
\hline
e&n&o&v&a&l& &s&e& &O&p\\
\braillebox{1578}&\braillebox{1345}&\braillebox{135}&\braillebox{1236}&\braillebox{1}&\braillebox{123}&\braillebox{}&\braillebox{234}&\braillebox{15}&\braillebox{}&\braillebox{1357}&\braillebox{1234}\\
\hline
\end{tabular}

První `o' četl nejdřív jako `e'. Druhý `O' četl nejdřív jako `k'. `p' četl nejdřív jako `l'.

Hlavní je, že poslední dvě řádky používal selektor samostatně.

Když jsem se zeptal, jestli si mysli, že by se dokázal naučit samostatně číst odpovídal \uv{no, snad ano}. Nijak nepřesvědčeně to nezněl. Ještě když jsem se zeptal na jeho dojmy, řekl:\em \uv{No, spíš tahle jsem trochu zmaten z toho, že posouvám tou ... tabulkou jestli řeknu takhle tou klávesnicí jako by tady tou pravou rukou že jako by spíš jak to funguje když dam každé to písmeno to znamená jeden ten jakoby posun.  Todle pro mně zkouška jak jsem občas myslel, že to je to písmeno zrcadlově obraceně, i když jsem fakt občas měl jsem pocit, nevím proč.}\em
