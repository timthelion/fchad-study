\subsubsection{Žák 9}
Setkání 23.4.2013

Žák 9 je ve věku mezi 20 a 25 lety.  Jeho nejvyšší dosažené vzdělání je maturita.  Je úplně nevidomý a má problémy se sluchem.  Při našem setkání jsem na sobě měl mikrofon, který byl bezdrátově připojen k jeho sluchadlům.  Rozuměl mi bez znatelných potíží.

Četl tištěný text jednou (pravou) rukou.

%začatek 197.8 otočení 230.2 konec 279
Čas čtení tištěného textu: 81.2 vteřin

CPS: 5

Čas čtení tištěného textu po otočení listu: 48.8 vteřin

CPS: 5.4

Když jsem žákovi ukazoval zobrazovač, říkal jsem: \uv{Tady se zobrazí to písmeno. Tady je osm děr.} On vzal zobrazovač do obou rukou a začal dírky počítat. Nepočítal je ve standardním pořadí, jen je počítal.  Během celého setkání mluvil milým a překvapeným hlasem. Když si byl jistý, že je tam skutečně osm děr, vyzval jsem ho, aby položil levou ruku na zobrazovač.  Odpověděl se smíchem: \uv{Já jsem sice levák, ale čtu pravou}. Vysvětlil jsem mu, že současný stroj má jenom jedno nastavení.  Potom jsem mu ukázal selektor.  Řekl jsem, že zvuk, který vydává zobrazovač, je zvuk změny písmen, a že by měl taky cítit, jak se to písmeno změní.  On si začal se selektorem hrát, a tak jsem si myslel, že se samostatně pokusí o čtení.  Měl ruku správně položenou na zobrazovači a prst na prvním senzoru selektoru.  Občas říkal \uv{uf}.

Protože jsem se chtěl dozvědět, jestli bude číst sám, čekal jsem 146 vteřin.  Bohužel na audionahrávce slyším, jak na začátku této pauzy říkám \uv{um}, takže je docela možné, že žák čekal na mě.  Když jsem pochopil, že sám číst nebude, vrátil jsem se k výkladu o použití nástroje.

Vysvětlil jsem mu, že \uv{ty písmena `k'} jsou senzory.  Začal je počítat a když spočítal, že je jich 12, tak řekl \uv{aha}, jako že si je spojil s těmi dvanácti senzory zmíněnými v úvodním textu. Zase jsem ho nechal si je osahat a tentokrát jsem nedělal žádné matoucí \uv{um}.  Skutečně si s nástrojem hrál a já věřil, že se snaží číst. O 210 vteřin později však stále ještě nepokračoval.  Hmatal na zobrazovač, jako by se snažil z něj číst, ale nečetl. %881-671.9

\paragraph{První řádek}

\begin{tabular}{|c|c|c|c|c|c|c|c|c|c|c|c|}
\hline
B&r&a&i&l&l&s&&&&&\\
\braillebox{1278}&\braillebox{1235}&\braillebox{1}&\braillebox{24}&\braillebox{123}&\braillebox{123}&\braillebox{234}&\braillebox{}&\braillebox{2358}&\braillebox{123}&\braillebox{}&\braillebox{}\\
\hline
\end{tabular}


Zeptal jsem se ho, jaké je první písmeno, a on se zeptal \uv{Jak je to poznat?}.  Řekl jsem mu, ať počítá dírky.  Nejdřív mě nepochopil, zřejmě kvůli chybě mé češtiny.  Počítal, že jsou dva body nahoře, ale nepochopil ještě, o co jde.  Uchopil jsem jeho levou ruku a ukazoval mu \uv{To je jedna a dva, tak které písmeno to je?} Na to už bez přemýšlení odpověděl `b'.  

Bohužel žák 8 kymácel trupem tak, že jeho ruka na selektoru nebyla stabilní, a dále trpěl drobnými zrcadlovými pohyby, když se snažil číst na zobrazovači.  Myslím si, že už rozuměl, že by měl jít na druhé písmeno, i bez toho, že bych mu to řekl, ale nebyl toho fyzicky schopný.  Věděl, že jeho prsty se pohybují bez ohledu na jeho přání, a brzo ho to začalo frustrovat.  Držel jsem jeho prsty tak, aby se samovolně nepohybovaly, a vedl jsem je na selektoru.  Na `r' reagoval: \uv{první a pátý koukám}. Řekl jsem mu \uv{ještě}. Když objevil další body, řekl překvapeně: \uv{Druhý a třetí, aha, takže erko}.

`a' četl opět způsobem, že nejdřív jmenoval čísla bodů a písmeno až potom.  `l' už četl bez jmenování bodů. Další písmeno četl jako `p'. Když jsem ho řekl, že četl nesprávně, opravil se na `s': \uv{Aha, tohle zmizelo, tak pozor na to, co zmizí.} Body nespadnou, pokud na ně netlačíte, takže při změně z `l'\braillebox{123} na `s'\braillebox{234} se skutečně na chvíli zobrazilo `p'\braillebox{1234}.

\paragraph{Druhý řádek}
\begin{tabular}{|c|c|c|c|c|c|c|c|c|c|c|c|}
\hline
k&ý& &ř&á&d&e&k&,& &k&t\\
\braillebox{1378}&\braillebox{12346}&\braillebox{}&\braillebox{2456}&\braillebox{16}&\braillebox{145}&\braillebox{15}&\braillebox{13}&\braillebox{2}&\braillebox{}&\braillebox{13}&\braillebox{2345}\\
\hline
\end{tabular}

Přešli jsme na další řádek.  `k' četl bez problému. `ý' četl zase tak, že očísloval body.  Když jsme dorazili k mezeře, řekl \uv{nic} a rozesmál se. Písmena slova \uv{řádek} četl po jednotlivých písmenech, ale bez chyb a bez číslování bodů.  Četl tak dobře, že jsem na chvíli pustil jeho pravou ruku a nechal ho používat selektor samostatně.  Ruka se zase rytmicky pohybovala a on ji nedovedl ovládnout.  Znovu jsem ho tedy za ni uchopil, našli jsme místo, kde jsme skončili, a pokračovali jsme.

`k' četl nejdřív jako `a', ale sám se opravil, když jsem mu naznačil, že to není správně.

\paragraph{Třetí řádek}

\begin{tabular}{|c|c|c|c|c|c|c|c|c|c|c|c|}
\hline
e&r&ý& &t&e&ď& &p&o&u&ž\\
\braillebox{1578}&\braillebox{1235}&\braillebox{12346}&\braillebox{}&\braillebox{2345}&\braillebox{15}&\braillebox{1456}&\braillebox{}&\braillebox{1234}&\braillebox{135}&\braillebox{136}&\braillebox{2346}\\
\hline
\end{tabular}

Přešli jsme na třetí řádek.  Vždy, když se dostal k mezeře a nahmatal, že tam nic není zobrazeno, usmíval se.  Řekl \uv{mezera}, zřejmě rozradostněn tím, jak lehké je poznat mezery.  Četl správně až na `ž'\braillebox{2346}. Zaměnil ho za `z'\braillebox{1356}, které je jeho zrcadlovým obrazem. Řekl jsem: \uv{To je jako `z', ale obráceně.}  Nahlas vyjmenoval body: \uv{jedna tři pět šest, to je `z'.} čímž vyšlo najevo, že je čísluje obráceně. Upozornil jsem ho na to a on se polekal: \uv{Takže jsem to celou dobu říkal blbě!}  Uklidnil jsem ho, že do té doby četl správně, a on se krasně usmíval a říkal, že je úplně zmatený.

\paragraph{Čtvrtý řádek}

\begin{tabular}{|c|c|c|c|c|c|c|c|c|c|c|c|}
\hline
í&v&á&t&e&,& &n&e&n&í& \\
\braillebox{3478}&\braillebox{1236}&\braillebox{16}&\braillebox{2345}&\braillebox{15}&\braillebox{2}&\braillebox{}&\braillebox{1345}&\braillebox{15}&\braillebox{1345}&\braillebox{34}&\braillebox{}\\
\hline
\end{tabular}

Když jsme pokračovali na další řádek, `í'\braillebox{34} četl jako `á'\braillebox{16}.  Když jsem ho upozornil, že to není správně, tentokrát se úspěšně opravil sám.  Zase se u toho usmíval.  Následující `v' a `á' četl bez chyb.  `e' četl jako `i', ale tentokrát jsem udělal chybu já a neopravil ho. Čárku `,' četl správně.  Když jsme byli u `n'\braillebox{1345}, přemýšlel a čísloval body hodně dlouho (39 vteřin), než hádal `ž', které vzápětí sám odvolal. Čekal jsem dalších 18 vteřin a říkal mu: \uv{Je to `ž' vzhůru nohama.}  Nakonec jsem mu prozradil, že je to `n'.  Nevěřil mi a řekl: \uv{Mate to, že to je zrcadlově obrácený}. Konečně řekl \uv{Aha, už to vidím} a zase se krásně usmíval.  %1838.4

Během této diskuze o `n' jsme přeskočili dvě písmena.  Dostali jsme se k písmenu `í', což četl nejdřív jako `t', pak `á' a konečně `í'.

Zeptal jsem se ho, jestli není unavený, ale řekl, že je \uv{v pohodě}.

\paragraph{Pátý řádek}

\begin{tabular}{|c|c|c|c|c|c|c|c|c|c|c|c|}
\hline
p&r&v&n&í& &ř&á&d&e&k&,\\
\braillebox{123478}&\braillebox{1235}&\braillebox{1236}&\braillebox{1345}&\braillebox{34}&\braillebox{}&\braillebox{2456}&\braillebox{16}&\braillebox{145}&\braillebox{15}&\braillebox{13}&\braillebox{2}\\
\hline
\end{tabular}

Na dalším řádku četl písmena \uv{první ř} bez chyb. Dal velký důraz na `n'.  `á' opět četl nejdříve jako `í'. `d' a `e' četl správně, ale `k' četl nejdříve jako `a'. Čárku `,' četl správně.

\paragraph{Šestý řádek}
\begin{tabular}{|c|c|c|c|c|c|c|c|c|c|c|c|}
\hline
 &k&t&e&r&ý& &z&o&b&r&a\\
\braillebox{78}&\braillebox{13}&\braillebox{2345}&\braillebox{15}&\braillebox{1235}&\braillebox{12346}&\braillebox{}&\braillebox{1356}&\braillebox{135}&\braillebox{12}&\braillebox{1235}&\braillebox{1}\\
\hline
\end{tabular}

Šestý řádek četl správně až na `z', které nejdřív četl jako `n'.  `o' a `b' četl správně, ale `r' četl nejdřív jako `h'.

Je možné, že do této chvíle nepochopil selektor, protože jsem musel jeho pravou ruku vést. Když jsme byli u konce řádku, myslel, že se zobrazuje mezera.  Nic se nezobrazovalo, protože nebyl v kontaktu s žádným senzorem.

\paragraph{Sedmý řádek}
\begin{tabular}{|c|c|c|c|c|c|c|c|c|c|c|c|}
\hline
z&u&j&e& &j&e&n&o&m& &j\\
\braillebox{135678}&\braillebox{136}&\braillebox{245}&\braillebox{15}&\braillebox{}&\braillebox{245}&\braillebox{15}&\braillebox{1345}&\braillebox{135}&\braillebox{134}&\braillebox{}&\braillebox{245}\\
\hline
\end{tabular}

`z' četl jako `n', ale sám se hned opravil.  Vyzval jsem ho, ať zkusí posunovat pravou ruku na selektoru samostatně a přestal jsem ji držet.  Opět se pohybovala samovolně.  Zdál se mi tím být frustrovaný, tak jsem ji zase uchopil.   `u' četl nejdřív jako `a', ale když jsem mu řekl, že to není správně, opravil se.  `j'\braillebox{245} četl jako `h'\braillebox{125}.  Opět se opravil sám, když jsem mu řekl, že zaměnil zrcadlové obrazy. `e' a mezeru četl správně. `j' opět obrátil.  `e' četl správně, ale `n' četl jako `d'. Když jsem mu řekl, že to není správně, nejdřív si myslel, že si opět spletl zrcadlové obrazy. Vysvětlil jsem mu, že ne, a že je ještě nějaký bod dole.  Rychle poznal, že se zobrazuje `n'. `o' četl správně, ale `m' četl nejdřív jako `k'.  `j' četl jako čárku `,' Připomněl jsem mu, že je ještě další řádek bodů, než svůj odhad opravil. Jinak četl správně.

\paragraph{Osmý řádek}
\begin{tabular}{|c|c|c|c|c|c|c|c|c|c|c|c|}
\hline
e&d&n&o& &p&í&s&m&e&n&o\\
\braillebox{1578}&\braillebox{145}&\braillebox{1345}&\braillebox{135}&\braillebox{}&\braillebox{1234}&\braillebox{34}&\braillebox{234}&\braillebox{134}&\braillebox{15}&\braillebox{1345}&\braillebox{135}\\
\hline
\end{tabular}

Na osmém řádku textu četl správně až na `í', což zase obrátil jako `á'. `s' a `m' četl správně.  Myslel si, že už jsme na dalším písmenu, i když jsme nebyli.  Podruhé to samé `m'\braillebox{134} přečetl jako `c'\braillebox{14}.  `e' četl jako `o', ale sám objevil chybu.  `o' četl nejdřív jako `e'.

\paragraph{Devátý řádek}
\begin{tabular}{|c|c|c|c|c|c|c|c|c|c|c|c|}
\hline
.& & &P&r&v&n&í& &b&y&l\\
\braillebox{378}&\braillebox{}&\braillebox{}&\braillebox{12347}&\braillebox{1235}&\braillebox{1236}&\braillebox{1345}&\braillebox{34}&\braillebox{}&\braillebox{12}&\braillebox{13456}&\braillebox{123}\\
\hline
\end{tabular}

Devátý řádek začíná tečkou `.', ale byla zobrazena jako \braillebox{378}, protože jsme byli na začátku řádku.  Četl ji jako závorku, což je \braillebox{236}.  Připomněl jsem mu, že body sedm a osm jsou nahoře, protože se jedná o začátek řádku. Zeptal jsem se ho, které body jsou ještě nahoře.  Jeho druhý odhad, že to je znak pro velké písmeno\braillebox{6}, byl opět zrcadlově obrácený. Když jsem mu řekl, že velké písmeno to není, konečně uhodl správně.  Nebyl důvod, aby si myslel, že znak pro velké písmeno je bod šest a ne bod tři.  Tento znak se používá jenom v tištěném Braillovu písmu, kde nejsou žádné rámečky a mezi znaky tvořenými jediným bodem není znatelný rozdíl.  Zbytek řádku četl bez problému.

\paragraph{Desátý řádek}
\begin{tabular}{|c|c|c|c|c|c|c|c|c|c|c|c|}
\hline
 &v&y&n&a&l&e&z&e&n& &v\\
\braillebox{78}&\braillebox{1236}&\braillebox{13456}&\braillebox{1345}&\braillebox{1}&\braillebox{123}&\braillebox{15}&\braillebox{1356}&\braillebox{15}&\braillebox{1345}&\braillebox{}&\braillebox{1236}\\
\hline
\end{tabular}

Desátý řádek četl skoro bez chyb. Za jedinou jeho chybu jsem částečně mohl já, protože jsem vedl jeho ruce na selektoru. Táhl jsem jeho ruku příliš daleko od `n' na `l' a museli jsme se vrátit. Když `l' četl podruhé, myslel, že je to `b'. Opravil se dříve, než jsem stihl něco říct.

\paragraph{Jedenáctý řádek}
\begin{tabular}{|c|c|c|c|c|c|c|c|c|c|c|c|}
\hline
 &r&o&c&e& &1&9&1&3& &v\\
\braillebox{78}&\braillebox{1235}&\braillebox{135}&\braillebox{14}&\braillebox{15}&\braillebox{}&\braillebox{18}&\braillebox{248}&\braillebox{18}&\braillebox{148}&\braillebox{}&\braillebox{1236}\\
\hline
\end{tabular}

`r' zaměnil za `ř'.  `o' četl nejdřív jako `e'. Když dočetl \uv{roce}, zeptal jsem se ho, jestli pozná, které slovo četl.  Nevěděl.  Vrátili jsme se na začátek slova a on ho přečetl s tou chybou, že `e' četl nejdříve jako `a'. Vyslovil první polovinu slova, \uv{ro}, ale nepokračoval. Místo toho, abych ho s tím zbytečně otravoval, jsem mu prozradil, že to bylo slovo \uv{roce}.  Napověděl jsem mu, že teď zřejmě přijde číslo nebo datum.  Myslím, že jsem ho tím dost zmátl.  Nejdřív číslici 1\braillebox{18} četl jako `a'.  Řekl jsem mu, že to má být číslo a že bod osm je číselný bod. Stále nechápal.  Přemýšlel 4.5 vteřiny a pak řekl \uv{aha, jedna}. Devět četl nejdřív jako `i'.  Potom se zeptal, jestli to je také ještě číslo.  Správně doplnil, že to je devět. `1' a `3' četl správně. Doplnil jsem, že to je \uv{devatenáct set třináct}.  Mezeru a `v' četl správně.  Předtím, než uhádl `v', řekl \uv{předpokládám, že tohle už není číslo}. Nevím, jestli k tomuto závěru dospěl proto, že žádné číslo tvaru\braillebox{1236} není, nebo proto, že pochopil, že jsme dočetli datum.

\paragraph{Dvanáctý řádek}
\begin{tabular}{|c|c|c|c|c|c|c|c|c|c|c|c|}
\hline
 &A&n&g&l&i&i&.& & &J&m\\
\braillebox{78}&\braillebox{17}&\braillebox{1345}&\braillebox{1245}&\braillebox{123}&\braillebox{24}&\braillebox{24}&\braillebox{3}&\braillebox{}&\braillebox{}&\braillebox{2457}&\braillebox{134}\\
\hline
\end{tabular}

Zase jsem žáka nechal číst samostatně. Už zvládl ovládnout samovolný pohyb pravé ruky.  Vyřešil ho tak, že na selektor tlačil hodně silně.  Jestli nějaký pohyb byl, pohyboval se celý selektor. `n' četl nejdřív jako `d'. Měl velký problém s písmenem `g'\braillebox{1245}, které četl jako `x'\braillebox{1346}.  Musel jsem mu připomenout, aby posunul prst na selektoru.  `l' četl původně jako `b'.  `j' četl jako čárku `,' a `m' jako `c'.  Ruka se mu na selektoru pohybovala sama, ale snažil se to překonat a úspěšně četl dál.  U konce řádku si myslel, že je to mezera.

\paragraph{Řádek 13}
\begin{tabular}{|c|c|c|c|c|c|c|c|c|c|c|c|}
\hline
e&n&o&v&a&l& &s&e& &O&p\\
\braillebox{1578}&\braillebox{1345}&\braillebox{135}&\braillebox{1236}&\braillebox{1}&\braillebox{123}&\braillebox{}&\braillebox{234}&\braillebox{15}&\braillebox{}&\braillebox{1357}&\braillebox{1234}\\
\hline
\end{tabular}

První `o' četl nejdřív jako `e'. Druhé `O' četl nejdřív jako `k'. `p' četl nejdřív jako `l'.

Za úspěch považuji, že poslední dva řádky používal selektor samostatně. Když jsem se zeptal, jestli si myslí, že by se dokázal naučit samostatně číst, odpověděl \uv{no, snad ano}. Nezněl přesvědčeně. Když jsem se ho zeptal na jeho dojmy, řekl:\em \uv{No, spíš takhle jsem trochu zmaten z toho, že posouvám tou ... tabulkou, jestli to řeknu takhle, tou klávesnicí, jakoby tady tou pravou rukou, že jakoby spíš jak to funguje, když dám každé to písmeno, to znamená jeden ten jakoby posun.  Todle je pro mně zkouška, jak jsem občas myslel, že to písmeno je zrcadlově obráceně, i když jsem fakt občas měl pocit, nevím proč.}\em
