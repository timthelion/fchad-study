\subsubsection{Žáci 1 a 2}

První \uv{žák}, o kterém budu psát, jsem já sám.  Číst rukama jsem se naučil před dvěma lety. Zajímal jsem se o Braillovo písmo, protože mívám při klasickém čtení migrény.  Nějakou dobu jsem si pohrával s myšlenkou, že až se naučím číst Braillovo písmo, už nebudu vůbec číst očima.  Byla to ale spíše emoční reakce na frustrace než skutečný plán.  Brzy jsem zjistil, že Braillovo písmo nelze číst tak rychle, jak bych chtěl, a že braillský řádek je příliš drahý.  Můj zájem o Braillovo písmo vyústil v rozčilení z toho, jak jsou všechny pomůcky pro nevidomé předražené, a ve snahu vytvořit něco levnějšího.

Umím číst rukama, ale ne zcela plynule.  Daleko lépe čtu na FCHADu než na papíře, protože písmena zobrazovaná FCHADem jsou velká.

Když jsem se čtením na FCHADu začínal, četl jsem velmi pomalu.  Ještě jsem neznal Braillovo písmo dokonale a často jsem spletl f \braillebox{124}, d \braillebox{145}, h \braillebox{125}, a j \braillebox{245}. Dnes už je zvládám bez problému.  Dozvěděl jsem se, že mezi \uv{psychologickou rychlostí čtení} a \uv{skutečnou rychlostí čtení} je velký rozdíl.  Čtení se zdá tím rychlejší, čím méně dělám chyb a čím víc rozumím textu. To ale neznamená, že skutečně čtu rychleji. Často jsem si myslel, že jsem udělal pokrok, ale po zkonzultování nahrávky jsem musel konstatovat, že rychlost stoupla pouze nepatrně nebo vůbec.

Selektor už používám bez přemýšlení, ale rozpoznání znaků na zobrazovači mi stále někdy trvá.  Nastavil jsem stroj tak, že se selektor používá pravou rukou a zobrazovač levou. Dělal jsem to tak, protože selektor se podobá počítačové myši, kterou praváci používají pravou rukou.  Dnes soudím, že by to bylo lepší naopak, tedy používat zobrazovač dominantní rukou.  Nastavení bylo stejné pro všechny účastníky studie.

\uv{Žáky 1 a 2} uvádím společně, protože když jsem FCHAD ukazoval žákovi 2, vůbec nefungoval. Z dodnes nevyjasněných důvodů stroj v jeho rukách bláznil, ačkoliv v mých rukách fungoval bezvadně.



