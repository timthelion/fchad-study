\subsubsection{Žák 8}
Setkání 23.4.2013

Žák 8 je ve věku mezi 20 a 25 lety. Braillovo písmo čte od začátku školní docházky. Jeho nejvyšší dosažené vzdělání je základní škola. Je úplně nevidomý. Je pravák.  Nepoužívá braillský řádek.

Četl tištěný text jednou (pravou) rukou.

% začátek 37.6 otočení 93.8 konec 180.7

Čas čtení tištěného textu: 143.1 vteřin

CPS: 2.84

Čas čtení tištěného textu po otočení listu: 87.1  vteřin

CPS: 3

Když mě představili osmému žákovi, hned zaregistroval, že Timothy není české jméno. Když jsem mu pokládal úvodní otázky, zkusil na mě mluvit anglicky, což dělal skoro bez přízvuku.

\paragraph{První řádek}

\begin{tabular}{|c|c|c|c|c|c|c|c|c|c|c|c|}
\hline
B&r&a&i&l&l&s&&&&&\\
\braillebox{1278}&\braillebox{1235}&\braillebox{1}&\braillebox{24}&\braillebox{123}&\braillebox{123}&\braillebox{234}&\braillebox{}&\braillebox{2358}&\braillebox{123}&\braillebox{}&\braillebox{}\\
\hline
\end{tabular}

Žák tentokrát sám inicioval četbu na FCHADu: Ptal se, \uv{jak s tím, jak na to}.  Ukázal jsem mu díry na zobrazovači a když jeho pravá ruka byla na selektoru, zeptal jsem se ho, které body jsou nahoře.  Řekl, že \uv{první}, a hádal `a'. Snažil jsem se ho nasměrovat, ale mluvil jsem u toho dost zmateně. Nechápal jsem, jak došel ke své chybě, a nevěděl jsem tudíž, jak ho mám nasměrovat ke správné odpovědi. Napadlo mě, že si třeba myslí, že se zobrazuje `á', a je zmatený z bodů 7 a 8. Chtěl jsem mu vysvětlit jejich funkci, ale sám jsem při svém vysvětlování udělal chybu: Když měl žák prst na bodě 8, řekl jsem mu, že označuje velká písmena, ačkoliv velká písmena označuje bod 7. Bod 8 značí číslice. Žák hádal `v' a `u'. Zřejmě neznal osmibodové Braillovo písmo, ale to jsem ještě nevěděl.  Začal jsem mu vykládat o selektoru. Když jsem zmínil, že senzory mají tvar písmene `k', žák hned hádal, že písmeno zobrazené na zobrazovači je také `k'.

V tu chvíli selektor začal zlobit, což mě znepokojilo. Záhada se brzy vyřešila: Žák měl příliš mokré ruce. Zatímco jsem řešil tuto komplikaci, žák se ozval: \uv{Já nechci urazit, ale nejdřív bych se zeptal, jestli už s tím umíte?}  Odpověděl jsem, že umím, a že stroj nefungoval kvůli jeho příliš mokrým rukám.

Opět jsem se mu snažil vysvětlit zobrazovač.  Položil jsem jeho prst na každou dírku a řekl mu číslo bodu.  Žák se mě zeptal na \uv{ty zbylý dva body}, z čehož jsem poznal, že se žák s osmibodovým Braillovým písmem dosud nesetkal.  Vysvětlil jsem mu je takto: \uv{Na počítači děláme velké písmeno tak, že dáváme nahoru sedmý bod, a čísla tak, že dáváme nahoru osmý bod}.

Otřením mokré ruky papírovým kapesníkem se mi nepodařilo ji vysušit dostatečně, tak jsem snížil citlivost senzorů.  Kvůli tomu jsem musel restartovat veškerý software.

Potom jsme se vrátili ke čtení písmen.  U `B' jsem opět označil body čísly a žákův prst jsem položil na každou tyčku nebo dírku. Označil a ukázal jsem mu takto jen prvních šest bodů, abych ho zbytečně nemátl.  Jasně musel cítit, že nahoře jsou body jedna a dva a žádné jiné, ale stále hádal nesprávně `včko'.  Vysvětlil jsem mu znovu, že body sedm a osm nepoužíváme k poznání písmene, a on konečně odhadl `bčko'.

`r' četl nejdřív jako `h', ale poté, co jsem mu poradil, ať hledá dole, se správně opravil.

Řekl jsem mu, ať postoupí na další písmeno, a zase jsem utřel jeho ruku.  V místnosti bylo velké horko.  Otřel jsem i selektor. Žák bez jakékoli souvislosti prohlásil: \uv{Bych měl radši braillský řádek, na kterém by se to, co bych napsal na počítači, automaticky zobrazilo.} Z této věty jsem usoudil, že žák princip braillského řádku nechápe.

Řekl jsem mu, jak se má přesouvat na selektoru. `a' četl bez pomoci.

Odstranil ruku ze selektoru a já mu vysvětlil, že když se ho nedotýká, tak se nic nezobrazí.  Dal ruku zpátky na selektor na přibližně správné místo a četl `i' jako čárku `,', než jsem mu připomněl další řádek bodů.

Zase odstranil ruku ze selektoru, ale vrátil ji zpět, když jsem ho vyzval, ať čte další písmeno. Položil ruku na selektor o tři písmena dále než měl, ale četl správně `s'. Vrátil se zpátky a `l' četl nejdřív jako `k'.

Začal si hrát se selektorem a zobrazovačem.  Odstranil ruku ze selektoru a pak poklepal na každou tyčku, která byla nahoře, aby spadla.  Nechal jsem ho si takhle hrát 46 vteřin a on pak řekl: \uv{Tohle je `lko'.} (Měl pravdu, skutečně se `l' zobrazovalo.)  Pak jsem navrhl, že postoupíme na další řádek.

\paragraph{Druhý řádek}
\begin{tabular}{|c|c|c|c|c|c|c|c|c|c|c|c|}
\hline
k&ý& &ř&á&d&e&k&,& &k&t\\
\braillebox{1378}&\braillebox{12346}&\braillebox{}&\braillebox{2456}&\braillebox{16}&\braillebox{145}&\braillebox{15}&\braillebox{13}&\braillebox{2}&\braillebox{}&\braillebox{13}&\braillebox{2345}\\
\hline
\end{tabular}

První písmeno četl jako `l', napodruhé hádal `v'.  Vysvětlil jsem mu, že body sedm a osm jsou tam jenom proto, že jsme na začátku řádku.  Hádal `h' a pak `u'.  Zase jsem položil jeho prst na každý bod zvlášť a očísloval je. Pak jsem se ho zeptal, které body jsou nahoře.  Žák ukázal na první bod a řekl: \uv{To je ten sedmý.} Opravil jsem ho, že není, že to je první.  Pak řekl: \uv{Těžko říct, závorka?}.  Naváděl jsem ho: \uv{Počítejte se mnou, jedna, dva, tři, čtyři, pět, šest, tak nahoře je bod jedna a bod tři, které písmeno to je?} Konečně správně odpověděl `káčko'.

Pokračovali jsme na další písmeno.  Nedržel ruku na selektoru, jen poklepal na senzor a pak prst stáhl.  Písmeno četl jako `l', pak `p' a konečně `ý'.

Zase odstranil ruku ze selektoru, tak jsem se opět snažil mu vysvětlit, jak funguje.  Vedl jsem jeho prst nad \uv{těmi káčky} (senzory).

Vedl jsem jeho ruku na čtvrté písmeno, to četl nejdřív jako `t' a potom správně jako `ř'.  Další písmeno, `á', četl správně. Zase jsem musel otřít selektor. Žák si otřel ruce o kalhoty a prohlásil: \uv{To je strašně přemodernizovaný braillský řádek, vůbec se v tom nevyznám.} Vedl jsem jeho ruku na správné místo a pokračovali jsme.  `d' četl nejdřív jako `č' a poté správně.  Četl tak, že klepal na selektor a pak klepal na každý bod na zobrazovači.  `e' četl nejdřív jako `á', pak jako `o' a pak jako `i'.  Prozradil jsem mu správnou odpověď.  On prohlásil: \uv{Vůbec to není poznat.} Odpověděl jsem, že to jsou stejná písmena, jenom větší. On řekl: \uv{Já vím ale.}  Pořád klepal na senzory, ale přibližně správně pokračoval. Chvíli myslel, že se zobrazuje `ě'.  `k' četl správně, čárku taktéž.

Upozornil jsem ho na to, že už přečetl dvě slova.  Řekl, že to nepoznal.  Zeptal jsem se ho, jestli je unavený, ale odpověděl, že není.

Vedl jsem jeho ruku na předposlední znak, který přečetl správně jako `k'.  Pak opět přestaly fungovat senzory a já je opět utřel.  Žák řekl anglicky: \uv{I will get drunk} (Opiju se.)  Vedl jsem jeho ruku zpátky na `t', které nejdřív četl jako středník `;'\braillebox{23} a poté správně.

\paragraph{Třetí řádek}
\begin{tabular}{|c|c|c|c|c|c|c|c|c|c|c|c|}
\hline
e&r&ý& &t&e&ď& &p&o&u&ž\\
\braillebox{1578}&\braillebox{1235}&\braillebox{12346}&\braillebox{}&\braillebox{2345}&\braillebox{15}&\braillebox{1456}&\braillebox{}&\braillebox{1234}&\braillebox{135}&\braillebox{136}&\braillebox{2346}\\
\hline
\end{tabular}

Na začátku třetího řádku odhadl nejdřív `k', pak `A', `t' a `z'. Pak řekl, že neví. Hádal ještě `x'. Zeptal jsem se ho, které body jsou nahoře. Jmenoval \uv{první, třetí.} Opravil jsem ho, že to je už sedmý.  Poznamenal, že číslování postrádá logiku. Pak hádal `ě' a já  řekl \uv{éčko, ale bez háčku}.  Opět mumlal něco o logice.  Další písmeno odhadl nejdřív jako `l' a pak správně jako `ý'. `t' četl správně.  Pořád četl tím způsobem, že klepal krátce na selektor a pak klepal prsty na zobrazovač, aby zjistil, které tyčky zůstaly zvednuté. Řekl \uv{Tohle nefunguje vůbec}. Vysvětlil jsem mu, že jenom body, které jsou pevné, jsou významné. Četl `e' správně.  Pak četl `ď' jako `š'\braillebox{156}.  Připomněl jsem mu, že ještě nezkusil hmatat na bod čtyři.  Poté odhadl správně `ď'.  Mezeru četl jako čárku. `p' četl nejdřív jako `t', pak `e' a pak konečně jako `p'.  `o' četl správně.  `u' přeskočil a `ř' četl správně.

\paragraph{Čtvrtý řádek}
\begin{tabular}{|c|c|c|c|c|c|c|c|c|c|c|c|}
\hline
í&v&á&t&e&,& &n&e&n&í& \\
\braillebox{3478}&\braillebox{1236}&\braillebox{16}&\braillebox{2345}&\braillebox{15}&\braillebox{2}&\braillebox{}&\braillebox{1345}&\braillebox{15}&\braillebox{1345}&\braillebox{34}&\braillebox{}\\
\hline
\end{tabular}

První písmeno četl nejdřív jako `ž', pak `i', až jsem mu řekl, že je to `í'.  `v' a `á' četl správně, `t' přeskočil a `e' četl správně.  Řekl, že neví, jaké to mělo být slovo. Vysvětlil jsem mu, že je to slovo \uv{používáte}, ale že přeskočil písmena `u' a `t'.  Zeptal se, \uv{kde je učko}, tak jsme se vrátili na předchozí řádek a já vedl jeho ruku k `o', `u' a `ž'. Všechna poznal.

Pak jsme začali číst od začátku řádku. Nejdřív si myslel, že `í' je `u', ale napodruhé odhadl správně. Pak si myslel, že se zobrazuje čárka `,', ale ještě bylo zobrazené `í'.  Zeptal se, kolik je hodin, a v té samé chvíli se ve dveřích objevila sekretářka a zeptala se, \uv{Jak jste na tom?}. Řekl jsem, že už můžeme končit a žák prohlásil, že by už měl jít.  Sekretářka řekla, že ještě má pět minut, a tak jsme pokračovali.

Přeskočil ještě pár písmen a byli jsme u `e', které označil správně. O `n' nejdřív myslel, že je to `g'. `í' přeskočil. Mezeru poznal správně.  Neřekl jsem mu, že něco přeskočil, protože by to bylo zbytečné ztížení úkolu.

Dal jsem mu číst další řádek.

\paragraph{Pátý řádek}
\begin{tabular}{|c|c|c|c|c|c|c|c|c|c|c|c|}
\hline
p&r&v&n&í& &ř&á&d&e&k&,\\
\braillebox{123478}&\braillebox{1235}&\braillebox{1236}&\braillebox{1345}&\braillebox{34}&\braillebox{}&\braillebox{2456}&\braillebox{16}&\braillebox{145}&\braillebox{15}&\braillebox{13}&\braillebox{2}\\
\hline
\end{tabular}

Zeptal se, \uv{Jak se ty řádky zadávají?}  Ukázal jsem mu svůj notebook a vysvětlil, že používám šipky, abych přešel na další řádek.  Poté, co jsem mu ukázal počítač, nemohl znovu najít selektor, tak jsem mu na něj položil ruku.

Nejdříve myslel, že řádek začal s `ý'. Připomněl jsem mu funkci bodů 7 a 8 a on se správně opravil na `p'.  `r' určil správně.  Když přesunul ruku, zřejmě slyšel cvaknutí, protože se zeptal, jestli něco nepřeskočil.  Řekl jsem mu, že skutečně přeskočil písmeno. Úspěšně se vrátil na `v'. Myslel chvíli, že je to `ě'\braillebox{126}, a trvalo mu trochu, než napodruhé správně odhadl `v'.  Správně odhadl `n' a `í'.  Zeptal jsem se ho, jestli pozná, které slovo to je, ale říkal, že ne.  Zeptal jsem se, jestli si pamatuje první písmeno v řádku.  Řekl `y'.  Zase jsme se vrátili na začátek řádku a mluvili o tom, že jsou nahoře body sedm a osm.  Napodruhé uhodl správně.  Zeptal se, \uv{druhý bylo co?} Vyzval jsem ho, ať to najde sám.  Začal číst druhé písmeno jako `h'. U toho jsme skončili, protože přišla sekretářka s dalším účastníkem.  Žák 8 poznamenal \uv{To je nad mou logiku} a rychle odešel.


