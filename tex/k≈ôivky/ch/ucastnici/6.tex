\subsubsection{Žák 6}
Setkání 12:45 20.3.2013

Žák 6 je ve věku mezi 15 a 20 lety. Braillovo písmo čte od začátku školní docházky. Používá braillský řádek. Její nejvyšší úroveň vzdělávání je základní škola.  Je úplně nevidomá.

Četla tištěný text oběma rukama.

% start 133.5 otočení 159.5 konec 193.9

Čas čtení tištěného textu:60.4 vteřin

CPS: 6.7

Čas čtení tištěného textu po otočení listu: 34.4 vteřin

CPS: 7.7

Pote co jsem ji zeptal otázky namočil jsem její pravou ukazovátku.  Vysvětlil jsem ji, že má dat ruku na první písmeno a odhadnout, které písmeno se zobrazuje.

\paragraph{První Řádek}
\begin{tabular}{|c|c|c|c|c|c|c|c|c|c|c|c|}
\hline
B&r&a&i&l&l&s&&&&&\\
\braillebox{1278}&\braillebox{1235}&\braillebox{1}&\braillebox{24}&\braillebox{123}&\braillebox{123}&\braillebox{234}&\braillebox{}&\braillebox{2358}&\braillebox{123}&\braillebox{}&\braillebox{}\\
\hline
\end{tabular}

Myslel nejdřív, že body jedna a tři jsou nahoře tak jsem se rozhodl vysvětlit lip zobrazovač.  Vysvětlil jsem, že jsou osm děr a jsem je ukazoval tím, že jsem její prst dával na ty díry.

Mumlala pro sebe \uv{to je} a sevřela na zobrazovače. Četla správně `l'.  Vysvětlil jsem ji, že pravou ukazovátka by měla byt na první senzor a dal jsem její prst tam.

Odhadla `v'.  Řekl jsem ji, které body byly nahoře.  Vysvětlil jsem ji, že to je osmibodové Braillovo písmo a, že nemusí dávat pozor na body sedm a osm.  Správně ztotožnila `b'.  Řekl jsem ji, ať posouvá ruku na selektor. Zeptala se \uv{do práva?}. Odpověděl jsem kladně a četla další písmeno správně.  Četla správně až na `i' což četla hodně pomalu ale správně.  Četla `l' správně podruhy. Není poznat na nahrávce její první pokus. Zbytek řádku četla správně.  Řekl jsem ji ať jde na další řádek potom co četla `s'.

\paragraph{Druhý Řádek}
\begin{tabular}{|c|c|c|c|c|c|c|c|c|c|c|c|}
\hline
k&ý& &ř&á&d&e&k&,& &k&t\\
\braillebox{1378}&\braillebox{12346}&\braillebox{}&\braillebox{2456}&\braillebox{16}&\braillebox{145}&\braillebox{15}&\braillebox{13}&\braillebox{2}&\braillebox{}&\braillebox{13}&\braillebox{2345}\\
\hline
\end{tabular}

Četla `k' nejdřív jako `a'.  Četla `ý' nejdřív(myslím) jako `p'.  Říkala \uv{nic} kdy dorazila na mezeru.  Říkal jsem ji, že \uv{nic} je mezera a ona odpověděla \uv{oh, aha}. Četla `ř' myslím nejdřív jako `t'.  Nevím co byl její první odhad pro `e' ale vysvětlil jsem ji, že body nespadnou samo a ona pak správně ztotožnila písmeno.  Tvářila se nějak překvapeně kdy se dorazila na `,' tak jsem ji řekl co to je.  Myslela, že je u konce řádku kdy dorazila na `k'. Četla `t' nejdřív jako `j'.

\paragraph{Třetí Řádek}
\begin{tabular}{|c|c|c|c|c|c|c|c|c|c|c|c|}
\hline
e&r&ý& &t&e&ď& &p&o&u&ž\\
\braillebox{1578}&\braillebox{1235}&\braillebox{12346}&\braillebox{}&\braillebox{2345}&\braillebox{15}&\braillebox{1456}&\braillebox{}&\braillebox{1234}&\braillebox{135}&\braillebox{136}&\braillebox{2346}\\
\hline
\end{tabular}

Její ruka byla moc malá na zobrazovače. Ačkoliv mohla pokryt celý zobrazovač, pokryla zobrazovače přesně.  Dala ruku na straně, tak že prsty hmatali na první tří body.  Body 4,5 a 6 hmatala sevřením. Četla všechno správně ale pomalu až na `o', což četla nejdřív jako `r'.  Kdy jsem ji řekl ať pojmenuje čísla bodů dělala to správně ale stále tvrdila, že se zobrazí `r'.  Říkal jsem ji, že je to `o'. Ona odpovídala \uv{jojojo, aha , a smála se}. Zbytek řádku četla správně.

\paragraph{Čtvrtý Řádek}
\begin{tabular}{|c|c|c|c|c|c|c|c|c|c|c|c|}
\hline
í&v&á&t&e&,& &n&e&n&í& \\
\braillebox{3478}&\braillebox{1236}&\braillebox{16}&\braillebox{2345}&\braillebox{15}&\braillebox{2}&\braillebox{}&\braillebox{1345}&\braillebox{15}&\braillebox{2345}&\braillebox{34}&\braillebox{}\\
\hline
\end{tabular}

Měla problém s posunem na selektor tím, že neposouvala dostatečně daleko.  Četla `v' nejdřív jako `l'.  Zbytek řádku četla správně.

\paragraph{Pátý Řádek}
\begin{tabular}{|c|c|c|c|c|c|c|c|c|c|c|c|}
\hline
p&r&v&n&í& &ř&á&d&e&k&,\\
\braillebox{123478}&\braillebox{1235}&\braillebox{1236}&\braillebox{1345}&\braillebox{34}&\braillebox{}&\braillebox{1235}&\braillebox{16}&\braillebox{145}&\braillebox{15}&\braillebox{13}&\braillebox{2}\\
\hline
\end{tabular}
Začatek řádku četla správně.  Četla `k' nejdřív jako `a'.  Zase nepoznala `,' ale věděla, že jenom druhý bod je nahoru.

\paragraph{Šestý Řádek}
\begin{tabular}{|c|c|c|c|c|c|c|c|c|c|c|c|}
\hline
 &k&t&e&r&ý& &z&o&b&r&a\\
\braillebox{78}&\braillebox{13}&\braillebox{2345}&\braillebox{15}&\braillebox{1235}&\braillebox{12346}&\braillebox{}&\braillebox{1356}&\braillebox{135}&\braillebox{12}&\braillebox{1235}&\braillebox{1}\\
\hline
\end{tabular}

Četla všechno správně.

\paragraph{Řádek 7}
\begin{tabular}{|c|c|c|c|c|c|c|c|c|c|c|c|}
\hline
z&u&j&e& &j&e&n&o&m& &j\\
\braillebox{135678}&\braillebox{136}&\braillebox{245}&\braillebox{15}&\braillebox{}&\braillebox{245}&\braillebox{15}&\braillebox{1345}&\braillebox{135}&\braillebox{134}&\braillebox{}&\braillebox{245}\\
\hline
\end{tabular}

Četla `n' nejdřív jako `d'. Jinak četla všechno správně.

\paragraph{Řádek 8}
\begin{tabular}{|c|c|c|c|c|c|c|c|c|c|c|c|}
\hline
e&d&n&o& &p&í&s&m&e&n&o\\
\braillebox{1578}&\braillebox{145}&\braillebox{1345}&\braillebox{135}&\braillebox{}&\braillebox{1234}&\braillebox{34}&\braillebox{234}&\braillebox{134}&\braillebox{15}&\braillebox{1345}&\braillebox{135}\\
\hline
\end{tabular}

Četla všechno správně.

\paragraph{Řádek 9}
\begin{tabular}{|c|c|c|c|c|c|c|c|c|c|c|c|}
\hline
.& & &P&r&v&n&í& &b&y&l\\
\braillebox{378}&\braillebox{}&\braillebox{}&\braillebox{12347}&\braillebox{1235}&\braillebox{1236}&\braillebox{1345}&\braillebox{34}&\braillebox{}&\braillebox{12}&\braillebox{13456}&\braillebox{123}\\
\hline
\end{tabular}

U první písmena koukala na mně s nepochopeným.  Vysvětlil jsme ji, že sedmi a osmi body nahoře naznačuji začátek řádku, a že třetí bod je tečka.  Četla zbytek správně. Nepoznala, že `P' je velký.  Anebo aspoň neřekla.  Četla `b'(správně) jako `,' protože vypadl drát v zobrazovače.  Spravil jsem to a pokračovala bez chyb.

\paragraph{Řádek 10}
\begin{tabular}{|c|c|c|c|c|c|c|c|c|c|c|c|}
\hline
 &v&y&n&a&l&e&z&e&n& &v\\
\braillebox{78}&\braillebox{1236}&\braillebox{13456}&\braillebox{1345}&\braillebox{1}&\braillebox{123}&\braillebox{15}&\braillebox{1356}&\braillebox{15}&\braillebox{1345}&\braillebox{}&\braillebox{1236}\\
\hline
\end{tabular}

Obecně nečetla mezery, jenom je přeskočila.  Je možný, že je nečetla protože bylo by zbytečně je číst.  Je taky možný, že řídila na selektor podle cvaknutí zobrazovače a ne podle citu senzoru.  Četla `y' původně nesprávně, myslím `ď' ale s tou nahrávkou jistý byt nemohu.  Zbytek řádku četla správně.

\paragraph{Řádek 11}
\begin{tabular}{|c|c|c|c|c|c|c|c|c|c|c|c|}
\hline
 &r&o&c&e& &1&9&1&3& &v\\
\braillebox{78}&\braillebox{1235}&\braillebox{135}&\braillebox{14}&\braillebox{15}&\braillebox{}&\braillebox{18}&\braillebox{248}&\braillebox{18}&\braillebox{148}&\braillebox{}&\braillebox{1236}\\
\hline
\end{tabular}

Četla začátek správně. Četla `1' jako `a'.  Vysvětlil jsem ji, že osmi bod naznačuje čísla.  Četla správně. Četla `9' jako `i'.  Zeptal jsem se ji, jestli osmi bod je nahoru.  Pak upravila se na `9'.  Zbytek četla správně.

\paragraph{Řádek 12}
\begin{tabular}{|c|c|c|c|c|c|c|c|c|c|c|c|}
\hline
 &A&n&g&l&i&i&.& & &J&m\\
\braillebox{78}&\braillebox{17}&\braillebox{1345}&\braillebox{1245}&\braillebox{123}&\braillebox{24}&\braillebox{24}&\braillebox{3}&\braillebox{}&\braillebox{}&\braillebox{2457}&\braillebox{134}\\
\hline
\end{tabular}

Četla začátek správně.  Nenaznačila, že `O' je velký. Myslela, že druhý `i' je `9' ale upravila se když jsem ji řekl, že není.  Kdy dorazila na `J' zase na mně koukala s nepochopením.  Viděl jsem, že pravě hmatala na body 2 a 5 ale ne na 4.  Řekl jsem ji \uv{ještě nahoře} a ona to správně odhadla.  Neřekla, že to je velký.

\paragraph{Řádek 13}
\begin{tabular}{|c|c|c|c|c|c|c|c|c|c|c|c|}
\hline
e&n&o&v&a&l& &s&e& &O&p\\
\braillebox{1578}&\braillebox{1345}&\braillebox{135}&\braillebox{1236}&\braillebox{1}&\braillebox{123}&\braillebox{}&\braillebox{234}&\braillebox{15}&\braillebox{}&\braillebox{1357}&\braillebox{1234}\\
\hline
\end{tabular}

Nevím co se přesně stal u začátek řádku, kvůli zavadám nahrávky.  Četla `o' opět jako `r' ale opravila se když jsem ji řekl, že to není správný.  Neřekla, že o je velké.

\paragraph{Řádek 14}
\begin{tabular}{|c|c|c|c|c|c|c|c|c|c|c|c|}
\hline
t&o&f&o&n& &a& &p&ř&e&v\\
\braillebox{234578}&\braillebox{135}&\braillebox{124}&\braillebox{135}&\braillebox{1345}&\braillebox{}&\braillebox{1}&\braillebox{}&\braillebox{1234}&\braillebox{2456}&\braillebox{15}&\braillebox{1236}\\
\hline
\end{tabular}

Četla `t' nejdřív jako `j'.  Četla `p' nejdřív jako `n'? Opravila se, když jsem ji řekl, že to není správný. Jinak četla správně.  Ale myslela, že je konec řádku už u písmena `e'.

\paragraph{Řádek 15}
\begin{tabular}{|c|c|c|c|c|c|c|c|c|c|c|c|}
\hline
á&d&ě&l& &s&v&ě&t&l&o& \\
\braillebox{1678}&\braillebox{145}&\braillebox{126}&\braillebox{123}&\braillebox{}&\braillebox{234}&\braillebox{1236}&\braillebox{126}&\braillebox{2345}&\braillebox{123}&\braillebox{135}&\braillebox{}\\
\hline
\end{tabular}

Myslela si, že `ě' je `v'.  Četla `s' jako `i' ale opravila se stejným dechem.  Taky četla `t' jako `j' a opravila se stejným dechem.

\paragraph{Řádek 16}
\begin{tabular}{|c|c|c|c|c|c|c|c|c|c|c|c|}
\hline
n&a& &z&v&u&k&.& & &K&d\\
\braillebox{134578}&\braillebox{1}&\braillebox{}&\braillebox{1356}&\braillebox{1236}&\braillebox{136}&\braillebox{13}&\braillebox{3}&\braillebox{}&\braillebox{}&\braillebox{137}&\braillebox{145}\\
\hline
\end{tabular}

Četla všechno správně.

\paragraph{Řádek 17}
\begin{tabular}{|c|c|c|c|c|c|c|c|c|c|c|c|}
\hline
y&ž& &č&t&e&n&á&ř& &p&o\\
\braillebox{1345678}&\braillebox{2346}&\braillebox{}&\braillebox{146}&\braillebox{2345}&\braillebox{15}&\braillebox{1345}&\braillebox{16}&\braillebox{2456}&\braillebox{}&\braillebox{1234}&\braillebox{135}\\
\hline
\end{tabular}

Četla `t' nejdřív jako `j' uvědomila, že to není správně stejným dechem. Četla `n' nejdřív jako `k'.  Vyslovila mezeru.

\paragraph{Řádek 18}
\begin{tabular}{|c|c|c|c|c|c|c|c|c|c|c|c|}
\hline
h&y&b&o&v&a&l& &s&p&e&c\\
\braillebox{12578}&\braillebox{13456}&\braillebox{12}&\braillebox{135}&\braillebox{1236}&\braillebox{1}&\braillebox{123}&\braillebox{}&\braillebox{234}&\braillebox{1234}&\braillebox{15}&\braillebox{14}\\
\hline
\end{tabular}

Četla všechno správně.

\paragraph{Řádek 19}
\begin{tabular}{|c|c|c|c|c|c|c|c|c|c|c|c|}
\hline
i&á&l&n&í&m& &p&e&r&e&m\\
\braillebox{2478}&\braillebox{16}&\braillebox{123}&\braillebox{1345}&\braillebox{34}&\braillebox{134}&\braillebox{}&\braillebox{1234}&\braillebox{15}&\braillebox{1235}&\braillebox{15}&\braillebox{134}\\
\hline
\end{tabular}

Četla všechno správně.

\paragraph{Řádek 20}
\begin{tabular}{|c|c|c|c|c|c|c|c|c|c|c|c|}
\hline
 &n&a&d& &p&í&s&m&e&n&e\\
\braillebox{78}&\braillebox{1345}&\braillebox{1}&\braillebox{145}&\braillebox{}&\braillebox{1234}&\braillebox{34}&\braillebox{234}&\braillebox{134}&\braillebox{15}&\braillebox{1345}&\braillebox{15}\\
\hline
\end{tabular}

Teď jsem se ji poprosil, aby přečetla slova, ne jednotlivá písmena.  V té chvílí, přestala reagovat nastroj, tak jsem to resetoval.  Četla \uv{nad} správně.

\paragraph{Řádek 21}
\begin{tabular}{|c|c|c|c|c|c|c|c|c|c|c|c|}
\hline
m&,& &p&ř&í&s&t&r&o&j& \\
\braillebox{13478}&\braillebox{2}&\braillebox{}&\braillebox{1234}&\braillebox{2456}&\braillebox{34}&\braillebox{234}&\braillebox{2345}&\braillebox{1235}&\braillebox{135}&\braillebox{245}&\braillebox{}\\
\hline
\end{tabular}

Četla \uv{písmenem} správně.  Četla \uv{přís} říkala něco(není rozumět na nahrávce), poradil jsem ji, ať hledá mezeru aby přečetla od začátku slova.  Našla mezeru, četla znovu, a správně řekla \uv{přístroj}.

\paragraph{Řádek 22}
\begin{tabular}{|c|c|c|c|c|c|c|c|c|c|c|c|}
\hline
b&z&u&č&e&l&.& & &T&o& \\
\braillebox{1278}&\braillebox{1356}&\braillebox{136}&\braillebox{146}&\braillebox{15}&\braillebox{123}&\braillebox{3}&\braillebox{}&\braillebox{}&\braillebox{23457}&\braillebox{135}&\braillebox{}\\
\hline
\end{tabular}

Četla \uv{bzučel} správně.  Řekl jsem ji, že tím ukončíme a zeptal jsem ji jestli má k tomu nějaké poznámky.  Ona řekla, že ne a rychle odešla.
