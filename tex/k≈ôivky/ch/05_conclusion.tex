\chapter{Závěr}
Čekal jsem, že bude žákům trvat déle porozumět stroji.  Také jsem čekal, že budou schopní číst rychleji. Byl jsem překvapen, že většina žáků rozuměla základnímu principu fungování stroje téměř okamžitě. Dalším překvapením bylo, že někteří žáci se během setkání skoro ve čtení nezrychlili.

Zahájil jsem studii s tím, že chci zkoumat, jak nevidomí sami od sebe objevují funkčnost nástroje. Moji žáci byli však příliš pasivní a vždy čekali na můj příkaz. Asi se stroje, který při dotyku vydá strašlivé cvaknutí, báli.

Nikdy jsem jim neřekl, že musejí čekat na moje potvrzení správného přečtení, než budou pokračovat. Nikdy jsem neřekl \uv{teď budeme drilovat čtení po jednotlivých písmenech}. Někteří zvládli stroj natolik dobře, že samostatně četli pár písmen. Do drilování jsme upadli bezděčně, aniž bych ho někdy explicitně vyžadoval. Čekal jsem, že jejich způsob čtení se během setkání zlepší, ale žádnou zásadnější adaptaci jsem nepozoroval. Během předběžné studie byl stroj ještě natolik nedokonalý a zranitelný, že přísné prostředí bylo namístě, během hlavní studie už k němu nebyl důvod. Soudím, že čtení jako takové v sobě obsahuje jistou míru strnulosti. Čteme zleva doprava, od začátku ke konci, není jiný způsob.

Žáci byli zvyklí čekat a poslouchat, neuměli se však adaptovat.

Vrátíme-li se k otázkám uvedeným v kapitole 1.5 Cíle Práce, už můžeme na některé z nich odpovědět.

\em Jakým způsobem lidé reagují na neznámý nástroj během prvních 45 minut používání? Jak se novému nástroji přizpůsobují? Jaké jsou jejich dojmy? \em  Moji účastníci velmi rychle pochopili fungování nástroje, ale nějakou dobu jim trvalo, než byli schopni toto pochopení využít v praxi.  Jejich důvěra ve vlastní schopnost naučit se používat FCHAD přímo odvisela od toho, jestli sami používali braillské řádky.

\em Jaká je role pedagoga v procesu podpory učení práce s novým nástrojem?\em Je jasné, že pro nevidomé je pedagog nezbytnou součástí procesu učení práce s novým nástrojem.  Všichni účastníci potřebovali pomoc s orientací na novém nástroji a opravou chyb při čtení.  Pedagogická intervence byla zvlášť nezbytná u žáka 9, který po většinu doby vůbec nezvládal číst samostatně.

Více informací o této studii naleznete v angličtině na webové stránce: \footnote{\url{https://github.com/timthelion/fchad-study}}.  S veškerými otázkami týkajícími se projektu FCHAD se obracejte na Timothyho Hobbse {\tt timothyhobbs@seznam.cz}.  Technické otázky týkající se návrhů braillských řádků směřujte na emailový seznam
\footnote{\url{ http://mielke.cc/brltty/contact.html#list}}.


