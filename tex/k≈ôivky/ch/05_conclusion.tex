\chapter{Závěr}
Čekal jsem, že to bude trvat déle rozumět stroj ale, že účastnici budou schopní rychlejší číst. Byl jsem překvapení z toho, že účastnici rozuměli základní fungování stroje skoro hned ale četli hodně pomalu. Některé účastnici skoro nezlepšili rychlost během setkání.

Začal jsem studie tím, že jsem chtěl vyzkoumat jak nevidomé lidí samy od sebe objevuji funkčnost nastroje. Moje účastnici byli až moc pasivní na to, a vždy čekali na můj příkaz. Asi se báli stroj, který vydá strašlivé cvaknutí když do ně dotykáte.

Měl jsem na mysl, že Andersonův kognitivní stadium bude trvat dlouho, že budou se strojem hrát. Třeba, kdybych čekal déle by hráli. Nikdy jsem je nedal přikázaní, že musejí čekat na moje potvrzení správné čtení aby pokračovali. Však, jsem je nikdy neříkal \uv{teď budeme drilovat čtení po jednotlivých písmenech}. Některé zvládly stroj dostatečně dobře, že samostatně četli par písmen ale do větší míru než ne spadli jsme do rigidní drilování, které jsem nikdy explicitně nepřikazoval. Během předběžní studium, stroj ještě byl dost vadní, že se rigidnost hodila ale u běžní studium už nebyl k tomu důvod. Asi \uv{čtení} má v sobě určíte míru rigiditu. Čteme z levá do práva od začátku ke konci není jiný způsob.

Účastnici byly snad jen tak zvykly čekat a poslouchat.

Vratíme-li k otázky uvedené v kapitole 1.5 Cíle Práce, už můžeme na některé z nich odpovědět.

\em Jakým způsobem lidé reagují na neznámým nastroj během prvních 45 minut používání? Jak oni přizpůsobují k novému nastroje? Jaké mají potom dojmy?\em  Moje účastnici velmi rychle pochopili fungování nastroje ale pomalu získali schopnost to pochopení používat.  Její důvěru v vlastní schopnost s naučit používat FCHAD byl přímo spojení tím jestli samy používali braillské řádky.

\em Jaká je role pedagoga v procesu podpory učení práce s novým nástrojem?\em Je snad jasný, že pro nevidomé lidí pedagog je nezbytný součásti proces učení neznámé nastroje.  Všechní účastníci potřebovali pomoc s orientaci s novým nástrojem a oprava chyb pří čtení.  Pedagogický intervence byl zvlášť nezbytný u žáka 9, který většina dobu čtení nezvládl vůbec číst samostatně.

Vice informace o teto studie nalezte v angličtině na webové strance: \footnote{\url{https://github.com/timthelion/fchad-study}}.  Veškeré otázky o projektu FCHAD pošlete Timothy Hobbs {\tt timothyhobbs@seznam.cz}.  Technické otázky o návrh braillských řádků pošlete na emailový seznam
\footnote{\url{ http://mielke.cc/brltty/contact.html#list}}.


