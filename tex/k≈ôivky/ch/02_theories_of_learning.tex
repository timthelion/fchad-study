\chapter{Teorie}

\section{Pedagogický a psychologický výzkum}

\subsection{Role pedagoga v drilování}

Když jsem si poseděl s těma nevidomým lidmi, rychle jsem se dozvěděl, že nevidomé lidi, které nechápou co mají dělat mají obrovskou schopnost sedět bez pochybu.  Musel jsem je nějak vest k poznání nového nastroje.  Postup, který jsem úplně přirozeně a bez myšlenek zvolil byl postupem drilování.  Když jsem nevidomému člověka říkal aby přečetl na nový nastroj neuměli.  Ale uměli číst první písmeno, které se zobrazuje kdy položili ruce na selektor, a s nějaký pomoci, uměli i přestupovat dal na druhý písmeno.  A tak jsme trénovali.

Někdy jste zkusil trénovat v matice sám doma?  S papírovou knihou těžko jde když nepochopíte zadaný úkolu, vůbec nemate možnost pokračovat.  Pokud rozumíte ale nejste jistý, může Vám pomoc ty "odpovědí v zádě."  Mnoho učebnic je nemají, protože méně motivované žáci by jen koukali v zádě, a nepočítali vůbec.  Nemůžete dat odpovědi v zádě když chcete aby méně motivované žáci něco získali z domácí úkolu.

Je jistý, že mrtvý papír nemůže byt dokonalým prostředkem drilování.  Lepší variant najdeme v počítače.  Počítač muže nám říct jestli jsme počítali správně až potom co zkusíme.  Dále nám dává emoční podporu.  Za každý pět správných odpovědí, hraje se zvuk a animace. Taylor v \em Theory of Learning fo r the Mobile Age\em protestuje, že počítač neumí dozvědět kdy žák něco špatně pochopil a pomáhat mu v pochopení protože neumí sám pochopit\citep{taylor2010theory}. Počítač nemůže Vám pomáhat v případě, že nevíte jak se začínat.

V tom případě, je nutný mít učitele.

Hlavní úkoly učitele v drilování jsou potvrzení odpovědí, vysvětlení nejasností a upřímné lhaní.  To je, \uv{ano Tome, máš to správně} nebo \uv{ne, to nebyl úplně přesný}. \uv{No, musíte nejdřív...} anebo \uv{Jak jste to vyřešil u čísla 5?} a konečně \uv{Děláš to moc dobře, jseš hodně chytrý kluk.}

\subsection{Vliv pedagoga v drilování}

Každý pedagog je jiný a to bude mít vliv na výsledky žáků.  Pedagogické vlastnosti jsou, emoční citlivost, schopnost rychle potvrdit odpovědí, jistotu v potvrzení či vysvětlení, správnost potvrzení nebo vysvětlení.

\subsection{Podstata prvních dojmů a neznámost materiálů}

Pedagog je výjimečně podstatní když jde o neznámy materiál.  Mnoho krát, pokročilý žák se učí líp sám než s pomoci.  Jim to upevňuje samostatnosti a schopnosti rozumět složité materiály.  Žák, který předmět ještě nerozumí naopak sám vůbec neumí dělat pokroky.  Sám neumí ani začít.

Je obecně známý fakt, že první dojmy jsou hodně důležité.  Často jen toho jak se mluví o novém oboru v halu mezi žáci ovlivňuje schopnost žáka to rozumět.

Obory a koncepty jsou často kulturně těžké rozumět.  Tím, že v kultuře se říká, že geometrie je těžká se stává těžká.  Kulturně předurčená složitost muže mít vliv i na moje experiment. Ačkoliv FCHAD je kulturně neznámý nastroj, \uv{nová technologie} je.  Člověk, který má rad novou technologie snad bude mít lehčí čas se učit než člověk, který ji nesnese.

\section{Vybrané teorie a modely učení}

V Průchově \em Přehledu Pedagogiky \em najdeme vymezení dva základní pojmy; vzdělávání a výchova.  Podle Průchy výchova je \uv{záměrné působení na osobnost jedince s cílem dosáhnout změn v různých složkách osobnosti} a vzdělávání je \uv{záměrné a organizované osvojování poznatků, dovedností, postojů, aj., typicky realizovaný prostřednictvím školního vyučování}\citep[str. 16-17]{prucha2009prehled}

Drilování může obsahovat implicitní a explicitní výchovu.  Drilování může implicitně upevňovat kázeň anebo s pomoci pedagoga explicitně potvrdit sebejistotu žáka a sílit jeho důvěr v trpělivosti a práci.  Moje studie by ale neměla mít výchovnou část anebo jestli má to je jenom v tom, že by moje žáci neměli odejít s pocitem, že novou technologie je jen zbyteční ztížení.

Primární účelem drilování je vzdělávací.  Však primární účelem pedagogika v svobodných zemí je vzdělávací a nikoliv výchovný. I když drilování je jeden z základních způsobů vzdělávání pedagogická věda moc o drilování nezajímá. Když jsem našel mnoho studie na křivek učení a drilování všechní byly z psychologických časopisů.

\subsection{Rozdělení učení podle Swifta}

V roce 1903 Edgar James Swift vydal tři studie, které měly 

\subsection{Andersonův(Fittův) stupně získávání dovednosti} %stages of skill acquisition

Základní rozdělení pamětí v psychologie je mezi deklarativní(explicitní) znalosti a procedurální(implicitní) znalost.  Když zapamatujete telefonní číslo můžete ho pamatovat či explicitně anebo implicitně. Explicitní paměť je to, když mohu odpovídat na otázku \uv{jaké je Váš telefonní čislo?} a implicitní paměť je když můžete zadat to číslo do telefonů bez myšlení. Někdy se Vám stálo, že jste zapomněl nějaké telefonní číslo ale ještě uměl ho zadat do telefonů?  Procedurální a deklarativní paměť jsou pevně rozdělení.

Procedurální paměť má nevýhodu, že se nemění.  Když umíte zadat telefonní číslo implicitně, děláte to po každý stejně.  Toho vlastnost procedurálního paměti se říká \em procedurální invariace\em . Abychom se učili něco procedurálně musíme nejdřív explicitně rozumět.  I když procedurální paměť je rozdělení od deklarativní paměť učíme procedurální věci přes deklarativní chápání.  Deklarativní paměť nám umožňuje učit se a vylepšovat naše dovednosti.  Procedurální paměť nám automatizovat dovednosti a činit je bez nepřiměřené námahy\citep[str. 105]{robert1994handbook}.

Andersonův model rozděluje tři stupni procedurální paměť:

\em Kognitivní stadium\em : V toto stadiu žák je vyučování anebo se snaží sám pochopit úkol.  Znalost je explicitní \uv{jak a co}.

\em Asociativní stadium\em : Stadium zlepšení a opravy pochopení.  Jak žák začíná procvičit jeho novou schopnost objevuje, že jeho původní pochopení dovednosti nebyl úplně správní.  Upravuje chyby v pochopení a jeho dovednost se postupně zdokonalí.

\em Autonomní stadium\em : Procedurální znalost.  Už žák osvojil dovednost. Všechno jde teď snadno a implicitně.  Umí činit bez myšlenek. Pracuje rychle a skoro bez chyb\citep{o1987some}.

\subsection{Vlastní model drilování}

\em Před znalost \em Člověk má určitý znalost už předtím než začíná cvičit. Například, jestli chci zapamatovat data o dějinný válek, mě pomůže znalost o dějinný umění.  Jestli chci se učit psát na klávesnice mě pomůže znalost abecedy. Pokud chci se učit hrát na klavír asi už vím co to je hudba.  Před-znalost není však stejně jako prerekvizitní znalost.  Některé před-znalosti mohou byt prerekvizitou pro osvojení novou dovednost ale jsou spousta znalosti, které pomůžou ale nejsou nutné znát.  Klavírista snaž se učit hrát na flétnu ale nemusíte umět hrát na klavír abyste se učil hrát na flétnu.

\em Chápání úkolu\em : Drilování funguje jenom když úkol je dostatečně jednoduchý, že chápání jednotlivých kroků úkolu není příliš náročné.  Jestli chci se učit hrát na housle, nemusím mít v hlavě model celého procesu, stačí když mám chytrý učitel, který umí rozdělit složitý dovednost na pochopitelné částky.

Často stadia chápání úkolu se probíhá jako demonstrace.  Nejdřív učitel ukazuje jak se něco děla, potom se snaží žák napodobit.  To jde těžko s nevidomým lidmi.

Další způsob chápání úkolu je přečtení anebo poslouchání návodu.

Třetí možnost je explicitní vedení.  Například, když žák se naučí chodit míč bereme jeho ruku a vedeme správnou cestu přes vzduchu aby věděl \uv{jak to má cítit.}

\em Obecný schopnost\em :

Můžeme pochopit drilování jako seznám podobných úkolu, který občas opakuje.

\em Konkretní schopnosti\em :

\em Dosažené úroveň\em :

Dosažitelný úroveň

Kognitivní bagáž

Emoční bagáž

\section{Role curve fitting v studii křivek učení}

