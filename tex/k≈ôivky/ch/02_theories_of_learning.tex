\chapter{Teorie}

\section{Pedagogický a psychologický výzkum}

\subsection{Role pedagoga v drilování}

Když jsem s účastníky studie usedl k práci, brzy jsem poznal, že nevidomí, kteří nechápou, co mají dělat, mají obzvláště vyvinutou schopnost sedět bez hnutí.  Musel jsem je k poznání nového nástroje vést.  Postup, který jsem zcela přirozeně a bezmyšlenkovitě zvolil, byla metoda drilování.  Když jsem nevidomému člověku řekl, aby na novém nástroji četli, nedovedli to.  Ale dovedli přečíst první písmeno, které se zobrazilo, když položili ruce na selektor, a s mou pomocí dovedli i přejít na další písmeno.  Takto jsme tedy trénovali.

Zkusili jste si někdy sami procvičovat matematiku?  Pokud nepochopíte zadání úkolu v učebnici, vůbec nemáte možnost pokračovat.  Pokud zadání rozumíte, ale nejste si jistí, můžou Vám pomoci výsledky vzadu v učebnici.  V mnoha učebnicích ale nejsou, protože méně motivovaní žáci by jen koukali dozadu a nepočítali vůbec.  Pokud se má méně motivovaný žák něco naučit z domácího úkolu, nesmí mít k dispozici výsledky.

Je jisté, že papír nemůže být dokonalým prostředkem drilování.  Lepší variantou je počítač.  Ten nám řekne, jestli jsme počítali správně, až poté, co to zkusíme.  Dále nám dává emoční podporu.  Za každých pět správných odpovědí se spustí znělka a animace. Taylor v \em Theory of Learning for the Mobile Age\em protestuje: Počítač se nedozví, že žák něco špatně pochopil, a pomoci mu v pochopení nemůže, protože sám nechápe \citep{taylor2010theory}. Počítač Vám také nepomůže v případě, že nevíte, jak začít.

V takovém případě je nutné mít učitele.

Hlavními úkoly učitele v drilování jsou potvrzení odpovědí, vysvětlení nejasností a někdy také lhaní.  Například: \uv{Ano Tome, máš to správně} nebo \uv{Ne, to nebylo úplně přesně}. \uv{No, musíš nejdřív...} anebo \uv{Jak jsi to vyřešil u čísla 5?} a konečně \uv{Děláš to moc dobře, jsi moc chytrý kluk.}

\subsection{Vliv pedagoga v drilování}

Každý pedagog je jiný a to bude mít vliv na výsledky žáků.  Žádoucími vlastnostmi pedagoga jsou emoční citlivost, schopnost rychle potvrdit odpovědi, správnost potvrzení nebo vysvětlení a sebejistota pokud jde o vlastní znalosti.

\subsection{Podstata prvních dojmů a neznámé látky}

Pedagog je obzvlášť důležitý, když jde o neznámou látku.  Pokročilý žák se často lépe učí sám než s cizí pomocí.  Upevňuje to jeho samostatnost a schopnost rozumět složité látce.  Žák, který látce ještě nerozumí, naopak sám vůbec neudělá pokroky.  Sám nedovede ani začít.

Je obecně známý fakt, že první dojmy jsou hodně důležité.  Často i to, jak se o novém oboru mluví na chodbě mezi žáky, ovlivňuje schopnost žáků mu porozumět.

Takto předurčená složitost může mít vliv i na můj experiment. Člověk, který má rád nové technologie, se na FCHADu pravděpodobně bude učit snadněji než člověk, který je nesnáší.

\section{Vybrané teorie a modely učení}

V Průchově \em Přehledu Pedagogiky \em najdeme vymezení dvou základních pojmů: vzdělávání a výchovy.  Podle Průchy je výchova \uv{záměrné působení na osobnost jedince s cílem dosáhnout změn v různých složkách osobnosti} a vzdělávání je \uv{záměrné a organizované osvojování poznatků, dovedností, postojů, aj., typicky realizovaný prostřednictvím školního vyučování} \citep[str. 16-17]{prucha2009prehled}

Drilování může obsahovat implicitní a explicitní výchovu.  Drilování může implicitně upevňovat kázeň anebo s pomocí pedagoga explicitně potvrdit sebejistotu žáka a posílit jeho důvěru v trpělivost a práci.  Moje studie by neměla mít výchovný aspekt. Pokud nějaký mít bude, pak jen v tom, že by mí žáci neměli odejít s pocitem, že nová technologie je jen zbytečná komplikace.

Primárním účelem drilování je vzdělání.  I když drilování je jeden ze základních způsobů vzdělávání, pedagogická věda se drilováním příliš nezabývá. Všechny studie křivek učení a drilování, které jsem našel, byly z psychologických časopisů.

\subsection{Rozdělení učení podle Swifta}

V roce 1903 Edgar James Swift vydal tři studie, které měly demonstrovat tři druhy učení. Uvedu zde jeho rozdělení, protože se přímo vztahuje k přísnému drilování.  Dalším důvodem, proč uvádím Swifta, je, že Swift psal před současnou módou nelidského \uv{objektivismu}.  Proto v jeho díle najdeme mnoho poznámek, které \uv{nepatří do objektivní vědy}.  Zajímavé na jeho studiích jsou především \uv{pedagogické rady a doporučení}, ve kterých výsledky studie dává do souvislosti s jejich uplatněním v pedagogice. %Pedagogic hints and suggestions

Swiftovy druhy učení a související studie jsou:

\em získání dovednosti, neboli naučit se něco dělat\em : Účastníci se učili házet dva míče jednou rukou. % The acquisiton of skill, or learning to do;

\em získání znalostí, uvědomování souvislostí \em : Účastník se učil psát a číst těsnopis.  % The formation of associations, or the acquisiotn of information

\em získání sebekontroly a potlačení reflexů \em : Účastníci se učili ovládat bezděčné mrkání - seděli a koukali přes skleněnou tabuli, jak na ni badatel klepá dřevěným kladívkem, přičemž se měli snažit co nejméně mrkat.  % The getting of control, or the formation of inhibitions

 \citep[str.201,224,230,231]{swift1903studies}

\subsection{Andersonovy (Fittovy) stupně získávání dovednosti} %stages of skill acquisition

Základní rozdělení paměti v psychologii je na deklarativní (explicitní) a procedurální (implicitní).  Když si zapamatujeme telefonní číslo, můžeme si ho pamatovat explicitně a/nebo implicitně. Explicitní paměť je to, když mohu odpovědět na otázku \uv{Jaké je Vaše telefonní číslo?}; implicitní paměť je to, když mohu číslo vytočit na telefonu bez myšlení. Už se Vám někdy stalo, že jste zapomněli nějaké telefonní číslo, ale ještě jste ho dovedli vytočit?  Procedurální a deklarativní paměť jsou pevně rozděleny.

Procedurální paměť má tu nevýhodu, že se nemění.  Když umíte telefonní číslo vytočit implicitně, děláte to pokaždé stejně.  Tato vlastnost procedurální paměti se nazývá \em procedurální invariace\em . Abychom se něco procedurálně naučili, musíme tomu nejdříve explicitně rozumět.  I když procedurální paměť je od deklarativní paměti oddělená, procedurální dovednosti se učíme přes deklarativní chápání.  Deklarativní paměť nám umožňuje učit se a vylepšovat naše dovednosti.  Procedurální paměť nám umožňuje dovednosti automatizovat a činit je bez nepřiměřené námahy \citep[str. 105]{robert1994handbook}.

Andersonův model rozlišuje tři stupně procedurální paměti:

\em Kognitivní stadium\em : V tomto stadiu je žák vyučován anebo se sám snaží pochopit úkol.  Znalost je explicitní - \uv{co a jak}.

\em Asociativní stadium\em : Stadium zlepšení a opravy pochopení.  Jak žák svou novou schopnost procvičuje, objevuje, že jeho původní pochopení nebylo úplně správné.  Opravuje chyby v pochopení a jeho dovednost se postupně zdokonalí.

\em Autonomní stadium\em : Žák si osvojil procedurální dovednost. Všechno mu jde snadno a implicitně.  Umí činit bez přemýšlení. Pracuje rychle a téměř bez chyb \citep{o1987some}.

\subsection{Vlastní model drilování}

Předcházející modely jsou příliš psychologické.  Věnují pozornost pouze žákovi a zapomínají na interakci mezi žákem a učitelem.  Proto zde navrhuji model drilování, kde jsou uvedena nejenom stadia učení, ale i role učitele v každém stadiu.

\em Předznalost \em : Člověk má určité znalosti už předtím, než se začne učit něco nového. Například když si chci zapamatovat data z dějin válek, pomůžou mi znalosti z dějin umění.  Když se chci naučit psát na klávesnici, pomůže mi znalost abecedy. Když se chci naučit hrát na klavír, vím přinejmenším, co to je hudba.  Předznalost není však to samé co prerekvizitní znalost.  Některé předznalosti mohou být prerekvizitou pro osvojení nové dovednosti. Je však spousta znalostí, které pomůžou, ale nejsou nutné.  Klavírista, který se snaží naučit hrát na flétnu, je ve výhodě, ale člověk nemusí umět hrát na klavír, aby se naučil hrát na flétnu.  Swift odkazuje na vliv předznalosti v jeho studii o házení míčů: Studenti, kteří měli zkušenost s basebalem, se lépe naučili nový druh házení než studenti bez zkušenosti. Jiná studie od Jacqueline Thomas naznačuje, že znalost španělštiny pomůže studentům, kteří se chtějí naučit francouzsky \citep{swift1903studies,thomas1988role}.

Předznalost však nemusí být vždy výhodou. Může také dojít ke konfliktu mezi předznalostí a novým úkolem. Ve své vlastní studii jsem pozoroval, že někteří účastníci se snažili používat selektor způsobem, který se podobal použití piezoelektrického braillského řádku, pro FCHAD to však byl způsob nesprávný.  I když znalost podobného jazyka žákům pomůže v učení nového, existuje jev zvaný jazyková interference: Člověk, který umí dva jazyky, si někdy plete jejich slova, přičemž míra jazykové interference je tím vyšší, čím podobnější jsou si ony jazyky \citep{brauer1998stroop}.

{\bf Role učitele}: Určení předznalosti žáka a přizpůsobení vyučování tak, aby žák zbytečně neopakoval věci, které už zná, aby nová látka co nejvíc navazovala na předznalost žáka, a aby látka odpovídala zájmům a sklonům žáka.

\em Chápání úkolu\em : Odpovídá kognitivnímu stadiu u Andersona. Drilování funguje, jen když úkol je dostatečně jednoduchý, takže chápání jednotlivých kroků úkolu není příliš náročné.  Když se chci naučit hrát na housle, nemusím mít v hlavě model celého procesu, stačí, když mám chytrého učitele, který umí rozdělit složitý úkol na pochopitelné části.

Stadium chápání úkolu často probíhá formou demonstrace.  Nejdříve učitel předvede, jak se něco dělá, žák se ho posléze snaží napodobit.  S nevidomými to však nelze.

Další formou stadia chápání úkolu je přečtení nebo poslechnutí návodu.

Třetí možností je explicitní vedení.  Například když se žák učí hodit míčem, uchopíme ho za ruku a naznačíme správný pohyb vzduchem.

{\bf Role učitele}: Jasné vyložení úkolu a také vysvětlení, jakou schopnost má úkol rozvíjet a k čemu je dobrý.

Je velmi důležité, aby vyložení úkolu nezanechalo v žákovi dojem, že úkol je příliš složitý, případně \uv{jenom pro holky, jenom pro sportovce, jenom pro šprty, atd.}

Swift ve své studii o házení míčů poznamenává, že způsob, který účastníci použili na samém začátku učení, byl silný indikátor úspěchu. Swift z toho usuzuje, že dobrý učitel by měl žákům dát dobrý začátek.   \citep[str. 215]{swift1903studies}  Moje osobní zkušenost se se Swiftovou rozchází v tom, že když jsem se snažil pomoci účastníkům správně položit ruce na FCHAD, často tak vydrželi jen pár vteřin nebo protestovali, že můj způsob použití FCHADu jim nevyhovuje.
%215 swift "as when the learner is assistet by a good teacher"

\em Obecná schopnost\em  a \em konkrétní schopnost\em :

Drilování můžeme chápat jako seznam podobných úkolů, které se občas opakují.  Například když se učím psát na klávesnici a opisuji řádek "Táta touží po tequile." Úkol je rozdělen na jednotlivé kroky, v každém kroku píši jedno písmeno.  Jednotlivá písmena se opakují, například písmeno 't'.  Obecná schopnost je psaní na klávesnici, konkrétní schopnost je psaní písmena 't'. Konkrétní schopnosti se zdokonalují dřív než obecná schopnost.  Toto stadium můžeme chápat jako kombinaci Andersonova asociativního a autonomního stadia.  Konkrétní schopnosti se zdokonalují a začínají být autonomní, zatímco obecná schopnost je stále ještě v asociativním stadiu.

{\bf Role učitele}: Učitel může zdokonalení obecné schopnosti urychlit tím, že zjistí, které konkrétní schopnosti jsou už zdokonalené.  Učitel podle toho vybírá vhodné úkoly.  Dynamický výběr úkolu je dobře prozkoumaná oblast \citep{hintzman1976repetition} a i počítačové programy umí vybírat pro žáka vhodné úkoly \citep{anki}.

Učitel by také měl monitorovat výkon žáka a zajistit, že žák dělá úkoly správně a vhodným způsobem.

Během stadia chápání úkolu žák nemusí vždy pochopit, jak zvládnout všechny podúkoly drilování.  Například když drilujeme matematiku, často začínáme s jednoduchými úkoly a poté přejdeme ke složitějším.  To, že žák zvládnul prvních pět úloh, neznamená, že zvládne všechny.  Učitel by měl žákovi pomoci, když nemůže dál.

\em Dosažená úroveň\em : Když žák dostatečně dlouho opakuje drilování, další drilování už nezlepšuje jeho výkon.  Tomu by mělo odpovídat Andersonovo autonomní stadium.  Každý žák má svou vlastní maximální dosažitelnou úroveň.  Úroveň, za kterou se žák už nevylepší, nemusí být jeho maximální dosažitelná.  Řada věcí mu může v dosažení maxima zabránit.  Jedním z takových škodlivých jevů je kognitivní zátěž. Když žák často opakuje určitou chybu, často si plete dvě věci, může se u něj vytvořit slabost. Například když se žák učí psát na klávesnici a opakovaně píše 'b' namísto 'v'(dvě písmena, která jsou přímo vedle sebe na klávesnici), v budoucnosti bude zpomalovat vždy, když má napsat 'v'. Podobným jevem je emoční zátěž: Žák, který má pocit, že část úkolu je těžká, má ve výsledku horší výkon.

{\bf Role učitele}: Učitel může pomoci v obraně proti kognitivní a emoční zátěži.  Učitel by měl vycítit, když je žák unavený a frustrovaný. Učitel by měl také pečlivě volit rady.  Pokud jich dává příliš, kognitivní zátěž tolika explicitních úvah může žáka spíše zpomalit než mu pomoci.
