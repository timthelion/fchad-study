\chapter{Výzkum}

\section{Popis studie}

\subsection{První část: čtení tištěného textu}

Účastníci studie, dále jen "žáci", usedli ke stolu.  Nevěděli, co se bude dít, kromě toho, že vyzkouší nový druh braillského řádku.  FCHAD ležel na stole položený stranou tak, aby nepřekážel rukám.  Prvním úkolem, který žáci dostali, bylo přečíst následující text.

\begin{verbatim}
Děkuji za Vaši účast v této
studii.
Vyzkoušíme nový braillský řádek!

V rámci této studie shromáždím ně-
jaká data o Vás a o tom, jak čtete
Braillovo písmo.  Tato data budou
při zveřejnění anonymní. Jedná se
o Váš přibližný věk, dosažené vzdě-
lání, zkušenost s Braillovým pís-
mem, audionahrávku našeho setkání a
záznam Vaší práce s nástrojem. Než
budeme pokračovat, potřebuji Váš
souhlas se shromážděním a zveřej-
něním těchto dat.

Následující text čtěte prosím na-
hlas a bez přestání až do konce.
Pokud budete mít nějaké otázky,
zeptejte se prosím až potom.

Začátek textu:
Dnes budete seznámen s novým druhem
braillského řádku, který se nazývá
FCHAD.  Tento řádek je navržen
s úmyslem vytvořit nejlevnější možný
                                   1
\end{verbatim}
---- konec stránky ----
\begin{verbatim}

zobrazovač vůbec. Zobrazuje text po
jednotlivých písmenech. Zobrazení
se ovládá pomocí řádku senzorů. Po-
ložením prstu na jeden z dvanácti
kovových bodíků vybíráte, které
z písmen v řádku chcete zobrazit.
Používáte obě ruce: jedna vybírá
písmeno, druhá ho čte.

Konec textu.

Teď Vám položím několik otázek
a poté vyzkoušíme nástroj.

\end{verbatim}

Žáci četli nahlas. Text byl vytištěn na tuhém listu braillského papíru o rozměrech 26.5 x 30.5cm.  Maximální délka řádku je 36 znaků.  Braillovo písmo je šestibodové.  Text byl vytištěn oboustranně. Levá strana listu je děrovaná kvůli případnému svázání.

Abych posoudil žákovu dovednost číst rukama, čtení určité části textu (označené slovy "Začátek textu:" a "Konec textu.") jsem stopoval. Tato část textu má na první straně 138 znaků včetně mezer plus čtyři znaky předznačující velká písmena, celkem tedy 142 znaků. Na druhé straně má 262 znaků včetně mezer plus 3 znaky předznačující velká písmena, celkem tedy 265 znaků. Celý měřený text má 407 znaků.

Text byl vytištěn v Knihovně a tiskárně pro nevidomé K.E. Macana, která je jediným veřejným tiskařským závodem pro nevidomé v Praze.  Všichni žáci již měli zkušenosti s materiály tištěnými v tiskárně a předpokládám, že Braillovo písmo na listě užitém v této části studie bylo standardní.

Poté, co žák dočetl, musel čekat pár vteřin, než jsem připravil FCHAD. Pokud ještě nedal souhlas se zveřejněním informací, požádal jsem ho o něj.  Nikdo z oslovených osob účast na studii ani udělení souhlasu neodmítl.

\subsection{Otázky}

Po čtení a přípravě jsem žákům položil následující otázky:

Kolik Vám je let? (Nabídka věkových rozmezí po pěti letech)

Jak dlouho čtete Braillovo písmo?

Jaké je Vaše nejvyšší dosažené vzdělání?

Jaká je Vaše úroveň zrakového postižení?

Všichni žáci kromě jednoho uvedli jako svou úroveň zrakového postižení \uv{úplně nevidomý}.

Světová zdravotnická organizace (WHO) dělí zrakové postižení do pěti kategorií. Nejnižší dvě, pro které je Braillovo pímo relevantní, jsou:

\textit \uv{\textbf{Praktická slepota}

zraková ostrost s nejlepší možnou korekcí 1/60 (0,02), 1/50 až světlocit nebo omezení zorného pole do 5 stupňů kolem centrální fixace, i když centrální ostrost není postižena, kategorie zrakového postižení 4

\textbf{Úplná slepota}

ztráta zraku zahrnující stavy od naprosté ztráty světlocitu až po zachování světlocitu s chybnou světelnou projekcí, kategorie zrakového postižení 5}} \citep{sonsklasifikace}

\subsection{Čtení na FCHADu}

Po zodpovězení otázek jsem umístil FCHAD před žáka.  Nejprve jsem musel vyzkoušet, zda funguje správně. To byla pro žáka první zkušenost s FCHADem.  Slyšel, jak cvakají kovové tyčky, které vytvářejí písmeno na zobrazovači.  Poté, co jsem se ujistil, že přístroj funguje správně, jsem žáka vyzval, aby začal číst.  Samozřejmě to nedovedl.  Když se vidomí lidé setkají s novým nástrojem, obvykle jsou aktivní přinejmenším v tom, že pozorují, co se děje. Nevidomí lidé pozorovat nemohou a tak jen nehybně sedí a nic nedělají. To pro mě jako pro pedagoga byla zcela nová zkušenost. Domnívám se, že se bojí nástroje dotýkat, aby ho nerozbili, což je poměrně oprávněná obava.

Musel jsem žáka podnítit k tomu, aby se snažil číst. Položil jsem mu ruce na jednotlivé komponenty nástroje tak, jak by měly správně být. Levou ruku jsem umístil na zobrazovač (viz Obrázek 1-A) a řekl: \uv{Tady se budou po jednom zobrazovat písmena.}  Pravou ruku jsem umístil na selektor (viz Obrázek 1-B,C).

Nyní je namístě vysvětlit, jak přesně vypadají a fungují FCHADy, které byly použity ve studii.

\section{Popis FCHADu}

FCHAD má dvě součásti: zobrazovač, který zobrazí jedno Braillovo písmeno, a selektor, kterým se zobrazované písmeno vybírá.

\subsection{První selektor}

Žáci 2, 3 a 4 používali jiný selektor než pozdější žáci.  To kvůli technickým závadám v prvním selektoru.   Technologie selektorů je velmi jednoduchá.  Selektor má několik elektrod: jednu katodu a více anod. Když žák spojí svým tělem anodu s katodou, elektrický proud přes něj proudí od anody ke katodě. Podle toho, přes kterou anodu  proud teče, poznáme, které písmeno je vybrané   (selektováné).   První selektor měl 12 anod z napínáčků(viz Obrázek 1-C) a katodu v náramku (Obrázek 3).

Napínáčky byly nepravidelně umístěné 1.2 až 2 cm od sebe.  Celý selektor měřil 19 cm od prvního napínáčku k poslednímu.

Tento systém moc dobře nefungoval. Když jsem to zkusil sám, fungoval bezvadně, ale když se o to pokusil někdo s více ochlupeným zápěstím, katoda v náramku nedostala dobrý kontakt a nástroj měl potíže rozpoznat dotyk.

\subsection{Druhý selektor}

Žáci 4-9 pracovali s lepším selektorem (viz Obrázek 1-B). Nový selektor má dva řádky elektrod. Horní řádek tvoří anody a dolní řádek katody.  Proud už tedy nemusí procházet tělem, ale jenom bříškem prstu.  Spoj mezi anodou a katodou je tak daleko lepší, někdy však stále ještě nedostatečný: V několika případech jsem žákům musel namočit prsty vodou, aby dostali spolehlivý spoj.

Další vylepšení spočívá v tom, že senzory jsou umístěny pravidelně 0.5 cm od sebe.  Jsou tedy daleko blíž u sebe než na prvním selektoru.

\subsection{Zobrazovač}
Další součástí FCHADu je zobrazovač (viz Obrázek 1-A a 2).  Zobrazovač, který jsem použil ve studii, zobrazuje jedno osmibodové písmeno.  Zobrazovací technologie je opět velmi jednoduchá.  Jedná se o cívky, které elektromagnetickou silou vytáhnou tyčky (body písma) směrem vzhůru.  Moje cívky vydají až 200 gramů síly\citep{multicomp}.  To je dost na to, aby pohyb tyček člověka zabolel.  Každý bod je jen 0.45 cm vysoký, takže pokud člověk nedrží ruku na zobrazovači příliš silně, nemusí se bolesti bát.  Body nespadnou samy, čtenář je musí dotykem ruky shodit zpátky dolů.

Řádky bodů jsou 2 cm od sebe a sloupce 3 cm od sebe (v případě žáků 2 a 3 byly 5 cm od sebe).  Celý zobrazovací prostor měří 6 x 3 cm.  Mé tři nejdelší prsty měří 7 až 8.5 cm, proto dokážu číst písmena na zobrazovači prsty.  Někteří žáci, zejména ženského pohlaví, však měli prsty kratší a takto číst nemohli.  Plastová báze zobrazovače je 5.9 cm široká (8 cm u dřívějších žáků), 9.2 cm dlouhá a 6 cm vysoká.

\subsection{Jak to jde dohromady}

FCHAD je připojený k počítači. Počítač má v paměti tabulku 12ti písmen.  Písmena do tabulky vyplní speciální software, který se nazývá \em čtečka obrazovky\em .  Ve studii jsem používal čtečku obrazovky Orca 3.6.3. Orca vyplní tabulku podle toho, jaký text je poblíž kurzoru na obrazovce.

Selektorem čtenář vybírá, které z těchto dvanácti písmen se zobrazí na zobrazovači. Když čtenář pohybuje rukou na selektoru, slyší hlasité cvakání, jak tyčky mění polohu.

Sám jsem navrhl kompletní elektroniku stroje, programoval ovladač a firmware. To mi umožnilo zaznamenávat činnost celého stroje.  Během výzkumu jsem tak mohl zaznamenávat lokace prstů uživatele, rychlost čtení a polohu ruky.  Napsal jsem ještě další software, který pomáhá shromažďovat data\footnote{\url{https://github.com/timthelion/anonGraph}}.  Shromážděná data si lze prohlédnout na odkazu uvedeném v poznámce\footnote{\url{https://github.com/timthelion/fchad-study}}.

Přesnější technické detaily nástroje jsou k dispozici v angličtině v apendixu 1.

\subsection{Text na FCHAD}
Text tištěný na papíře, který žáci četli na začátku experimentu, pojednává o FCHADu (je to vlastně návod). Pro čtení na FCHADu jsem chtěl mít srovnatelný obsah, proto jsem napsal krátké dějiny FCHADu.

Celý text, který jsem měl připravený, najdete v apendixu.

\section{Moji žáci}

Žáci byli všichni starší 18ti let, takže mohli dát souhlas s použitím osobních údajů samostatně.  Účast na studii nebyla placená.

\subsection{Předběžná studie}

\subsubsection{Žáci 1 a 2}

První \uv{žák} o čem budu psát jsem Já Váš badatel.  Já jsem se učil číst rukama před dvěma roky. Zajímal jsem o Braillovo písmo protože mívám migrény když čtu.  Nějakou dobu jsem si řekl: \uv{naučím číst Braillovo písmo a už nebudu vůbec číst očima}.  Bylo to spíš emoční reakce na frustrace než praktický plán.  Rychle jsem se dozvěděl, že Braillovo písmo vůbec nejde číst rychle a, že braillský řádek je daleko příliš drahý koupit jenom vztekem.  Můj zájem o Braillova písma se změnil na hněv o tom jak je všechno pro nevidomé předražený a snahu stvořit něco levnější.

Umím číst rukama ale nijak plynule.  Už čtu daleko lepe FCHADem než obyčejný tisknutým Braillským písmem.  Je lehčí číst FCHADem protože písmo je velké.

Když jsem se začal číst na FCHAD četl jsem velmi pomalu.  Ještě jsem neznal Braillova písma dokonalý a často jsem spletl f \braillebox{124}, d \braillebox{145}, h \braillebox{125}, a j \braillebox{245}. Teď je už zvládnu bez problému.  Dozvěděl jsem, že připojený mezi \uv{psychologický rychlost čtení} a \uv{skuteční rychlost čtení} není moc silné.  Čtení se zda rychlejší, tím miň dělám chyby a tím víc rozumím text ale to neznamená, že zkutečně čtu rychlejší.  Čtení zrychluje daleko pomalejší než psychologický rychlost čtení.  Často jsem si myslel, že jsem dělal pokrok v svému schopnosti čtení ale když jsem koukal do nahrávky jsem dozvěděl, že rychlost či nezměnil, anebo jen mírně stoupal.

Pro mně, použity selektor je skoro bez myšlenek ale poznání znaků na zobrazovače stále někdy trvá.  Nastavěl jsem stroj tak, že se používá selektor pravou rukou a zobrazovač se čte levou. Dělal jsem to tak protože selektor je podobný jako počítačový myš a myš se používá pravou rukou(pro praváci).  Myslím si, že jsem to nastavěl obraceně a byl by lepší používat zobrazovač dominantní rukou.  Nastavený byl stejný pro všechní účastnici studie.

Dával jsem \uv{žáci 1 a 2} dohromady.  To je protože můj FCHAD vůbec nefungoval když jsem ho ukázal prvnímu účastníkovi. Nevím jakým kozlem to bylo ale když jsem držel Já katodu v ruce a používal stroj selektor fungoval bezvadně. Když on to vzal do ruky začal celý stroj zbláznit.




\subsubsection{Žák 3}

Setkání 6.3.2013

Žák 3 je vysokoškolsky vzdělaná žena ve věku mezi 45 a 50 lety. Braillovo písmo čte od začátku školní docházky.  Pracuje jako vysokoškolská vyučující němčiny.  Je úplně nevidomá.

Navštívil jsem ji během jejích konzultačních hodin odpoledne brzy po obědě.  Výzkum byl zpožděn tím, že jsme museli hledat prodlužovací kabel, neboť blízko pracovního stolu neměla zásuvku.  Zdála se mi čilá a zdravá.

Než jsem nástroj nastavil, nedotýkala se ho. Během našeho setkání došlo k technickým problémům. Senzory někdy \uv{cítily} dotyky, ke kterým ve skutečnosti nedošlo.  Přestože FCHAD moc dobře nefungoval, žákyně byla schopná na něm číst.

Začali jsme s tištěným návodem:

Četla tištěný text oběma rukama.

Čas čtení tištěného textu: 93.3 vteřin (údaj naměřený zpětně ručně pomocí nahrávek)

Počet znaků: 407

Rychlost čtení tištěného Braillova písma: 4.36 CPS (počet znaků za sekundu)

Čas čtení tištěného textu po otočení listu: 51.8 vteřin

Počet znaků: 265

Rychlost čtení: 5.11 CPS

Žákyně zřejmě není zvyklá číst tištěné Braillovo písmo.  Není divu, když má doma braillský řádek.  V tištěném Braillově písmu je řada velkých písmen označena speciálním znakem \braillebox{23} tak, že \uv{FCHAD} je psáno jako \braillebox{23}\braillebox{124}\braillebox{14}\braillebox{125}\braillebox{1}\braillebox{145}. Žákyně 3 znak označující řadu velkých písmen nevnímala.  Začala ji číst po jednotlivých písmenech a pak řekla \uv{To budou asi čísla} a četla je jako čísla.  Když se dostala na konec stránky, vypadala překvapeně.

Kdy začala číst na FCHADu, katoda nebyla v dobrém kontaktu s rukou.  Musel jsem to upravit.  Četla správně, ale pomalu. Svírala tyčky prsty, aby cítila každou zvlášť.  To je sice dobrý způsob, jak bezpečně rozpoznat písmeno, ale velmi zdlouhavý. Snažil jsem se ji naučit používat zobrazovač správně.  Když jsem jí říkal, že by měla položit ruku rovně, a nastavil ji tak, namítla: \uv{Já nevím, jak ta ruka jen tak leží, mně to nevyhovuje, já si musím na ty body zvyknout a osahat si je trochu jinak.} %10:02.6

Ruku na selektoru měla příliš daleko od sebe. Místo toho, aby se dotýkala napínáčků, pokládala prsty na dráty, které z nich vedou.  Okraje napínáčků necítila správně a proto velmi těžko odhadovala vzdálenost potřebnou k posunu mezi písmeny.  Nepochopila hned, že anody jsou dotykové a že silnější zmáčknutí na ně nemá vliv.  To je ale pochopitelné, protože často nefungovaly. Snad by stroji porozuměla lépe, kdybych mluvil lépe česky.  Opakovaně jsem dělal chyby, například jsem pletl slova \uv{písmo} a \uv{písmeno}.

Občas udělala tu chybu, že při čtení písmene ze zobrazovače nahmatala jenom první sloupec bodů. Například když bylo zobrazené \braillebox{24} myslela, že se zobrazuje jenom \braillebox{2}.

Protože anody jsou dotykové, stačí, aby se okraj prstů dotýkal napínáčku, a stroj už registruje dotyk.  Pokud se člověk současně dotýká více napínáčků, zobrazuje se písmeno toho nejvíce vlevo.  Žákyni dělalo velké potíže to, že si myslela, že už posunula ruku na selektoru dostatečně a že by se mělo zobrazit další písmeno, její prst však stále zůstával v kontaktu s předcházejícím napínáčkem a zobrazené písmeno se tedy nezměnilo.

\paragraph{ První řádek:}

\begin{tabular}{|c|c|c|c|c|c|c|c|c|c|c|c|}
\hline
B&r&a&i&l&l&s&&&&&\\
\braillebox{1278}&\braillebox{1235}&\braillebox{1}&\braillebox{24}&\braillebox{123}&\braillebox{123}&\braillebox{234}&\braillebox{}&\braillebox{2358}&\braillebox{123}&\braillebox{}&\braillebox{}\\
\hline
\end{tabular}

První řádek textu, který žáci na FCHADu četli, byl kratší než selektor.  To byla určitě chyba na mojí straně.  Orca na konci řádku zobrazuje znaky \braillebox{2358}\braillebox{123}.  To znamená, že žáci dočetli \uv{Braills} a pak najednou byli konfrontováni se dvěma znaky, které nepoznali.  Řekl jsem jim, ať si jich nevšímají a pokračují na dalším řádku.

FCHAD, který byl použitý ve studii, nemá žádná navigační tlačítka.  Aby žáci mohli pokračovat na dalším řádku, stačilo přesunout ruku na začátek selektoru a pokračovat. Změnu řádku jsem ovládal sám, často bez jejich vědomí.

\paragraph{Druhý řádek:}

\begin{tabular}{|c|c|c|c|c|c|c|c|c|c|c|c|}
\hline
k&ý& &ř&á&d&e&k&,& &k&t\\
\braillebox{1378}&\braillebox{12346}&\braillebox{}&\braillebox{2456}&\braillebox{16}&\braillebox{145}&\braillebox{15}&\braillebox{13}&\braillebox{2}&\braillebox{}&\braillebox{13}&\braillebox{2345}\\
\hline
\end{tabular}


Další nezvyklost Orcy spočívá v tom, že podtrhává začátky řádků. Například `k' je na začátku řádku psáno jako \braillebox{1378} místo \braillebox{13}.  Do jisté míry to žáky mátlo.

Dalším problémem pro žákyni 3 bylo čtení `ý' \braillebox{12346}. Nesevřela bod 6 a proto písmeno četla jako `p' \braillebox{1234}.  Když jsem jí říkal, že čte špatně a že jí chybí ještě jeden bod, našla ho, ale zase nesprávně odhadovala, že se jedná o písmeno `q' \braillebox{12345}, rychle se však sama opravila.

Když dorazila k mezeře, \braillebox{} říkala \uv{Tady nemám nic}. Nebylo mi jasné, zda to pochopila jako mezeru, a tak jsem jí to vysvětlil.

Věděla, že jsem Američan.  Když narazila na \braillebox{1235}, myslela, že je to `w' - jako podle anglické Braillovy abecedy - a ne `ř' jako podle české.  Snad si neuvědomila, že text zobrazený na FCHADu je český.

Když dorazila k `á'\braillebox{16}, nahlas přemýšlela: \uv{To je první a šestý bod, to je a s čárkou}.  To znamená, že nebyla schopna pochopit význam přímo z tvarů znaků, ale musela o nich explicitně přemýšlet.  Také to znamená, že přemýšlení o číslech bodů jí pomohlo rozpoznat znak, v explicitní paměti tedy měla k dispozici informace o tom, které body má `á'.

Tímto způsobem pokračovala - hlásila nejprve čísla bodů a teprve potom, o které písmeno se jedná.

Když se dostala k čárce (`,')\braillebox{2}, správně poznala, který bod je nahoře, ale nedovedla říct, který znak to je, i když čárka je v Braillově písmu úplně obyčejný a standardní znak.  Když jsem jí vysvětlil, že to je čárka, ihned pochopila.

\paragraph{Třetí řádek}

\begin{tabular}{|c|c|c|c|c|c|c|c|c|c|c|c|}
\hline
e&r&ý& &t&e&ď& &p&o&u&ž\\
\braillebox{1578}&\braillebox{1235}&\braillebox{12346}&\braillebox{}&\braillebox{2345}&\braillebox{15}&\braillebox{1456}&\braillebox{}&\braillebox{1234}&\braillebox{135}&\braillebox{136}&\braillebox{2346}\\
\hline
\end{tabular}

Když se po druhém řádku chtěla vrátit na začátek selektoru, vrátila se jen asi tak do poloviny.  Byl jsem tím překvapen, protože selektor má jasně citelný okraj a předpokládal jsem, že žákyně se bude řídit podle citu a ne odhadem.  Je možné, že se nechtěla dotýkat senzorů, když zrovna nečetla. Jinak nevidím důvod, proč by po selektoru nepřejela prstem na začátek.

Na třetím řádku četla `e' tak, že pojmenovala nahlas čísla bodů, ale když se dostala k `r', stroj se začal střídavě vypínat a zapínat, protože její prst neměl dobrý kontakt se senzorem.  Ona však rozpoznala `r' i bez toho, aby hledala body.  Z toho usuzuji, že vibrace stroje jí spíše pomohla než zmátla.

Je to zajímavé z pedagogického hlediska, protože já sám jsem se lekl, že stroj nefunguje, a snažil se jí vysvětlit, že prst není správně položen na senzoru, a to způsobuje vibraci.  Měl jsem ji nechat číst, protože četla správně.

Když podruhé narazila na `ý', udělala stejnou chybu jako poprvé.

Třetí `e' (ve slově \uv{teď}) už četla bez toho, že by nahlas říkala čísla bodů.  Znamená to, že při třetím opakování už vidíme autonomizaci konkrétní schopnosti?

Když četla `ď'\braillebox{1456}, zmínila, že ho používá jako lomítko (normálně \braillebox{12456}).  Uznala, že ve standardním Braillově písmu to je `ď'.

\paragraph{Čtvrtý řádek}

\begin{tabular}{|c|c|c|c|c|c|c|c|c|c|c|c|}
\hline
í&v&á&t&e&,& &n&e&n&í& \\
\braillebox{3478}&\braillebox{1236}&\braillebox{16}&\braillebox{2345}&\braillebox{15}&\braillebox{2}&\braillebox{}&\braillebox{1345}&\braillebox{15}&\braillebox{1345}&\braillebox{34}&\braillebox{}\\
\hline
\end{tabular}

Když se vracela na začátek selektoru ke čtvrtému řádku, prohlásila, že se v tom stále neorientuje.  Zase se nevrátila dostatečně, ale tentokrát jsem ji neopravil.

I když předtím četla `e' bez problému, zase zapomněla sevřít body na druhém řádku a `e'\braillebox{15} četla jako `a'\braillebox{1}.

Když pokračovala, ruku na selektoru posunula příliš daleko (až na `n'). Dal jsem jí za úkol vrátit se a hledat `t'.  Písmena, která už četla dříve, nyní četla bez počítání bodů a rychle.

Když se znovu dostala k prvnímu `n'\braillebox{1345} na řádku, nejdříve ho odhadla jako `d'\braillebox{145}. `e' četla správně, ale když se dostala na následující `n', četla ho jako `m'\braillebox{134}.

U předposledního písmena řádku, `í', si myslela, že už je poslední.  To proto, že střed jejího prstu byl na posledním napínáčku, ale okrajem prstu se ještě dotýkala předposledního.

Je už obecně známo, že lidé se ve skutečnosti dotýkají jinde, než kde si myslí, že se dotýkají. Dotykové obrazovky používají speciální algoritmy, aby odhadly, kde se uživatel chtěl dotýkat. Bez takové "opravy" by dotyková technologie byla pro lidi nepřirozená. Psychologie dotyku je aktuální téma bádaní.

Stávající algoritmy byly vyvinuty na základě pozorování zrakových vjemů. Člověk se obrazovky vždy dotýký po větší ploše, ale počítač reaguje, jako by se jednalo o jediný bod.  Počítač musí odhadnout, kterého bodu se uživatel chtěl dotknout \citep{holz2011understanding}. Rozlišení selektoru FCHADu je oproti rozlišení dotykové obrazovky velmi malé.  FCHAD neví, jestli je prst v plném kontaktu se senzorem, nebo se dotýká jen okraje.  A co je snad ještě důležitější, psychologie dotyku se u nevidomých samozřejmě zakládá na hmatu, ne na zraku. Stávající algoritmy tedy není možné na FCHAD aplikovat.

\paragraph{Pátý řádek}

\begin{tabular}{|c|c|c|c|c|c|c|c|c|c|c|c|}
\hline
p&r&v&n&í& &ř&á&d&e&k&,\\
\braillebox{123478}&\braillebox{1235}&\braillebox{1236}&\braillebox{1345}&\braillebox{34}&\braillebox{}&\braillebox{2456}&\braillebox{16}&\braillebox{145}&\braillebox{15}&\braillebox{13}&\braillebox{2}\\
\hline
\end{tabular}

Na pátém řádku si žákyně myslela, že `v'\braillebox{1236} je `r'\braillebox{1235}, protože si zaměnila šestý bod za pátý.

Stále neuměla posouvat ruku na selektoru. Posunula ji při čtení příliš daleko doprava, takže přeskočila několik písmem naráz.

Během čtení pátého řádku přestal stroj fungovat a musel jsem restartovat ovládací software.  Žákyně byla extrémně trpělivá, za což jsem jí velice vděčný.

Když jsem stroj opravil, žákyně začala číst pátý řádek zase od začátku. Opakovanou část řádku přečetla rychle a v rychlém čtení pokračovala, až dorazila ke slovu \uv{řádek}.  Začátek slova přečetla správně, ale `e'\braillebox{15} četla jako `á'\braillebox{16}.

Po pátém řádku jsme setkání ukončili. Žákyně se ještě chtěla podělit o své dojmy.  O FCHADu řekla:

%330
\em \uv{Jistě víte sám, že běžný řádek je určitě komfortnější, ale určitě je lepší mít tenhle řádek, než nemít žádný. Já si myslím, že na to zobrazování písmen bych si zvykla. Asi by bylo dobré, kdyby si každý uživatel mohl vybrat velikost. Například vy to tady máte rozděleno tou látkou, ale třeba já osobně bych chtěla mít ty dva sloupce bodů naopak blíž. Protože já mám tu ruku menší a užší. Za svou osobu vidím hlavní problém v tom posunu, protože tam ty aktivní body se mi zatím velice špatně hledají. Za svoji osobu bych jako velikou překážku užívání viděla především hledání těch písmen.} \em



\subsubsection{Žák 4}

Setkání 8.3.2013

Čtvrtý žák je 20-25 let starý.  Jeho nejvyšší úroveň dosažené vzdělávání je maturita.  Je můj spolužák tady na katedře angličtiny. Je úplně nevidomé a taky trpí problémy se sluchem.  Pracuje s Braillovým písmem od základní školy. Používá braillský řádek doma na počítače.

Měl velké potíže s přesunem na selektor.

Když jsem ho dával číst úvodní papír on se mně zeptal jestli to má číst nahlas.  Říkal jsem ne, protože první část papíru se zkutečně nahlas číst nemusí.  To byla chyba.  Čekal jsem, že až dorazí na větu \uv{Následující text čtěte prosím nahlas a bez přestání až do konce.} začíná číst nahlas, ale četl dal po tichu.  Z toho důvodu mám jeho rychlost čtení tiskové Braillovo písmo jenom z druhý straně papíru.  Je to obecně zajímavý, že moje účastnicí nereagovali na psáné příkazy.  Žák 4 dal souhlas bez ústního zeptání ale některé z nich nepochopili z věty \uv{Než budeme pokračovat, potřebuji Váš souhlas se shromážděním a zveřejněním těchto dat.}, že mají nějak mě aktivně říct že s tím souhlasili.

Četl tiskový text obě ruce.

Čas na čtení celého textu:43 vteřin %169.2-126.1

CPS: 9.46*

Čas na čtení druhé strany tiskového textu:25.9 vteřin %169.2-143.3

CPS: 10.23**

*) přečetl to nahlas ale potom co to četl potichu

**) Přečetl to jenom nahlas.

Potom co jsem se zeptal otázky, přemístil jsem FCHAD před účastníkem a říkal jsem mu ať začíná prozkoumat stroj.  On seděl a nic nedělal.  Třeba jsem mu měl nechat díl. Kdy jsem tam seděl s něj měl jsem pocit, že sedí strašně dlouho ale na zvukové nahrávce jsem čekal jenom deset vteřin.

Protože žák 3 měla tolik problému s selektorem rozhodl jsem se vysvětlit čtvrtému žákovi selektor líp.  Dal jsem jeho ruku na selektor a táhl jsem jeho prsty nad senzory. Jak jeho prsty dotekly se senzory zobrazovač vydal zvuk cvaknutí jak se změnilo písmeno.  Říkal jsem ho, že ten zvuk je zvuk změna písmen.  Jenom až potom co jsem mu vysvětlil selektor jsem dal jeho levou ruku na zobrazovače.

Věnoval jsem větší čas vysvětlení fungování nastroje než u žáka 3. Kdy jsem ho ukazoval zobrazovače zobrazilo se písmeno `o' a jsem ho říkal, které písmeno to je.

Ještě během ukázku přestala stroj reagovat.  Naše setkání bylo usekaná mnoho krát technickými potíží.  Protože měl chlupatou kůži, katoda neměla dobrá kontakt.  Proto, body strašně zatřásly.

\paragraph{První řádek}

\begin{tabular}{|c|c|c|c|c|c|c|c|c|c|c|c|}
\hline
B&r&a&i&l&l&s&&&&&\\
\braillebox{1278}&\braillebox{1235}&\braillebox{1}&\braillebox{24}&\braillebox{123}&\braillebox{123}&\braillebox{234}&\braillebox{}&\braillebox{2358}&\braillebox{123}&\braillebox{}&\braillebox{}\\
\hline
\end{tabular}

Konečně jsme se začali číst.  Žák 4 nepoužíval způsob pojmenování bodů kdy četl.  Odhadl nejdřív, že první písmeno je `l'\braillebox{123} a kdy jsem ho říkal že to není opravil sebe, že je to `b'.

Kdy pokračoval v čtení stále katoda neměla dobrý spoj.  Protože jsem věděl, že budou problémy s katodou jsem ještě měl další katoda(ve formě měděnou samolepicí paskou) nalepený na zobrazovače. Snažil jsem se mu vysvětlit, že bude fungovat lip kdy je v kontaktu s druhou katodou.

Přečetl `r' správně a zase začal zbláznit stroj.  Vysvětlil jsem mu ten problém s kontaktem s katodou a poprosil jsem mu aby držel katodu v ruce.  Pak četl další dvě písmena správně a zase nefungoval stroj.

Po par pokusu resetnutí stroje konečně už to fungoval až ke konci setkání.  Četl od začátku prvního řádku a četl první tři písmena správně. Přeskočil jedno písmeno na selektor a jsem mu říkal ať se vrátí zpět.  Četl `i'\braillebox{24} jako `c'\braillebox{14}. Když jsem ho říkal, že četl nesprávně používal stejný způsob uvažování, které používala žák 3 celou dobu, říkal na hlas, které body jsou nahoru \uv{to je bod jedna čtyři} říkal. \uv{Není to bod jedna} jsem mu odpověděl. Potom už se upravil správně sám.

Na rozdíl od žáka 3 pochopil hned, že čte české slova.  Kdy viděl, že další písmeno `l' říkal \uv{Brail}. Četl ještě `l' a `s' správně a doplnil \uv{Braills}.

Jsem ho říkal ať se vrátí na začátek textu radši než, že bych mu vysvětlil jaké znaky používá Orca označit konec řádku, v retrospekce myslím si že to byla chyba.

\paragraph{Druhý řádek}

\begin{tabular}{|c|c|c|c|c|c|c|c|c|c|c|c|}
\hline
k&ý& &ř&á&d&e&k&,& &k&t\\
\braillebox{1378}&\braillebox{12346}&\braillebox{}&\braillebox{2456}&\braillebox{16}&\braillebox{145}&\braillebox{15}&\braillebox{13}&\braillebox{2}&\braillebox{}&\braillebox{13}&\braillebox{2345}\\
\hline
\end{tabular}

Kdy začal číst druhý řádek dostal na `s' a `ý' a doplnil, že dočetl \uv{Braillský} pak \uv{četl} dal. Domyslel, že se zobrazí písmeno je `ř'.  Ještě ale nepřesunul ruku na selektor.  Říkal jsem mu, že stále ze zobrazuje `ý'.  Přesunul ruku na selektor až k `ř' a říkal, \uv{aha, tak tohle je opravdu `ř'}.  Přečetl slovo \uv{řádek} skoro plynule, kromě tomu, že měl problém s přesunem na selektor.

\paragraph{Třetí řádek}

\begin{tabular}{|c|c|c|c|c|c|c|c|c|c|c|c|}
\hline
e&r&ý& &t&e&ď& &p&o&u&ž\\
\braillebox{1578}&\braillebox{1235}&\braillebox{12346}&\braillebox{}&\braillebox{2345}&\braillebox{15}&\braillebox{1456}&\braillebox{}&\braillebox{1234}&\braillebox{135}&\braillebox{136}&\braillebox{2346}\\
\hline
\end{tabular}

Vrátil se k začátku selektoru bez pomoci, ale první písmeno `e' četl jako `á'\braillebox{16}, pojmenoval čísla bodů jako jedna a šest a jsem ho říkal, že šesti bod není nahoru.  Hmatal prstem na díry na zobrazovače a počítal, který bod zkutečně nahoru je.  Pak se upravil sám.

Kdy četl další slovo \uv{teď}, četl `t' správně ale zase původně četl `e' jako `á'.  Měl taky problém s písmenem `ď'\braillebox{1456}, který původně četl jako `č'\braillebox{146}.  Neříkal nahlas ani \uv{který} ani \uv{teď} po jejich čtení.  Četl `p' správně, měl trochu problém s posunem na selektor ale sám našel `o'.  Myslel si, že je u konce řádku.  Říkal jsem mu, že není. Přečetl `u' a pak už opravdu věřil, že je u konce řádku. Říkal jsem mu, že mu zbývá ještě jedno písmeno.  Snažil se pokračovat. Písmeno na zobrazovače začal vibrovat chaotickým způsobem. Říkal jsem mu, že musím zkusit jestli stroj teď funguje správně. Zkoušel jsem stroj a fungoval. Říkal jsem mu ať pokračuje. Četl `ž' jako `t'. Říkal jsem, že to není správní. Opravil se sám.

Dokončili jsme u toho setkání.

Kdy jsme dokončili setkání ještě jsme mluví o FCHADech.  Nabídl další spolupráce a říkal, že \uv{kdyby to bylo opravdu levná alternativa, tak by to bylo opravdu super}. Říkal mně, že \uv{by to bylo lepší kdyby ty body byly blíž k sobě}.




\subsection{Hlavní studie}

Po předběžné studii jsem se dva víkendy intenzivně věnoval vylepšení nastroje a odstranění softwarové chyby.  Předělal jsem celý selektor a přenastavil zobrazovač na nejužší nastavení.

Hlavní studie proběhla 20.3.2013 a 23.4.2013 v odpoledních hodinách na Gymnáziu pro zrakově postižené a střední odborné škole pro zrakově postižené v Praze.

Žáci pracovali v učebně zeměpisu, která pro ně představovala známé prostředí.  Stůl, u kterého seděli, je vidět na Obrázku 4.

{\bf Poznámka:}  V případě prvních tří účastníků hlavní studie (žáci 5, 6 a 7) nebyl správně nastavený mikrofon a audionahrávky ze setkání jsou v nízké kvalitě.  Proto může popis setkání obsahovat chyby.

\subsubsection{Žák 5}
Setkání 12:00 20.3.2013

Pátý žák je ve věku mezi 20 a 25 lety.  Jeho nejvyšší dosažené vzdělání je maturita. Je úplně nevidomý.  Také čte Braillovo písmo od základní školy.  Používá braillský řádek.

Než začal číst tištěný text, zeptal se, jestli má číst z obou stran.  Během čtení  otočil stranu velmi rychle, takže celou dobu četl plynule bez jakéhokoliv přestání.  Jedno slovo přečetl nesprávně: \uv{nejlevnější} četl jako \uv{nejlepší}.

Četl tištěný text oběma rukama.

% začatek 176.8 otočení 192.9 konec 212.6

Čas čtení tištěného textu:35.8 vteřin

CPS: 11.37

Čas čtení tištěného textu po otočení listu:19.7 vteřin

CPS: 13.45

Nejdříve jsem mu ukázal selektor.  Popsal jsem ho jako \uv{dvanáct kovových bodů, které vypadají jako písmo[sic] `k'.}.  Pak jsem mu ukázal zobrazovač.  Řekl jsem, že \uv{to má osm dír[sic] a že z každé z nich skáče kovový tyč.}

\paragraph{První Řádek}
\begin{tabular}{|c|c|c|c|c|c|c|c|c|c|c|c|}
\hline
B&r&a&i&l&l&s&&&&&\\
\braillebox{1278}&\braillebox{1235}&\braillebox{1}&\braillebox{24}&\braillebox{123}&\braillebox{123}&\braillebox{234}&\braillebox{}&\braillebox{2358}&\braillebox{123}&\braillebox{}&\braillebox{}\\
\hline
\end{tabular}

Přečetl první písmeno správně, ale potom začaly tyčky v zobrazovači náhodně vyskakovat. Usoudil jsem, že senzory jsou přecitlivělé, a snížil jsem práh citlivosti tak, aby fungovaly. Protože jsem to dělal poprvé, trvalo mi to asi 10 minut, během kterých žák seděl a nic nedělal.

Vedl jsem jeho ruku nad prvními třemi senzory (abych ověřil, že fungují) a řekl jsem na hlas `b' `r' `a', takže druhá dvě písmena žák sám nečetl.  Potřeboval jsem namočit jeho prst, aby senzory fungovaly. Tak jsem se ho zeptal, kde najdu vodu, a on mi řekl o umyvadle.  Namočil jsem jeho ukazovák na pravé ruce a začali jsme číst.

Začátek řádku četl správně. Jedinou jeho chybou bylo, že četl `s' jako `i'. Opravil se poté, co jsem ho upozornil, že to není správně.

\paragraph{Druhý Řádek}
\begin{tabular}{|c|c|c|c|c|c|c|c|c|c|c|c|}
\hline
k&ý& &ř&á&d&e&k&,& &k&t\\
\braillebox{1378}&\braillebox{12346}&\braillebox{}&\braillebox{2456}&\braillebox{16}&\braillebox{145}&\braillebox{15}&\braillebox{13}&\braillebox{2}&\braillebox{}&\braillebox{13}&\braillebox{2345}\\
\hline
\end{tabular}

Četl správně až na `,'.  Choval se, jako že nerozumí.  Zeptal jsem se, který bod je nahoře.  On správně řekl, že druhý.  Vysvětlil jsem mu tedy, že druhý bod je čárka.  Když dorazil ke `k', myslel, že už je u konce řádku.

\paragraph{Třetí Řádek}
\begin{tabular}{|c|c|c|c|c|c|c|c|c|c|c|c|}
\hline
e&r&ý& &t&e&ď& &p&o&u&ž\\
\braillebox{1578}&\braillebox{1235}&\braillebox{12346}&\braillebox{}&\braillebox{2345}&\braillebox{15}&\braillebox{1456}&\braillebox{}&\braillebox{1234}&\braillebox{135}&\braillebox{136}&\braillebox{2346}\\
\hline
\end{tabular}

Četl správně až na druhé `e'.  Kvůli nekvalitní nahrávce nevím, co odhadl.  Zřejmě ho zmátlo, že body nespadly samy. Zeptal se na to a já odpověděl \uv{ano, nespadnou, když na ně netlačíte}. Zřejmě už sledoval slova, protože poté, co přečetl \uv{teď} po jednotlivých písmenech, řekl slovo \uv{teď} jako celek.

\paragraph{Čtvrtý Řádek}
\begin{tabular}{|c|c|c|c|c|c|c|c|c|c|c|c|}
\hline
í&v&á&t&e&,& &n&e&n&í& \\
\braillebox{3478}&\braillebox{1236}&\braillebox{16}&\braillebox{2345}&\braillebox{15}&\braillebox{2}&\braillebox{}&\braillebox{1345}&\braillebox{15}&\braillebox{1345}&\braillebox{34}&\braillebox{}\\
\hline
\end{tabular}

`t' četl nejdříve jako `j'.  První `n' četl nejdříve jako `d'.  Přeskočil druhé `e' a byl na dalším `n'.  Řekl jsem mu, ať se vrátí zpátky, a on to udělal a přečetl zbytek řádku správně.

\paragraph{Pátý Řádek}
\begin{tabular}{|c|c|c|c|c|c|c|c|c|c|c|c|}
\hline
p&r&v&n&í& &ř&á&d&e&k&,\\
\braillebox{123478}&\braillebox{1235}&\braillebox{1236}&\braillebox{1345}&\braillebox{34}&\braillebox{}&\braillebox{1235}&\braillebox{16}&\braillebox{145}&\braillebox{15}&\braillebox{13}&\braillebox{2}\\
\hline
\end{tabular}

Četl všechno správně.

\paragraph{Šestý Řádek}
\begin{tabular}{|c|c|c|c|c|c|c|c|c|c|c|c|}
\hline
 &k&t&e&r&ý& &z&o&b&r&a\\
\braillebox{78}&\braillebox{13}&\braillebox{2345}&\braillebox{15}&\braillebox{1235}&\braillebox{12346}&\braillebox{}&\braillebox{1356}&\braillebox{135}&\braillebox{12}&\braillebox{1235}&\braillebox{1}\\
\hline
\end{tabular}

U začátku řádku byl nějak překvapený.  Vysvětlil jsem mu, že body 7 a 8 \uv{znamenají} začátek řádku.  Jinak četl správně.

\paragraph{Řádek 7}
\begin{tabular}{|c|c|c|c|c|c|c|c|c|c|c|c|}
\hline
z&u&j&e& &j&e&n&o&m& &j\\
\braillebox{135678}&\braillebox{136}&\braillebox{245}&\braillebox{15}&\braillebox{}&\braillebox{245}&\braillebox{15}&\braillebox{1345}&\braillebox{135}&\braillebox{134}&\braillebox{}&\braillebox{245}\\
\hline
\end{tabular}

Četl všechno správně.

\paragraph{Řádek 8}
\begin{tabular}{|c|c|c|c|c|c|c|c|c|c|c|c|}
\hline
e&d&n&o& &p&í&s&m&e&n&o\\
\braillebox{1578}&\braillebox{145}&\braillebox{1345}&\braillebox{135}&\braillebox{}&\braillebox{1234}&\braillebox{34}&\braillebox{234}&\braillebox{134}&\braillebox{15}&\braillebox{1345}&\braillebox{135}\\
\hline
\end{tabular}

Četl všechno správně a já přestal potvrzovat, že čte správně. Zdálo se mi, že poté zrychlil.

\paragraph{Řádek 9}
\begin{tabular}{|c|c|c|c|c|c|c|c|c|c|c|c|}
\hline
.& & &P&r&v&n&í& &b&y&l\\
\braillebox{378}&\braillebox{}&\braillebox{}&\braillebox{12347}&\braillebox{1235}&\braillebox{1236}&\braillebox{1345}&\braillebox{34}&\braillebox{}&\braillebox{12}&\braillebox{13456}&\braillebox{123}\\
\hline
\end{tabular}

Četl všechno správně. Zeptal se, jestli má říct, že `P' je velké.

\paragraph{Řádek 10}
\begin{tabular}{|c|c|c|c|c|c|c|c|c|c|c|c|}
\hline
 &v&y&n&a&l&e&z&e&n& &v\\
\braillebox{78}&\braillebox{1236}&\braillebox{13456}&\braillebox{1345}&\braillebox{1}&\braillebox{123}&\braillebox{15}&\braillebox{1356}&\braillebox{15}&\braillebox{1345}&\braillebox{}&\braillebox{1236}\\
\hline
\end{tabular}

Přečetl všechno správně.

\paragraph{Řádek 11}
\begin{tabular}{|c|c|c|c|c|c|c|c|c|c|c|c|}
\hline
 &r&o&c&e& &1&9&1&3& &v\\
\braillebox{78}&\braillebox{1235}&\braillebox{135}&\braillebox{14}&\braillebox{15}&\braillebox{}&\braillebox{18}&\braillebox{248}&\braillebox{18}&\braillebox{148}&\braillebox{}&\braillebox{1236}\\
\hline
\end{tabular}

Když dorazil k datu, řekl: \uv{Předpokládám, že to asi bude číslo    jedna devět   to znamená devatenáct set třináct}. Četl všechno správně.

\paragraph{Řádek 12}
\begin{tabular}{|c|c|c|c|c|c|c|c|c|c|c|c|}
\hline
 &A&n&g&l&i&i&.& & &J&m\\
\braillebox{78}&\braillebox{17}&\braillebox{1345}&\braillebox{1245}&\braillebox{123}&\braillebox{24}&\braillebox{24}&\braillebox{3}&\braillebox{}&\braillebox{}&\braillebox{2457}&\braillebox{134}\\
\hline
\end{tabular}

`n' četl nejdřív jako `d' a tečku `.' nejdřív jako mezeru.  Jinak četl správně.

\paragraph{Řádek 13}
\begin{tabular}{|c|c|c|c|c|c|c|c|c|c|c|c|}
\hline
e&n&o&v&a&l& &s&e& &O&p\\
\braillebox{1578}&\braillebox{1345}&\braillebox{135}&\braillebox{1236}&\braillebox{1}&\braillebox{123}&\braillebox{}&\braillebox{234}&\braillebox{15}&\braillebox{}&\braillebox{1357}&\braillebox{1234}\\
\hline
\end{tabular}

Četl všechno správně kromě velkého `O', které původně četl jako `n'.

\paragraph{Řádek 14}
\begin{tabular}{|c|c|c|c|c|c|c|c|c|c|c|c|}
\hline
t&o&f&o&n& &a& &p&ř&e&v\\
\braillebox{234578}&\braillebox{135}&\braillebox{124}&\braillebox{135}&\braillebox{1345}&\braillebox{}&\braillebox{1}&\braillebox{}&\braillebox{1234}&\braillebox{2456}&\braillebox{15}&\braillebox{1236}\\
\hline
\end{tabular}

Četl všechno správně.

\paragraph{Řádek 15}
\begin{tabular}{|c|c|c|c|c|c|c|c|c|c|c|c|}
\hline
á&d&ě&l& &s&v&ě&t&l&o& \\
\braillebox{1678}&\braillebox{145}&\braillebox{126}&\braillebox{123}&\braillebox{}&\braillebox{234}&\braillebox{1236}&\braillebox{126}&\braillebox{2345}&\braillebox{123}&\braillebox{135}&\braillebox{}\\
\hline
\end{tabular}

Četl všechno správně.

\paragraph{Řádek 16}
\begin{tabular}{|c|c|c|c|c|c|c|c|c|c|c|c|}
\hline
n&a& &z&v&u&k&.& & &K&d\\
\braillebox{134578}&\braillebox{1}&\braillebox{}&\braillebox{1356}&\braillebox{1236}&\braillebox{136}&\braillebox{13}&\braillebox{3}&\braillebox{}&\braillebox{}&\braillebox{137}&\braillebox{145}\\
\hline
\end{tabular}

`z' četl nejdřív jako `e'. Jinak četl správně.

\paragraph{Řádek 17}
\begin{tabular}{|c|c|c|c|c|c|c|c|c|c|c|c|}
\hline
y&ž& &č&t&e&n&á&ř& &p&o\\
\braillebox{1345678}&\braillebox{2346}&\braillebox{}&\braillebox{146}&\braillebox{2345}&\braillebox{15}&\braillebox{1345}&\braillebox{16}&\braillebox{2456}&\braillebox{}&\braillebox{1234}&\braillebox{135}\\
\hline
\end{tabular}

Četl všechno správně.

Ukončil jsem setkání a on řekl:
\uv{To se hrozně špatně poznává, když člověk je zvyklý číst Braillovo písmo ... Tím, že ten taktilní bod má určité zvyklosti, pro mě je to těžší poznat}

Tim: \uv{Vy jste moc nepoložil tu ruku rovně na ten článek, musím říct, že já čtu docela rychlejší, než jste četl vy teď na ten stroj, i když určitě čtu rukama pomalejší než vy}

Žák : \uv{Já si myslím, že někomu kdo už má zkušenosti s normálním Braillovým písmem od začátku je to trošku problém}

Tim: \uv{Poznal jste, že jsou tam slova?}

Žák: \uv{Ano}
Tim: \uv{Rozuměl jste?}
Žák: \uv{Ano}

Pozoroval jsem, že pro něj bylo těžké zvyknout si na malý selektor.  Byl zvyklý používat dlouhý braillský řádek. Když se vrátil na začátek selektoru poté, co dočetl řádek textu, často pohyboval svou rukou příliš vlevo.

\subsubsection{Žák 6}
Setkání 12:45 20.3.2013

Žák 6 je mezi 15-20 let stará. Čte Braillovo písmo od začátku školní docházky. Používá braillský řádek. Její nejvyšší dosažený úroveň vzdělávání je základní škola.  Je úplně nevidomá.

Četla tiskový text obě ruce.

% start 133.5 otočení 159.5 konec 193.9

Čas na čtení celého textu:60.4 vteřin

CPS: 6.7

Čas na čtení druhé strany tiskového textu: 34.4 vteřin

CPS: 7.7

Po otázek jsem namočil její pravou ukazovátku.  Vysvětlil jsem ji, že má dat ruku na první písmeno a odhadnout, které písmeno se zobrazuje.

\paragraph{První Řádek}
\begin{tabular}{|c|c|c|c|c|c|c|c|c|c|c|c|}
\hline
B&r&a&i&l&l&s&&&&&\\
\braillebox{1278}&\braillebox{1235}&\braillebox{1}&\braillebox{24}&\braillebox{123}&\braillebox{123}&\braillebox{234}&\braillebox{}&\braillebox{2358}&\braillebox{123}&\braillebox{}&\braillebox{}\\
\hline
\end{tabular}

Myslel nejdřív, že body jedna a tři jsou nahoře tak jsem se rozhodl vysvětlit lip zobrazovač.  Vysvětlil jsem, že jsou osm děr a jsem je ukazoval tím, že jsem její prst dával na ty díry.

Mumlala pro sebe \uv{to je} a sevřela na zobrazovače. Četla správně `l'.  Vysvětlil jsem ji, že pravou ukazovátka by měla byt na první senzor a dal jsem její prst tam.

Odhadla `v'.  Řekl jsem ji, které body byly nahoře.  Vysvětlil jsem ji, že to je osmibodové Braillovo písmo a, že nemusí dávat pozor na body sedm a osm.  Správně ztotožnila `b'.  Řekl jsem ji, ať posouvá ruku na selektor. Zeptala se \uv{do práva?}. Odpověděl jsem kladně a četla další písmeno správně.  Četla správně až na `i' což četla hodně pomalu ale správně.  Četla `l' správně podruhy. Není poznat na nahrávce její první pokus. Zbytek řádku četla správně.  Řekl jsem ji ať jde na další řádek potom co četla `s'.

\paragraph{Druhý Řádek}
\begin{tabular}{|c|c|c|c|c|c|c|c|c|c|c|c|}
\hline
k&ý& &ř&á&d&e&k&,& &k&t\\
\braillebox{1378}&\braillebox{12346}&\braillebox{}&\braillebox{2456}&\braillebox{16}&\braillebox{145}&\braillebox{15}&\braillebox{13}&\braillebox{2}&\braillebox{}&\braillebox{13}&\braillebox{2345}\\
\hline
\end{tabular}

Četla `k' nejdřív jako `a'.  Četla `ý' nejdřív(myslím) jako `p'.  Říkala \uv{nic} kdy dorazila na mezeru.  Říkal jsem ji, že \uv{nic} je mezera a ona odpověděla \uv{oh, aha}. Četla `ř' myslím nejdřív jako `t'.  Nevím co byl její první odhad pro `e' ale vysvětlil jsem ji, že body nespadnou samo a ona pak správně ztotožnila písmeno.  Tvářila se nějak překvapeně kdy se dorazila na `,' tak jsem ji řekl co to je.  Myslela, že je u konce řádku kdy dorazila na `k'. Četla `t' nejdřív jako `j'.

\paragraph{Třetí Řádek}
\begin{tabular}{|c|c|c|c|c|c|c|c|c|c|c|c|}
\hline
e&r&ý& &t&e&ď& &p&o&u&ž\\
\braillebox{1578}&\braillebox{1235}&\braillebox{12346}&\braillebox{}&\braillebox{2345}&\braillebox{15}&\braillebox{1456}&\braillebox{}&\braillebox{1234}&\braillebox{135}&\braillebox{136}&\braillebox{2346}\\
\hline
\end{tabular}

Její ruka byla moc malá na zobrazovače. Ačkoliv mohla pokryt celý zobrazovač, pokryla zobrazovače přesně.  Dala ruku na straně, tak že prsty hmatali na první tří body.  Body 4,5 a 6 hmatala sevřením. Četla všechno správně ale pomalu až na `o', což četla nejdřív jako `r'.  Kdy jsem ji řekl ať pojmenuje čísla bodů dělala to správně ale stále tvrdila, že se zobrazí `r'.  Říkal jsem ji, že je to `o'. Ona odpovídala \uv{jojojo, aha , a smála se}. Zbytek řádku četla správně.

\paragraph{Čtvrtý Řádek}
\begin{tabular}{|c|c|c|c|c|c|c|c|c|c|c|c|}
\hline
í&v&á&t&e&,& &n&e&n&í& \\
\braillebox{3478}&\braillebox{1236}&\braillebox{16}&\braillebox{2345}&\braillebox{15}&\braillebox{2}&\braillebox{}&\braillebox{1345}&\braillebox{15}&\braillebox{2345}&\braillebox{34}&\braillebox{}\\
\hline
\end{tabular}

Měla problém s posunem na selektor tím, že neposouvala dostatečně daleko.  Četla `v' nejdřív jako `l'.  Zbytek řádku četla správně.

\paragraph{Pátý Řádek}
\begin{tabular}{|c|c|c|c|c|c|c|c|c|c|c|c|}
\hline
p&r&v&n&í& &ř&á&d&e&k&,\\
\braillebox{123478}&\braillebox{1235}&\braillebox{1236}&\braillebox{1345}&\braillebox{34}&\braillebox{}&\braillebox{1235}&\braillebox{16}&\braillebox{145}&\braillebox{15}&\braillebox{13}&\braillebox{2}\\
\hline
\end{tabular}
Začatek řádku četla správně.  Četla `k' nejdřív jako `a'.  Zase nepoznala `,' ale věděla, že jenom druhý bod je nahoru.

\paragraph{Šestý Řádek}
\begin{tabular}{|c|c|c|c|c|c|c|c|c|c|c|c|}
\hline
 &k&t&e&r&ý& &z&o&b&r&a\\
\braillebox{78}&\braillebox{13}&\braillebox{2345}&\braillebox{15}&\braillebox{1235}&\braillebox{12346}&\braillebox{}&\braillebox{1356}&\braillebox{135}&\braillebox{12}&\braillebox{1235}&\braillebox{1}\\
\hline
\end{tabular}

Četla všechno správně.

\paragraph{Řádek 7}
\begin{tabular}{|c|c|c|c|c|c|c|c|c|c|c|c|}
\hline
z&u&j&e& &j&e&n&o&m& &j\\
\braillebox{135678}&\braillebox{136}&\braillebox{245}&\braillebox{15}&\braillebox{}&\braillebox{245}&\braillebox{15}&\braillebox{1345}&\braillebox{135}&\braillebox{134}&\braillebox{}&\braillebox{245}\\
\hline
\end{tabular}

Četla `n' nejdřív jako `d'. Jinak četla všechno správně.

\paragraph{Řádek 8}
\begin{tabular}{|c|c|c|c|c|c|c|c|c|c|c|c|}
\hline
e&d&n&o& &p&í&s&m&e&n&o\\
\braillebox{1578}&\braillebox{145}&\braillebox{1345}&\braillebox{135}&\braillebox{}&\braillebox{1234}&\braillebox{24}&\braillebox{234}&\braillebox{134}&\braillebox{15}&\braillebox{1345}&\braillebox{135}\\
\hline
\end{tabular}

Četla všechno správně.

\paragraph{Řádek 9}
\begin{tabular}{|c|c|c|c|c|c|c|c|c|c|c|c|}
\hline
.& & &P&r&v&n&í& &b&y&l\\
\braillebox{378}&\braillebox{}&\braillebox{}&\braillebox{12347}&\braillebox{1235}&\braillebox{1236}&\braillebox{1345}&\braillebox{34}&\braillebox{}&\braillebox{12}&\braillebox{13456}&\braillebox{123}\\
\hline
\end{tabular}

U první písmena koukala na mně s nepochopeným.  Vysvětlil jsme ji, že sedmi a osmi body nahoře naznačuji začátek řádku, a že třetí bod je tečka.  Četla zbytek správně. Nepoznala, že `P' je velký.  Anebo aspoň neřekla.  Četla `b'(správně) jako `,' protože vypadl drát v zobrazovače.  Spravil jsem to a pokračovala bez chyb.

\paragraph{Řádek 10}
\begin{tabular}{|c|c|c|c|c|c|c|c|c|c|c|c|}
\hline
 &v&y&n&a&l&e&z&e&n& &v\\
\braillebox{78}&\braillebox{1236}&\braillebox{13456}&\braillebox{1345}&\braillebox{1}&\braillebox{123}&\braillebox{15}&\braillebox{1356}&\braillebox{15}&\braillebox{1345}&\braillebox{}&\braillebox{1236}\\
\hline
\end{tabular}

Obecně nečetla mezery, jenom je přeskočila.  Je možný, že je nečetla protože bylo by zbytečně je číst.  Je taky možný, že řídila na selektor podle cvaknutí zobrazovače a ne podle citu senzoru.  Četla `y' původně nesprávně, myslím `ď' ale s tou nahrávkou jistý byt nemohu.  Zbytek řádku četla správně.

\paragraph{Řádek 11}
\begin{tabular}{|c|c|c|c|c|c|c|c|c|c|c|c|}
\hline
 &r&o&c&e& &1&9&1&3& &v\\
\braillebox{78}&\braillebox{1235}&\braillebox{135}&\braillebox{14}&\braillebox{15}&\braillebox{}&\braillebox{18}&\braillebox{248}&\braillebox{18}&\braillebox{148}&\braillebox{}&\braillebox{1236}\\
\hline
\end{tabular}

Četla začátek správně. Četla `1' jako `a'.  Vysvětlil jsem ji, že osmi bod naznačuje čísla.  Četla správně. Četla `9' jako `i'.  Zeptal jsem se ji, jestli osmi bod je nahoru.  Pak upravila se na `9'.  Zbytek četla správně.

\paragraph{Řádek 12}
\begin{tabular}{|c|c|c|c|c|c|c|c|c|c|c|c|}
\hline
 &A&n&g&l&i&i&.& & &J&m\\
\braillebox{78}&\braillebox{17}&\braillebox{1345}&\braillebox{1245}&\braillebox{123}&\braillebox{24}&\braillebox{24}&\braillebox{3}&\braillebox{}&\braillebox{}&\braillebox{2457}&\braillebox{134}\\
\hline
\end{tabular}

Četla začátek správně.  Nenaznačila, že `O' je velký. Myslela, že druhý `i' je `9' ale upravila se když jsem ji řekl, že není.  Kdy dorazila na `J' zase na mně koukala s nepochopením.  Viděl jsem, že pravě hmatala na body 2 a 5 ale ne na 4.  Řekl jsem ji \uv{ještě nahoře} a ona to správně odhadla.  Neřekla, že to je velký.

\paragraph{Řádek 13}
\begin{tabular}{|c|c|c|c|c|c|c|c|c|c|c|c|}
\hline
e&n&o&v&a&l& &s&e& &O&p\\
\braillebox{1578}&\braillebox{1345}&\braillebox{135}&\braillebox{1236}&\braillebox{1}&\braillebox{123}&\braillebox{}&\braillebox{234}&\braillebox{15}&\braillebox{}&\braillebox{1357}&\braillebox{1234}\\
\hline
\end{tabular}

Nevím co se přesně stal u začátek řádku, kvůli zavadám nahrávky.  Četla `o' opět jako `r' ale opravila se když jsem ji řekl, že to není správný.  Neřekla, že o je velké.

\paragraph{Řádek 14}
\begin{tabular}{|c|c|c|c|c|c|c|c|c|c|c|c|}
\hline
t&o&f&o&n& &a& &p&ř&e&v\\
\braillebox{234578}&\braillebox{135}&\braillebox{124}&\braillebox{135}&\braillebox{1345}&\braillebox{}&\braillebox{1}&\braillebox{}&\braillebox{1234}&\braillebox{2456}&\braillebox{15}&\braillebox{1236}\\
\hline
\end{tabular}

Četla `t' nejdřív jako `j'.  Četla `p' nejdřív jako `n'? Opravila se, když jsem ji řekl, že to není správný. Jinak četla správně.  Ale myslela, že je konec řádku už u písmena `e'.

\paragraph{Řádek 15}
\begin{tabular}{|c|c|c|c|c|c|c|c|c|c|c|c|}
\hline
á&d&ě&l& &s&v&ě&t&l&o& \\
\braillebox{1678}&\braillebox{145}&\braillebox{126}&\braillebox{123}&\braillebox{}&\braillebox{234}&\braillebox{1236}&\braillebox{126}&\braillebox{2345}&\braillebox{123}&\braillebox{135}&\braillebox{}\\
\hline
\end{tabular}

Myslela si, že `ě' je `v'.  Četla `s' jako `i' ale opravila se stejným dechem.  Taky četla `t' jako `j' a opravila se stejným dechem.

\paragraph{Řádek 16}
\begin{tabular}{|c|c|c|c|c|c|c|c|c|c|c|c|}
\hline
n&a& &z&v&u&k&.& & &K&d\\
\braillebox{134578}&\braillebox{1}&\braillebox{}&\braillebox{1356}&\braillebox{1236}&\braillebox{136}&\braillebox{13}&\braillebox{3}&\braillebox{}&\braillebox{}&\braillebox{137}&\braillebox{145}\\
\hline
\end{tabular}

Četla všechno správně.

\paragraph{Řádek 17}
\begin{tabular}{|c|c|c|c|c|c|c|c|c|c|c|c|}
\hline
y&ž& &č&t&e&n&á&ř& &p&o\\
\braillebox{1345678}&\braillebox{2346}&\braillebox{}&\braillebox{146}&\braillebox{2345}&\braillebox{15}&\braillebox{1345}&\braillebox{16}&\braillebox{2456}&\braillebox{}&\braillebox{1234}&\braillebox{135}\\
\hline
\end{tabular}

Četla `t' nejdřív jako `j' uvědomila, že to není správně stejným dechem. Četla `n' nejdřív jako `k'.  Vyslovila mezeru.

\paragraph{Řádek 18}
\begin{tabular}{|c|c|c|c|c|c|c|c|c|c|c|c|}
\hline
h&y&b&o&v&a&l& &s&p&e&c\\
\braillebox{12578}&\braillebox{13456}&\braillebox{12}&\braillebox{135}&\braillebox{1236}&\braillebox{1}&\braillebox{123}&\braillebox{}&\braillebox{234}&\braillebox{1234}&\braillebox{15}&\braillebox{14}\\
\hline
\end{tabular}

Četla všechno správně.

\paragraph{Řádek 19}
\begin{tabular}{|c|c|c|c|c|c|c|c|c|c|c|c|}
\hline
i&á&l&n&í&m& &p&e&r&e&m\\
\braillebox{2478}&\braillebox{16}&\braillebox{123}&\braillebox{1345}&\braillebox{34}&\braillebox{134}&\braillebox{}&\braillebox{1234}&\braillebox{15}&\braillebox{1235}&\braillebox{15}&\braillebox{134}\\
\hline
\end{tabular}

Četla všechno správně.

\paragraph{Řádek 20}
\begin{tabular}{|c|c|c|c|c|c|c|c|c|c|c|c|}
\hline
 &n&a&d& &p&í&s&m&e&n&e\\
\braillebox{78}&\braillebox{1345}&\braillebox{1}&\braillebox{145}&\braillebox{}&\braillebox{1234}&\braillebox{34}&\braillebox{234}&\braillebox{134}&\braillebox{15}&\braillebox{1345}&\braillebox{15}\\
\hline
\end{tabular}

Teď jsem se ji poprosil, aby přečetla slova, ne jednotlivá písmena.  V té chvílí, přestala reagovat nastroj, tak jsem to resetoval.  Četla \uv{nad} správně.

\paragraph{Řádek 21}
\begin{tabular}{|c|c|c|c|c|c|c|c|c|c|c|c|}
\hline
m&,& &p&ř&í&s&t&r&o&j& \\
\braillebox{13478}&\braillebox{2}&\braillebox{}&\braillebox{1234}&\braillebox{2456}&\braillebox{34}&\braillebox{234}&\braillebox{2345}&\braillebox{1235}&\braillebox{135}&\braillebox{245}&\braillebox{}\\
\hline
\end{tabular}

Četla \uv{písmenem} správně.  Četla \uv{přís} říkala něco(není rozumět na nahrávce), poradil jsem ji, ať hledá mezeru aby přečetla od začátku slova.  Našla mezeru, četla znovu, a správně řekla \uv{přístroj}.

\paragraph{Řádek 22}
\begin{tabular}{|c|c|c|c|c|c|c|c|c|c|c|c|}
\hline
b&z&u&č&e&l&.& & &T&o& \\
\braillebox{1278}&\braillebox{1356}&\braillebox{136}&\braillebox{146}&\braillebox{15}&\braillebox{123}&\braillebox{3}&\braillebox{}&\braillebox{}&\braillebox{23457}&\braillebox{135}&\braillebox{}\\
\hline
\end{tabular}

Četla \uv{bzučel} správně.  Řekl jsem ji, že tím ukončíme a zeptal jsem ji jestli má k tomu nějaké poznámky.  Ona řekla, že ne a rychle odešla.

\subsubsection{Žák 7}
Setkání 13:40 20.3.2013

Žák 7 je mezi 20-25 let starý.  Čte Braillovo písmo od začátku školní docházky.  Jeho nejvyšší dosažení úroveň vzdělávání je základní škola.

Četl tiskový text obě ruce.

% začátek 85.2 otočení 106.7 konec 134.6

Čas na čtení celého textu: 49.4 vteřin

CPS: 8.24

Čas na čtení druhé strany tiskového textu: 27.9 vteřin

CPS: 9.5

Namočil jsem vodou jeho ukazovací prst na pravou ruku aby se lepe fungovaly senzory. Ukázal jsem ho selektor a pak zobrazovač stručně bez delší popsání.

\paragraph{První řádek}
\begin{tabular}{|c|c|c|c|c|c|c|c|c|c|c|c|}
\hline
B&r&a&i&l&l&s&&&&&\\
\braillebox{1278}&\braillebox{1235}&\braillebox{1}&\braillebox{24}&\braillebox{123}&\braillebox{123}&\braillebox{234}&\braillebox{}&\braillebox{2358}&\braillebox{123}&\braillebox{}&\braillebox{}\\
\hline
\end{tabular}

Říkal jsem ať čte první \uv{písmo} a on říkal, že tam nic není.  Říkal jsem, že musí mít ruku na selektor a on tam ji dal ale ne na první senzor jen náhodně.  Začal hádat `a' ale jsem ho zastavil ať začíná na první písmeno.  Odhadl nejdřív jako `ě'\braillebox{126} ale podruhy odhadl správně.

Nehledal body jen položil ruku na selektor a říkal co si myslí.  Často odstranil ruku ze zobrazovače a pohyboval hlavu v stereotypickým pohybem tak aby tvař koukal nahoru a doleva. Hlavu pohyboval takhle po skoro každém písmenu ale pohyby nebyly formou rozptýlení. Samostatně se vrátil ke čtení potom co takhle pohyboval jeden až tři krát.

Druhý písmeno odhadl nejdřív jako `h' ale odhadl správně podruhy.  Četl `a' správně, a `i' četl na druhý pokus(první říkal čárka ale ani nedokončil slovo než samostatně se upravil).  Četl `l' správně, přeskočil další `l'. Myslím si, že používal zvuk cvaknutí od zobrazovače aby věděl o tom, kdy se změnil písmeno, ale protože nic se neměnil mezi `l' a `l' k žádnému cvaknutí nedošlo.  Neopravil jsem ho.  Četl `s' správně.  Říkal jsem mu, ať se vrátí na začátek selektoru abychom četli další řádek.

\paragraph{Druhý řádek}
\begin{tabular}{|c|c|c|c|c|c|c|c|c|c|c|c|}
\hline
k&ý& &ř&á&d&e&k&,& &k&t\\
\braillebox{1378}&\braillebox{12346}&\braillebox{}&\braillebox{2456}&\braillebox{16}&\braillebox{145}&\braillebox{15}&\braillebox{13}&\braillebox{2}&\braillebox{}&\braillebox{13}&\braillebox{2345}\\
\hline
\end{tabular}

Druhý řádek četl bez chyb kromě tomu, že jedno se vrátil na začátek selektorů.  Dělal jsem dost važnou chybu.  Upravil jsem ho nesprávně kvůli neznalost češtiny. Nerozuměl jsem ho kdy mě říkal `ř' jako `eř' a říkal jsem ho, že četl nesprávně.  Nikdy nevyslovoval mezery nahlas(ne, že bych to přikazoval).

\paragraph{Třetí řádek}
\begin{tabular}{|c|c|c|c|c|c|c|c|c|c|c|c|}
\hline
e&r&ý& &t&e&ď& &p&o&u&ž\\
\braillebox{1578}&\braillebox{1235}&\braillebox{12346}&\braillebox{}&\braillebox{2345}&\braillebox{15}&\braillebox{1456}&\braillebox{}&\braillebox{1234}&\braillebox{135}&\braillebox{136}&\braillebox{2346}\\
\hline
\end{tabular}

Četl `e' jako `o'.  To byl protože nespadnou body samou. Samostatně to pochopil a opravil se.  Potom co se přečetl `p', přeskočil několik písmen a četl správně `ž'. Kdy se začal číst znovu od `p', četl `o' nejdřív jako `k' ale opravil se než jsem ho říkal, že to není správný.

\paragraph{Čtvrtý řádek}
\begin{tabular}{|c|c|c|c|c|c|c|c|c|c|c|c|}
\hline
í&v&á&t&e&,& &n&e&n&í& \\
\braillebox{3478}&\braillebox{1236}&\braillebox{16}&\braillebox{2345}&\braillebox{15}&\braillebox{2}&\braillebox{}&\braillebox{1345}&\braillebox{15}&\braillebox{2345}&\braillebox{34}&\braillebox{}\\
\hline
\end{tabular}

První dvě písmena četl správně. Četl `á' jako `u'\braillebox{136}.  Nějak jsem se bal, že jeho chyba může byt kvůli vadný spojení v zobrazovače ale zjistil jsem, že ne.  Kdy se vrátil na začátek řádku ještě četl `v' jako `u'. Zase četl `á' jako `u' ale konečně správně říkal `á'.  Četl `e' jako `o'. Připomněl jsem ho, že body nespadnou samo.  Zbytek řádku četl správně ale četl `n' jako `d'(upravil se stejném dechem).

\paragraph{Pátý řádek}
\begin{tabular}{|c|c|c|c|c|c|c|c|c|c|c|c|}
\hline
p&r&v&n&í& &ř&á&d&e&k&,\\
\braillebox{123478}&\braillebox{1235}&\braillebox{1236}&\braillebox{1345}&\braillebox{34}&\braillebox{}&\braillebox{1235}&\braillebox{16}&\braillebox{145}&\braillebox{15}&\braillebox{13}&\braillebox{2}\\
\hline
\end{tabular}

Četl slovo \uv{první} bez chyb.  Audio je zaseknutí, a nevím co se stalo dal na řádce.

\paragraph{Šesti řádek}
\begin{tabular}{|c|c|c|c|c|c|c|c|c|c|c|c|}
\hline
 &k&t&e&r&ý& &z&o&b&r&a\\
\braillebox{78}&\braillebox{13}&\braillebox{2345}&\braillebox{15}&\braillebox{1235}&\braillebox{12346}&\braillebox{}&\braillebox{1356}&\braillebox{135}&\braillebox{12}&\braillebox{1235}&\braillebox{1}\\
\hline
\end{tabular}

První část četl správné ale myslel, že `z' je `ř'. Zbytek taky četl správně.

Přesouval ruku na selektor úplně samostatně a správně.

\paragraph{Sedmi řádek}
\begin{tabular}{|c|c|c|c|c|c|c|c|c|c|c|c|}
\hline
z&u&j&e& &j&e&n&o&m& &j\\
\braillebox{135678}&\braillebox{136}&\braillebox{245}&\braillebox{15}&\braillebox{}&\braillebox{245}&\braillebox{15}&\braillebox{1345}&\braillebox{135}&\braillebox{134}&\braillebox{}&\braillebox{245}\\
\hline
\end{tabular}

Četl správně.

\paragraph{Osmi řádek}
\begin{tabular}{|c|c|c|c|c|c|c|c|c|c|c|c|}
\hline
e&d&n&o& &p&í&s&m&e&n&o\\
\braillebox{1578}&\braillebox{145}&\braillebox{1345}&\braillebox{135}&\braillebox{}&\braillebox{1234}&\braillebox{24}&\braillebox{234}&\braillebox{134}&\braillebox{15}&\braillebox{1345}&\braillebox{135}\\
\hline
\end{tabular}

Četl správně kromě `s', nevím přesně co odhadl první kvůli špatné kvalitě zvuku.  `o' četl nejdřív jako  `a' ale opravil se stejným dechem.

\paragraph{Devátý řádek}
\begin{tabular}{|c|c|c|c|c|c|c|c|c|c|c|c|}
\hline
.& & &P&r&v&n&í& &b&y&l\\
\braillebox{378}&\braillebox{}&\braillebox{}&\braillebox{12347}&\braillebox{1235}&\braillebox{1236}&\braillebox{1345}&\braillebox{34}&\braillebox{}&\braillebox{12}&\braillebox{13456}&\braillebox{123}\\
\hline
\end{tabular}

Četl bez chyb.

\paragraph{Desátý řádek}
\begin{tabular}{|c|c|c|c|c|c|c|c|c|c|c|c|}
\hline
 &v&y&n&a&l&e&z&e&n& &v\\
\braillebox{78}&\braillebox{1236}&\braillebox{13456}&\braillebox{1345}&\braillebox{1}&\braillebox{123}&\braillebox{15}&\braillebox{1356}&\braillebox{15}&\braillebox{1345}&\braillebox{}&\braillebox{1236}\\
\hline
\end{tabular}

Četl `v' nejdřív nejsprávně ale kvalita zvuku mě nedovolí vám říct jak.  Četl až na `e' správně ale selektor začal zle fungovat. Správil jsem to a řekl jsem ho ať přečte od začatku.  Četl správně kromě `e' což četl nejdřív jako `o' a mezera, která chvílí myslel je tečka.  

\paragraph{Jedenacti řádek}
\begin{tabular}{|c|c|c|c|c|c|c|c|c|c|c|c|}
\hline
 &r&o&c&e& &1&9&1&3& &v\\
\braillebox{78}&\braillebox{1235}&\braillebox{135}&\braillebox{14}&\braillebox{15}&\braillebox{}&\braillebox{18}&\braillebox{248}&\braillebox{18}&\braillebox{148}&\braillebox{}&\braillebox{1236}\\
\hline
\end{tabular}

Četl `c' nejdřív jako `n'.  Četl `1' nejdřív jako `a'. Vysvětlil jsem, že osmi bod je pro čísla.  Nevím zda pochopil, protože `9' nejdřív četl jako `i'.  Když jsem ho říkal, že `i' je nesprávně, četl zbytek čísel správně.  Četl `v' jako `l' ale odvolal stejným dechem.  Pak odhadl `r' a konečně správně `v'.

\paragraph{Dvanácti řádek}
\begin{tabular}{|c|c|c|c|c|c|c|c|c|c|c|c|}
\hline
 &A&n&g&l&i&i&.& & &J&m\\
\braillebox{78}&\braillebox{17}&\braillebox{1345}&\braillebox{1245}&\braillebox{123}&\braillebox{24}&\braillebox{24}&\braillebox{3}&\braillebox{}&\braillebox{}&\braillebox{2457}&\braillebox{134}\\
\hline
\end{tabular}

Řešil jsem problém se selektorem u začátku řádku.  Musím poznamenat, že on poznal, že selektor nefungoval a čekal na mně. Četl `i' nejdřív jako `e'.  Vím, že už dobře rozuměl selektor, protože poznal, že jsou dvě `i' a dvě mezery po tečce.  Zbytek řádku četl správně.

\paragraph{Řádek 13}
\begin{tabular}{|c|c|c|c|c|c|c|c|c|c|c|c|}
\hline
e&n&o&v&a&l& &s&e& &O&p\\
\braillebox{1578}&\braillebox{1345}&\braillebox{135}&\braillebox{1236}&\braillebox{1}&\braillebox{123}&\braillebox{}&\braillebox{234}&\braillebox{15}&\braillebox{}&\braillebox{1357}&\braillebox{1234}\\
\hline
\end{tabular}

Četl `v' nejdřív jako `l' ale opravil se stejným dechem. Jinak četl správně.

\paragraph{Řádek 14}
\begin{tabular}{|c|c|c|c|c|c|c|c|c|c|c|c|}
\hline
t&o&f&o&n& &a& &p&ř&e&v\\
\braillebox{234578}&\braillebox{135}&\braillebox{124}&\braillebox{135}&\braillebox{1345}&\braillebox{}&\braillebox{1}&\braillebox{}&\braillebox{1234}&\braillebox{2456}&\braillebox{15}&\braillebox{1236}\\
\hline
\end{tabular}

Jedno jsem si myslel, že selektor nefunguje, protože nějak rychle cvakl zobrazovač. Ale kdy jsem zkusil, jsem zjistil, že zkutečně funguje správně.  Jinak četl bez chyb.

\paragraph{Řádek 15}
\begin{tabular}{|c|c|c|c|c|c|c|c|c|c|c|c|}
\hline
á&d&ě&l& &s&v&ě&t&l&o& \\
\braillebox{1678}&\braillebox{145}&\braillebox{126}&\braillebox{123}&\braillebox{}&\braillebox{234}&\braillebox{1236}&\braillebox{126}&\braillebox{2345}&\braillebox{123}&\braillebox{135}&\braillebox{}\\
\hline
\end{tabular}

Četl bez chyb.

\paragraph{Řádek 16}
\begin{tabular}{|c|c|c|c|c|c|c|c|c|c|c|c|}
\hline
n&a& &z&v&u&k&.& & &K&d\\
\braillebox{134578}&\braillebox{1}&\braillebox{}&\braillebox{1356}&\braillebox{1236}&\braillebox{136}&\braillebox{13}&\braillebox{3}&\braillebox{}&\braillebox{}&\braillebox{137}&\braillebox{145}\\
\hline
\end{tabular}

Četl bez chyb a poznal `K' jako velké.

\paragraph{Řádek 17}
\begin{tabular}{|c|c|c|c|c|c|c|c|c|c|c|c|}
\hline
y&ž& &č&t&e&n&á&ř& &p&o\\
\braillebox{1345678}&\braillebox{2346}&\braillebox{}&\braillebox{146}&\braillebox{2345}&\braillebox{15}&\braillebox{1345}&\braillebox{16}&\braillebox{2456}&\braillebox{}&\braillebox{1234}&\braillebox{135}\\
\hline
\end{tabular}

Četl správně až na `ř' a u toho jsem mu říkal, že budeme zkusit něco jiné a jsem se zkusil ho naučit nechat ruku položený na selektor.  Musel jsem zase resetovat stroj.

Audio je zase zpřeházený ale je rozumět kdy četl písmeno `y'. Odhadl `ď'\braillebox{1456} ale dal najevo, že nebyl jistý.  Teď, už nechal ruku na zobrazovač.  Taky četl `č' nesprávně ale není rozumět jak.

\paragraph{Řádek 18}
\begin{tabular}{|c|c|c|c|c|c|c|c|c|c|c|c|}
\hline
h&y&b&o&v&a&l& &s&p&e&c\\
\braillebox{12578}&\braillebox{13456}&\braillebox{12}&\braillebox{135}&\braillebox{1236}&\braillebox{1}&\braillebox{123}&\braillebox{}&\braillebox{234}&\braillebox{1234}&\braillebox{15}&\braillebox{14}\\
\hline
\end{tabular}
Četl `h' správně na druhý pokus, první odhad není rozumět.  Četl `y' jako `d' ale uvědomoval stejném dechem, že to není správně odhadl správně na druhý pokus.

\paragraph{Řádek 19}
\begin{tabular}{|c|c|c|c|c|c|c|c|c|c|c|c|}
\hline
i&á&l&n&í&m& &p&e&r&e&m\\
\braillebox{2478}&\braillebox{16}&\braillebox{123}&\braillebox{1345}&\braillebox{34}&\braillebox{134}&\braillebox{}&\braillebox{1234}&\braillebox{15}&\braillebox{1235}&\braillebox{15}&\braillebox{134}\\
\hline
\end{tabular}
Četl `i' nejdřív jako `s'.  Četl `á' nejdřív jako `e'. Četl druhý `e' nejdřív jako `o'. Četl `m' jako `p' ale upravil se stejným dechem.

U toho jsem končil setkání.  Zeptal jsem se ho \uv{myslíte, že byste se naučil to používat lip s praxi?} on odpovídal \uv{ne, to je těžký, já mám na to notebook, psací stroj.} Zeptal jsem se \uv{a používáte braillský řádek?} On: \uv{Ne, hlasový vystup.}


\subsubsection{Žák 8}
Setkání 23.4.2013

Žák 8 je ve věku mezi 20 a 25 lety. Braillovo písmo čte od začátku školní docházky. Je úplně nevidomý. Je pravák.  Nepoužívá braillský řádek.

Četl tištěný text jednou(pravou) rukou.

% začátek 37.6 otočení 93.8 konec 180.7

Čas čtení tištěného textu: 143.1 vteřin

CPS: 2.84

Čas čtení tištěného textu po otočení listu: 87.1  vteřin

CPS: 3

Kdy jsem byl představený osmi žák hned chytil na to, že Timothy není české jméno a kdy jsem se zeptal osmi žákovi úvodní otázky, zkusil na mně mluvit anglicky, což dělal skoro bez přízvuku.

Když jsem ho ukazoval selektor, selektor začal poznat náhodné dotyky.  Po krátké době určení problém, jsem zjistil, že je tím, že účastník měl mokré ruce.

\paragraph{První řádek}

\begin{tabular}{|c|c|c|c|c|c|c|c|c|c|c|c|}
\hline
B&r&a&i&l&l&s&&&&&\\
\braillebox{1278}&\braillebox{1235}&\braillebox{1}&\braillebox{24}&\braillebox{123}&\braillebox{123}&\braillebox{234}&\braillebox{}&\braillebox{2358}&\braillebox{123}&\braillebox{}&\braillebox{}\\
\hline
\end{tabular}

Začali jsme číst první řádek tím, že se zeptal \uv{jak s tím na to}.  Začal jsem vysvětlení tím, že jsem ho ukázal ty díry na zobrazovače a pak kdy jeho pravá ruka byla na selektor zeptal jsem se které body jsou nahoře.  On říkal \uv{první} a odhadl `a', asi moje vysvětlení byla dost zmatený nepochopil jsem proč odhadl `ačko'.  Myslel, jsem si, že třeba myslí, že se zobrazí `á' a je zmatený z těch body 7 a 8.  Kdy měl prst na bod 8 jsem ho říkal, že to je pro velké písmena(což není pravda, bod 7 je pro velké písmena, bod 8 je pro čísla). Odhadl `v' a ještě `u'. Zřejmě nevěděl co to je osmibodové Braillovo písmo ale jsem to ještě nevěděl.  Začal jsem ho vykládat o selektoru.  Ale jak jsem ho vysvětlil, že jsou písmena `k' na selektor, hned odhadl, že zobrazené písmeno je `k'.

Jsem ještě vyřešil tomu, že měl mokré ruce a jsem malém leh tím, že selektor nefungoval.  Říkal jsem \uv{to mně dost překvapuje, aha!} když jsem viděl tu vodu.  On odpovídal \uv{Já nechci urazit, ale bych se nejdřív zeptal jestli už s tím umíte}.  Odpovídal jsem, že umím ale, že měl příliš mokré ruce na to a to chvíli nefungoval.

Zase jsem ho snažil vysvětlit zobrazovač.  Dal jsem jeho prst na každé díře a řekl jsem ho, které to je.  Teď jsem se dozvěděl, že snad v životě nesetkal s osmibodové Braillovo písmo protože se mně rovně zeptal na \uv{ty zbyli dva body}.  Jsem ho vysvětlil: \uv{Na počítače děláme velké písmeno tím, že dáváme nahoru asi sedmi bod a čísla tím že dáváme nahoru osmi bod}.

U toho jsem si rozhodl snížit citlivost senzorů, protože jenom tím jak jsem jeho ruku natřel papírovým kapesníkem nestačil.  Musel jsem restartovat celý softwarový sestavení.

Potom jsme se začaly znovu s poznání písmen.  Ještě na `B' jsem zase pojmenoval číslema body a dával jsem jeho prst na každý tyč nebo díru. Pojmenoval jsem a ukazoval jsem jenom první šest bodů abych ho zbytečně nezmátl.  Jasně musel cítit, že body jedna a dva jsou nahoře a žádné jiné ale stále odhadl nesprávně `včko'.  Vysvětlil jsem ho zase, že body sedm a osm nepoužíváme k poznání písmena a konečně on odhadl `bčko'.

Četl `r' nejdřív jako `h' ale odhadl správně potom co jsem ho říkal ať hledá dolu.

Říkal jsem ho, ať přesouvá na další písmeno a jsem musel zase utřít jeho ruce.  Byl tam dost vedro.  Natřel jsem selektor a on mě říkal: \uv{Bych měl radši braillský řádek na, kterém by to co bych napsal na počítači automatický zobrazoval.}

Říkal jsem ho jak má přesouvat na selektor ale četl `a' bez pomoci.

Odstranil ruku ze selektoru a jsem ho vysvětlil, že když nedotyká selektorem, tak se nezobrazí nic.  Dal ruku zpátky na selektor v přibližně správní místo a četl `i' jako `,' než jsem ho připomněl o další řádek bodů.

Zase odstranil ruku z selektoru ale vrátil ji zpět když jsem ho říkal ať čte další písmeno. Dal tu ruku na selektor o tři písmen dal než měl, ale četl správně `s'. Vrátil zpátky a četl `l' nejdřív jako `k'.

On začal hrát s selektorem a zobrazovačem.  Odstranil ruku z selektoru a pak poklepl na každý tyč který byl nahoru aby spadl.  Nechal jsem ho hrát 46 vteřin a pak on říkal tohle je `lko'(měl pravdu, tam se zkutečně `l' zobrazoval).  Pak jsem navrhoval, že čteme další řádek.

\paragraph{Druhý řádek}
\begin{tabular}{|c|c|c|c|c|c|c|c|c|c|c|c|}
\hline
k&ý& &ř&á&d&e&k&,& &k&t\\
\braillebox{1378}&\braillebox{12346}&\braillebox{}&\braillebox{2456}&\braillebox{16}&\braillebox{145}&\braillebox{15}&\braillebox{13}&\braillebox{2}&\braillebox{}&\braillebox{13}&\braillebox{2345}\\
\hline
\end{tabular}

Četl první písmeno jako `l', druhý odhad měl jako `v'.  Vysvětlil jsem mu, že body sedm a osm jsou tam jenom protože jsme u začátku řádek.  Odhadl `h' pak `u'.  Zase jsem dával jeho prst na každý bod a pojmenoval jsem je čísly. Pak jsem se zeptal, které body jsou nahoře.  On ukazoval na první bod a říkal \uv{To je ten sedmi} říkal jsem ho, že to není, že to je první.  Pak on říkal \uv{těžko říct, on je závorky}.  Říkal jsem ho \uv{Počítejte se mnou, jedna, dva, tři, čtyři, pět, šest, tak nahoře je bod jedna a bod tři, které písmeno to je} konečně správně říkal `kačko'.

Pokračovali jsme na další písmeno.  Nedržel ruku na selektor ale poklepal na senzor a pak prst odstranil.  Četl písmeno jako `l', pak `p' a konečně `ý'.

Odstranil zase ruku z selektoru, tak jsem si rozhodl zase snažit mu vysvětlit jak funguje.  Vedl jsem jeho prst nad `\uv{ty kačky}(ty senzory).

Vedl jsem jeho ruku na čtvrté písmeno, a četl to nejdřív jako `t' a potom, správně, jako `ř'.  Četl další písmeno `á' správně. Zase jsem musel utřít selektor a on utřel ruky na kalhotách a mně vykládal \uv{To je strašně přemodernizováné braillský řádek, vůbec se vtom nevyznám.} Vedl jsem jeho ruku na správné místo a jsme pokračovali.  Četl `d' nejdřív jako `č' než to četl správně.  On četl, tak že klepal na selektor a pak klepal každý bod na zobrazovač dole.  Četl `e' nejdřív jako `á', pak `o', pak `i'.  Konečně jsem ho jen říkal odpověď.  On říkal \uv{vůbec to není poznat} odpověděl jsem, že to jsou stejné písmena jenom větší, a on mě řek \uv{Já vím ale}.  Pořád klepal na ty senzory ale přibližně správně překračoval. Myslel, že chvili se zobrazoval `ě'.  Četl `k' správně.  Četl čárku správně.

Vysvětlil jsem mu, že už četl dvě slova.  Říkal mně, že to nepoznal.  Zeptal jsem se ho, jestli je unavený, ale říkal, že není.

Vedl jsem jeho ruku na předposlední znak, který četl správně jako `k'.  Pak začal zase ty senzory nefungovat a jsem je zase utřel.  On říkal anglicky \uv{I will get drunk}(budu opilý).  Vedl jsem jeho ruku zpátky na `t' který nejdřív četl jako `;'\braillebox{23} a pak správně.

\paragraph{Třetí řádek}
\begin{tabular}{|c|c|c|c|c|c|c|c|c|c|c|c|}
\hline
e&r&ý& &t&e&ď& &p&o&u&ž\\
\braillebox{1578}&\braillebox{1235}&\braillebox{12346}&\braillebox{}&\braillebox{2345}&\braillebox{15}&\braillebox{1456}&\braillebox{}&\braillebox{1234}&\braillebox{135}&\braillebox{136}&\braillebox{2346}\\
\hline
\end{tabular}

Na začátek třetí řádku odhadl nejdřív `k', pak `A', `t', `z'. Pak řekl, že neví. Odhadl `x' a jsem se zeptal, které body jsou nahoru. On říkal \uv{první, třetí} řekl jsem ho, že to je \uv{sedmi už}.  On říkal(správně) \uv{tuto logiku mizí}. Pak odhadl `ě' a jsem ho říkal \uv{ečko ale bez háčku}.  On mumlal \uv{tu logiku prostě mizí když}.  Další písmeno odhadl nejdřív jako `l' a pak správně jako `ý'. Četl `t' správně.  Pořád četl tím způsobem, že klepal krátce na selektor, a pak klepal na zobrazovače prstem. Je důležité poznamenat teď, že body na můj FCHAD nespadnou samy, musíte je pomoct. Říkal \uv{tohle nefunguje vůbec}. Vysvětlil jsem mu, že jenom body, které jsou pevné jsou významné. Četl `e' správně.  Pak četl `ď' jako `š'\braillebox{156}.  Připomněl jsem ho, že ještě ne zkusil hmatat na bodu čtyři.  Tento krát odhadl správně `ď'.  Četl tu mezeru jako čárka. Četl `p' nejdřív jako `t' pak `e', konečně jako `p'.  Četl `o' správně.  Přeskočil `u' ale četl `ř' správně.

\paragraph{Čtvrtý řádek}
\begin{tabular}{|c|c|c|c|c|c|c|c|c|c|c|c|}
\hline
í&v&á&t&e&,& &n&e&n&í& \\
\braillebox{3478}&\braillebox{1236}&\braillebox{16}&\braillebox{2345}&\braillebox{15}&\braillebox{2}&\braillebox{}&\braillebox{1345}&\braillebox{15}&\braillebox{2345}&\braillebox{34}&\braillebox{}\\
\hline
\end{tabular}
Četl, první písmeno nejdřív jako `ž', pak `i', a jsem ho řekl, že je to `í'.  Četl `v' a `á' správně přeskočil `t' ale četl `e' správně.  Řekl, že neví jaké to mělo byt slovo. Vysvětlil jsem mu, že je to slovo \uv{používáte} ale, že přeskočil písmena `u' a `t'.  Zeptal se, \uv{kde je učko} tak jsme se vrátili na předchozí řádku a jsem vedl jeho ruku k `o' pak `u' a `ž', písmena správně ztotožnil.

Pak jsme se začali číst od začátku řádku. Nejdřív, myslel, že `í' je `u' ale odhadl správně po druhý. Pak myslel, že se zobrazuje `,' ale ještě bylo nahoru `í'.  Zeptal se kolik je hodin a v stejném chvilce objevila v dveře sekretářka a zeptala se \uv{jak jste na to}. Říkal jsem, že už můžeme končit a žák 8 říkal, že by měl už jít.  Sekretářka ho řekla, že ještě má pět minut a jsme pokračovali.

Přeskočil ještě par písmen a jsme byli u `e' co ztotožnil správně. Myslel nejdřív je `n' je `g'. Přeskočil `í' a správně ztotožnil mezeru.  Neřekl jsem ho, že přeskočil, protože to by bylo zbytečné obtížení úkolu.

Dal jsem hu číst další řádek.

\paragraph{Pátý řádek}
\begin{tabular}{|c|c|c|c|c|c|c|c|c|c|c|c|}
\hline
p&r&v&n&í& &ř&á&d&e&k&,\\
\braillebox{123478}&\braillebox{1235}&\braillebox{1236}&\braillebox{1345}&\braillebox{34}&\braillebox{}&\braillebox{1235}&\braillebox{16}&\braillebox{145}&\braillebox{15}&\braillebox{13}&\braillebox{2}\\
\hline
\end{tabular}

Zeptal se \uv{jak ty řádky zadávají?}  Ukazoval jsem ho svůj notebook, a řekl jsem ho, že používám ty šípky abych přešel na další řádek.  Potom co jsem ho ukazoval ten počítač, nemohl znovu najít selektor, tak jsem musel jeho ruku tam dat.

Nejdřív myslel, že řádek začal `ý' ale pak správně řekl `p' když jsem ho připomněl o bodech sedm a osm.  Správně ztotožnil `r'.  Kdy přesunul ruku, zřejmě slyšel cvaknutí, protože zeptal se, jestli něco přeskočil.  Řekl jsem ho, že zkutečně přeskočil písmeno a vrátil se úspěšně na `v', myslel chvílí, že je `ě'\braillebox{126} a trval mu trochu, ale odhadl správně `v' podruhy.  Správně odhadl `n' a `í'.  Zeptal jsem ho, jestli pozná, které slovo to je, ale říkal, že ne.  Zeptal jsem se, jestli pamatuje první písmeno v řádku.  Řekl `y'.  Zase jsme se vrátili na začátek řádku a jsme zase uvažovali přes to, že jsou tam nahoru body sedm a osm.  Správně odhadl podruhy.  Zeptal se pak \uv{druhý byl co} a jsem ho říkal ať to najde sám.  Začal číst druhý písmeno jako `h'. Tím jsme ukončili, protože přišla sekretářka s dalším účastníkem.  Žák 8 říkal \uv{to je nad mou logiku} a rychle odešel.



\subsubsection{Žák 9}
Setkání 23.4.2013

Žák 9 je 20-25 let starý.  Nejvyšší dosažení úroveň vzdělávání je maturita.  Je úplně nevidomý a má problémy se sluchem.  Při naše setkání jsem nosil na sebe mikrofon, který byl bezdrátově přípojení na jeho sluchadla.  Rozuměl mě bez znatelný potíže.

Četl tiskový text jednou(pravou) rukou.

%začatek 197.8 otočení 230.2 konec 279
Čas na čtení celého textu: 81.2 vteřin

CPS: 5

Čas na čtení druhé strany tiskového textu: 48.8 vteřin

CPS: 5.4

Kdy jsem ho ukazoval zobrazovač, říkal jsem mu \uv{tady se zobrazí to písmeno. Tady jsou osm děr.} On vzal zobrazovače obě ruce a začal je počítat, ale nepočítal je v standardní pořadí číslování jen je počítal.  Mluvil milým a překvapeným hlasem po celé setkání. Kdy byl jistý, že jsou tam zkutečně osm děr, říkal jsem ho, že by měl položit levou ruku na zobrazovače.  On mě odpovídal smíchem, že \uv{Já jsem sice levák ale čtu pravou}. Vysvětlil jsem mu, že současní stroj má jenom jeden nastavení.  Potom jsem ho ukazoval selektor.  Říkal jsem mu, že zvuk, který vydá zobrazovač je zvuk změna písmen a že by měl taky cítit jak se to písmeno změní.  On začal \uv{hrát} se selektorem a jsem si myslel, že bude samostatně pokusit o čtení.  Měl ruku správně položenou na zobrazovače a správně byl u první písmena na selektor.  Občas říkal \uv{uf}.

Protože jsem chtěl dozvědět, jestli bude číst sám, čekal jsem 146 vteřin.  Bohužel v analýze audio-nahrávku vidím, že jsem říkal \uv{um} u začátku toho času, tak je docela možný, že on sám čekal na mně.  Kdy mně bylo jasno, že sám číst nebude, jsem se začal vykládat o použitý nastroje dal.

Vysvětlil jsem mu, že \uv{ty písmena `k'} jsou senzory.  On začal je počítat a když počítal, že jsou jích 12 tak říkal \uv{aha} jak že už spojil je s těmi dvanácti senzory vysvětlené v úvodním textu. Zase jsem čekal ať hraje, a tento krát jsem neříkal žádný zmítající \uv{um}.  Zkutečně hrál s nástrojem a věřil jsem, že se snaží čist. 210 vteřin později ještě nepokračoval.  Stále hmatal na zobrazovače jako by se snažil z něj číst ale nečetl. %881-671.9

\paragraph{První řádek}

\begin{tabular}{|c|c|c|c|c|c|c|c|c|c|c|c|}
\hline
B&r&a&i&l&l&s&&&&&\\
\braillebox{1278}&\braillebox{1235}&\braillebox{1}&\braillebox{24}&\braillebox{123}&\braillebox{123}&\braillebox{234}&\braillebox{}&\braillebox{2358}&\braillebox{123}&\braillebox{}&\braillebox{}\\
\hline
\end{tabular}


Zeptal jsem se mu co je první písmeno a on se mě zeptal \uv{jak to poznat?}.  Říkal jsem mu, ať počítá díry.  Nejdřív se mně nepochopil, zřejmě kvůli chybě mé češtiny.  Počítal, že jsou dva body nahoru, ale nepochopil ještě o co jde.  Vzal jsem jeho levou ruku a ukazoval jsem ho \uv{to je jedna a dva tak které písmeno to je?} a u toho už říkal bez myšlení `b'.  Protože jsme byli u začátek řádku ještě byly nahoře body 7 a 8.  Bohužel žák 8 se kymácel trupem tak, že jeho ruce na selektor nebyl stabilní a dále trpěl malé zrcadloví pohyby kdy se snažil číst na zobrazovače.  Myslím si, že rozuměl už, že by měl jít na druhé písmeno i bez toho, že bych mu řekl, ale nebyl fyzické schopní.  Věděl, že jeho prsty pohybují bez jeho přání a začal brzo byt frustrovaní tím.  Držel jsem jeho prstí tak aby nepohybovaly a vedl jsem jeho prsti na selektor.  Přečetl `r' \uv{první a pátý koukám} jsem mu říkal \uv{ještě} a on překvapeně, když objevoval ještě \uv{druhý a třetí aha, takže rko}.

Dal četl `a' stále způsobem, že nejdřív říkal čísla bodů a písmeno až potom.  `l' už četl bez pojmenování bodů. Četl další písmeno jako `p' a pak opravil, že je `s' když jsem ho říkal že četl nesprávně \uv{aha,tohle zmizelo, tak dávat pozor na to co se zmizí} on myslel tím, že nejdřív četl `l'\braillebox{123} a pak `s'\braillebox{234} protože body nespadnou než na něj tlačíte, chvílí zkutečně se zobrazilo `p'\braillebox{1234}.

\paragraph{Druhý řádek}
\begin{tabular}{|c|c|c|c|c|c|c|c|c|c|c|c|}
\hline
k&ý& &ř&á&d&e&k&,& &k&t\\
\braillebox{1378}&\braillebox{12346}&\braillebox{}&\braillebox{2456}&\braillebox{16}&\braillebox{145}&\braillebox{15}&\braillebox{13}&\braillebox{2}&\braillebox{}&\braillebox{13}&\braillebox{2345}\\
\hline
\end{tabular}

Přišli jsme na další řádek a četl `k' bez problému, ale četl `ý' zase tím, že pojmenoval body.  Kdy jsme dorazili na mezeru, on říkal \uv{nic} a začal se smát.   Četl slovo \uv{řádek} po jednotlivých písmenech ale bez chyb a bez pojmenování čísla bodů.  Četl tak dobře, že jsem chvíli pustil jeho pravou ruku ať používá selektor samostatně ale zase začala pochybovat rytmický bez jeho ovládání.  Našli jsme místo kde jsme skončili a pokračovali jsme.

Četl `k' nejdřív jako `a' ale sám se opravil když jsem ho říkal, že to není správný.

\paragraph{Třetí řádek}

\begin{tabular}{|c|c|c|c|c|c|c|c|c|c|c|c|}
\hline
e&r&ý& &t&e&ď& &p&o&u&ž\\
\braillebox{1578}&\braillebox{1235}&\braillebox{12346}&\braillebox{}&\braillebox{2345}&\braillebox{15}&\braillebox{1456}&\braillebox{}&\braillebox{1234}&\braillebox{135}&\braillebox{136}&\braillebox{2346}\\
\hline
\end{tabular}

Přešli jsme na třetí řádek.  Začátek četl správně. Vždy, kdy dostal na mezera, hmatal jak tam nic nahoru není a usmíval se a řekl \uv{mezera} s radosti tím jak je lehký poznat mezery až na `ž', ale četl `ž'\braillebox{2346} jako `z'\braillebox{1356}. Říkal jsem \uv{To je jako `z' ale obraceně} a on četl(obraceně) \uv{jedna tři pět šest, to je `z'} jsem ho ukazoval, že pojmenoval čísla ty body obraceně a on odpovídal rozčileně \uv{Tak, že jsem to celou dobu říkal blbě!}  Uklidnil jsem ho, že četl správně do té doby a on se krasně usmíval a říkal, že je úplně zmatený.

\paragraph{Čtvrtý řádek}

\begin{tabular}{|c|c|c|c|c|c|c|c|c|c|c|c|}
\hline
í&v&á&t&e&,& &n&e&n&í& \\
\braillebox{3478}&\braillebox{1236}&\braillebox{16}&\braillebox{2345}&\braillebox{15}&\braillebox{2}&\braillebox{}&\braillebox{1345}&\braillebox{15}&\braillebox{2345}&\braillebox{34}&\braillebox{}\\
\hline
\end{tabular}

Kdy jsme pokračovali na další řádek, četl `í'\braillebox{34} jako `á'\braillebox{16} ale tento krát upravil se sám když jsem ho říkal, že to není správní.  Zase u toho usmíval.  Pak četl `v' a `á' bez chybně.  Četl `e' jako `i' ale tento krát jsem Já dělal chybu a neopravil jsem ho. Četl `,' správně a když jsme byli na `n'\braillebox{1345} přemyslel a pojmenoval body hodně dlouho(39 vteřin) než odhadl `ž', který hned sám odvolal. Čekal jsem další 18 vteřin a říkal jsem ho \uv{je to `ž' z horu nohama}(I Já jsem byl zřejmě trochu zmatený, protože `ž' z horu nohama je `ň').  Konečně jsem ho jen tak říkal, že je to `n' ale mně nevěřil a říkal \uv{mate to, že to je zrcadlově obracený} ale konečně říkal \uv{aha, už to vidím} a zase krasně usmíval.  %1838.4

Nějak během toho, jsme přeskočili dvě písmena a jsme byly u `í', což četl nejdřív jako `t', pak `á' a konečně `í'.

Zeptal jsem ho jestli není unavený, ale říkal že je \uv{v pohodě}.

\paragraph{Pátý řádek}

\begin{tabular}{|c|c|c|c|c|c|c|c|c|c|c|c|}
\hline
p&r&v&n&í& &ř&á&d&e&k&,\\
\braillebox{123478}&\braillebox{1235}&\braillebox{1236}&\braillebox{1345}&\braillebox{34}&\braillebox{}&\braillebox{1235}&\braillebox{16}&\braillebox{145}&\braillebox{15}&\braillebox{13}&\braillebox{2}\\
\hline
\end{tabular}

Na další řádek četl písmena \uv{první ř} bez chyb a dal velký důraz na `n'.  Zase četl `á' nejdřív `í'. Četl `d' a `e' správně ale `k' četl nejdřív jako `a'. Četl `,' správně.

\paragraph{Šesti řádek}
\begin{tabular}{|c|c|c|c|c|c|c|c|c|c|c|c|}
\hline
 &k&t&e&r&ý& &z&o&b&r&a\\
\braillebox{78}&\braillebox{13}&\braillebox{2345}&\braillebox{15}&\braillebox{1235}&\braillebox{12346}&\braillebox{}&\braillebox{1356}&\braillebox{135}&\braillebox{12}&\braillebox{1235}&\braillebox{1}\\
\hline
\end{tabular}

Šesti řádek četl správně až na `z' který nejdřív četl jako `n'.  Četl `o' a `b' správně ale četl `r' nejdřív jako `h'.

Je možní, že protože jsem musel vest jeho pravou ruku ještě nerozuměl selektor. Kdy jsme byli u konce řádku myslel, že se zobrazuje mezera.  Nic se nezobrazoval, ale to byl protože nebyl v kontaktu s žádným senzorem.

\paragraph{Sedmi řádek}
\begin{tabular}{|c|c|c|c|c|c|c|c|c|c|c|c|}
\hline
z&u&j&e& &j&e&n&o&m& &j\\
\braillebox{135678}&\braillebox{136}&\braillebox{245}&\braillebox{15}&\braillebox{}&\braillebox{245}&\braillebox{15}&\braillebox{1345}&\braillebox{135}&\braillebox{134}&\braillebox{}&\braillebox{245}\\
\hline
\end{tabular}

Četl `z' jako `n' ale sám se upravil.  Říkal jsem ho ať zkusí posunut pravou ruku na selektor samostatně, a přestal jsem držet jeho ruku ale jak jsem nedržel jeho ruka tak pohybovala sama a on si povzdechl frustraci, tak jsem zase tu ruku držel.   Četl `u' nejdřív jako `a' ale upravil se když jsem ho říkal, že to není správný.  Četl `j'\braillebox{245} jako `h'\braillebox{125} ale upravil se sám když jsem ho říkal, že je obracený. Četl `e' a mezera správně ale zase obrátil `j'.  Četl `e` správně ale `n' jako `d'. Když jsem ho říkal, že to není správný, nejdřív si myslel, že četl obraceně. Říkal jsem ho, že neobracel to a, že ještě nějaký bod dolu a rychle už našel, že je to `n'. Četl `o' správně ale četl `m' nejdřív jako `k'.  Četl `j' jako `,' a jsem ho připomněl, že je ještě další řádek bodu než svůj odhad opravil.

\paragraph{Osmi řádek}
\begin{tabular}{|c|c|c|c|c|c|c|c|c|c|c|c|}
\hline
e&d&n&o& &p&í&s&m&e&n&o\\
\braillebox{1578}&\braillebox{145}&\braillebox{1345}&\braillebox{135}&\braillebox{}&\braillebox{1234}&\braillebox{24}&\braillebox{234}&\braillebox{134}&\braillebox{15}&\braillebox{1345}&\braillebox{135}\\
\hline
\end{tabular}

Na osmi řádek textu četl správně až na `í' což zase obrátil jako `á'. Četl `s' a `m' správně ale pak si myslel, že už jsme na další písmeno, když jsme v zkutečnosti nebyli.  Četl `m'\braillebox{134} jako `c'\braillebox{14}.  Četl `e' jako `o' ale sám si objevil tu chybu.  A četl `o' jako `e'.

\paragraph{Devátý řádek}
\begin{tabular}{|c|c|c|c|c|c|c|c|c|c|c|c|}
\hline
.& & &P&r&v&n&í& &b&y&l\\
\braillebox{378}&\braillebox{}&\braillebox{}&\braillebox{12347}&\braillebox{1235}&\braillebox{1236}&\braillebox{1345}&\braillebox{34}&\braillebox{}&\braillebox{12}&\braillebox{13456}&\braillebox{123}\\
\hline
\end{tabular}

Devátý řádek začíná s `.' ale bylo zobrazený jako \braillebox{378} protože jsme byli na začátek řádku.  On to četl jako závorka což je \braillebox{236}.  Připomněl jsem ho, že body sedm a osm jsou nahoru protože je to začátek řádku a zeptal jsem se, které body jsou ještě nahoru.  Jeho první odhad byl obracený, že to je znak pro velké písmeno\braillebox{6} a konečně odhadl správně když jsem ho říkal že to není.  Není důvod proč by věděl, že znak pro velké písmeno je bod šest a ne bod 3, protože ten znak je použitý jenom v tiskovém Braillovu písmu a bez rámečku není podstatní rozdíl.  Četl zbytek řádku bez problému.

\paragraph{Desátý řádek}
\begin{tabular}{|c|c|c|c|c|c|c|c|c|c|c|c|}
\hline
 &v&y&n&a&l&e&z&e&n& &v\\
\braillebox{78}&\braillebox{1236}&\braillebox{13456}&\braillebox{1345}&\braillebox{1}&\braillebox{123}&\braillebox{15}&\braillebox{1356}&\braillebox{15}&\braillebox{1345}&\braillebox{}&\braillebox{1236}\\
\hline
\end{tabular}

Četl desátý řádek skoro bez chyb, jedině byl částečně moje vína(protože jsem vedl jeho ruce na selektor). Já jsem táhl jeho ruku příliš daleko od `n' na `l' a museli jsme se vrátit. Kdy četl `l' podruhy četl to jako `b' ale opravil se hned než jsem stihl něco říct.

\paragraph{Jedenáctý řádek}
\begin{tabular}{|c|c|c|c|c|c|c|c|c|c|c|c|}
\hline
 &r&o&c&e& &1&9&1&3& &v\\
\braillebox{78}&\braillebox{1235}&\braillebox{135}&\braillebox{14}&\braillebox{15}&\braillebox{}&\braillebox{18}&\braillebox{248}&\braillebox{18}&\braillebox{148}&\braillebox{}&\braillebox{1236}\\
\hline
\end{tabular}

Jedenácti řádek obracel `r' jako `ř'.  Četl `o' nejdřív jako `e'. Kdy dočetl \uv{roce} zeptal jsem ho jestli pozná, které slovo to bylo ale nevěděl.  I potom co jsme se vrátili a on to četl znovu(s chybou, že nejdřív četl `e' jako `a') začal vykládat slovo jako \uv{ro} ale nepokračoval. Místo tomu, že bych ho zbytečně otrávil, jsem ho říkal, že to bylo slovo \uv{roce} a jsem ho říkal, že teď zřejmě bude číslo nebo datum.  Myslím, že jsem ho zmátl dost, protože nejdřív četl 1\braillebox{18} jako `a'.  Když jsem ho říkal, že to má byt číslo a, že bod osm je číselní bod stále to nepochopil.  Přemyslel o tom 4.5 vteřin a pak, řekl \uv{aha, jedna}. Devět četl nejdřív jako `i' a pak se zeptal jestli to je ještě číslo a správně doplnil, že to je devět. Četl `1' a `3' správně a jsem doplnil, že to je \uv{devatenátset třináct}.  Četl mezera a `v' správně.  Předtím, že řekl `v' říkal \uv{předpokládám, že tohle už není číslo}. Nevím jestli je to protože žádný číslo je v tvaru\braillebox{1236} anebo protože pochopil, že jsme dočetli datum.

\paragraph{Dvanáctý řádek}
\begin{tabular}{|c|c|c|c|c|c|c|c|c|c|c|c|}
\hline
 &A&n&g&l&i&i&.& & &J&m\\
\braillebox{78}&\braillebox{17}&\braillebox{1345}&\braillebox{1245}&\braillebox{123}&\braillebox{24}&\braillebox{24}&\braillebox{3}&\braillebox{}&\braillebox{}&\braillebox{2457}&\braillebox{134}\\
\hline
\end{tabular}

Dvanáctý řádek.  Zase jsem ho nechal číst samostatně. Už zvládl nepohybovat pravou ruku.  Přizpůsoboval tak, že tlačil hodně silně prstem, tak že jestli nějaký pohyb byl, pohyboval celý selektor. Četl `n' nejdřív jako `d' a měl velký problém s `g'\braillebox{1245} protože to četl jako `x'\braillebox{1346}.  Musel jsem mu říct ať pohybuje pravou ruku na selektor.  Četl `l' původně jako `b'.  Četl `j' jako `,' a `m' jako `c'.  Ještě mu pohyboval ruku na selektor ale snažil se, úspěšně to překonat.  U konce řádku myslel, že je mezera.

\paragraph{Řádek 13}
\begin{tabular}{|c|c|c|c|c|c|c|c|c|c|c|c|}
\hline
e&n&o&v&a&l& &s&e& &O&p\\
\braillebox{1578}&\braillebox{1345}&\braillebox{135}&\braillebox{1236}&\braillebox{1}&\braillebox{123}&\braillebox{}&\braillebox{234}&\braillebox{15}&\braillebox{}&\braillebox{1357}&\braillebox{1234}\\
\hline
\end{tabular}

První `o' četl nejdřív jako `e' a druhý `o' četl nejdřív jako `k', a četl `p' nejdřív jako `l'.

Hlavní je, že poslední dvě řádky používal selektor samostatně.

Když jsem se zeptal, jestli si mysli jestli by se dokázal naučit samostatně číst odpovídal \uv{no, snad ano} ale nijak nepřesvědčeně. Ještě když jsem se zeptal na jeho dojmy z toho, říkal:\em \uv{No, spíš tahle jsem trochu zmaten z toho, že posouvám tou ... tabulkou jestli řeknu takhle tou klávesnicí jako by tady tou pravou rukou že jako by spíš jak to funguje když dam každé to písmeno to znamená jeden ten jakoby posun.  Todle pro mně zkouška jak jsem občas myslel, že to je to písmeno zrcadlově obraceně, i když jsem fakt občas měl jsem pocit, nevím proč.}\em

