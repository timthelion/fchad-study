\chapter{Výzkum}

\section{Popis studie}

\subsection{První Část: Čtení Tisku}

Účastnici nebo možno lepe žáci v teto studiu nejdřív poseděli u prázdného stolu.  Nevěděli co se stane kromě tomu, že budou vyzkoušet nový druh braillského řádku.  FCHAD byl na stole ale položený do zády. Tak aby nepřekážel ruk.  Hned od začátku studie účastnici měli úkol.  Měli číst text(v apendixu 3.1).  Četli na hlas. Text požádá souhlas o shromáždění dat taky vysvětli co to je FCHAD a jak se to používá. Text byl tisknutí na tuhém kusu braillského papíru velký 26.5 x 30.5cm.  Maximální délka řádku je 36 písmen.  Braillovo písmo je šestibodové.  Text byl tisknutí na obě straně, a na jedné straně papíru je sloupec děr aby papíry mohly byt vázané.

Text byl tisknutí v Knihovně a tiskárna pro nevidomé K.E.Macana, která je jediný veřejný tiskařský závod pro nevidomé v Praze.  Všechní účastnici studie již měli zkušenosti s materiály tisknutí u tiskárny a předpokládám, že tiskové Braillovo písmo bylo velmi standardní.

Když účastník dočte text musel čekat par vteřinu abych připravil FCHAD. Pokud ještě nedali souhlas k zveřejnění informaci jsem požádal o souhlas.  Žádný z mých účastníků odmítly, a nikdo který jsem pozval k studie odmítl.

\subsection{Otázky}

Po čtení a přípravě jsem se zeptal účastníků následující otázky:

Jak jste starý?

Jak dlouho čtete braillské písmo?

Jaký je Váš nejvyšší dosažený úroveň vzdělání?

Jaký je Váš úroveň zrakové postižený?

Když jsem se je zeptal na věk, jsem vždy objasnil, že nepotřebuji přesný věk.  Věky uvedené v teto studie jsou přesné jenom do pěti let.

Zajímavě, všechní účastnici kromě jeden odpověděli "úplně nevidomý" když jsem se je zeptal na jejich úroveň zrakové postižení.

WHO rozdělí zrakové postižený na 5 kategorii. Nejnižší dvě jsou:

\em "Praktická slepota
zraková ostrost s nejlepší možnou korekcí 1/60 (0,02), 1/50 až světlocit nebo omezení zorného pole do 5 stupňů kolem centrální fixace, i když centrální ostrost není postižena, kategorie zrakového postižení 4

Úplná slepota ztráta zraku zahrnující stavy od naprosté ztráty světlocitu až po zachování světlocitu s chybnou světelnou projekcí, kategorie zrakového postižení 5" \em \citep{sonsklasifikace}

To, že nikdo nehlásil "Praktická slepota" nebo "praktická nevidomost" muže znamenat, že oni otázku nepochopili a mysleli si, že nemusejí odpovědět přesně.  Anebo to muže znamenat, že zkutečně byly všechní úplně nevidomé.

\subsection{Čtení na FCHADu}

Po otázek jsem přemístil FCHAD aby byl před účastnici.  Musel jsem taky vyzkoušet jestli to funguje správně. To byl první zkušenost pro účastnici s FCHADem.  Slyšeli jak cvakne ty body na zobrazovače.  Potom co jsem zjistil, že software je spouštění správně jsem řekl účastníkům aby začali číst.  Jenom, že samozřejmě neuměli.  Vidomé lidí když setkají s novým nastroje činní. Oni koukají!  Nevidomé lidí nekoukají.  Sedí a nic nedělají.  Koukají do vzdálenosti bez pochybu.  Myslím si, že se boji hrabat na stroje, by ho mohli rozbit.  Není to nerozumná obava.

Musel jsem něco dělat aby svůj účastnici se snažili číst. Dal jsem jejích ruce na stroje tak jak by měly byt k správné čtení.

Vzal jsem je levou rukou, a jsem dal tu levou ruku na zobrazovač(viz Obrázek 1-A). Řekl jsem je \uv{tady se zobrazí to jedno písmeno.}  Pak jsem je vzal pravou rukou a dal jsem ji na selektor(vis Obrázek 1-B,C).

Teď je dobrý čas vysvětlit jak přesně vypadá a funguje FCHADy, které byli použití v studie.

\section{Popis FCHADu}

\subsection{První selektor}

První účastnici(2 a 3) používali jiný selektor než poslední.  To je kvůli technickým závadám v prvním selektoru.   Technologie selektorů je velmi jednoduchý.  Selektor má několik elektrod. Jedena katoda a vice anod. Když účastník tělem spojuje anod s katodou, elektrický proud jde přes tělo z anody k katodě. Podle toho, přes kterou anodou teče proud víme které písmeno je vybrané(selektováné).   První selektor měl 12 anod z napínáčků(vis Obrázek 1-C), a katoda v náramku(Obrázek 3).

Napínáčky jsou nepravidelně umístěné mezi 1.2 a 2 cm od sebe.  Celý selektor je 19cm od prvního senzoru k poslednímu.

Tento systém moc dobře nefungoval. Když jsem to zkusil doma fungoval to bezvadně ale když to pokusil nějaký chlupatější muž náramní katoda nedostala dobrý kontakt a stroj nevěděl kdy je dotyk a kdy ne.

\subsection{Druhý selektor}

Účastníci 4-8 měli lepší selektor(Vis obrázek 1-B). Nový selektor má dvě řádky elektrod. Horní řádek jsou anody a dolní řádek jsou katody.  Už proud nemusí projet tělem ale jenom kůži na prstu.  Tím způsobem, je daleko lepší spoj mezi anodou a katodou.  To ještě nebyl dostatečný.  Jsem musel namočit prsty účastníků vodou aby dostali spolehlivý spoj.

Další vylepšení je v tom, že senzory jsou v pravidelné umístění 0.5cm od sebe.  Jsou taky daleko blíž k sobě.

\subsection{Zobrazovač}
Další součást FCHADu je zobrazovač(vis Obrazy 1-A a 2).  Zobrazovač, který jsem použil v studie zobrazuje jedno osmibodové písmeno.  Zobrazovací technologie je zase velmi jednoduchá.  Jde to o cívka, která vybaví tyč aby pohyboval nahoru.  Taková cívka se říká solenoid.  Moje solenoidy vydají až 200 gramů síly\citep{multicomp}.  To je dost aby jejích pohyb bolelo.  Každý bod je jenom 0.45cm vysoký. Neboli to když nedržíte ruku na zobrazovače příliš silně.  Body nespadnou samy.  Čtenář musí položit ruku na zobrazovače aby správně fungoval.

Řádky bodů jsou 2cm od sebe a sloupci 3cm od sebe(pro účastnici 2 a 3, byly 5cm od sebe).  Celý zobrazovací prostor je 6 x 3cm.  Moje tři delší prsti jsou mezi 7 a 8.5cm dlouhé a proto dokážu číst písmena na zobrazovače prsty.  Některé účastnici zejména ženské měli kratší prsty a to nešlo.  Plastový báze Zobrazovače je 5.9cm široký(8cm u dřívější účastníci) 9.2cm dlouhý 6cm vysoký.

\subsection{Jak to jde dohromady}

FCHAD je přípojený na počítač a počítač má v pamětí tabulku 12ti písmen.  Písmena v tabulce vyplní speciální software, která se jmenuje \em čtečka obrazovky\em .  V studie používal jsem čtečku obrazovky Orca 3.6.3. Orca vyplní tabulku podle toho jaký text je blízko kursoru na obrazovce.

Selektorem čtenář vybírá, které z těch dvanácti písmen bude zobrazené na zobrazovače. Když pohybuje ruku na selektor čtenář slyší hlasitý cvaknutí jak solenoidy změní polohu.

Navrhoval jsem celou elektroniku stroje. Programoval jsem ovladač a firmware.  To mě dal možnost nahrávat fungování celý stroj.  Když jsem dělal výzkum, jsem mohl nahrávat lokace prstů uživatele, rychlost čtení, polohu ruky.  Napsal jsem další software, který pomáhá v shromáždění dat\footnote{\url{https://github.com/timthelion/anonGraph}}.  Mužete prohlidnout shromaždění data zde\footnote{\url{https://github.com/timthelion/fchad-study}}.

Přesnější detaily nastroje najdete v apendix A1.

\subsection{Text na FCHAD}
Tiskový text je na témě FCHADech(je to návod) a protože jsem chtěl mít srovnatelný obsah taky na nastroje jsem napsal krátký dějinný FCHADu.

Celý text, který jsem měl připravený najdete v apendixu.

\section{Moje žáci}

Účastnici studie byly všechní starší 18, tak že mohli dávat souhlas samostatně.  Účastnici nebyli zaplacené.

\subsection{Předběžná studie}

\subsubsection{Žáci 1 a 2}

První "žák" o čem budu psát jsem Já Váš badatel.  Já jsem se učil číst rukama před dvěma roky. Zajímal jsem o Braillovo písmo protože mívám migrény když čtu.  Nějakou dobu jsem si řekl: "naučím číst Braillovo písmo a už nebudu vůbec číst očima".  Bylo to spíš emoční reakce na frustrace než praktický plán.  Rychle jsem se dozvěděl, že Braillovo písmo vůbec nejde číst rychle a, že braillský řádek je daleko příliš drahý koupit jenom vztekem.  Můj zájem o Braillova písma se změnil na hněv o tom jak je všechno pro nevidomé předražený a snahu stvořit něco levnější.

Umím číst rukama ale nijak plynule.  Už čtu daleko lepe FCHADem než obyčejný tisknutým Braillským písmem.  Je lehčí číst FCHADem protože písmo je velké.

Když jsem se začal číst na FCHAD četl jsem velmi pomalu.  Ještě jsem neznal Braillova písma dokonalý a často jsem spletl f \braillebox{124}, d \braillebox{145}, h \braillebox{125}, a j \braillebox{245}. Teď je už zvládnu bez problému.  Dozvěděl jsem, že připojený mezi "psychologický rychlost čtení" a "skuteční rychlost čtení" není moc silné.  Čtení se zda rychlejší, tím miň dělám chyby a tím víc rozumím text ale to neznamená, že zkutečně čtu rychlejší.  Čtení zrychluje daleko pomalejší než psychologický rychlost čtení.  Často jsem si myslel, že jsem dělal pokrok v svému schopnosti čtení ale když jsem koukal do nahrávky jsem dozvěděl, že rychlost či nezměnil, anebo jen mírně stoupal.

Pro mně, použity selektor je skoro bez myšlenek ale poznání znaků na zobrazovače stále někdy trvá.  Nastavěl jsem stroj tak, že se používá selektor pravou rukou a zobrazovač se čte levou. Dělal jsem to tak protože selektor je podobný jako počítačový myš a myš se používá pravou rukou(pro praváci).  Myslím si, že jsem to nastavěl obraceně a byl by lepší používat zobrazovač dominantní rukou.  Nastavený byl stejný pro všechní účastnici studie.

Dával jsem "žáci 1 a 2" dohromady.  To je protože můj FCHAD vůbec nefungoval když jsem ho ukázal prvnímu účastníkovi. Nevím jakým kozlem to bylo ale když jsem držel Já katodu v ruce a používal stroj selektor fungoval bezvadně. Když on to vzal do ruky začal celý stroj zbláznit.


\subsubsection{Žák 3}

Setkání 6.3.2013

\uv{Žák} 3 je vysokoškolské vzdělaná žena ve věku mezi 45 a 50 let. Čte Braillovo písmo od začátku školní docházky.  Pracuje jako vysokoškolskou profesorkou němčiny.  Je úplně nevidomá.

Navštívil jsem ji během její konzultační hodiny což byly odpoledne hned po obědě.  Výzkum byl zpoždění tím, že jsme museli hledat prodlužovací kabel. Ona neměla zásuvku blízko pracovního stolu.  Ona se mě zdála čilá a zdravá.

Nehrabala s nástrojem když jsem to nastavil. Během naše setkání FCHAD trpěl technické problémy. Senzory někdy \uv{cítily} dotyky, které v zkutečnosti nebyly.  Zajímavě i když FCHAD moc dobře nefungoval ona byla schopná na to číst.

Začali jsme s tiskovém návodem:

Četla tiskový text obě ruce.

Čas na čtení celý tisknutý text: 93.3 vteřin (časy zaměřené ručně z nahrávek)

Počet znaků: 407

Rychlost čtení tisku: 4.36 CPS (počet znaků za sekundu)

Čas na čtení tisknutý text po otočený stranu: 51.8 vteřin

Počet znaků: 265

Rychlost čtení: 5.11 CPS

Ona jistě nebyla ve formě pro čtení tiskové Braillova písma.  Není divů kdy má doma braillský řádek.  V tiskovém Braillovu písmu řetězci velkých písmen jsou naznačen speciálním znakem: \braillebox{23} tak, že \uv{FCHAD} je psáno jako \braillebox{23}\braillebox{124}\braillebox{14}\braillebox{1}\braillebox{145}. Žák 3 snad nevnímala znak na řetězec velkých písmen.  Začala číst řetězec po jednotlivých písmen a pak říkala \uv{to budou asi čísla} a četla je jako čísla.  Dále, kdy dostala na konec stránky se chovala nějak překvapená.

Kdy se začala číst katoda nebyla v dobrém kontaktu s rukou.  Musel jsem to upravit.  Četla správně ale pomalu. Používala sevření prstu aby cítila každého tyče zvlášť.  Jelikož to funguje pro poznání písmena nejde tím způsobem číst rychle. Snažil jsem ji naučit používat zobrazovač správně.  Když jsem ji říkal, že by měla položit ruku rovně a dal jsem její ruku na zobrazovače správně říkala \uv{Já nevím, jak ta ruka jen tak leží mně to nevyhovuje, Já se musím ty body zvyknout a osahat trochu jinak.} %10:02.6

Položila ruku na selektor příliš daleko od sebe. Měla mít ruku napínači ale dala ruce na dráty, které vedou z nich.  Necítila správně kde jsou kraje mezi napínači a proto velmi těžko dozvěděla vzdálenost potřební posun mezi písmenem.  Nepochopila hned, že anody jsou dotykové a, že silnější zmačknutí nemá na ně vliv.  To je ale pochopitelní protože často nefungovaly. Snad by pochopila stroj lip kdybych mluvil líp česky.  Opakovaně jsem dělal chyby. Například jsem spletl slova \uv{písmo} a \uv{písmeno}.

Občas dělala chybu, že hmatala jenom na první sloupce bodů kdy četla písmena ze zobrazovače. Například, když bylo zobrazené \braillebox{24} myslela, že se zobrazuje jenom \braillebox{2}.

Protože anody jsou dotykové stačí když okraj prstů je v kontaktu s anodou aby se zaregistroval dotyk.  Zobrazuje se písmeno od nejlevější anody, která je dotýkaná.  Ona měla velké potíže, tím že si myslela že posunula ruku na selektor, a že by mělo zobrazit další písmeno ale protože její prst ještě byl v kontaktu s předcházející písmenem zobrazené písmeno zkutečně se nezměnilo.

\paragraph{ První řádek:}

\begin{tabular}{|c|c|c|c|c|c|c|c|c|c|c|c|}
\hline
B&r&a&i&l&l&s&&&&&\\
\braillebox{1278}&\braillebox{1235}&\braillebox{1}&\braillebox{24}&\braillebox{123}&\braillebox{123}&\braillebox{234}&\braillebox{}&\braillebox{2358}&\braillebox{123}&\braillebox{}&\braillebox{}\\
\hline
\end{tabular}

První řádek textu, který účastnicí četli na FCHADU byl kratší než selektor.  To byl určitě chyba na moje straně.  Orca se zobrazuje znaků \braillebox{2358}\braillebox{123} u konce řádků.  To znamená, že účastnici dočetli \uv{Braills} a pak najedeno byli konfrontovaný s tím, že jsou dvě znaky, které nepoznali.  Jsem jím říkal ať je nečtou a, že by měli pokračovat na další řádce.

FCHAD, který byl použitý v studie nemá žádné navigační tlačítka.  Aby účastníci pokračovali na další řádce museli jenom přesunout ruku k začátku řádku a pokračovat. Ovládal jsem změna řádku často bez jejích vědomi.

\paragraph{Druhý řádek:}

\begin{tabular}{|c|c|c|c|c|c|c|c|c|c|c|c|}
\hline
k&ý& &ř&á&d&e&k&,& &k&t\\
\braillebox{1378}&\braillebox{12346}&\braillebox{}&\braillebox{2456}&\braillebox{16}&\braillebox{145}&\braillebox{15}&\braillebox{13}&\braillebox{2}&\braillebox{}&\braillebox{13}&\braillebox{2345}\\
\hline
\end{tabular}


Další nezvyklost Orcy je, že pod čaruje začátky řádků. Například kdy 'k' je u začátku řádku, je psáno jako \braillebox{1378} a ne \braillebox{13}.  Do jisté míru to účastnici zmátlo.

Další problém pro žáka 5 byl kdy četla 'ý' \braillebox{12346}. Nesevřela na bodu 6 a proto to původně četla jako 'p' \braillebox{1234}.  Kdy jsem ji říkal, že četla nesprávně a že ji chyby ještě jeden bod, ona to našla, ale zase nesprávně odhadla, že písmeno bude 'q' \braillebox{12345} ale rychle sama upravila.

Zajímavě, kdy dorazila na mezera \braillebox{} říkala \uv{tady nemám nic}. Nebylo mi jasný, že pochopila to jako mezera než jsem ji to říkal.

Ona věděla, že jsem Američan.  Kdy dorazila na \braillebox{1235} myslela, že to je Americký 'w' a ne český 'ř'.  Snad nepochopila, že text zobrazený na FCHAD je český.

Kdy dorazila na 'á'\braillebox{16} říkala nahlas \uv{to je první a šesti bod, to je a s čárkou}.  Znamená to, že ona nemohla přímo z tvarů znaků pochopit vyznám. Musela explicitně o tom přemyslet.  Taky to znamená, že ji to pomohla přemyslet o čísla bodů, a že měla v explicitní paměti hned k dispozicí informace o tom, které body má 'á'.

Pokračovala tím, že volala nejdřív čísla bodů a potom, které písmeno to má byt.

Kdy dostala na ','\braillebox{2}, správně poznala, který bod je nahoru, ale nemohla říct, který znak to je i když čárka je úplně obyčejný a standardní znak v Braillovu písmu.  Když jsem ji říkal, že to je čárka hned pochopila.  Musel byt někdy v mozku tomu, že to má byt čárka, ale nějak ona to nespojila.

\paragraph{Třetí řádek}

\begin{tabular}{|c|c|c|c|c|c|c|c|c|c|c|c|}
\hline
e&r&ý& &t&e&ď& &p&o&u&ž\\
\braillebox{1578}&\braillebox{1235}&\braillebox{12346}&\braillebox{}&\braillebox{2345}&\braillebox{15}&\braillebox{1456}&\braillebox{}&\braillebox{1234}&\braillebox{135}&\braillebox{136}&\braillebox{2346}\\
\hline
\end{tabular}

Kdy se vrátila na začátek selektoru potom co četla druhý řádek textu vrátila se jenom asi polovinu tak daleko jak musela.  Byl jsem překvapení z toho protože selektor má jasně citelný okraj a jsem si myslel, že bude řídit podle citu a ne odhadem.  Je možný, že nechtěla dotykat na dotykové senzory kdy nečetla. Jinak nevidím důvod proč nepřejela prstem na začátku.

Na třetí řádce četla 'e', tím že pojmenovala na hlas čísla bodů ale kdy dostal na 'r' stroj začal rychle vypnout a zapnout(protože její prst neměl dobrý kontakt s senzorem).  Říkala 'r' bez toho, že by nějak hledala body!  Tak myslím si, že vibrace ji pomohlo víc než zmátl.

Je to zajímavý z pedagogické hlediska, protože jsem sám lekl tím, že stroj nefunguje a jsem se snažil ji vysvětlit, že prst není správně položení na senzor(a proto to tak vibruje).  Měl jsem ji jen nechat číst, protože četla správně.

Podruhy kdy četla 'ý' dělala stejnou chybu jako poprvé.

Třetí 'e' v slovu \uv{teď} už četla bez toho, že by říkala nahlas čísla bodů.  Znamená to, že třetím opakování už vidíme autonomizovace konkretní schopnosti?

Kdy četla 'ď'\braillebox{1456} říkala, že to používá jako lomítko(normálně \braillebox{12456}).  Uznala, že v standardní Braillovo písmo zobrazuje se 'ď'.

\paragraph{Čtvrtý řádek}

\begin{tabular}{|c|c|c|c|c|c|c|c|c|c|c|c|}
\hline
í&v&á&t&e&,& &n&e&n&í& \\
\braillebox{3478}&\braillebox{1236}&\braillebox{16}&\braillebox{2345}&\braillebox{15}&\braillebox{2}&\braillebox{}&\braillebox{1345}&\braillebox{15}&\braillebox{2345}&\braillebox{34}&\braillebox{}\\
\hline
\end{tabular}

Kdy se vrátila na začátku selektoru u čtvrtý řádek říkala, že moc na to nezorientuje.  Zase posunula krátce ale jsem ji neopravil.

I když dřív četla 'e' bez problému, zase zapomněla sevřít na druhém řádku bodů a četla 'e'\braillebox{15} jako 'a'\braillebox{1}.

Kdy pokračovala přesunula na selektor až moc(až na 'n') a jsem ji dal za úkol jít zpátky a hledat 't'.  Kdy četla písmena, které už četla dříve, četla je bez počítání bodů a rychle.

Kdy se vrátila na 'n'\braillebox{1345} nejdřív odhadla to jako 'd'\braillebox{145}. Četla 'e' správně, ale kdy dostala zase na další 'n' četla to jako 'm'\braillebox{134}.

Teď byla u předposledního písmena řádku 'í' a dělala chybu, že si myslela že už je konec řádku.  To je protože střed jejího prstu byl na poslední napínače ale okraj jejího prstu dotekl ještě jinde.

Je už obecně známý, že neopravené dotykové ovládání je nepřirozený pro lidí. Je normálně rozdíl mezi tím kde lidí si myslí že dotykají a kde kontakt mezí prstem a senzor zkutečně je.  Dotykové obrazovky používají speciální algoritmy odhadnout kde uživatel chtěl dotykat a psychologie dotyku je aktuální téma bádaní.

I když algoritmy jsou dostatečně rozvinuté, že používáte smartphony bez větší úsilí většina algoritmy jsou na základě zrakové vnímání dotyků.  Dotyk na obrazovce je vždy po nějaké oblasti ale počítač reaguje na dotyk jenom v jediném bodu.  Počítač musí odhadnout, který bod uživatel chtěl dotykat\citep{holz2011understanding}. Rozlišení selektor FCHADu je velmi málo na rozdílu od rozlišení dotykové obrazovky a psychologie dotyku u nevidomých je samozřejmě na základě hmatu a ne zraku.  FCHAD neví jestli je prst v plném kontaktu s senzorem anebo je senzor dotykován jenom z okraji.  Dále, vnímání dotyků uživatele FCHADu je hmatní a ne zrakový.  Nemůžeme bohužel jen kopírovat stavějící algoritmy.

\paragraph{paty řádek}

\begin{tabular}{|c|c|c|c|c|c|c|c|c|c|c|c|}
\hline
p&r&v&n&í& &ř&á&d&e&k&,\\
\braillebox{123478}&\braillebox{1235}&\braillebox{1236}&\braillebox{1345}&\braillebox{34}&\braillebox{}&\braillebox{1235}&\braillebox{16}&\braillebox{145}&\braillebox{15}&\braillebox{13}&\braillebox{2}\\
\hline
\end{tabular}

Na paty řádku myslela, že 'v'\braillebox{1236} je 'r'\braillebox{1235} protože si myslela, že paty bod je nahoru a ne šesti.

Stále neuměla přesunout ruku na selektor a stále ji přesunula příliš daleko do práva jak četla, tak že přeskočila několik písmem na raz.

Během čtení pátého řádku přestal stroj fungovat a musel jsem restartovat ovládací software.  Třetí účastník byla extrémně trpělivá a bych ji chtěl poděkovat za to!

Kdy jsem stroj upravil začala číst paty řádek zase od začátku. Opakování část řádku četla rychle a pokračovala v rychle čtení až dorazila na slovo \uv{řádek}.  Přečetla začátek slovo správně, ale četla 'e'\braillebox{15} jako 'á'\braillebox{16}.

Po paty řádku dokončili jsme setkání. Můj účastník ještě chtěla sdílet své dojmy.  O FCHADu říkala:

%330
\em \uv{Já si myslím, že ... Taklhe Vám řeknu, že jistě to víte sám, že ten řádek běžní je určitě komfortnější ale určitě než mít žádný řádek je lepší tenhle. Já si myslím, že na ten zobrazování těch písmen bych se zvykla myslím, že asi by to bylo dobré, kdyby každý uživatel by mohl trochu vybrat velikost, a například vy tady to máte rozděleno tou látkou a třeba osobně bych chtěla mít ty dva sloupce bodů naopak blíž. Protože Já mám tu ruku menší a užší. Teda jako za svou osobu, hlavní problém je v tom posunu jo, protože tam nějak ty aktivní body mě se zatím velice špatně hledají za svojí osobu jako veliko překážkou toho užívání bych především bych viděla v hledání těch písmena} \em

\subsubsection{Žák 4}

Setkání 8.3.2013

Čtvrtý žák je 20-25 let starý.  Jeho nejvyšší úroveň dosažené vzdělávání je maturita.  Je můj spolužák tady na katedře angličtiny. Je úplně nevidomé a taky trpí problémy se sluchem.  Pracuje s Braillovým písmem od základní školy. Používá braillský řádek doma na počítače.

Měl velké potíže s přesunem na selektor.

Když jsem ho dával číst úvodní papír on se mně zeptal jestli to má číst nahlas.  Říkal jsem ne, protože první část papíru se zkutečně nahlas číst nemusí.  To byla chyba.  Čekal jsem, že až dorazí na větu \uv{Následující text čtěte prosím nahlas a bez přestání až do konce.} začíná číst nahlas, ale četl dal po tichu.  Z toho důvodu mám jeho rychlost čtení tiskové Braillovo písmo jenom z druhý straně papíru.  Je to obecně zajímavý, že moje účastnicí nereagovali na psáné příkazy.  Žák 4 dal souhlas bez ústního zeptání ale některé z nich nepochopili z věty \uv{Než budeme pokračovat, potřebuji Váš souhlas se shromážděním a zveřejněním těchto dat.}, že mají nějak mě aktivně říct že s tím souhlasili.

Četl tiskový text obě ruce.

Čas na čtení celého textu:43 vteřin %169.2-126.1

CPS: 9.46*

Čas na čtení druhé strany tiskového textu:25.9 vteřin %169.2-143.3

CPS: 10.23**

*) přečetl to nahlas ale potom co to četl potichu

**) Přečetl to jenom nahlas.

Potom co jsem se zeptal otázky, přemístil jsem FCHAD před účastníkem a říkal jsem mu ať začíná prozkoumat stroj.  On seděl a nic nedělal.  Třeba jsem mu měl nechat díl. Kdy jsem tam seděl s něj měl jsem pocit, že sedí strašně dlouho ale na zvukové nahrávce jsem čekal jenom deset vteřin.

Protože žák 3 měla tolik problému s selektorem rozhodl jsem se vysvětlit čtvrtému žákovi selektor líp.  Dal jsem jeho ruku na selektor a táhl jsem jeho prsty nad senzory. Jak jeho prsty dotekly se senzory zobrazovač vydal zvuk cvaknutí jak se změnilo písmeno.  Říkal jsem ho, že ten zvuk je zvuk změna písmen.  Jenom až potom co jsem mu vysvětlil selektor jsem dal jeho levou ruku na zobrazovače.

Věnoval jsem větší čas vysvětlení fungování nastroje než u žáka 3. Kdy jsem ho ukazoval zobrazovače zobrazilo se písmeno 'o' a jsem ho říkal, které písmeno to je.

Ještě během ukázku přestala stroj reagovat.  Naše setkání bylo usekaná mnoho krát technickými potíží.  Protože měl chlupatou kůži, katoda neměla dobrá kontakt.  Proto, body strašně zatřásly.

\paragraph{První řádek}

\begin{tabular}{|c|c|c|c|c|c|c|c|c|c|c|c|}
\hline
B&r&a&i&l&l&s&&&&&\\
\braillebox{1278}&\braillebox{1235}&\braillebox{1}&\braillebox{24}&\braillebox{123}&\braillebox{123}&\braillebox{234}&\braillebox{}&\braillebox{2358}&\braillebox{123}&\braillebox{}&\braillebox{}\\
\hline
\end{tabular}

Konečně jsme se začali číst.  Žák 4 nepoužíval způsob pojmenování bodů kdy četl.  Odhadl nejdřív, že první písmeno je 'l'\braillebox{123} a kdy jsem ho říkal že to není opravil sebe, že je to 'b'.

Kdy pokračoval v čtení stále katoda neměla dobrý spoj.  Protože jsem věděl, že budou problémy s katodou jsem ještě měl další katoda(ve formě měděnou samolepicí paskou) nalepený na zobrazovače. Snažil jsem se mu vysvětlit, že bude fungovat lip kdy je v kontaktu s druhou katodou.

Přečetl 'r' správně a zase začal zbláznit stroj.  Vysvětlil jsem mu ten problém s kontaktem s katodou a poprosil jsem mu aby držel katodu v ruce.  Pak četl další dvě písmena správně a zase nefungoval stroj.

Po par pokusu resetnutí stroje konečně už to fungoval až ke konci setkání.  Četl od začátku prvního řádku a četl první tři písmena správně. Přeskočil jedno písmeno na selektor a jsem mu říkal ať se vrátí zpět.  Četl 'i'\braillebox{24} jako 'c'\braillebox{14}. Když jsem ho říkal, že četl nesprávně používal stejný způsob uvažování, které používala žák 3 celou dobu, říkal na hlas, které body jsou nahoru \uv{to je bod jedna čtyři} říkal. \uv{Není to bod jedna} jsem mu odpověděl. Potom už se upravil správně sám.

Na rozdíl od žáka 3 pochopil hned, že čte české slova.  Kdy viděl, že další písmeno 'l' říkal \uv{Brail}. Četl ještě 'l' a 's' správně a doplnil \uv{Braills}.

Jsem ho říkal ať se vrátí na začátek textu radši než, že bych mu vysvětlil jaké znaky používá Orca označit konec řádku, v retrospekce myslím si že to byla chyba.

\paragraph{Druhý řádek}

\begin{tabular}{|c|c|c|c|c|c|c|c|c|c|c|c|}
\hline
k&ý& &ř&á&d&e&k&,& &k&t\\
\braillebox{1378}&\braillebox{12346}&\braillebox{}&\braillebox{2456}&\braillebox{16}&\braillebox{145}&\braillebox{15}&\braillebox{13}&\braillebox{2}&\braillebox{}&\braillebox{13}&\braillebox{2345}\\
\hline
\end{tabular}

Kdy začal číst druhý řádek dostal na 's' a 'ý' a doplnil, že dočetl \uv{Braillský} pak \uv{četl} dal. Domyslel, že se zobrazí písmeno je 'ř'.  Ještě ale nepřesunul ruku na selektor.  Říkal jsem mu, že stále ze zobrazuje 'ý'.  Přesunul ruku na selektor až k 'ř' a říkal, \uv{aha, tak tohle je opravdu 'ř'}.  Přečetl slovo \uv{řádek} skoro plynule, kromě tomu, že měl problém s přesunem na selektor.

\paragraph{Třetí řádek}

\begin{tabular}{|c|c|c|c|c|c|c|c|c|c|c|c|}
\hline
e&r&ý& &t&e&ď& &p&o&u&ž\\
\braillebox{1578}&\braillebox{1235}&\braillebox{12346}&\braillebox{}&\braillebox{2345}&\braillebox{15}&\braillebox{1456}&\braillebox{}&\braillebox{1234}&\braillebox{135}&\braillebox{136}&\braillebox{2346}\\
\hline
\end{tabular}

Vrátil se k začátku selektoru bez pomoci, ale první písmeno 'e' četl jako 'á'\braillebox{16}, pojmenoval čísla bodů jako jedna a šest a jsem ho říkal, že šesti bod není nahoru.  Hmatal prstem na díry na zobrazovače a počítal, který bod zkutečně nahoru je.  Pak se upravil sám.

Kdy četl další slovo \uv{teď}, četl 't' správně ale zase původně četl 'e' jako 'á'.  Měl taky problém s písmenem 'ď'\braillebox{1456}, který původně četl jako 'č'\braillebox{146}.  Neříkal nahlas ani \uv{který} ani \uv{teď} po jejich čtení.  Četl 'p' správně, měl trochu problém s posunem na selektor ale sám našel 'o'.  Myslel si, že je u konce řádku.  Říkal jsem mu, že není. Přečetl 'u' a pak už opravdu věřil, že je u konce řádku. Říkal jsem mu, že mu zbývá ještě jedno písmeno.  Snažil se pokračovat. Písmeno na zobrazovače začal vibrovat chaotickým způsobem. Říkal jsem mu, že musím zkusit jestli stroj teď funguje správně. Zkoušel jsem stroj a fungoval. Říkal jsem mu ať pokračuje. Četl 'ž' jako 't'. Říkal jsem, že to není správní. Opravil se sám.

Dokončili jsme u toho setkání.

Kdy jsme dokončili setkání ještě jsme mluví o FCHADech.  Nabídl další spolupráce a říkal, že \uv{kdyby to bylo opravdu levná alternativa, tak by to bylo opravdu super}. Říkal mně, že \uv{by to bylo lepší kdyby ty body byly blíž k sobě}.

\subsection{Běžná studie}

Po předběžné studie jsem věnoval dva intenzivních víkendů vylepšení nastroj a odstranění softwarové chyby.  Předělal jsem celý selektor a přenastavil jsem zobrazovač na nejužší nastavení.

Běžná studie se proběhla v gymnasium pro zrakově postižené a střední odborná škola pro zrakově postižení v Praze odpoledne 20.3.2013 a odpoledne 23.4.2013.

Účastnici posadili v učebně zeměpisu(která pro ně bylo známé místo).  Stole kde seděli je zobrazený v Obrazce 4.

{\bf Poznámka:}  U první tři účastníci běžného studie(Žáci 5,6, a 7) nebyl správně nastavení mikrofon a audio nahrávky z setkání jsou nízko-kvalitně.  Proto, popis setkání může obsahovat chyby.

\subsubsection{Žák 5}
Setkání 12:00 20.3.2013

Paty žák je 20-25 let starý.  Jeho nejvyšší úroveň dosažené vzdělávání je maturita. Je úplně nevidomý.  Taký čte Braillova písma od základní školy.  Používá braillský řádek.

Kdy začal číst, zase selektor zle choval.  Musel jsem přenastavit prahy.  To mně trval asi deset minut.  To znamenalo, že potom co on četl o stroje, a potom co on teprve dal ruce na stroje, on seděl a čekal.

Kdy četl tisknutí text nejdřív se zeptal, jestli má číst z obou stran.  Kdy četl, otočil stranu velmi rychle, tak že celou dobu plynule četl bez jakékoliv přestání.  Přečetl jedno slovo nesprávně: četl \uv{nejlevnější} jako \uv{nejlepší}.

Četl tiskový text obě ruce.

% začatek 176.8 otočení 192.9 konec 212.6

Čas na čtení celého textu:35.8 vteřin

CPS: 11.37

Čas na čtení druhé strany tiskového textu:19.7 vteřin

CPS: 13.45

\subsubsection{Žák 6}
Setkání 12:45 20.3.2013

Žák 6 je mezi 15-20 let stará. Čte Braillovo písmo od začátku školní docházky. Používá braillský řádek. Její nejvyšší dosažený úroveň vzdělávání je základní škola.  Je úplně nevidomá.

Četla tiskový text obě ruce.

% start 133.5 otočení 159.5 konec 193.9

Čas na čtení celého textu:60.4 vteřin

CPS: 6.7

Čas na čtení druhé strany tiskového textu: 34.4 vteřin

CPS: 7.7

\subsubsection{Žák 7}
Setkání 13:40 20.3.2013

Žák 7 je mezi 20-25 let starý.  Čte Braillovo písmo od začátku školní docházky.  Jeho nejvyšší dosažení úroveň vzdělávání je základní škola.

Četl tiskový text obě ruce.

% začátek 85.2 otočení 106.7 konec 134.6

Čas na čtení celého textu: 49.4 vteřin

CPS: 8.24

Čas na čtení druhé strany tiskového textu: 27.9 vteřin

CPS: 9.5

Ukázal jsem ho selektor a pak zobrazovač stručně bez delší popsání.

\paragraph{První řádek}
\begin{tabular}{|c|c|c|c|c|c|c|c|c|c|c|c|}
\hline
B&r&a&i&l&l&s&&&&&\\
\braillebox{1278}&\braillebox{1235}&\braillebox{1}&\braillebox{24}&\braillebox{123}&\braillebox{123}&\braillebox{234}&\braillebox{}&\braillebox{2358}&\braillebox{123}&\braillebox{}&\braillebox{}\\
\hline
\end{tabular}

Říkal jsem ať čte první \uv{písmo} a on říkal, že tam nic není.  Říkal jsem, že musí mít ruku na selektor a on tam ji dal ale ne na první senzor jen náhodně.  Začal hádat `a' ale jsem ho zastavil ať začíná na první písmeno.  Odhadl nejdřív jako `ě'\braillebox{126} ale podruhy odhadl správně.

Nehledal body jen položil ruku na selektor a říkal co si myslí.  Často odstranil ruku ze zobrazovače a pohyboval hlavu v stereotypickým pohybem tak aby tvař koukal nahoru a doleva. Hlavu pohyboval takhle po skoro každém písmenu ale pohyby nebyly formou rozptýlení. Samostatně se vrátil ke čtení potom co takhle pohyboval jeden až tři krát.

Druhý písmeno odhadl nejdřív jako `h' ale odhadl správně podruhy.  Četl `a' správně, a `i' četl na druhý pokus(první říkal čárka ale ani nedokončil slovo než samostatně se upravil).  Četl `l' správně, přeskočil další `l'. Myslím si, že používal zvuk cvaknutí od zobrazovače aby věděl o tom, kdy se změnil písmeno, ale protože nic se neměnil mezi `l' a `l' k žádnému cvaknutí nedošlo.  Neopravil jsem ho.  Četl `s' správně.  Říkal jsem mu, ať se vrátí na začátek selektoru abychom četli další řádek.

\paragraph{Druhý řádek}
\begin{tabular}{|c|c|c|c|c|c|c|c|c|c|c|c|}
\hline
k&ý& &ř&á&d&e&k&,& &k&t\\
\braillebox{1378}&\braillebox{12346}&\braillebox{}&\braillebox{2456}&\braillebox{16}&\braillebox{145}&\braillebox{15}&\braillebox{13}&\braillebox{2}&\braillebox{}&\braillebox{13}&\braillebox{2345}\\
\hline
\end{tabular}

Druhý řádek četl bez chyb kromě tomu, že jedno se vrátil na začátek selektorů.  Dělal jsem dost važnou chybu.  Upravil jsem ho nesprávně kvůli neznalost češtiny. Nerozuměl jsem ho kdy mě říkal `ř' jako `eř' a říkal jsem ho, že četl nesprávně.  Nikdy nevyslovoval mezery nahlas(ne, že bych to přikazoval).

\paragraph{Třetí řádek}
\begin{tabular}{|c|c|c|c|c|c|c|c|c|c|c|c|}
\hline
e&r&ý& &t&e&ď& &p&o&u&ž\\
\braillebox{1578}&\braillebox{1235}&\braillebox{12346}&\braillebox{}&\braillebox{2345}&\braillebox{15}&\braillebox{1456}&\braillebox{}&\braillebox{1234}&\braillebox{135}&\braillebox{136}&\braillebox{2346}\\
\hline
\end{tabular}

Četl `e' jako `o'.  To byl protože nespadnou body samou. Samostatně to pochopil a opravil se.  Potom co se přečetl `p', přeskočil několik písmen a četl správně `ž'. Kdy se začal číst znovu od `p', četl `o' nejdřív jako `k' ale opravil se než jsem ho říkal, že to není správný.

\paragraph{Čtvrtý řádek}
\begin{tabular}{|c|c|c|c|c|c|c|c|c|c|c|c|}
\hline
í&v&á&t&e&,& &n&e&n&í& \\
\braillebox{3478}&\braillebox{1236}&\braillebox{16}&\braillebox{2345}&\braillebox{15}&\braillebox{2}&\braillebox{}&\braillebox{1345}&\braillebox{15}&\braillebox{2345}&\braillebox{34}&\braillebox{}\\
\hline
\end{tabular}

První dvě písmena četl správně. Četl `á' jako `u'\braillebox{136}.  Nějak jsem se bal, že jeho chyba může byt kvůli vadný spojení v zobrazovače ale zjistil jsem, že ne.  Kdy se vrátil na začátek řádku ještě četl `v' jako `u'. Zase četl `á' jako `u' ale konečně správně říkal `á'.  Četl `e' jako `o'. Připomněl jsem ho, že body nespadnou samo.  Zbytek řádku četl správně ale četl `n' jako `d'(upravil se stejném dechem).

\paragraph{Pátý řádek}
\begin{tabular}{|c|c|c|c|c|c|c|c|c|c|c|c|}
\hline
p&r&v&n&í& &ř&á&d&e&k&,\\
\braillebox{123478}&\braillebox{1235}&\braillebox{1236}&\braillebox{1345}&\braillebox{34}&\braillebox{}&\braillebox{1235}&\braillebox{16}&\braillebox{145}&\braillebox{15}&\braillebox{13}&\braillebox{2}\\
\hline
\end{tabular}

Četl slovo \uv{první} bez chyb.  Audio je zaseknutí, a nevím co se stalo dal na řádce.

\paragraph{Šesti řádek}
\begin{tabular}{|c|c|c|c|c|c|c|c|c|c|c|c|}
\hline
 &k&t&e&r&ý& &z&o&b&r&a\\
\braillebox{78}&\braillebox{13}&\braillebox{2345}&\braillebox{15}&\braillebox{1235}&\braillebox{12346}&\braillebox{}&\braillebox{1356}&\braillebox{135}&\braillebox{12}&\braillebox{1235}&\braillebox{1}\\
\hline
\end{tabular}

První část četl správné ale myslel, že `z' je `ř'. Zbytek taky četl správně.

Přesouval ruku na selektor úplně samostatně a správně.

\paragraph{Sedmi řádek}
\begin{tabular}{|c|c|c|c|c|c|c|c|c|c|c|c|}
\hline
z&u&j&e& &j&e&n&o&m& &j\\
\braillebox{135678}&\braillebox{136}&\braillebox{245}&\braillebox{15}&\braillebox{}&\braillebox{245}&\braillebox{15}&\braillebox{1345}&\braillebox{135}&\braillebox{134}&\braillebox{}&\braillebox{245}\\
\hline
\end{tabular}

Četl správně.

\paragraph{Osmi řádek}
\begin{tabular}{|c|c|c|c|c|c|c|c|c|c|c|c|}
\hline
e&d&n&o& &p&í&s&m&e&n&o\\
\braillebox{1578}&\braillebox{145}&\braillebox{1345}&\braillebox{135}&\braillebox{}&\braillebox{1234}&\braillebox{24}&\braillebox{234}&\braillebox{134}&\braillebox{15}&\braillebox{1345}&\braillebox{135}\\
\hline
\end{tabular}

Četl správně kromě `s', nevím přesně co odhadl první kvůli špatné kvalitě zvuku.  `o' četl nejdřív jako  `a' ale opravil se stejným dechem.

\paragraph{Devátý řádek}
\begin{tabular}{|c|c|c|c|c|c|c|c|c|c|c|c|}
\hline
.& & &P&r&v&n&í& &b&y&l\\
\braillebox{378}&\braillebox{}&\braillebox{}&\braillebox{12347}&\braillebox{1235}&\braillebox{1236}&\braillebox{1345}&\braillebox{34}&\braillebox{}&\braillebox{12}&\braillebox{13456}&\braillebox{123}\\
\hline
\end{tabular}

Četl bez chyb.

\paragraph{Desátý řádek}
\begin{tabular}{|c|c|c|c|c|c|c|c|c|c|c|c|}
\hline
 &v&y&n&a&l&e&z&e&n& &v\\
\braillebox{78}&\braillebox{1236}&\braillebox{13456}&\braillebox{1345}&\braillebox{1}&\braillebox{123}&\braillebox{15}&\braillebox{1356}&\braillebox{15}&\braillebox{1345}&\braillebox{}&\braillebox{1236}\\
\hline
\end{tabular}

Četl `v' nejdřív nejsprávně ale kvalita zvuku mě nedovolí vám říct jak.  Četl až na `e' správně ale selektor začal zle fungovat. Správil jsem to a řekl jsem ho ať přečte od začatku.  Četl správně kromě `e' což četl nejdřív jako `o' a mezera, která chvílí myslel je tečka.  

\paragraph{Jedenacti řádek}
\begin{tabular}{|c|c|c|c|c|c|c|c|c|c|c|c|}
\hline
 &r&o&c&e& &1&9&1&3& &v\\
\braillebox{78}&\braillebox{1235}&\braillebox{135}&\braillebox{14}&\braillebox{15}&\braillebox{}&\braillebox{18}&\braillebox{248}&\braillebox{18}&\braillebox{148}&\braillebox{}&\braillebox{1236}\\
\hline
\end{tabular}

Četl `c' nejdřív jako `n'.  Četl `1' nejdřív jako `a'. Vysvětlil jsem, že osmi bod je pro čísla.  Nevím zda pochopil, protože `9' nejdřív četl jako `i'.  Když jsem ho říkal, že `i' je nesprávně, četl zbytek čísel správně.  Četl `v' jako `l' ale odvolal stejným dechem.  Pak odhadl `r' a konečně správně `v'.

\paragraph{Dvanácti řádek}
\begin{tabular}{|c|c|c|c|c|c|c|c|c|c|c|c|}
\hline
 &A&n&g&l&i&i&.& & &J&m\\
\braillebox{78}&\braillebox{17}&\braillebox{1345}&\braillebox{1245}&\braillebox{123}&\braillebox{24}&\braillebox{24}&\braillebox{3}&\braillebox{}&\braillebox{}&\braillebox{2457}&\braillebox{134}\\
\hline
\end{tabular}

Řešil jsem problém se selektorem u začátku řádku.  Musím poznamenat, že on poznal, že selektor nefungoval a čekal na mně. Četl `i' nejdřív jako `e'.  Vím, že už dobře rozuměl selektor, protože poznal, že jsou dvě `i' a dvě mezery po tečce.  Zbytek řádku četl správně.

\paragraph{Řádek 13}
\begin{tabular}{|c|c|c|c|c|c|c|c|c|c|c|c|}
\hline
e&n&o&v&a&l& &s&e& &O&p\\
\braillebox{1578}&\braillebox{1345}&\braillebox{135}&\braillebox{1236}&\braillebox{1}&\braillebox{123}&\braillebox{}&\braillebox{234}&\braillebox{15}&\braillebox{}&\braillebox{1357}&\braillebox{1234}\\
\hline
\end{tabular}

Četl `v' nejdřív jako `l' ale opravil se stejným dechem. Jinak četl správně.

\paragraph{Řádek 14}
\begin{tabular}{|c|c|c|c|c|c|c|c|c|c|c|c|}
\hline
t&o&f&o&n& &a& &p&ř&e&v\\
\braillebox{234578}&\braillebox{135}&\braillebox{124}&\braillebox{135}&\braillebox{1345}&\braillebox{}&\braillebox{1}&\braillebox{}&\braillebox{1234}&\braillebox{2456}&\braillebox{15}&\braillebox{1236}\\
\hline
\end{tabular}

Jedno jsem si myslel, že selektor nefunguje, protože nějak rychle cvakl zobrazovač. Ale kdy jsem zkusil, jsem zjistil, že zkutečně funguje správně.  Jinak četl bez chyb.

\paragraph{Řádek 15}
\begin{tabular}{|c|c|c|c|c|c|c|c|c|c|c|c|}
\hline
á&d&ě&l& &s&v&ě&t&l&o& \\
\braillebox{1678}&\braillebox{145}&\braillebox{126}&\braillebox{123}&\braillebox{}&\braillebox{234}&\braillebox{1236}&\braillebox{126}&\braillebox{2345}&\braillebox{123}&\braillebox{135}&\braillebox{}\\
\hline
\end{tabular}

Četl bez chyb.

\paragraph{Řádek 16}
\begin{tabular}{|c|c|c|c|c|c|c|c|c|c|c|c|}
\hline
n&a& &z&v&u&k&.& & &K&d\\
\braillebox{134578}&\braillebox{1}&\braillebox{}&\braillebox{1356}&\braillebox{1236}&\braillebox{136}&\braillebox{13}&\braillebox{3}&\braillebox{}&\braillebox{}&\braillebox{137}&\braillebox{145}\\
\hline
\end{tabular}

Četl bez chyb a poznal `K' jako velké.

\paragraph{Řádek 17}
\begin{tabular}{|c|c|c|c|c|c|c|c|c|c|c|c|}
\hline
y&ž& &č&t&e&n&á&ř& &p&o\\
\braillebox{1345678}&\braillebox{2346}&\braillebox{}&\braillebox{146}&\braillebox{2345}&\braillebox{15}&\braillebox{1345}&\braillebox{16}&\braillebox{2456}&\braillebox{}&\braillebox{1234}&\braillebox{135}\\
\hline
\end{tabular}

Četl správně až na `ř' a u toho jsem mu říkal, že budeme zkusit něco jiné a jsem se zkusil ho naučit nechat ruku položený na selektor.  Musel jsem zase resetovat stroj.

Audio je zase zpřeházený ale je rozumět kdy četl písmeno `y'. Odhadl `ď'\braillebox{1456} ale dal najevo, že nebyl jistý.  Teď, už nechal ruku na zobrazovač.  Taky četl `č' nesprávně ale není rozumět jak.

\paragraph{Řádek 18}
\begin{tabular}{|c|c|c|c|c|c|c|c|c|c|c|c|}
\hline
h&y&b&o&v&a&l& &s&p&e&c\\
\braillebox{12578}&\braillebox{13456}&\braillebox{12}&\braillebox{135}&\braillebox{1236}&\braillebox{1}&\braillebox{123}&\braillebox{}&\braillebox{234}&\braillebox{1234}&\braillebox{15}&\braillebox{14}\\
\hline
\end{tabular}
Četl `h' správně na druhý pokus, první odhad není rozumět.  Četl `y' jako `d' ale uvědomoval stejném dechem, že to není správně odhadl správně na druhý pokus.

\paragraph{Řádek 19}
\begin{tabular}{|c|c|c|c|c|c|c|c|c|c|c|c|}
\hline
i&á&l&n&í&m& &p&e&r&e&m\\
\braillebox{2478}&\braillebox{16}&\braillebox{123}&\braillebox{1345}&\braillebox{34}&\braillebox{134}&\braillebox{}&\braillebox{1234}&\braillebox{15}&\braillebox{1235}&\braillebox{15}&\braillebox{134}\\
\hline
\end{tabular}
Četl `i' nejdřív jako `s'.  Četl `á' nejdřív jako `e'. Četl druhý `e' nejdřív jako `o'. Četl `m' jako `p' ale upravil se stejným dechem.

U toho jsem končil setkání.  Zeptal jsem se ho \uv{myslíte, že byste se naučil to používat lip s praxi?} on odpovídal \uv{ne, to je těžký, já mám na to notebook, psací stroj.} Zeptal jsem se \uv{a používáte braillský řádek?} On: \uv{Ne, hlasový vystup.}

\subsubsection{Žák 8}
Setkání 23.4.2013

Žák 8 je 20-25 let starý. Čte Braillovo písmo od začátku školní docházky. Je úplně nevidomý. Je pravák.  Nepoužívá braillský řádek.

Četl tiskový text jednou(pravou) rukou.

% začátek 37.6 otočení 93.8 konec 180.7

Čas na čtení celého textu: 143.1 vteřin

CPS: 2.84

Čas na čtení druhé strany tiskového textu: 87.1  vteřin

CPS: 3

Kdy jsem byl představený osmi žák hned chytil na to, že Timothy není české jméno a kdy jsem se zeptal osmi žákovi úvodní otázky, zkusil na mně mluvit anglicky, což dělal skoro bez přízvuku.

Když jsem ho ukazoval selektor, selektor začal poznat náhodné dotyky.  Po krátké době určení problém, jsem zjistil, že je tím, že účastník měl mokré ruce.

\paragraph{První řádek}

\begin{tabular}{|c|c|c|c|c|c|c|c|c|c|c|c|}
\hline
B&r&a&i&l&l&s&&&&&\\
\braillebox{1278}&\braillebox{1235}&\braillebox{1}&\braillebox{24}&\braillebox{123}&\braillebox{123}&\braillebox{234}&\braillebox{}&\braillebox{2358}&\braillebox{123}&\braillebox{}&\braillebox{}\\
\hline
\end{tabular}

Začali jsme číst první řádek tím, že se zeptal \uv{jak s tím na to}.  Začal jsem vysvětlení tím, že jsem ho ukázal ty díry na zobrazovače a pak kdy jeho pravá ruka byla na selektor zeptal jsem se které body jsou nahoře.  On říkal \uv{první} a odhadl `a', asi moje vysvětlení byla dost zmatený nepochopil jsem proč odhadl `ačko'.  Myslel, jsem si, že třeba myslí, že se zobrazí `á' a je zmatený z těch body 7 a 8.  Kdy měl prst na bod 8 jsem ho říkal, že to je pro velké písmena(což není pravda, bod 7 je pro velké písmena, bod 8 je pro čísla). Odhadl `v' a ještě `u'. Zřejmě nevěděl co to je osmibodové Braillovo písmo ale jsem to ještě nevěděl.  Začal jsem ho vykládat o selektoru.  Ale jak jsem ho vysvětlil, že jsou písmena `k' na selektor, hned odhadl, že zobrazené písmeno je `k'.

Jsem ještě vyřešil tomu, že měl mokré ruce a jsem malém leh tím, že selektor nefungoval.  Říkal jsem \uv{to mně dost překvapuje, aha!} když jsem viděl tu vodu.  On odpovídal \uv{Já nechci urazit, ale bych se nejdřív zeptal jestli už s tím umíte}.  Odpovídal jsem, že umím ale, že měl příliš mokré ruce na to a to chvíli nefungoval.

Zase jsem ho snažil vysvětlit zobrazovač.  Dal jsem jeho prst na každé díře a řekl jsem ho, které to je.  Teď jsem se dozvěděl, že snad v životě nesetkal s osmibodové Braillovo písmo protože se mně rovně zeptal na \uv{ty zbyli dva body}.  Jsem ho vysvětlil: \uv{Na počítače děláme velké písmeno tím, že dáváme nahoru asi sedmi bod a čísla tím že dáváme nahoru osmi bod}.

U toho jsem si rozhodl snížit citlivost senzorů, protože jenom tím jak jsem jeho ruku natřel papírovým kapesníkem nestačil.  Musel jsem restartovat celý softwarový sestavení.

Potom jsme se začaly znovu s poznání písmen.  Ještě na `B' jsem zase pojmenoval číslema body a dával jsem jeho prst na každý tyč nebo díru. Pojmenoval jsem a ukazoval jsem jenom první šest bodů abych ho zbytečně nezmátl.  Jasně musel cítit, že body jedna a dva jsou nahoře a žádné jiné ale stále odhadl nesprávně `včko'.  Vysvětlil jsem ho zase, že body sedm a osm nepoužíváme k poznání písmena a konečně on odhadl `bčko'.

Četl `r' nejdřív jako `h' ale odhadl správně potom co jsem ho říkal ať hledá dolu.

Říkal jsem ho, ať přesouvá na další písmeno a jsem musel zase utřít jeho ruce.  Byl tam dost vedro.  Natřel jsem selektor a on mě říkal: \uv{Bych měl radši braillský řádek na, kterém by to co bych napsal na počítači automatický zobrazoval.}

Říkal jsem ho jak má přesouvat na selektor ale četl `a' bez pomoci.

Odstranil ruku ze selektoru a jsem ho vysvětlil, že když nedotyká selektorem, tak se nezobrazí nic.  Dal ruku zpátky na selektor v přibližně správní místo a četl `i' jako `,' než jsem ho připomněl o další řádek bodů.

Zase odstranil ruku z selektoru ale vrátil ji zpět když jsem ho říkal ať čte další písmeno. Dal tu ruku na selektor o tři písmen dal než měl, ale četl správně `s'. Vrátil zpátky a četl `l' nejdřív jako `k'.

On začal hrát s selektorem a zobrazovačem.  Odstranil ruku z selektoru a pak poklepl na každý tyč který byl nahoru aby spadl.  Nechal jsem ho hrát 46 vteřin a pak on říkal tohle je `lko'(měl pravdu, tam se zkutečně `l' zobrazoval).  Pak jsem navrhoval, že čteme další řádek.

\paragraph{Druhý řádek}
\begin{tabular}{|c|c|c|c|c|c|c|c|c|c|c|c|}
\hline
k&ý& &ř&á&d&e&k&,& &k&t\\
\braillebox{1378}&\braillebox{12346}&\braillebox{}&\braillebox{2456}&\braillebox{16}&\braillebox{145}&\braillebox{15}&\braillebox{13}&\braillebox{2}&\braillebox{}&\braillebox{13}&\braillebox{2345}\\
\hline
\end{tabular}

Četl první písmeno jako `l', druhý odhad měl jako `v'.  Vysvětlil jsem mu, že body sedm a osm jsou tam jenom protože jsme u začátku řádek.  Odhadl `h' pak `u'.  Zase jsem dával jeho prst na každý bod a pojmenoval jsem je čísly. Pak jsem se zeptal, které body jsou nahoře.  On ukazoval na první bod a říkal \uv{To je ten sedmi} říkal jsem ho, že to není, že to je první.  Pak on říkal \uv{těžko říct, on je závorky}.  Říkal jsem ho \uv{Počítejte se mnou, jedna, dva, tři, čtyři, pět, šest, tak nahoře je bod jedna a bod tři, které písmeno to je} konečně správně říkal `kačko'.

Pokračovali jsme na další písmeno.  Nedržel ruku na selektor ale poklepal na senzor a pak prst odstranil.  Četl písmeno jako `l', pak `p' a konečně `ý'.

Odstranil zase ruku z selektoru, tak jsem si rozhodl zase snažit mu vysvětlit jak funguje.  Vedl jsem jeho prst nad `\uv{ty kačky}(ty senzory).

Vedl jsem jeho ruku na čtvrté písmeno, a četl to nejdřív jako `t' a potom, správně, jako `ř'.  Četl další písmeno `á' správně. Zase jsem musel utřít selektor a on utřel ruky na kalhotách a mně vykládal \uv{To je strašně přemodernizováné braillský řádek, vůbec se vtom nevyznám.} Vedl jsem jeho ruku na správné místo a jsme pokračovali.  Četl `d' nejdřív jako `č' než to četl správně.  On četl, tak že klepal na selektor a pak klepal každý bod na zobrazovač dole.  Četl `e' nejdřív jako `á', pak `o', pak `i'.  Konečně jsem ho jen říkal odpověď.  On říkal \uv{vůbec to není poznat} odpověděl jsem, že to jsou stejné písmena jenom větší, a on mě řek \uv{Já vím ale}.  Pořád klepal na ty senzory ale přibližně správně překračoval. Myslel, že chvili se zobrazoval `ě'.  Četl `k' správně.  Četl čárku správně.

Vysvětlil jsem mu, že už četl dvě slova.  Říkal mně, že to nepoznal.  Zeptal jsem se ho, jestli je unavený, ale říkal, že není.

Vedl jsem jeho ruku na předposlední znak, který četl správně jako `k'.  Pak začal zase ty senzory nefungovat a jsem je zase utřel.  On říkal anglicky \uv{I will get drunk}(budu opilý).  Vedl jsem jeho ruku zpátky na `t' který nejdřív četl jako `;'\braillebox{23} a pak správně.

\paragraph{Třetí řádek}
\begin{tabular}{|c|c|c|c|c|c|c|c|c|c|c|c|}
\hline
e&r&ý& &t&e&ď& &p&o&u&ž\\
\braillebox{1578}&\braillebox{1235}&\braillebox{12346}&\braillebox{}&\braillebox{2345}&\braillebox{15}&\braillebox{1456}&\braillebox{}&\braillebox{1234}&\braillebox{135}&\braillebox{136}&\braillebox{2346}\\
\hline
\end{tabular}

Na začátek třetí řádku odhadl nejdřív `k', pak `A', `t', `z'. Pak řekl, že neví. Odhadl `x' a jsem se zeptal, které body jsou nahoru. On říkal \uv{první, třetí} řekl jsem ho, že to je \uv{sedmi už}.  On říkal(správně) \uv{tuto logiku mizí}. Pak odhadl `ě' a jsem ho říkal \uv{ečko ale bez háčku}.  On mumlal \uv{tu logiku prostě mizí když}.  Další písmeno odhadl nejdřív jako `l' a pak správně jako `ý'. Četl `t' správně.  Pořád četl tím způsobem, že klepal krátce na selektor, a pak klepal na zobrazovače prstem. Je důležité poznamenat teď, že body na můj FCHAD nespadnou samy, musíte je pomoct. Říkal \uv{tohle nefunguje vůbec}. Vysvětlil jsem mu, že jenom body, které jsou pevné jsou významné. Četl `e' správně.  Pak četl `ď' jako `š'\braillebox{156}.  Připomněl jsem ho, že ještě ne zkusil hmatat na bodu čtyři.  Tento krát odhadl správně `ď'.  Četl tu mezeru jako čárka. Četl `p' nejdřív jako `t' pak `e', konečně jako `p'.  Četl `o' správně.  Přeskočil `u' ale četl `ř' správně.

\paragraph{Čtvrtý řádek}
\begin{tabular}{|c|c|c|c|c|c|c|c|c|c|c|c|}
\hline
í&v&á&t&e&,& &n&e&n&í& \\
\braillebox{3478}&\braillebox{1236}&\braillebox{16}&\braillebox{2345}&\braillebox{15}&\braillebox{2}&\braillebox{}&\braillebox{1345}&\braillebox{15}&\braillebox{2345}&\braillebox{34}&\braillebox{}\\
\hline
\end{tabular}
Četl, první písmeno nejdřív jako `ž', pak `i', a jsem ho řekl, že je to `í'.  Četl `v' a `á' správně přeskočil `t' ale četl `e' správně.  Řekl, že neví jaké to mělo byt slovo. Vysvětlil jsem mu, že je to slovo \uv{používáte} ale, že přeskočil písmena `u' a `t'.  Zeptal se, \uv{kde je učko} tak jsme se vrátili na předchozí řádku a jsem vedl jeho ruku k `o' pak `u' a `ž', písmena správně ztotožnil.

Pak jsme se začali číst od začátku řádku. Nejdřív, myslel, že `í' je `u' ale odhadl správně po druhý. Pak myslel, že se zobrazuje `,' ale ještě bylo nahoru `í'.  Zeptal se kolik je hodin a v stejném chvilce objevila v dveře sekretářka a zeptala se \uv{jak jste na to}. Říkal jsem, že už můžeme končit a žák 8 říkal, že by měl už jít.  Sekretářka ho řekla, že ještě má pět minut a jsme pokračovali.

Přeskočil ještě par písmen a jsme byli u `e' co ztotožnil správně. Myslel nejdřív je `n' je `g'. Přeskočil `í' a správně ztotožnil mezeru.  Neřekl jsem ho, že přeskočil, protože to by bylo zbytečné obtížení úkolu.

Dal jsem hu číst další řádek.

\paragraph{Pátý řádek}
\begin{tabular}{|c|c|c|c|c|c|c|c|c|c|c|c|}
\hline
p&r&v&n&í& &ř&á&d&e&k&,\\
\braillebox{123478}&\braillebox{1235}&\braillebox{1236}&\braillebox{1345}&\braillebox{34}&\braillebox{}&\braillebox{1235}&\braillebox{16}&\braillebox{145}&\braillebox{15}&\braillebox{13}&\braillebox{2}\\
\hline
\end{tabular}

Zeptal se \uv{jak ty řádky zadávají?}  Ukazoval jsem ho svůj notebook, a řekl jsem ho, že používám ty šípky abych přešel na další řádek.  Potom co jsem ho ukazoval ten počítač, nemohl znovu najít selektor, tak jsem musel jeho ruku tam dat.

Nejdřív myslel, že řádek začal `ý' ale pak správně řekl `p' když jsem ho připomněl o bodech sedm a osm.  Správně ztotožnil `r'.  Kdy přesunul ruku, zřejmě slyšel cvaknutí, protože zeptal se, jestli něco přeskočil.  Řekl jsem ho, že zkutečně přeskočil písmeno a vrátil se úspěšně na `v', myslel chvílí, že je `ě'\braillebox{126} a trval mu trochu, ale odhadl správně `v' podruhy.  Správně odhadl `n' a `í'.  Zeptal jsem ho, jestli pozná, které slovo to je, ale říkal, že ne.  Zeptal jsem se, jestli pamatuje první písmeno v řádku.  Řekl `y'.  Zase jsme se vrátili na začátek řádku a jsme zase uvažovali přes to, že jsou tam nahoru body sedm a osm.  Správně odhadl podruhy.  Zeptal se pak \uv{druhý byl co} a jsem ho říkal ať to najde sám.  Začal číst druhý písmeno jako `h'. Tím jsme ukončili, protože přišla sekretářka s dalším účastníkem.  Žák 8 říkal \uv{to je nad mou logiku} a rychle odešel.

\subsubsection{Žák 9}
Setkání 23.4.2013

Žák 9 je 20-25 let starý.  Nejvyšší dosažení úroveň vzdělávání je maturita.  Je úplně nevidomý a má problémy se sluchem.  Při naše setkání jsem nosil na sebe mikrofon, který byl bezdrátově přípojení na jeho sluchadla.  Rozuměl mě bez znatelný potíže.

Četl tiskový text jednou(pravou) rukou.

%začatek 197.8 otočení 230.2 konec 279
Čas na čtení celého textu: 81.2 vteřin

CPS: 5

Čas na čtení druhé strany tiskového textu: 48.8 vteřin

CPS: 5.4

Kdy jsem ho ukazoval zobrazovač, říkal jsem mu \uv{tady se zobrazí to písmeno. Tady jsou osm děr.} On vzal zobrazovače obě ruce a začal je počítat, ale nepočítal je v standardní pořadí číslování jen je počítal.  Mluvil milým a překvapeným hlasem po celé setkání. Kdy byl jistý, že jsou tam zkutečně osm děr, říkal jsem ho, že by měl položit levou ruku na zobrazovače.  On mě odpovídal smíchem, že \uv{Já jsem sice levák ale čtu pravou}. Vysvětlil jsem mu, že současní stroj má jenom jeden nastavení.  Potom jsem ho ukazoval selektor.  Říkal jsem mu, že zvuk, který vydá zobrazovač je zvuk změna písmen a že by měl taky cítit jak se to písmeno změní.  On začal \uv{hrát} se selektorem a jsem si myslel, že bude samostatně pokusit o čtení.  Měl ruku správně položenou na zobrazovače a správně byl u první písmena na selektor.  Občas říkal \uv{uf}.

Protože jsem chtěl dozvědět, jestli bude číst sám, čekal jsem 146 vteřin.  Bohužel v analýze audio-nahrávku vidím, že jsem říkal \uv{um} u začátku toho času, tak je docela možný, že on sám čekal na mně.  Kdy mně bylo jasno, že sám číst nebude, jsem se začal vykládat o použitý nastroje dal.

Vysvětlil jsem mu, že \uv{ty písmena `k'} jsou senzory.  On začal je počítat a když počítal, že jsou jích 12 tak říkal \uv{aha} jak že už spojil je s těmi dvanácti senzory vysvětlené v úvodním textu. Zase jsem čekal ať hraje, a tento krát jsem neříkal žádný zmítající \uv{um}.  Zkutečně hrál s nástrojem a věřil jsem, že se snaží čist. 210 vteřin později ještě nepokračoval.  Stále hmatal na zobrazovače jako by se snažil z něj číst ale nečetl. %881-671.9

\paragraph{První řádek}

\begin{tabular}{|c|c|c|c|c|c|c|c|c|c|c|c|}
\hline
B&r&a&i&l&l&s&&&&&\\
\braillebox{1278}&\braillebox{1235}&\braillebox{1}&\braillebox{24}&\braillebox{123}&\braillebox{123}&\braillebox{234}&\braillebox{}&\braillebox{2358}&\braillebox{123}&\braillebox{}&\braillebox{}\\
\hline
\end{tabular}


Zeptal jsem se mu co je první písmeno a on se mě zeptal \uv{jak to poznat?}.  Říkal jsem mu, ať počítá díry.  Nejdřív se mně nepochopil, zřejmě kvůli chybě mé češtiny.  Počítal, že jsou dva body nahoru, ale nepochopil ještě o co jde.  Vzal jsem jeho levou ruku a ukazoval jsem ho \uv{to je jedna a dva tak které písmeno to je?} a u toho už říkal bez myšlení `b'.  Protože jsme byli u začátek řádku ještě byly nahoře body 7 a 8.  Bohužel žák 8 se kymácel trupem tak, že jeho ruce na selektor nebyl stabilní a dále trpěl malé zrcadloví pohyby kdy se snažil číst na zobrazovače.  Myslím si, že rozuměl už, že by měl jít na druhé písmeno i bez toho, že bych mu řekl, ale nebyl fyzické schopní.  Věděl, že jeho prsty pohybují bez jeho přání a začal brzo byt frustrovaní tím.  Držel jsem jeho prstí tak aby nepohybovaly a vedl jsem jeho prsti na selektor.  Přečetl `r' \uv{první a pátý koukám} jsem mu říkal \uv{ještě} a on překvapeně, když objevoval ještě \uv{druhý a třetí aha, takže rko}.

Dal četl `a' stále způsobem, že nejdřív říkal čísla bodů a písmeno až potom.  `l' už četl bez pojmenování bodů. Četl další písmeno jako `p' a pak opravil, že je `s' když jsem ho říkal že četl nesprávně \uv{aha,tohle zmizelo, tak dávat pozor na to co se zmizí} on myslel tím, že nejdřív četl `l'\braillebox{123} a pak `s'\braillebox{234} protože body nespadnou než na něj tlačíte, chvílí zkutečně se zobrazilo `p'\braillebox{1234}.

\paragraph{Druhý řádek}
\begin{tabular}{|c|c|c|c|c|c|c|c|c|c|c|c|}
\hline
k&ý& &ř&á&d&e&k&,& &k&t\\
\braillebox{1378}&\braillebox{12346}&\braillebox{}&\braillebox{2456}&\braillebox{16}&\braillebox{145}&\braillebox{15}&\braillebox{13}&\braillebox{2}&\braillebox{}&\braillebox{13}&\braillebox{2345}\\
\hline
\end{tabular}

Přišli jsme na další řádek a četl `k' bez problému, ale četl `ý' zase tím, že pojmenoval body.  Kdy jsme dorazili na mezeru, on říkal \uv{nic} a začal se smát.   Četl slovo \uv{řádek} po jednotlivých písmenech ale bez chyb a bez pojmenování čísla bodů.  Četl tak dobře, že jsem chvíli pustil jeho pravou ruku ať používá selektor samostatně ale zase začala pochybovat rytmický bez jeho ovládání.  Našli jsme místo kde jsme skončili a pokračovali jsme.

Četl `k' nejdřív jako `a' ale sám se opravil když jsem ho říkal, že to není správný.

\paragraph{Třetí řádek}

\begin{tabular}{|c|c|c|c|c|c|c|c|c|c|c|c|}
\hline
e&r&ý& &t&e&ď& &p&o&u&ž\\
\braillebox{1578}&\braillebox{1235}&\braillebox{12346}&\braillebox{}&\braillebox{2345}&\braillebox{15}&\braillebox{1456}&\braillebox{}&\braillebox{1234}&\braillebox{135}&\braillebox{136}&\braillebox{2346}\\
\hline
\end{tabular}

Přešli jsme na třetí řádek.  Začátek četl správně. Vždy, kdy dostal na mezera, hmatal jak tam nic nahoru není a usmíval se a řekl \uv{mezera} s radosti tím jak je lehký poznat mezery až na `ž', ale četl `ž'\braillebox{2346} jako `z'\braillebox{1356}. Říkal jsem \uv{To je jako `z' ale obraceně} a on četl(obraceně) \uv{jedna tři pět šest, to je `z'} jsem ho ukazoval, že pojmenoval čísla ty body obraceně a on odpovídal rozčileně \uv{Tak, že jsem to celou dobu říkal blbě!}  Uklidnil jsem ho, že četl správně do té doby a on se krasně usmíval a říkal, že je úplně zmatený.

\paragraph{Čtvrtý řádek}

\begin{tabular}{|c|c|c|c|c|c|c|c|c|c|c|c|}
\hline
í&v&á&t&e&,& &n&e&n&í& \\
\braillebox{3478}&\braillebox{1236}&\braillebox{16}&\braillebox{2345}&\braillebox{15}&\braillebox{2}&\braillebox{}&\braillebox{1345}&\braillebox{15}&\braillebox{2345}&\braillebox{34}&\braillebox{}\\
\hline
\end{tabular}

Kdy jsme pokračovali na další řádek, četl `í'\braillebox{34} jako `á'\braillebox{16} ale tento krát upravil se sám když jsem ho říkal, že to není správní.  Zase u toho usmíval.  Pak četl `v' a `á' bez chybně.  Četl `e' jako `i' ale tento krát jsem Já dělal chybu a neopravil jsem ho. Četl `,' správně a když jsme byli na `n'\braillebox{1345} přemyslel a pojmenoval body hodně dlouho(39 vteřin) než odhadl `ž', který hned sám odvolal. Čekal jsem další 18 vteřin a říkal jsem ho \uv{je to `ž' z horu nohama}(I Já jsem byl zřejmě trochu zmatený, protože `ž' z horu nohama je `ň').  Konečně jsem ho jen tak říkal, že je to `n' ale mně nevěřil a říkal \uv{mate to, že to je zrcadlově obracený} ale konečně říkal \uv{aha, už to vidím} a zase krasně usmíval.  %1838.4

Nějak během toho, jsme přeskočili dvě písmena a jsme byly u `í', což četl nejdřív jako `t', pak `á' a konečně `í'.

Zeptal jsem ho jestli není unavený, ale říkal že je \uv{v pohodě}.

\paragraph{Pátý řádek}

\begin{tabular}{|c|c|c|c|c|c|c|c|c|c|c|c|}
\hline
p&r&v&n&í& &ř&á&d&e&k&,\\
\braillebox{123478}&\braillebox{1235}&\braillebox{1236}&\braillebox{1345}&\braillebox{34}&\braillebox{}&\braillebox{1235}&\braillebox{16}&\braillebox{145}&\braillebox{15}&\braillebox{13}&\braillebox{2}\\
\hline
\end{tabular}

Na další řádek četl písmena \uv{první ř} bez chyb a dal velký důraz na `n'.  Zase četl `á' nejdřív `í'. Četl `d' a `e' správně ale `k' četl nejdřív jako `a'. Četl `,' správně.

\paragraph{Šesti řádek}
\begin{tabular}{|c|c|c|c|c|c|c|c|c|c|c|c|}
\hline
 &k&t&e&r&ý& &z&o&b&r&a\\
\braillebox{78}&\braillebox{13}&\braillebox{2345}&\braillebox{15}&\braillebox{1235}&\braillebox{12346}&\braillebox{}&\braillebox{1356}&\braillebox{135}&\braillebox{12}&\braillebox{1235}&\braillebox{1}\\
\hline
\end{tabular}

Šesti řádek četl správně až na `z' který nejdřív četl jako `n'.  Četl `o' a `b' správně ale četl `r' nejdřív jako `h'.

Je možní, že protože jsem musel vest jeho pravou ruku ještě nerozuměl selektor. Kdy jsme byli u konce řádku myslel, že se zobrazuje mezera.  Nic se nezobrazoval, ale to byl protože nebyl v kontaktu s žádným senzorem.

\paragraph{Sedmi řádek}
\begin{tabular}{|c|c|c|c|c|c|c|c|c|c|c|c|}
\hline
z&u&j&e& &j&e&n&o&m& &j\\
\braillebox{135678}&\braillebox{136}&\braillebox{245}&\braillebox{15}&\braillebox{}&\braillebox{245}&\braillebox{15}&\braillebox{1345}&\braillebox{135}&\braillebox{134}&\braillebox{}&\braillebox{245}\\
\hline
\end{tabular}

Četl `z' jako `n' ale sám se upravil.  Říkal jsem ho ať zkusí posunut pravou ruku na selektor samostatně, a přestal jsem držet jeho ruku ale jak jsem nedržel jeho ruka tak pohybovala sama a on si povzdechl frustraci, tak jsem zase tu ruku držel.   Četl `u' nejdřív jako `a' ale upravil se když jsem ho říkal, že to není správný.  Četl `j'\braillebox{245} jako `h'\braillebox{125} ale upravil se sám když jsem ho říkal, že je obracený. Četl `e' a mezera správně ale zase obrátil `j'.  Četl `e` správně ale `n' jako `d'. Když jsem ho říkal, že to není správný, nejdřív si myslel, že četl obraceně. Říkal jsem ho, že neobracel to a, že ještě nějaký bod dolu a rychle už našel, že je to `n'. Četl `o' správně ale četl `m' nejdřív jako `k'.  Četl `j' jako `,' a jsem ho připomněl, že je ještě další řádek bodu než svůj odhad opravil.

\paragraph{Osmi řádek}
\begin{tabular}{|c|c|c|c|c|c|c|c|c|c|c|c|}
\hline
e&d&n&o& &p&í&s&m&e&n&o\\
\braillebox{1578}&\braillebox{145}&\braillebox{1345}&\braillebox{135}&\braillebox{}&\braillebox{1234}&\braillebox{24}&\braillebox{234}&\braillebox{134}&\braillebox{15}&\braillebox{1345}&\braillebox{135}\\
\hline
\end{tabular}

Na osmi řádek textu četl správně až na `í' což zase obrátil jako `á'. Četl `s' a `m' správně ale pak si myslel, že už jsme na další písmeno, když jsme v zkutečnosti nebyli.  Četl `m'\braillebox{134} jako `c'\braillebox{14}.  Četl `e' jako `o' ale sám si objevil tu chybu.  A četl `o' jako `e'.

\paragraph{Devátý řádek}
\begin{tabular}{|c|c|c|c|c|c|c|c|c|c|c|c|}
\hline
.& & &P&r&v&n&í& &b&y&l\\
\braillebox{378}&\braillebox{}&\braillebox{}&\braillebox{12347}&\braillebox{1235}&\braillebox{1236}&\braillebox{1345}&\braillebox{34}&\braillebox{}&\braillebox{12}&\braillebox{13456}&\braillebox{123}\\
\hline
\end{tabular}

Devátý řádek začíná s `.' ale bylo zobrazený jako \braillebox{378} protože jsme byli na začátek řádku.  On to četl jako závorka což je \braillebox{236}.  Připomněl jsem ho, že body sedm a osm jsou nahoru protože je to začátek řádku a zeptal jsem se, které body jsou ještě nahoru.  Jeho první odhad byl obracený, že to je znak pro velké písmeno\braillebox{6} a konečně odhadl správně když jsem ho říkal že to není.  Není důvod proč by věděl, že znak pro velké písmeno je bod šest a ne bod 3, protože ten znak je použitý jenom v tiskovém Braillovu písmu a bez rámečku není podstatní rozdíl.  Četl zbytek řádku bez problému.

\paragraph{Desátý řádek}
\begin{tabular}{|c|c|c|c|c|c|c|c|c|c|c|c|}
\hline
 &v&y&n&a&l&e&z&e&n& &v\\
\braillebox{78}&\braillebox{1236}&\braillebox{13456}&\braillebox{1345}&\braillebox{1}&\braillebox{123}&\braillebox{15}&\braillebox{1356}&\braillebox{15}&\braillebox{1345}&\braillebox{}&\braillebox{1236}\\
\hline
\end{tabular}

Četl desátý řádek skoro bez chyb, jedině byl částečně moje vína(protože jsem vedl jeho ruce na selektor). Já jsem táhl jeho ruku příliš daleko od `n' na `l' a museli jsme se vrátit. Kdy četl `l' podruhy četl to jako `b' ale opravil se hned než jsem stihl něco říct.

\paragraph{Jedenáctý řádek}
\begin{tabular}{|c|c|c|c|c|c|c|c|c|c|c|c|}
\hline
 &r&o&c&e& &1&9&1&3& &v\\
\braillebox{78}&\braillebox{1235}&\braillebox{135}&\braillebox{14}&\braillebox{15}&\braillebox{}&\braillebox{18}&\braillebox{248}&\braillebox{18}&\braillebox{148}&\braillebox{}&\braillebox{1236}\\
\hline
\end{tabular}

Jedenácti řádek obracel `r' jako `ř'.  Četl `o' nejdřív jako `e'. Kdy dočetl \uv{roce} zeptal jsem ho jestli pozná, které slovo to bylo ale nevěděl.  I potom co jsme se vrátili a on to četl znovu(s chybou, že nejdřív četl `e' jako `a') začal vykládat slovo jako \uv{ro} ale nepokračoval. Místo tomu, že bych ho zbytečně otrávil, jsem ho říkal, že to bylo slovo \uv{roce} a jsem ho říkal, že teď zřejmě bude číslo nebo datum.  Myslím, že jsem ho zmátl dost, protože nejdřív četl 1\braillebox{18} jako `a'.  Když jsem ho říkal, že to má byt číslo a, že bod osm je číselní bod stále to nepochopil.  Přemyslel o tom 4.5 vteřin a pak, řekl \uv{aha, jedna}. Devět četl nejdřív jako `i' a pak se zeptal jestli to je ještě číslo a správně doplnil, že to je devět. Četl `1' a `3' správně a jsem doplnil, že to je \uv{devatenátset třináct}.  Četl mezera a `v' správně.  Předtím, že řekl `v' říkal \uv{předpokládám, že tohle už není číslo}. Nevím jestli je to protože žádný číslo je v tvaru\braillebox{1236} anebo protože pochopil, že jsme dočetli datum.

\paragraph{Dvanáctý řádek}
\begin{tabular}{|c|c|c|c|c|c|c|c|c|c|c|c|}
\hline
 &A&n&g&l&i&i&.& & &J&m\\
\braillebox{78}&\braillebox{17}&\braillebox{1345}&\braillebox{1245}&\braillebox{123}&\braillebox{24}&\braillebox{24}&\braillebox{3}&\braillebox{}&\braillebox{}&\braillebox{2457}&\braillebox{134}\\
\hline
\end{tabular}

Dvanáctý řádek.  Zase jsem ho nechal číst samostatně. Už zvládl nepohybovat pravou ruku.  Přizpůsoboval tak, že tlačil hodně silně prstem, tak že jestli nějaký pohyb byl, pohyboval celý selektor. Četl `n' nejdřív jako `d' a měl velký problém s `g'\braillebox{1245} protože to četl jako `x'\braillebox{1346}.  Musel jsem mu říct ať pohybuje pravou ruku na selektor.  Četl `l' původně jako `b'.  Četl `j' jako `,' a `m' jako `c'.  Ještě mu pohyboval ruku na selektor ale snažil se, úspěšně to překonat.  U konce řádku myslel, že je mezera.

\paragraph{Řádek 13}
\begin{tabular}{|c|c|c|c|c|c|c|c|c|c|c|c|}
\hline
e&n&o&v&a&l& &s&e& &O&p\\
\braillebox{1578}&\braillebox{1345}&\braillebox{135}&\braillebox{1236}&\braillebox{1}&\braillebox{123}&\braillebox{}&\braillebox{234}&\braillebox{15}&\braillebox{}&\braillebox{1357}&\braillebox{1234}\\
\hline
\end{tabular}

První `o' četl nejdřív jako `e' a druhý `o' četl nejdřív jako `k', a četl `p' nejdřív jako `l'.

Hlavní je, že poslední dvě řádky používal selektor samostatně.

Když jsem se zeptal, jestli si mysli jestli by se dokázal naučit samostatně číst odpovídal \uv{no, snad ano} ale nijak nepřesvědčeně. Ještě když jsem se zeptal na jeho dojmy z toho, říkal:\em \uv{No, spíš tahle jsem trochu zmaten z toho, že posouvám tou ... tabulkou jestli řeknu takhle tou klávesnicí jako by tady tou pravou rukou že jako by spíš jak to funguje když dam každé to písmeno to znamená jeden ten jakoby posun.  Todle pro mně zkouška jak jsem občas myslel, že to je to písmeno zrcadlově obraceně, i když jsem fakt občas měl jsem pocit, nevím proč.}\em

\section{Shrnutí výsledků}

