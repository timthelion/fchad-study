Tato práce se zabývá prvními dojmy, adaptací a učením s podporou nového nástroje. V této studii jsem zkoumal, jak lidé se zrakovým postižením reagují na nový druh braillského řádku typu Fast Character Display (FCHAD) a jak se ho naučí používat. Pedagogické prostředí této studie je kontrolované a učení je založené na drilování. Zkoumání se zvláště zaměří na otázku adaptace během drilování. Studie se podobá psychologickým výzkumům křivek učení s tím rozdílem, že pedagogické působení je plnohodnotnou součástí studie.
