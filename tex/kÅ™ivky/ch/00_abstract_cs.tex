Tato práce se zabývá prvními dojmy, adaptací a učením s neznámým výukovým materiálem během krátké doby.
V této studii jsem zkoumal, jak zrakově postižení lidé reagují na nový druh braillsého řádku FCHAD a jak se ho naučí používat. Pedagogické prostředí této studie je přísné a učení je založené na drilování. Zkoumání se zvláště zaměří na otázku adaptace během přísného drilování.  Studie se podobá psychologickým výzkumům křivek učení s tím rozdílem, že pedagog je plnohodnotná součást studie.
