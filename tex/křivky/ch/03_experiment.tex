\chapter{Výzkum}

\section{Popis studie}

\subsection{První Část: Čtení Tisku}

Účastnici nebo možno lepe žáci v teto studiu nejdřív poseděli u prázdného stolu.  Nevěděli co se stane kromě tomu, že budou vyzkoušet nový druh braillského řádku.  FCHAD byl na stole ale položený do zády. Tak aby nepřekážel ruk.  Hned od začátku studie účastnici měli úkol.  Měli číst následující text.

\begin{verbatim}
Děkuji za Vaši účast v této
studii.
Vyzkoušíme nový braillský řádek!

V rámci této studie shromáždím ně-
jaká data o Vás a o tom, jak čtete
Braillovo písmo.  Tato data budou
při zveřejnění anonymní. Jedná se
o Váš přibližný věk, dosažené vzdě-
lání, zkušenost s Braillovým pís-
mem, audionahrávku našeho setkání a
záznam Vaší práce s nástrojem. Než
budeme pokračovat, potřebuji Váš
souhlas se shromážděním a zveřej-
něním těchto dat.

Následující text čtěte prosím na-
hlas a bez přestání až do konce.
Pokud budete mít nějaké otázky,
zeptejte se prosím až potom.

Začátek textu:
Dnes budete seznámen s novým druhem
braillského řádku, který se nazývá
FCHAD.  Tento řádek je navržen
s úmyslem vytvořit nejlevnější možný
                                   1

---- konec stranky ----
\end{verbatim}
\begin{verbatim}

zobrazovač vůbec. Zobrazuje text po
jednotlivých písmenech. Zobrazení
se ovládá pomocí řádku senzorů. Po-
ložením prstu na jeden z dvanácti
kovových bodíků vybíráte, které
z písmen v řádku chcete zobrazit.
Používáte obě ruce: jedna vybírá
písmeno, druhá ho čte.

Konec textu.

Teď Vám položím několik otázek
a poté vyzkoušíme nástroj.
 ---------------------
\end{verbatim}

Text má 138 písmen včetně mezery na první straně a 4 znaků velkých písmen, celkem 142 znaků.

Druhý část má 262 písmen včetně mezery a 3 znaků velkých písmen, celkem 265 znaků.

Celý text má 407 znaků.

Četli na hlas.  Uvedení v podsekcí žáků je rychlost jejích čtení. Text byl tisknutí na tuhém kusu braillského papíru velký 26.5 x 30.5cm.  Maximální délka řádku je 36 písmen.  Braillovo písmo je šestibodové.  Text byl tisknutí na obě straně, a na jedné straně papíru je sloupec děr aby papíry mohly byt vázané.

Text byl tisknutí v Knihovně a tiskárna pro nevidomé K.E.Macana, která je jediný veřejný tiskařský závod pro nevidomé v Praze.  Všechní účastnici studie již měli zkušenosti s materiály tisknutí u tiskárny a předpokládám, že tiskové Braillovo písmo bylo velmi standardní.

Když účastník dočetl text musel čekat par vteřinu abych připravil FCHAD. Pokud ještě nedali souhlas k zveřejnění informaci jsem požádal o souhlas.  Žádný z mých účastníků odmítly, a nikdo který jsem pozval k studie odmítl.

\subsection{Otázky}

Po čtení a přípravě jsem se zeptal účastníků následující otázky:

Jak jste starý?

Jak dlouho čtete braillské písmo?

Jaký je Váš nejvyšší dosažený úroveň vzdělání?

Jaký je Váš úroveň zrakové postižený?

Když jsem se je zeptal na věk, jsem vždy objasnil, že nepotřebuji přesný věk.  Věky uvedené v teto studie jsou přesné jenom do pěti let.

Zajímavě, všechní účastnici kromě jeden odpověděli \uv{úplně nevidomý} když jsem se je zeptal na jejich úroveň zrakové postižení.

WHO rozdělí zrakové postižený na 5 kategorii. Nejnižší dvě jsou:

\em \uv{Praktická slepota
zraková ostrost s nejlepší možnou korekcí 1/60 (0,02), 1/50 až světlocit nebo omezení zorného pole do 5 stupňů kolem centrální fixace, i když centrální ostrost není postižena, kategorie zrakového postižení 4

Úplná slepota ztráta zraku zahrnující stavy od naprosté ztráty světlocitu až po zachování světlocitu s chybnou světelnou projekcí, kategorie zrakového postižení 5} \em \citep{sonsklasifikace}

To, že nikdo nehlásil \uv{Praktická slepota} nebo \uv{praktická nevidomost} muže znamenat, že oni otázku nepochopili a mysleli si, že nemusejí odpovědět přesně.  Anebo to muže znamenat, že zkutečně byly všechní úplně nevidomé.

\subsection{Čtení na FCHADu}

Po otázek jsem přemístil FCHAD aby byl před účastnici.  Musel jsem taky vyzkoušet jestli to funguje správně. To byl první zkušenost pro účastnici s FCHADem.  Slyšeli jak cvakne ty kovové tyče, které vytvářejí písmeno na zobrazovače.  Potom co jsem zjistil, že software je spouštění správně jsem řekl účastníkům aby začali číst.  Jenom, že samozřejmě neuměli.  Vidomé lidí když setkají s novým nastroje činní. Oni koukají!  Nevidomé lidí nekoukají.  Sedí a nic nedělají.  Koukají do vzdálenosti bez pochybu.  Myslím si, že se boji hrabat na stroje, by ho mohli rozbit.  Není to nerozumná obava.

Musel jsem něco dělat aby svůj účastnici se snažili číst. Dal jsem jejích ruce na stroje tak jak by měly byt k správné čtení. Vzal jsem je levou rukou a dal jsem tu levou ruku na zobrazovač(viz Obrázek 1-A). Řekl jsem je \uv{tady se zobrazí to jedno písmeno.}  Pak jsem je vzal pravou rukou a dal jsem ji na selektor(vis Obrázek 1-B,C).

Teď je dobrý čas vysvětlit jak přesně vypadá a funguje FCHADy, které byli použití v studie.

\section{Popis FCHADu}

FCHAD má dvě součásti. Jeden je zobrazovač, kde se zobrazí jedno Braillovo písmeno, druhý je selektor s čím se to písmeno mění.

\subsection{První selektor}

První účastnici(2 a 3) používali jiný selektor než poslední.  To je kvůli technickým závadám v prvním selektoru.   Technologie selektorů je velmi jednoduchý.  Selektor má několik elektrod. Jedena katoda a vice anod. Když účastník tělem spojuje anod s katodou, elektrický proud jde přes tělo z anody k katodě. Podle toho, přes kterou anodou teče proud víme které písmeno je vybrané(selektováné).   První selektor měl 12 anod z napínáčků(vis Obrázek 1-C), a katoda v náramku(Obrázek 3).

Napínáčky jsou nepravidelně umístěné mezi 1.2 a 2 cm od sebe.  Celý selektor je 19cm od prvního senzoru k poslednímu.

Tento systém moc dobře nefungoval. Když jsem to zkusil doma fungoval to bezvadně ale když to pokusil nějaký chlupatější muž náramní katoda nedostala dobrý kontakt a stroj nevěděl kdy je dotyk a kdy ne.

\subsection{Druhý selektor}

Účastníci 4-8 měli lepší selektor(Vis obrázek 1-B). Nový selektor má dvě řádky elektrod. Horní řádek jsou anody a dolní řádek jsou katody.  Už proud nemusí projet tělem ale jenom kůži na prstu.  Tím způsobem, je daleko lepší spoj mezi anodou a katodou.  To ještě nebyl dostatečný.  Jsem musel namočit prsty účastníků vodou aby dostali spolehlivý spoj.

Další vylepšení je v tom, že senzory jsou v pravidelné umístění 0.5cm od sebe.  Jsou taky daleko blíž k sobě.

\subsection{Zobrazovač}
Další součást FCHADu je zobrazovač(vis Obrazy 1-A a 2).  Zobrazovač, který jsem použil v studie zobrazuje jedno osmibodové písmeno.  Zobrazovací technologie je zase velmi jednoduchá.  Jde to o cívka, která vybaví tyč aby pohyboval nahoru.  Taková cívka se říká solenoid.  Moje solenoidy vydají až 200 gramů síly\citep{multicomp}.  To je dost aby jejích pohyb bolelo.  Každý bod je jenom 0.45cm vysoký. Neboli to když nedržíte ruku na zobrazovače příliš silně.  Body nespadnou samy.  Čtenář musí položit ruku na zobrazovače aby správně fungoval.

Řádky bodů jsou 2cm od sebe a sloupci 3cm od sebe(pro účastnici 2 a 3, byly 5cm od sebe).  Celý zobrazovací prostor je 6 x 3cm.  Moje tři delší prsti jsou mezi 7 a 8.5cm dlouhé a proto dokážu číst písmena na zobrazovače prsty.  Některé účastnici zejména ženské měli kratší prsty a to nešlo.  Plastový báze Zobrazovače je 5.9cm široký(8cm u dřívější účastníci) 9.2cm dlouhý 6cm vysoký.

\subsection{Jak to jde dohromady}

FCHAD je přípojený na počítač a počítač má v pamětí tabulku 12ti písmen.  Písmena v tabulce vyplní speciální software, která se jmenuje \em čtečka obrazovky\em .  V studie používal jsem čtečku obrazovky Orca 3.6.3. Orca vyplní tabulku podle toho jaký text je blízko kursoru na obrazovce.

Selektorem čtenář vybírá, které z těch dvanácti písmen bude zobrazené na zobrazovače. Když pohybuje ruku na selektor čtenář slyší hlasitý cvaknutí jak solenoidy změní polohu.

Navrhoval jsem celou elektroniku stroje. Programoval jsem ovladač a firmware.  To mě dal možnost nahrávat fungování celý stroj.  Když jsem dělal výzkum, jsem mohl nahrávat lokace prstů uživatele, rychlost čtení, polohu ruky.  Napsal jsem další software, který pomáhá v shromáždění dat\footnote{\url{https://github.com/timthelion/anonGraph}}.  Mužete prohlidnout shromaždění data zde\footnote{\url{https://github.com/timthelion/fchad-study}}.

Přesnější detaily nastroje najdete v apendix A1.

\subsection{Text na FCHAD}
Tiskový text je na témě FCHADech(je to návod) a protože jsem chtěl mít srovnatelný obsah taky na nastroje jsem napsal krátký dějinný FCHADu.

Celý text, který jsem měl připravený najdete v apendixu.

\section{Moje žáci}

Účastnici studie byly všechní starší 18, tak že mohli dávat souhlas samostatně.  Účastnici nebyli zaplacené.

\subsection{Předběžná studie}

\subsubsection{Žáci 1 a 2}

První \uv{žák} o čem budu psát jsem Já Váš badatel.  Já jsem se učil číst rukama před dvěma roky. Zajímal jsem o Braillovo písmo protože mívám migrény když čtu.  Nějakou dobu jsem si řekl: \uv{naučím číst Braillovo písmo a už nebudu vůbec číst očima}.  Bylo to spíš emoční reakce na frustrace než praktický plán.  Rychle jsem se dozvěděl, že Braillovo písmo vůbec nejde číst rychle a, že braillský řádek je daleko příliš drahý koupit jenom vztekem.  Můj zájem o Braillova písma se změnil na hněv o tom jak je všechno pro nevidomé předražený a snahu stvořit něco levnější.

Umím číst rukama ale nijak plynule.  Už čtu daleko lepe FCHADem než obyčejný tisknutým Braillským písmem.  Je lehčí číst FCHADem protože písmo je velké.

Když jsem se začal číst na FCHAD četl jsem velmi pomalu.  Ještě jsem neznal Braillova písma dokonalý a často jsem spletl f \braillebox{124}, d \braillebox{145}, h \braillebox{125}, a j \braillebox{245}. Teď je už zvládnu bez problému.  Dozvěděl jsem, že připojený mezi \uv{psychologický rychlost čtení} a \uv{skuteční rychlost čtení} není moc silné.  Čtení se zda rychlejší, tím miň dělám chyby a tím víc rozumím text ale to neznamená, že zkutečně čtu rychlejší.  Čtení zrychluje daleko pomalejší než psychologický rychlost čtení.  Často jsem si myslel, že jsem dělal pokrok v svému schopnosti čtení ale když jsem koukal do nahrávky jsem dozvěděl, že rychlost či nezměnil, anebo jen mírně stoupal.

Pro mně, použity selektor je skoro bez myšlenek ale poznání znaků na zobrazovače stále někdy trvá.  Nastavěl jsem stroj tak, že se používá selektor pravou rukou a zobrazovač se čte levou. Dělal jsem to tak protože selektor je podobný jako počítačový myš a myš se používá pravou rukou(pro praváci).  Myslím si, že jsem to nastavěl obraceně a byl by lepší používat zobrazovač dominantní rukou.  Nastavený byl stejný pro všechní účastnici studie.

Dával jsem \uv{žáci 1 a 2} dohromady.  To je protože můj FCHAD vůbec nefungoval když jsem ho ukázal prvnímu účastníkovi. Nevím jakým kozlem to bylo ale když jsem držel Já katodu v ruce a používal stroj selektor fungoval bezvadně. Když on to vzal do ruky začal celý stroj zbláznit.




\subsubsection{Žák 3}

Setkání 6.3.2013

\uv{Žák} 3 je vysokoškolské vzdělaná žena ve věku mezi 45 a 50 let. Čte Braillovo písmo od začátku školní docházky.  Pracuje jako vysokoškolskou profesorkou němčiny.  Je úplně nevidomá.

Navštívil jsem ji během její konzultační hodiny což byly odpoledne hned po obědě.  Výzkum byl zpoždění tím, že jsme museli hledat prodlužovací kabel. Ona neměla zásuvku blízko pracovního stolu.  Ona se mě zdála čilá a zdravá.

Nehrabala s nástrojem když jsem to nastavil. Během naše setkání FCHAD trpěl technické problémy. Senzory někdy \uv{cítily} dotyky, které v zkutečnosti nebyly.  Zajímavě i když FCHAD moc dobře nefungoval ona byla schopná na to číst.

Začali jsme s tiskovém návodem:

Četla tiskový text obě ruce.

Čas na čtení celý tisknutý text: 93.3 vteřin (časy zaměřené ručně z nahrávek)

Počet znaků: 407

Rychlost čtení tisku: 4.36 CPS (počet znaků za sekundu)

Čas na čtení tisknutý text po otočený stranu: 51.8 vteřin

Počet znaků: 265

Rychlost čtení: 5.11 CPS

Ona jistě nebyla ve formě pro čtení tiskové Braillova písma.  Není divů kdy má doma braillský řádek.  V tiskovém Braillovu písmu řetězci velkých písmen jsou naznačen speciálním znakem: \braillebox{23} tak, že \uv{FCHAD} je psáno jako \braillebox{23}\braillebox{124}\braillebox{14}\braillebox{1}\braillebox{145}. Žák 3 snad nevnímala znak na řetězec velkých písmen.  Začala číst řetězec po jednotlivých písmen a pak říkala \uv{to budou asi čísla} a četla je jako čísla.  Dále, kdy dostala na konec stránky se chovala nějak překvapená.

Kdy se začala číst katoda nebyla v dobrém kontaktu s rukou.  Musel jsem to upravit.  Četla správně ale pomalu. Používala sevření prstu aby cítila každého tyče zvlášť.  Jelikož to funguje pro poznání písmena nejde tím způsobem číst rychle. Snažil jsem ji naučit používat zobrazovač správně.  Když jsem ji říkal, že by měla položit ruku rovně a dal jsem její ruku na zobrazovače správně říkala \uv{Já nevím, jak ta ruka jen tak leží mně to nevyhovuje, Já se musím ty body zvyknout a osahat trochu jinak.} %10:02.6

Položila ruku na selektor příliš daleko od sebe. Měla mít ruku napínači ale dala ruce na dráty, které vedou z nich.  Necítila správně kde jsou kraje mezi napínači a proto velmi těžko dozvěděla vzdálenost potřební posun mezi písmenem.  Nepochopila hned, že anody jsou dotykové a, že silnější zmačknutí nemá na ně vliv.  To je ale pochopitelní protože často nefungovaly. Snad by pochopila stroj lip kdybych mluvil líp česky.  Opakovaně jsem dělal chyby. Například jsem spletl slova \uv{písmo} a \uv{písmeno}.

Občas dělala chybu, že hmatala jenom na první sloupce bodů kdy četla písmena ze zobrazovače. Například, když bylo zobrazené \braillebox{24} myslela, že se zobrazuje jenom \braillebox{2}.

Protože anody jsou dotykové stačí když okraj prstů je v kontaktu s anodou aby se zaregistroval dotyk.  Zobrazuje se písmeno od nejlevější anody, která je dotýkaná.  Ona měla velké potíže, tím že si myslela že posunula ruku na selektor, a že by mělo zobrazit další písmeno ale protože její prst ještě byl v kontaktu s předcházející písmenem zobrazené písmeno zkutečně se nezměnilo.

\paragraph{ První řádek:}

\begin{tabular}{|c|c|c|c|c|c|c|c|c|c|c|c|}
\hline
B&r&a&i&l&l&s&&&&&\\
\braillebox{1278}&\braillebox{1235}&\braillebox{1}&\braillebox{24}&\braillebox{123}&\braillebox{123}&\braillebox{234}&\braillebox{}&\braillebox{2358}&\braillebox{123}&\braillebox{}&\braillebox{}\\
\hline
\end{tabular}

První řádek textu, který účastnicí četli na FCHADU byl kratší než selektor.  To byl určitě chyba na moje straně.  Orca se zobrazuje znaků \braillebox{2358}\braillebox{123} u konce řádků.  To znamená, že účastnici dočetli \uv{Braills} a pak najedeno byli konfrontovaný s tím, že jsou dvě znaky, které nepoznali.  Jsem jím říkal ať je nečtou a, že by měli pokračovat na další řádce.

FCHAD, který byl použitý v studie nemá žádné navigační tlačítka.  Aby účastníci pokračovali na další řádce museli jenom přesunout ruku k začátku řádku a pokračovat. Ovládal jsem změna řádku často bez jejích vědomi.

\paragraph{Druhý řádek:}

\begin{tabular}{|c|c|c|c|c|c|c|c|c|c|c|c|}
\hline
k&ý& &ř&á&d&e&k&,& &k&t\\
\braillebox{1378}&\braillebox{12346}&\braillebox{}&\braillebox{2456}&\braillebox{16}&\braillebox{145}&\braillebox{15}&\braillebox{13}&\braillebox{2}&\braillebox{}&\braillebox{13}&\braillebox{2345}\\
\hline
\end{tabular}


Další nezvyklost Orcy je, že pod čaruje začátky řádků. Například kdy 'k' je u začátku řádku, je psáno jako \braillebox{1378} a ne \braillebox{13}.  Do jisté míru to účastnici zmátlo.

Další problém pro žáka 5 byl kdy četla 'ý' \braillebox{12346}. Nesevřela na bodu 6 a proto to původně četla jako 'p' \braillebox{1234}.  Kdy jsem ji říkal, že četla nesprávně a že ji chyby ještě jeden bod, ona to našla, ale zase nesprávně odhadla, že písmeno bude 'q' \braillebox{12345} ale rychle sama upravila.

Zajímavě, kdy dorazila na mezera \braillebox{} říkala \uv{tady nemám nic}. Nebylo mi jasný, že pochopila to jako mezera než jsem ji to říkal.

Ona věděla, že jsem Američan.  Kdy dorazila na \braillebox{1235} myslela, že to je Americký 'w' a ne český 'ř'.  Snad nepochopila, že text zobrazený na FCHAD je český.

Kdy dorazila na 'á'\braillebox{16} říkala nahlas \uv{to je první a šesti bod, to je a s čárkou}.  Znamená to, že ona nemohla přímo z tvarů znaků pochopit vyznám. Musela explicitně o tom přemyslet.  Taky to znamená, že ji to pomohla přemyslet o čísla bodů, a že měla v explicitní paměti hned k dispozicí informace o tom, které body má 'á'.

Pokračovala tím, že volala nejdřív čísla bodů a potom, které písmeno to má byt.

Kdy dostala na ','\braillebox{2}, správně poznala, který bod je nahoru, ale nemohla říct, který znak to je i když čárka je úplně obyčejný a standardní znak v Braillovu písmu.  Když jsem ji říkal, že to je čárka hned pochopila.  Musel byt někdy v mozku tomu, že to má byt čárka, ale nějak ona to nespojila.

\paragraph{Třetí řádek}

\begin{tabular}{|c|c|c|c|c|c|c|c|c|c|c|c|}
\hline
e&r&ý& &t&e&ď& &p&o&u&ž\\
\braillebox{1578}&\braillebox{1235}&\braillebox{12346}&\braillebox{}&\braillebox{2345}&\braillebox{15}&\braillebox{1456}&\braillebox{}&\braillebox{1234}&\braillebox{135}&\braillebox{136}&\braillebox{2346}\\
\hline
\end{tabular}

Kdy se vrátila na začátek selektoru potom co četla druhý řádek textu vrátila se jenom asi polovinu tak daleko jak musela.  Byl jsem překvapení z toho protože selektor má jasně citelný okraj a jsem si myslel, že bude řídit podle citu a ne odhadem.  Je možný, že nechtěla dotykat na dotykové senzory kdy nečetla. Jinak nevidím důvod proč nepřejela prstem na začátku.

Na třetí řádce četla 'e', tím že pojmenovala na hlas čísla bodů ale kdy dostal na 'r' stroj začal rychle vypnout a zapnout(protože její prst neměl dobrý kontakt s senzorem).  Říkala 'r' bez toho, že by nějak hledala body!  Tak myslím si, že vibrace ji pomohlo víc než zmátl.

Je to zajímavý z pedagogické hlediska, protože jsem sám lekl tím, že stroj nefunguje a jsem se snažil ji vysvětlit, že prst není správně položení na senzor(a proto to tak vibruje).  Měl jsem ji jen nechat číst, protože četla správně.

Podruhy kdy četla 'ý' dělala stejnou chybu jako poprvé.

Třetí 'e' v slovu \uv{teď} už četla bez toho, že by říkala nahlas čísla bodů.  Znamená to, že třetím opakování už vidíme autonomizovace konkretní schopnosti?

Kdy četla 'ď'\braillebox{1456} říkala, že to používá jako lomítko(normálně \braillebox{12456}).  Uznala, že v standardní Braillovo písmo zobrazuje se 'ď'.

\paragraph{Čtvrtý řádek}

\begin{tabular}{|c|c|c|c|c|c|c|c|c|c|c|c|}
\hline
í&v&á&t&e&,& &n&e&n&í& \\
\braillebox{3478}&\braillebox{1236}&\braillebox{16}&\braillebox{2345}&\braillebox{15}&\braillebox{2}&\braillebox{}&\braillebox{1345}&\braillebox{15}&\braillebox{2345}&\braillebox{34}&\braillebox{}\\
\hline
\end{tabular}

Kdy se vrátila na začátku selektoru u čtvrtý řádek říkala, že moc na to nezorientuje.  Zase posunula krátce ale jsem ji neopravil.

I když dřív četla 'e' bez problému, zase zapomněla sevřít na druhém řádku bodů a četla 'e'\braillebox{15} jako 'a'\braillebox{1}.

Kdy pokračovala přesunula na selektor až moc(až na 'n') a jsem ji dal za úkol jít zpátky a hledat 't'.  Kdy četla písmena, které už četla dříve, četla je bez počítání bodů a rychle.

Kdy se vrátila na 'n'\braillebox{1345} nejdřív odhadla to jako 'd'\braillebox{145}. Četla 'e' správně, ale kdy dostala zase na další 'n' četla to jako 'm'\braillebox{134}.

Teď byla u předposledního písmena řádku 'í' a dělala chybu, že si myslela že už je konec řádku.  To je protože střed jejího prstu byl na poslední napínače ale okraj jejího prstu dotekl ještě jinde.

Je už obecně známý, že neopravené dotykové ovládání je nepřirozený pro lidí. Je normálně rozdíl mezi tím kde lidí si myslí že dotykají a kde kontakt mezí prstem a senzor zkutečně je.  Dotykové obrazovky používají speciální algoritmy odhadnout kde uživatel chtěl dotykat a psychologie dotyku je aktuální téma bádaní.

I když algoritmy jsou dostatečně rozvinuté, že používáte smartphony bez větší úsilí většina algoritmy jsou na základě zrakové vnímání dotyků.  Dotyk na obrazovce je vždy po nějaké oblasti ale počítač reaguje na dotyk jenom v jediném bodu.  Počítač musí odhadnout, který bod uživatel chtěl dotykat\citep{holz2011understanding}. Rozlišení selektor FCHADu je velmi málo na rozdílu od rozlišení dotykové obrazovky a psychologie dotyku u nevidomých je samozřejmě na základě hmatu a ne zraku.  FCHAD neví jestli je prst v plném kontaktu s senzorem anebo je senzor dotykován jenom z okraji.  Dále, vnímání dotyků uživatele FCHADu je hmatní a ne zrakový.  Nemůžeme bohužel jen kopírovat stavějící algoritmy.

\paragraph{paty řádek}

\begin{tabular}{|c|c|c|c|c|c|c|c|c|c|c|c|}
\hline
p&r&v&n&í& &ř&á&d&e&k&,\\
\braillebox{123478}&\braillebox{1235}&\braillebox{1236}&\braillebox{1345}&\braillebox{34}&\braillebox{}&\braillebox{1235}&\braillebox{16}&\braillebox{145}&\braillebox{15}&\braillebox{13}&\braillebox{2}\\
\hline
\end{tabular}

Na paty řádku myslela, že 'v'\braillebox{1236} je 'r'\braillebox{1235} protože si myslela, že paty bod je nahoru a ne šesti.

Stále neuměla přesunout ruku na selektor a stále ji přesunula příliš daleko do práva jak četla, tak že přeskočila několik písmem na raz.

Během čtení pátého řádku přestal stroj fungovat a musel jsem restartovat ovládací software.  Třetí účastník byla extrémně trpělivá a bych ji chtěl poděkovat za to!

Kdy jsem stroj upravil začala číst paty řádek zase od začátku. Opakování část řádku četla rychle a pokračovala v rychle čtení až dorazila na slovo \uv{řádek}.  Přečetla začátek slovo správně, ale četla 'e'\braillebox{15} jako 'á'\braillebox{16}.

Po paty řádku dokončili jsme setkání. Můj účastník ještě chtěla sdílet své dojmy.  O FCHADu říkala:

%330
\em \uv{Já si myslím, že ... Taklhe Vám řeknu, že jistě to víte sám, že ten řádek běžní je určitě komfortnější ale určitě než mít žádný řádek je lepší tenhle. Já si myslím, že na ten zobrazování těch písmen bych se zvykla myslím, že asi by to bylo dobré, kdyby každý uživatel by mohl trochu vybrat velikost, a například vy tady to máte rozděleno tou látkou a třeba osobně bych chtěla mít ty dva sloupce bodů naopak blíž. Protože Já mám tu ruku menší a užší. Teda jako za svou osobu, hlavní problém je v tom posunu jo, protože tam nějak ty aktivní body mě se zatím velice špatně hledají za svojí osobu jako veliko překážkou toho užívání bych především bych viděla v hledání těch písmena} \em



\subsubsection{Žák 4}

Setkání 8.3.2013

Čtvrtý žák je ve věku mezi 20 a 25 lety.  Jeho nejvyšší dosažené vzdělání je maturita.  Je můj spolužák na katedře angličtiny PedF UK. Je úplně nevidomý a také trpí problémy se sluchem.  Pracuje s Braillovým písmem od základní školy. Používá braillský řádek doma na počítači.

Měl velké potíže s přesunem na selektoru.

Když jsem mu dal číst tištěný návod, zeptal se mě, jestli má číst nahlas.  Řekl jsem, že ne. To byla chyba. První část listu se skutečně nahlas číst nemusí, a tak jsem čekal, až dorazí k větě \uv{Následující text čtěte prosím nahlas a bez přestání až do konce.} a začne číst nahlas. Jenže on četl dál potichu.  Z toho důvodu jsem měřil jeho rychlost čtení tištěného Braillova písma jenom od druhé strany listu. Je zajímavé, že žáci často nereagovali na příkazy psané ve textu. Když žák 4 skončil se čtením, potvrdil, že souhlasí se zveřejněním osobních údajů.  Oproti mým očekáváním žádný z mých žáků nedal svůj souhlas během čtení v momentě, když k tomu byl textem vyzván. Někteří z nich nepochopili z věty \uv{Než budeme pokračovat, potřebuji Váš souhlas se shromážděním a zveřejněním těchto dat.}, že by ten souhlas měli vůbec vyslovit nahlas.

Žák 4 četl tištěný text oběma rukama.

Čas čtení tištěného textu: 43 vteřin %169.2-126.1

CPS: 9.46*

Čas čtení tištěného textu po otočení listu:25.9 vteřin %169.2-143.3

CPS: 10.23**

*) Na moji žádost přečetl text dvakrát, abych ho mohl změřit. Jednou četl potichu a podruhé nahlas.

**) Četl jenom nahlas.

Poté, co jsem žákovi položil otázky, přemístil jsem před něj FCHAD.  Řekl jsem mu, ať nástroj prozkoumá.  On však seděl a nic nedělal.  Třeba jsem ho měl nechat sedět déle. Jak jsem tam s ním seděl, zdálo se mi, že tam sedíme velmi dlouho. Podle zvukové nahrávky jsem však čekal jenom deset vteřin.

Protože žákyně 3 měla tolik problémů se selektorem, rozhodl jsem se čtvrtému žákovi lépe vysvětlit fungování selektoru.  Umístil jsem jeho ruku na selektor a vedl jeho prsty nad senzory. Tím, že se jeho prsty dotkly senzorů, zobrazovač cvakal jako obvykle při změně písmene.  Řekl jsem mu, že ten zvuk je zvuk změny písmene.  Až poté, co jsem mu vysvětlil fungování selektoru, jsem jeho levou ruku umístil na zobrazovač.

Zrovna když jsem mu ukazoval zobrazovač, zobrazilo se na něm písmeno `o'. Zmínil jsem, které písmeno se zobrazuje.

Ještě během ukázky přestal nástroj reagovat. Naše setkání bylo mnohokrát přerušeno technickými potížemi.  Katoda neměla dobrý kontakt, protože žák měl ochlupené zápěstí.  Body se proto strašně třásly.

\paragraph{První řádek}

\begin{tabular}{|c|c|c|c|c|c|c|c|c|c|c|c|}
\hline
B&r&a&i&l&l&s&&&&&\\
\braillebox{1278}&\braillebox{1235}&\braillebox{1}&\braillebox{24}&\braillebox{123}&\braillebox{123}&\braillebox{234}&\braillebox{}&\braillebox{2358}&\braillebox{123}&\braillebox{}&\braillebox{}\\
\hline
\end{tabular}

Konečně jsme začali číst.  Žák 4 během čtení nepojmenovával body.  Odhadl nejdřív, že první písmeno je `l'\braillebox{123}, a když jsem mu řekl, že není, opravil se, že je to `b'.

Když pokračoval ve čtení, katoda stále neměla dobrý spoj.  Protože jsem věděl, že budou s katodou problémy, měl jsem ještě jednu (ve formě měděné samolepicí pásky) nalepenou na zobrazovači. Snažil jsem se žákovi vysvětlit, že stroj bude fungovat lépe, když bude v kontaktu s druhou katodou.

Přečetl `r' správně a stroj zase začal bláznit.  Vysvětlil jsem mu problém s kontaktem s katodou a poprosil ho, aby držel katodu v ruce.  Pak četl další dvě písmena správně a poté stroj zase nefungoval.

Po pár pokusech resetování stroje už fungoval až do konce setkání. Žák četl znovu od začátku prvního řádku a četl první tři písmena správně. Přeskočil jedno písmeno na selektoru a já mu řekl, ať se vrátí zpět.  Četl `i'\braillebox{24} jako `c'\braillebox{14}. Když jsem mu řekl, že četl nesprávně, používal stejný způsob uvažování, který žákyně 3 používala po celou dobu - říkal nahlas, které body jsou zvýšené: \uv{To je bod jedna čtyři}. \uv{Není to bod jedna,} odpověděl jsem. Potom už se opravil správně sám.

Na rozdíl od žákyně 3 pochopil hned, že čte česká slova.  Když viděl, že další písmeno je `l', vyslovil \uv{Brail}. Četl ještě `l' a `s' správně a doplnil \uv{Braills}.

Než abych mu vysvětloval, jaké znaky používá Orca k označení konce řádku, jednoduše jsem mu řekl, ať se vrátí na začátek textu. Zpětně si myslím, že to byla chyba.

\paragraph{Druhý řádek}

\begin{tabular}{|c|c|c|c|c|c|c|c|c|c|c|c|}
\hline
k&ý& &ř&á&d&e&k&,& &k&t\\
\braillebox{1378}&\braillebox{12346}&\braillebox{}&\braillebox{2456}&\braillebox{16}&\braillebox{145}&\braillebox{15}&\braillebox{13}&\braillebox{2}&\braillebox{}&\braillebox{13}&\braillebox{2345}\\
\hline
\end{tabular}

Když začal číst druhý řádek a dostal se k `s' a `ý', doplnil, že dočetl \uv{Braillský} a pak domýšlel, co se zobrazí dál (`ř'), aniž by přesouval ruku na selektoru.  Upozornil jsem ho, že stále zobrazuje `ý'.  Přesunul ruku na selektoru až k `ř' a řekl: \uv{Aha, tak tohle je opravdu `ř'}.  Přečetl slovo \uv{řádek} téměř plynule, kromě toho, že měl problém s přesunem na selektoru.

\paragraph{Třetí řádek}

\begin{tabular}{|c|c|c|c|c|c|c|c|c|c|c|c|}
\hline
e&r&ý& &t&e&ď& &p&o&u&ž\\
\braillebox{1578}&\braillebox{1235}&\braillebox{12346}&\braillebox{}&\braillebox{2345}&\braillebox{15}&\braillebox{1456}&\braillebox{}&\braillebox{1234}&\braillebox{135}&\braillebox{136}&\braillebox{2346}\\
\hline
\end{tabular}

Vrátil se k začátku selektoru bez pomoci, ale první písmeno `e' četl jako `á'\braillebox{16}. Označil body jako jedna a šest a já ho upozornil, že šestý bod není nahoře.  Nahmatal prstem díry na zobrazovači a počítal, který bod skutečně nahoře je.  Pak se opravil sám.

Kdy četl další slovo, \uv{teď}, `t' četl správně, ale zase zaměnil `e' za `á'.  Měl také problém s písmenem `ď'\braillebox{1456}, které původně četl jako `č'\braillebox{146}.  Po přečtení slov \uv{který} a \uv{teď} neřekl ani jedno z nich nahlas.  Následující `p' četl správně. Ačkoli měl trochu problém s posunem na selektoru, sám našel `o'.  Myslel si, že je u konce řádku.  Řekl jsem mu, že není. Přečetl `u' a pak už byl přesvědčený, že je u konce řádku. Upozornil jsem ho, že mu zbývá ještě jedno písmeno.  Snažil se pokračovat. Písmeno na zobrazovači začalo chaoticky vibrovat. Prohlásil jsem, že musím zjistit, jestli teď stroj funguje správně. Vyzkoušel jsem ho a fungoval. Vyzval jsem žáka, ať pokračuje. Četl `ž' jako `t'. Řekl jsem, že to není správně. Opravil se sám.

Ukončili jsme setkání.

Poté jsme ještě mluvili o FCHADech.  Nabídl mi další spolupráci a prohlásil, že \uv{kdyby to byla opravdu levná alternativa, tak by to bylo opravdu super}. Zmínil, že \uv{by bylo lepší, kdyby ty body byly blíž k sobě}.




\subsection{Běžná studie}

Po předběžné studie jsem věnoval dva intenzivních víkendů vylepšení nastroj a odstranění softwarové chyby.  Předělal jsem celý selektor a přenastavil jsem zobrazovač na nejužší nastavení.

Běžná studie se proběhla v gymnasium pro zrakově postižené a střední odborná škola pro zrakově postižení v Praze odpoledne 20.3.2013 a odpoledne 23.4.2013.

Účastnici posadili v učebně zeměpisu(která pro ně bylo známé místo).  Stole kde seděli je zobrazený v Obrazce 4.

{\bf Poznámka:}  U první tři účastníci běžného studie(Žáci 5,6, a 7) nebyl správně nastavení mikrofon a audio nahrávky z setkání jsou nízko-kvalitně.  Proto, popis setkání může obsahovat chyby.

\subsubsection{Žák 5}
Setkání 12:00 20.3.2013

Pátý žák je 20-25 let starý.  Jeho nejvyšší úroveň vzdělávání je maturita. Je úplně nevidomý.  Taký čte Braillova písma od základní školy.  Používá braillský řádek.

Když začal číst selektor zle choval.  Musel jsem přenastavit prahy.  To mně trval asi deset minut.  To znamenalo, že potom co on četl o stroje, a potom co on teprve dal ruce na stroje, on seděl a čekal.

Kdy četl tisknutí text nejdřív se zeptal, jestli má číst z obou stran.  Kdy četl, otočil stranu velmi rychle, tak že celou dobu plynule četl bez jakékoliv přestání.  Přečetl jedno slovo nesprávně: četl \uv{nejlevnější} jako \uv{nejlepší}.

Četl tištěný text oběma rukama.

% začatek 176.8 otočení 192.9 konec 212.6

Čas čtení tištěného textu:35.8 vteřin

CPS: 11.37

Čas čtení tištěného textu po otočení listu:19.7 vteřin

CPS: 13.45

Nejdřív jsem ho ukazoval selektor.  Popisoval jsem to jako \uv{dvanáct kovových bodů, vypadají jako písmo[sic] `k'.}.  Pak jsem ho ukazoval zobrazovač.  Řekl jsem ho, že \uv{má to osm dír[sic] a, že z každé z nich skáče kovový tyč.}

\paragraph{První Řádek}
\begin{tabular}{|c|c|c|c|c|c|c|c|c|c|c|c|}
\hline
B&r&a&i&l&l&s&&&&&\\
\braillebox{1278}&\braillebox{1235}&\braillebox{1}&\braillebox{24}&\braillebox{123}&\braillebox{123}&\braillebox{234}&\braillebox{}&\braillebox{2358}&\braillebox{123}&\braillebox{}&\braillebox{}\\
\hline
\end{tabular}

Přečetl první písmeno správně ale potom začali skákat tyče na zobrazovače náhodně, jsem uvědomoval, že senzory byly přecitlivé.  Musel jsem snížit citlivost aby fungovaly a protože jsem to dělal poprvé, trvalo mě to asi 10 minut.

Vedl jsem jeho ruku nad první tří senzory(abych věděl, že fungují) a řekl jsem na hlas `b' `r' `a'.  Tak že druhý dvě písmena sám nečetl.  Potřeboval jsem namočit jeho prst, aby ty senzory fungovaly. Tak jsem se ho zeptal kde mohu najít vodu a on mě řekl o umyvadle.  Namočil jsem jeho pravou ukazovátku a začaly jsme číst.

Četl začátek řádku správně. Jedinou chybu byl, že četl `s' jako `i'. Řekl to správně potom co jsem ho říkal, že to není správný.

\paragraph{Druhý Řádek}
\begin{tabular}{|c|c|c|c|c|c|c|c|c|c|c|c|}
\hline
k&ý& &ř&á&d&e&k&,& &k&t\\
\braillebox{1378}&\braillebox{12346}&\braillebox{}&\braillebox{2456}&\braillebox{16}&\braillebox{145}&\braillebox{15}&\braillebox{13}&\braillebox{2}&\braillebox{}&\braillebox{13}&\braillebox{2345}\\
\hline
\end{tabular}

Četl správně až na `,'.  Choval se, jak že nerozumí.  Zeptal jsem se, který bod je nahoru.  On správně říkal, že druhý.  Říkal jsem ho, že druhý bod je čárka.  Myslel, že je u konce řádku kdy dorazil na `k'.

\paragraph{Třetí Řádek}
\begin{tabular}{|c|c|c|c|c|c|c|c|c|c|c|c|}
\hline
e&r&ý& &t&e&ď& &p&o&u&ž\\
\braillebox{1578}&\braillebox{1235}&\braillebox{12346}&\braillebox{}&\braillebox{2345}&\braillebox{15}&\braillebox{1456}&\braillebox{}&\braillebox{1234}&\braillebox{135}&\braillebox{136}&\braillebox{2346}\\
\hline
\end{tabular}

Četl správně až na druhý `e'.  Ho zřejmě zmátl, to že body nespadnou samo. Kvůli nekvalitě nahrávky nevím co odhadl ale říkal \uv{nějaké body} a jsem odpověděl \uv{ano, nepadnou když netlačíte na ně}.  Zřejmě už sledoval slova, protože potom co přečetl \uv{teď} po jednotlivých písmenech, říkal \uv{teď}.

\paragraph{Čtvrtý Řádek}
\begin{tabular}{|c|c|c|c|c|c|c|c|c|c|c|c|}
\hline
í&v&á&t&e&,& &n&e&n&í& \\
\braillebox{3478}&\braillebox{1236}&\braillebox{16}&\braillebox{2345}&\braillebox{15}&\braillebox{2}&\braillebox{}&\braillebox{1345}&\braillebox{15}&\braillebox{2345}&\braillebox{34}&\braillebox{}\\
\hline
\end{tabular}

Četl `t' nejdřív jako `j'.  Četl první `n' nejdřív jako `d'.  Přeskočil druhý `e' a byl zase na `n'.  Říkal jsem ho, ať se vrátí zpátky a on to dělal a četl zbytek řádku správně.

\paragraph{Pátý Řádek}
\begin{tabular}{|c|c|c|c|c|c|c|c|c|c|c|c|}
\hline
p&r&v&n&í& &ř&á&d&e&k&,\\
\braillebox{123478}&\braillebox{1235}&\braillebox{1236}&\braillebox{1345}&\braillebox{34}&\braillebox{}&\braillebox{1235}&\braillebox{16}&\braillebox{145}&\braillebox{15}&\braillebox{13}&\braillebox{2}\\
\hline
\end{tabular}

Četl všechno správně.

\paragraph{Šestý Řádek}
\begin{tabular}{|c|c|c|c|c|c|c|c|c|c|c|c|}
\hline
 &k&t&e&r&ý& &z&o&b&r&a\\
\braillebox{78}&\braillebox{13}&\braillebox{2345}&\braillebox{15}&\braillebox{1235}&\braillebox{12346}&\braillebox{}&\braillebox{1356}&\braillebox{135}&\braillebox{12}&\braillebox{1235}&\braillebox{1}\\
\hline
\end{tabular}

U začátek řádku byl nějak překvapený.  Vysvětlil jsem mu, že body 7 a 8 \uv{znamenají} začátek řádku.  Jinak četl správně.

\paragraph{Řádek 7}
\begin{tabular}{|c|c|c|c|c|c|c|c|c|c|c|c|}
\hline
z&u&j&e& &j&e&n&o&m& &j\\
\braillebox{135678}&\braillebox{136}&\braillebox{245}&\braillebox{15}&\braillebox{}&\braillebox{245}&\braillebox{15}&\braillebox{1345}&\braillebox{135}&\braillebox{134}&\braillebox{}&\braillebox{245}\\
\hline
\end{tabular}

Četl všechno správně.

\paragraph{Řádek 8}
\begin{tabular}{|c|c|c|c|c|c|c|c|c|c|c|c|}
\hline
e&d&n&o& &p&í&s&m&e&n&o\\
\braillebox{1578}&\braillebox{145}&\braillebox{1345}&\braillebox{135}&\braillebox{}&\braillebox{1234}&\braillebox{34}&\braillebox{234}&\braillebox{134}&\braillebox{15}&\braillebox{1345}&\braillebox{135}\\
\hline
\end{tabular}

Četl všechno správně a jsem přestal potvrdit jeho čtení. Pote, zřejmě zrychlil.

\paragraph{Řádek 9}
\begin{tabular}{|c|c|c|c|c|c|c|c|c|c|c|c|}
\hline
.& & &P&r&v&n&í& &b&y&l\\
\braillebox{378}&\braillebox{}&\braillebox{}&\braillebox{12347}&\braillebox{1235}&\braillebox{1236}&\braillebox{1345}&\braillebox{34}&\braillebox{}&\braillebox{12}&\braillebox{13456}&\braillebox{123}\\
\hline
\end{tabular}

Četl všechno správně. Zeptal se, jestli má říct, že `P' byl velký.

\paragraph{Řádek 10}
\begin{tabular}{|c|c|c|c|c|c|c|c|c|c|c|c|}
\hline
 &v&y&n&a&l&e&z&e&n& &v\\
\braillebox{78}&\braillebox{1236}&\braillebox{13456}&\braillebox{1345}&\braillebox{1}&\braillebox{123}&\braillebox{15}&\braillebox{1356}&\braillebox{15}&\braillebox{1345}&\braillebox{}&\braillebox{1236}\\
\hline
\end{tabular}

Přečetl všechno správně.

\paragraph{Řádek 11}
\begin{tabular}{|c|c|c|c|c|c|c|c|c|c|c|c|}
\hline
 &r&o&c&e& &1&9&1&3& &v\\
\braillebox{78}&\braillebox{1235}&\braillebox{135}&\braillebox{14}&\braillebox{15}&\braillebox{}&\braillebox{18}&\braillebox{248}&\braillebox{18}&\braillebox{148}&\braillebox{}&\braillebox{1236}\\
\hline
\end{tabular}

Kdy dorazil na datum, říkal \uv{předpokládám, že to asi bude číslo    jedna devět   to znamená devatenáct set třináct}. Četl všechno správně.

\paragraph{Řádek 12}
\begin{tabular}{|c|c|c|c|c|c|c|c|c|c|c|c|}
\hline
 &A&n&g&l&i&i&.& & &J&m\\
\braillebox{78}&\braillebox{17}&\braillebox{1345}&\braillebox{1245}&\braillebox{123}&\braillebox{24}&\braillebox{24}&\braillebox{3}&\braillebox{}&\braillebox{}&\braillebox{2457}&\braillebox{134}\\
\hline
\end{tabular}

Četl `n' nejdřív jako `d'.  Četl `.' nejdřív jako mezera.  Jinak četl správně.

\paragraph{Řádek 13}
\begin{tabular}{|c|c|c|c|c|c|c|c|c|c|c|c|}
\hline
e&n&o&v&a&l& &s&e& &O&p\\
\braillebox{1578}&\braillebox{1345}&\braillebox{135}&\braillebox{1236}&\braillebox{1}&\braillebox{123}&\braillebox{}&\braillebox{234}&\braillebox{15}&\braillebox{}&\braillebox{1357}&\braillebox{1234}\\
\hline
\end{tabular}

Četl všechno správně kromě `O' což původně četl jako `n'.

\paragraph{Řádek 14}
\begin{tabular}{|c|c|c|c|c|c|c|c|c|c|c|c|}
\hline
t&o&f&o&n& &a& &p&ř&e&v\\
\braillebox{234578}&\braillebox{135}&\braillebox{124}&\braillebox{135}&\braillebox{1345}&\braillebox{}&\braillebox{1}&\braillebox{}&\braillebox{1234}&\braillebox{2456}&\braillebox{15}&\braillebox{1236}\\
\hline
\end{tabular}

Četl všechno správně.

\paragraph{Řádek 15}
\begin{tabular}{|c|c|c|c|c|c|c|c|c|c|c|c|}
\hline
á&d&ě&l& &s&v&ě&t&l&o& \\
\braillebox{1678}&\braillebox{145}&\braillebox{126}&\braillebox{123}&\braillebox{}&\braillebox{234}&\braillebox{1236}&\braillebox{126}&\braillebox{2345}&\braillebox{123}&\braillebox{135}&\braillebox{}\\
\hline
\end{tabular}

Četl všechno správně.

\paragraph{Řádek 16}
\begin{tabular}{|c|c|c|c|c|c|c|c|c|c|c|c|}
\hline
n&a& &z&v&u&k&.& & &K&d\\
\braillebox{134578}&\braillebox{1}&\braillebox{}&\braillebox{1356}&\braillebox{1236}&\braillebox{136}&\braillebox{13}&\braillebox{3}&\braillebox{}&\braillebox{}&\braillebox{137}&\braillebox{145}\\
\hline
\end{tabular}

Četl `z' nejdřív jako `e'. Jinak četl správně.

\paragraph{Řádek 17}
\begin{tabular}{|c|c|c|c|c|c|c|c|c|c|c|c|}
\hline
y&ž& &č&t&e&n&á&ř& &p&o\\
\braillebox{1345678}&\braillebox{2346}&\braillebox{}&\braillebox{146}&\braillebox{2345}&\braillebox{15}&\braillebox{1345}&\braillebox{16}&\braillebox{2456}&\braillebox{}&\braillebox{1234}&\braillebox{135}\\
\hline
\end{tabular}

Četl všechno správně.

Dokončil jsem setkání a on mě říkal:
\uv{to se hrozně špatně jako totiž poznává když člověk je zvyklý číst Braillovo písmo ... Nevlastně jenom tím, že ten taktilní bod má určitý zvyklost vlastně, jo, že pro mě je to těžší poznat}

Tim: \uv{Vy jste moc nepoložil tu ruku rovně na ten článek musím říct, že Já čtu docela rychlejší než jste četl vy teď na ten stroj tak m I když určitě čtu rukama pomalejší než vy}

Žák : \uv{jo jako jo jenom Já si myslím, že prostě, nevím no ale prostě, jenom že někomu kdo už má zkušenosti s normálním Braillovo písmo od začátku trošku problém}

Tim: \uv{poznal jste, že jsou tam slova?}

Žák: \uv{ano}
Tim: \uv{Rozuměl jste?}
Žák: \uv{Ano}

Pozoroval jsem, že pro něj bylo těžký zvyknout na malý selektor.  Byl zvyklý používat dlouhý braillský řádek. Kdy se vrátil na začátek selektorů kdy dočetl řádek textu často pohyboval svou ruku příliš daleko do levá.

\subsubsection{Žák 6}
Setkání 12:45 20.3.2013

Žák 6 je mezi 15-20 let stará. Čte Braillovo písmo od začátku školní docházky. Používá braillský řádek. Její nejvyšší dosažený úroveň vzdělávání je základní škola.  Je úplně nevidomá.

Četla tiskový text obě ruce.

% start 133.5 otočení 159.5 konec 193.9

Čas na čtení celého textu:60.4 vteřin

CPS: 6.7

Čas na čtení druhé strany tiskového textu: 34.4 vteřin

CPS: 7.7

Po otázek jsem namočil její pravou ukazovátku.  Vysvětlil jsem ji, že má dat ruku na první písmeno a odhadnout, které písmeno se zobrazuje.

\paragraph{První Řádek}
\begin{tabular}{|c|c|c|c|c|c|c|c|c|c|c|c|}
\hline
B&r&a&i&l&l&s&&&&&\\
\braillebox{1278}&\braillebox{1235}&\braillebox{1}&\braillebox{24}&\braillebox{123}&\braillebox{123}&\braillebox{234}&\braillebox{}&\braillebox{2358}&\braillebox{123}&\braillebox{}&\braillebox{}\\
\hline
\end{tabular}

Myslel nejdřív, že body jedna a tři jsou nahoře tak jsem se rozhodl vysvětlit lip zobrazovač.  Vysvětlil jsem, že jsou osm děr a jsem je ukazoval tím, že jsem její prst dával na ty díry.

Mumlala pro sebe \uv{to je} a sevřela na zobrazovače. Četla správně `l'.  Vysvětlil jsem ji, že pravou ukazovátka by měla byt na první senzor a dal jsem její prst tam.

Odhadla `v'.  Řekl jsem ji, které body byly nahoře.  Vysvětlil jsem ji, že to je osmibodové Braillovo písmo a, že nemusí dávat pozor na body sedm a osm.  Správně ztotožnila `b'.  Řekl jsem ji, ať posouvá ruku na selektor. Zeptala se \uv{do práva?}. Odpověděl jsem kladně a četla další písmeno správně.  Četla správně až na `i' což četla hodně pomalu ale správně.  Četla `l' správně podruhy. Není poznat na nahrávce její první pokus. Zbytek řádku četla správně.  Řekl jsem ji ať jde na další řádek potom co četla `s'.

\paragraph{Druhý Řádek}
\begin{tabular}{|c|c|c|c|c|c|c|c|c|c|c|c|}
\hline
k&ý& &ř&á&d&e&k&,& &k&t\\
\braillebox{1378}&\braillebox{12346}&\braillebox{}&\braillebox{2456}&\braillebox{16}&\braillebox{145}&\braillebox{15}&\braillebox{13}&\braillebox{2}&\braillebox{}&\braillebox{13}&\braillebox{2345}\\
\hline
\end{tabular}

Četla `k' nejdřív jako `a'.  Četla `ý' nejdřív(myslím) jako `p'.  Říkala \uv{nic} kdy dorazila na mezeru.  Říkal jsem ji, že \uv{nic} je mezera a ona odpověděla \uv{oh, aha}. Četla `ř' myslím nejdřív jako `t'.  Nevím co byl její první odhad pro `e' ale vysvětlil jsem ji, že body nespadnou samo a ona pak správně ztotožnila písmeno.  Tvářila se nějak překvapeně kdy se dorazila na `,' tak jsem ji řekl co to je.  Myslela, že je u konce řádku kdy dorazila na `k'. Četla `t' nejdřív jako `j'.

\paragraph{Třetí Řádek}
\begin{tabular}{|c|c|c|c|c|c|c|c|c|c|c|c|}
\hline
e&r&ý& &t&e&ď& &p&o&u&ž\\
\braillebox{1578}&\braillebox{1235}&\braillebox{12346}&\braillebox{}&\braillebox{2345}&\braillebox{15}&\braillebox{1456}&\braillebox{}&\braillebox{1234}&\braillebox{135}&\braillebox{136}&\braillebox{2346}\\
\hline
\end{tabular}

Její ruka byla moc malá na zobrazovače. Ačkoliv mohla pokryt celý zobrazovač, pokryla zobrazovače přesně.  Dala ruku na straně, tak že prsty hmatali na první tří body.  Body 4,5 a 6 hmatala sevřením. Četla všechno správně ale pomalu až na `o', což četla nejdřív jako `r'.  Kdy jsem ji řekl ať pojmenuje čísla bodů dělala to správně ale stále tvrdila, že se zobrazí `r'.  Říkal jsem ji, že je to `o'. Ona odpovídala \uv{jojojo, aha , a smála se}. Zbytek řádku četla správně.

\paragraph{Čtvrtý Řádek}
\begin{tabular}{|c|c|c|c|c|c|c|c|c|c|c|c|}
\hline
í&v&á&t&e&,& &n&e&n&í& \\
\braillebox{3478}&\braillebox{1236}&\braillebox{16}&\braillebox{2345}&\braillebox{15}&\braillebox{2}&\braillebox{}&\braillebox{1345}&\braillebox{15}&\braillebox{2345}&\braillebox{34}&\braillebox{}\\
\hline
\end{tabular}

Měla problém s posunem na selektor tím, že neposouvala dostatečně daleko.  Četla `v' nejdřív jako `l'.  Zbytek řádku četla správně.

\paragraph{Pátý Řádek}
\begin{tabular}{|c|c|c|c|c|c|c|c|c|c|c|c|}
\hline
p&r&v&n&í& &ř&á&d&e&k&,\\
\braillebox{123478}&\braillebox{1235}&\braillebox{1236}&\braillebox{1345}&\braillebox{34}&\braillebox{}&\braillebox{1235}&\braillebox{16}&\braillebox{145}&\braillebox{15}&\braillebox{13}&\braillebox{2}\\
\hline
\end{tabular}
Začatek řádku četla správně.  Četla `k' nejdřív jako `a'.  Zase nepoznala `,' ale věděla, že jenom druhý bod je nahoru.

\paragraph{Šestý Řádek}
\begin{tabular}{|c|c|c|c|c|c|c|c|c|c|c|c|}
\hline
 &k&t&e&r&ý& &z&o&b&r&a\\
\braillebox{78}&\braillebox{13}&\braillebox{2345}&\braillebox{15}&\braillebox{1235}&\braillebox{12346}&\braillebox{}&\braillebox{1356}&\braillebox{135}&\braillebox{12}&\braillebox{1235}&\braillebox{1}\\
\hline
\end{tabular}

Četla všechno správně.

\paragraph{Řádek 7}
\begin{tabular}{|c|c|c|c|c|c|c|c|c|c|c|c|}
\hline
z&u&j&e& &j&e&n&o&m& &j\\
\braillebox{135678}&\braillebox{136}&\braillebox{245}&\braillebox{15}&\braillebox{}&\braillebox{245}&\braillebox{15}&\braillebox{1345}&\braillebox{135}&\braillebox{134}&\braillebox{}&\braillebox{245}\\
\hline
\end{tabular}

Četla `n' nejdřív jako `d'. Jinak četla všechno správně.

\paragraph{Řádek 8}
\begin{tabular}{|c|c|c|c|c|c|c|c|c|c|c|c|}
\hline
e&d&n&o& &p&í&s&m&e&n&o\\
\braillebox{1578}&\braillebox{145}&\braillebox{1345}&\braillebox{135}&\braillebox{}&\braillebox{1234}&\braillebox{34}&\braillebox{234}&\braillebox{134}&\braillebox{15}&\braillebox{1345}&\braillebox{135}\\
\hline
\end{tabular}

Četla všechno správně.

\paragraph{Řádek 9}
\begin{tabular}{|c|c|c|c|c|c|c|c|c|c|c|c|}
\hline
.& & &P&r&v&n&í& &b&y&l\\
\braillebox{378}&\braillebox{}&\braillebox{}&\braillebox{12347}&\braillebox{1235}&\braillebox{1236}&\braillebox{1345}&\braillebox{34}&\braillebox{}&\braillebox{12}&\braillebox{13456}&\braillebox{123}\\
\hline
\end{tabular}

U první písmena koukala na mně s nepochopeným.  Vysvětlil jsme ji, že sedmi a osmi body nahoře naznačuji začátek řádku, a že třetí bod je tečka.  Četla zbytek správně. Nepoznala, že `P' je velký.  Anebo aspoň neřekla.  Četla `b'(správně) jako `,' protože vypadl drát v zobrazovače.  Spravil jsem to a pokračovala bez chyb.

\paragraph{Řádek 10}
\begin{tabular}{|c|c|c|c|c|c|c|c|c|c|c|c|}
\hline
 &v&y&n&a&l&e&z&e&n& &v\\
\braillebox{78}&\braillebox{1236}&\braillebox{13456}&\braillebox{1345}&\braillebox{1}&\braillebox{123}&\braillebox{15}&\braillebox{1356}&\braillebox{15}&\braillebox{1345}&\braillebox{}&\braillebox{1236}\\
\hline
\end{tabular}

Obecně nečetla mezery, jenom je přeskočila.  Je možný, že je nečetla protože bylo by zbytečně je číst.  Je taky možný, že řídila na selektor podle cvaknutí zobrazovače a ne podle citu senzoru.  Četla `y' původně nesprávně, myslím `ď' ale s tou nahrávkou jistý byt nemohu.  Zbytek řádku četla správně.

\paragraph{Řádek 11}
\begin{tabular}{|c|c|c|c|c|c|c|c|c|c|c|c|}
\hline
 &r&o&c&e& &1&9&1&3& &v\\
\braillebox{78}&\braillebox{1235}&\braillebox{135}&\braillebox{14}&\braillebox{15}&\braillebox{}&\braillebox{18}&\braillebox{248}&\braillebox{18}&\braillebox{148}&\braillebox{}&\braillebox{1236}\\
\hline
\end{tabular}

Četla začátek správně. Četla `1' jako `a'.  Vysvětlil jsem ji, že osmi bod naznačuje čísla.  Četla správně. Četla `9' jako `i'.  Zeptal jsem se ji, jestli osmi bod je nahoru.  Pak upravila se na `9'.  Zbytek četla správně.

\paragraph{Řádek 12}
\begin{tabular}{|c|c|c|c|c|c|c|c|c|c|c|c|}
\hline
 &A&n&g&l&i&i&.& & &J&m\\
\braillebox{78}&\braillebox{17}&\braillebox{1345}&\braillebox{1245}&\braillebox{123}&\braillebox{24}&\braillebox{24}&\braillebox{3}&\braillebox{}&\braillebox{}&\braillebox{2457}&\braillebox{134}\\
\hline
\end{tabular}

Četla začátek správně.  Nenaznačila, že `O' je velký. Myslela, že druhý `i' je `9' ale upravila se když jsem ji řekl, že není.  Kdy dorazila na `J' zase na mně koukala s nepochopením.  Viděl jsem, že pravě hmatala na body 2 a 5 ale ne na 4.  Řekl jsem ji \uv{ještě nahoře} a ona to správně odhadla.  Neřekla, že to je velký.

\paragraph{Řádek 13}
\begin{tabular}{|c|c|c|c|c|c|c|c|c|c|c|c|}
\hline
e&n&o&v&a&l& &s&e& &O&p\\
\braillebox{1578}&\braillebox{1345}&\braillebox{135}&\braillebox{1236}&\braillebox{1}&\braillebox{123}&\braillebox{}&\braillebox{234}&\braillebox{15}&\braillebox{}&\braillebox{1357}&\braillebox{1234}\\
\hline
\end{tabular}

Nevím co se přesně stal u začátek řádku, kvůli zavadám nahrávky.  Četla `o' opět jako `r' ale opravila se když jsem ji řekl, že to není správný.  Neřekla, že o je velké.

\paragraph{Řádek 14}
\begin{tabular}{|c|c|c|c|c|c|c|c|c|c|c|c|}
\hline
t&o&f&o&n& &a& &p&ř&e&v\\
\braillebox{234578}&\braillebox{135}&\braillebox{124}&\braillebox{135}&\braillebox{1345}&\braillebox{}&\braillebox{1}&\braillebox{}&\braillebox{1234}&\braillebox{2456}&\braillebox{15}&\braillebox{1236}\\
\hline
\end{tabular}

Četla `t' nejdřív jako `j'.  Četla `p' nejdřív jako `n'? Opravila se, když jsem ji řekl, že to není správný. Jinak četla správně.  Ale myslela, že je konec řádku už u písmena `e'.

\paragraph{Řádek 15}
\begin{tabular}{|c|c|c|c|c|c|c|c|c|c|c|c|}
\hline
á&d&ě&l& &s&v&ě&t&l&o& \\
\braillebox{1678}&\braillebox{145}&\braillebox{126}&\braillebox{123}&\braillebox{}&\braillebox{234}&\braillebox{1236}&\braillebox{126}&\braillebox{2345}&\braillebox{123}&\braillebox{135}&\braillebox{}\\
\hline
\end{tabular}

Myslela si, že `ě' je `v'.  Četla `s' jako `i' ale opravila se stejným dechem.  Taky četla `t' jako `j' a opravila se stejným dechem.

\paragraph{Řádek 16}
\begin{tabular}{|c|c|c|c|c|c|c|c|c|c|c|c|}
\hline
n&a& &z&v&u&k&.& & &K&d\\
\braillebox{134578}&\braillebox{1}&\braillebox{}&\braillebox{1356}&\braillebox{1236}&\braillebox{136}&\braillebox{13}&\braillebox{3}&\braillebox{}&\braillebox{}&\braillebox{137}&\braillebox{145}\\
\hline
\end{tabular}

Četla všechno správně.

\paragraph{Řádek 17}
\begin{tabular}{|c|c|c|c|c|c|c|c|c|c|c|c|}
\hline
y&ž& &č&t&e&n&á&ř& &p&o\\
\braillebox{1345678}&\braillebox{2346}&\braillebox{}&\braillebox{146}&\braillebox{2345}&\braillebox{15}&\braillebox{1345}&\braillebox{16}&\braillebox{2456}&\braillebox{}&\braillebox{1234}&\braillebox{135}\\
\hline
\end{tabular}

Četla `t' nejdřív jako `j' uvědomila, že to není správně stejným dechem. Četla `n' nejdřív jako `k'.  Vyslovila mezeru.

\paragraph{Řádek 18}
\begin{tabular}{|c|c|c|c|c|c|c|c|c|c|c|c|}
\hline
h&y&b&o&v&a&l& &s&p&e&c\\
\braillebox{12578}&\braillebox{13456}&\braillebox{12}&\braillebox{135}&\braillebox{1236}&\braillebox{1}&\braillebox{123}&\braillebox{}&\braillebox{234}&\braillebox{1234}&\braillebox{15}&\braillebox{14}\\
\hline
\end{tabular}

Četla všechno správně.

\paragraph{Řádek 19}
\begin{tabular}{|c|c|c|c|c|c|c|c|c|c|c|c|}
\hline
i&á&l&n&í&m& &p&e&r&e&m\\
\braillebox{2478}&\braillebox{16}&\braillebox{123}&\braillebox{1345}&\braillebox{34}&\braillebox{134}&\braillebox{}&\braillebox{1234}&\braillebox{15}&\braillebox{1235}&\braillebox{15}&\braillebox{134}\\
\hline
\end{tabular}

Četla všechno správně.

\paragraph{Řádek 20}
\begin{tabular}{|c|c|c|c|c|c|c|c|c|c|c|c|}
\hline
 &n&a&d& &p&í&s&m&e&n&e\\
\braillebox{78}&\braillebox{1345}&\braillebox{1}&\braillebox{145}&\braillebox{}&\braillebox{1234}&\braillebox{34}&\braillebox{234}&\braillebox{134}&\braillebox{15}&\braillebox{1345}&\braillebox{15}\\
\hline
\end{tabular}

Teď jsem se ji poprosil, aby přečetla slova, ne jednotlivá písmena.  V té chvílí, přestala reagovat nastroj, tak jsem to resetoval.  Četla \uv{nad} správně.

\paragraph{Řádek 21}
\begin{tabular}{|c|c|c|c|c|c|c|c|c|c|c|c|}
\hline
m&,& &p&ř&í&s&t&r&o&j& \\
\braillebox{13478}&\braillebox{2}&\braillebox{}&\braillebox{1234}&\braillebox{2456}&\braillebox{34}&\braillebox{234}&\braillebox{2345}&\braillebox{1235}&\braillebox{135}&\braillebox{245}&\braillebox{}\\
\hline
\end{tabular}

Četla \uv{písmenem} správně.  Četla \uv{přís} říkala něco(není rozumět na nahrávce), poradil jsem ji, ať hledá mezeru aby přečetla od začátku slova.  Našla mezeru, četla znovu, a správně řekla \uv{přístroj}.

\paragraph{Řádek 22}
\begin{tabular}{|c|c|c|c|c|c|c|c|c|c|c|c|}
\hline
b&z&u&č&e&l&.& & &T&o& \\
\braillebox{1278}&\braillebox{1356}&\braillebox{136}&\braillebox{146}&\braillebox{15}&\braillebox{123}&\braillebox{3}&\braillebox{}&\braillebox{}&\braillebox{23457}&\braillebox{135}&\braillebox{}\\
\hline
\end{tabular}

Četla \uv{bzučel} správně.  Řekl jsem ji, že tím ukončíme a zeptal jsem ji jestli má k tomu nějaké poznámky.  Ona řekla, že ne a rychle odešla.

\subsubsection{Žák 7}
Setkání 13:40 20.3.2013

Žák 7 je ve věku mezi 20 a 25 lety.  Braillovo písmo čte od začátku školní docházky.  Jeho nejvyšší dosažené vzdělání je maturita.  Je úplně nevidomý.

Četl tištěný text oběma rukama.

% začátek 85.2 otočení 106.7 konec 134.6

Čas čtení tištěného textu: 49.4 vteřin

CPS: 8.24

Čas čtení tištěného textu po otočení listu: 27.9 vteřin

CPS: 9.5

Ukazovák pravé ruky jsem žákovi namočil, aby lépe fungovaly senzory. Ukázal jsem mu selektor a pak zobrazovač stručně a bez delšího popsání.

\paragraph{První řádek}
\begin{tabular}{|c|c|c|c|c|c|c|c|c|c|c|c|}
\hline
B&r&a&i&l&l&s&&&&&\\
\braillebox{1278}&\braillebox{1235}&\braillebox{1}&\braillebox{24}&\braillebox{123}&\braillebox{123}&\braillebox{234}&\braillebox{}&\braillebox{2358}&\braillebox{123}&\braillebox{}&\braillebox{}\\
\hline
\end{tabular}

Řekl jsem žákovi, ať čte první \uv{písmo} a on namítl, že tam nic není.  Upozornil jsem ho, že musí mít ruku na selektoru. Položil ji na něj, ale ne na první senzor, jen náhodně.  Hádal `a', ale já jsem ho zastavil a vyzval, ať začne na prvním písmenu.  Nejdřív hádal `ě'\braillebox{126}, ale napodruhé uhádl správně.

Nehledal body, jen položil ruku na selektor a říkal, co si myslí.  Často odstranil ruku ze zobrazovače a pohyboval hlavou monotónním pohybem tak, aby tvář směřovala nahoru a doleva. Hlavou takto pohyboval skoro po každém písmenu, ale ne proto, že by byl rozptýlený - samostatně se vrátil ke čtení poté, co tento pohyb vykonal jednou až třikrát.

Druhé písmeno odhadl nejdříve jako `h', ale napodruhé odhadl správně. `a' četl správně. `i' přečetl na druhý pokus (napoprvé hádal čárku, ale opravil se, ještě něž to dořekl). `l' četl správně, další `l' přeskočil. Myslím si, že změnu písmena odhadoval podle cvaknutí zobrazovače a protože mezi `l' a `l' se nic nezměnilo, k žádnému cvaknutí nedošlo.  Neopravil jsem ho.  `s' četl správně.  Řekl jsem mu, ať se vrátí na začátek selektoru, abychom četli další řádek.

\paragraph{Druhý řádek}
\begin{tabular}{|c|c|c|c|c|c|c|c|c|c|c|c|}
\hline
k&ý& &ř&á&d&e&k&,& &k&t\\
\braillebox{1378}&\braillebox{12346}&\braillebox{}&\braillebox{2456}&\braillebox{16}&\braillebox{145}&\braillebox{15}&\braillebox{13}&\braillebox{2}&\braillebox{}&\braillebox{13}&\braillebox{2345}\\
\hline
\end{tabular}

Druhý řádek četl bez chyb kromě toho, že se jednou vrátil na začátek selektoru.  Udělal jsem dost vážnou chybu:  Opravil jsem ho nesprávně kvůli své nedostatečné znalosti češtiny. Nerozuměl jsem mu, když `ř' četl jako `eř', a tvrdil mu, že četl nesprávně.  Nikdy nevyslovoval mezery nahlas (nevyžadoval jsem to).

\paragraph{Třetí řádek}
\begin{tabular}{|c|c|c|c|c|c|c|c|c|c|c|c|}
\hline
e&r&ý& &t&e&ď& &p&o&u&ž\\
\braillebox{1578}&\braillebox{1235}&\braillebox{12346}&\braillebox{}&\braillebox{2345}&\braillebox{15}&\braillebox{1456}&\braillebox{}&\braillebox{1234}&\braillebox{135}&\braillebox{136}&\braillebox{2346}\\
\hline
\end{tabular}

Druhé `e' četl jako `o'.  To proto, že body nespadnou samy. Samostatně to pochopil a opravil se.  Poté, co přečetl `p', přeskočil několik písmen a četl správně `ž'. Když začal číst znovu od `p', četl `o' nejdřív jako `k', ale opravil se ještě dřív, než jsem  mu řekl, že to není správně.  Zbytek četl správně.

\paragraph{Čtvrtý řádek}
\begin{tabular}{|c|c|c|c|c|c|c|c|c|c|c|c|}
\hline
í&v&á&t&e&,& &n&e&n&í& \\
\braillebox{3478}&\braillebox{1236}&\braillebox{16}&\braillebox{2345}&\braillebox{15}&\braillebox{2}&\braillebox{}&\braillebox{1345}&\braillebox{15}&\braillebox{1345}&\braillebox{34}&\braillebox{}\\
\hline
\end{tabular}

První dvě písmena četl správně. `á' četl jako `u'\braillebox{136}.  Bál jsem se, že jeho chyba může být zapříčiněna vadným spojením v zobrazovači, ale zjistil jsem, že není.  Když se vrátil na začátek řádku, `v' četl jako `u'. `á' četl zase jako `u', ale nakonec správně uhádl `á'.  `e' četl jako `o'. Připomněl jsem mu, že body nespadnou samy.  Zbytek řádku četl správně až na `n', které četl jako `d', ale opravil se jedním dechem.

\paragraph{Pátý řádek}
\begin{tabular}{|c|c|c|c|c|c|c|c|c|c|c|c|}
\hline
p&r&v&n&í& &ř&á&d&e&k&,\\
\braillebox{123478}&\braillebox{1235}&\braillebox{1236}&\braillebox{1345}&\braillebox{34}&\braillebox{}&\braillebox{2456}&\braillebox{16}&\braillebox{145}&\braillebox{15}&\braillebox{13}&\braillebox{2}\\
\hline
\end{tabular}

Slovo \uv{první} četl bez chyb.  Audionahrávka se zasekla a nevím tedy, jak četl zbytek řádku.

\paragraph{Šestý řádek}
\begin{tabular}{|c|c|c|c|c|c|c|c|c|c|c|c|}
\hline
 &k&t&e&r&ý& &z&o&b&r&a\\
\braillebox{78}&\braillebox{13}&\braillebox{2345}&\braillebox{15}&\braillebox{1235}&\braillebox{12346}&\braillebox{}&\braillebox{1356}&\braillebox{135}&\braillebox{12}&\braillebox{1235}&\braillebox{1}\\
\hline
\end{tabular}

První část četl správně. `z' si spletl s `ř'. Zbytek četl správně.

Přesouval ruku na selektoru úplně samostatně a správně.

\paragraph{Sedmý řádek}
\begin{tabular}{|c|c|c|c|c|c|c|c|c|c|c|c|}
\hline
z&u&j&e& &j&e&n&o&m& &j\\
\braillebox{135678}&\braillebox{136}&\braillebox{245}&\braillebox{15}&\braillebox{}&\braillebox{245}&\braillebox{15}&\braillebox{1345}&\braillebox{135}&\braillebox{134}&\braillebox{}&\braillebox{245}\\
\hline
\end{tabular}

Četl správně.

\paragraph{Osmý řádek}
\begin{tabular}{|c|c|c|c|c|c|c|c|c|c|c|c|}
\hline
e&d&n&o& &p&í&s&m&e&n&o\\
\braillebox{1578}&\braillebox{145}&\braillebox{1345}&\braillebox{135}&\braillebox{}&\braillebox{1234}&\braillebox{34}&\braillebox{234}&\braillebox{134}&\braillebox{15}&\braillebox{1345}&\braillebox{135}\\
\hline
\end{tabular}

Četl správně až na `s', kvůli špatné kvalitě zvuku nevím přesně, co odhadl napoprvé.  `o' četl nejdřív jako  `a', ale opravil se jedním dechem.

\paragraph{Devátý řádek}
\begin{tabular}{|c|c|c|c|c|c|c|c|c|c|c|c|}
\hline
.& & &P&r&v&n&í& &b&y&l\\
\braillebox{378}&\braillebox{}&\braillebox{}&\braillebox{12347}&\braillebox{1235}&\braillebox{1236}&\braillebox{1345}&\braillebox{34}&\braillebox{}&\braillebox{12}&\braillebox{13456}&\braillebox{123}\\
\hline
\end{tabular}

Četl bez chyb.

\paragraph{Desátý řádek}
\begin{tabular}{|c|c|c|c|c|c|c|c|c|c|c|c|}
\hline
 &v&y&n&a&l&e&z&e&n& &v\\
\braillebox{78}&\braillebox{1236}&\braillebox{13456}&\braillebox{1345}&\braillebox{1}&\braillebox{123}&\braillebox{15}&\braillebox{1356}&\braillebox{15}&\braillebox{1345}&\braillebox{}&\braillebox{1236}\\
\hline
\end{tabular}

`v' četl nejdřív nesprávně, ale kvalita zvuku mi nedovolí říct jak.  Četl až na `e' správně, ale selektor začal špatně fungovat. Spravil jsem to a řekl mu, ať čte od začátku řádku.  Četl správně kromě prvního `e', které četl nejdřív jako `o', a mezery, kterou chvíli považoval za tečku.

\paragraph{Jedenáctý řádek}
\begin{tabular}{|c|c|c|c|c|c|c|c|c|c|c|c|}
\hline
 &r&o&c&e& &1&9&1&3& &v\\
\braillebox{78}&\braillebox{1235}&\braillebox{135}&\braillebox{14}&\braillebox{15}&\braillebox{}&\braillebox{18}&\braillebox{248}&\braillebox{18}&\braillebox{148}&\braillebox{}&\braillebox{1236}\\
\hline
\end{tabular}

`c' četl nejdřív jako `n'.  Číslici `1' četl nejdřív jako `a'. Vysvětlil jsem mu, že osmý bod signalizuje čísla.  Nevím, zda to pochopil, protože `9' nejdříve četl jako `i'.  Když jsem ho upozornil, že `i' je nesprávně, četl zbytek čísel správně.  `v' četl jako `l', ale hned to odvolal.  Pak hádal `r' a konečně správně `v'.

\paragraph{Dvanáctý řádek}
\begin{tabular}{|c|c|c|c|c|c|c|c|c|c|c|c|}
\hline
 &A&n&g&l&i&i&.& & &J&m\\
\braillebox{78}&\braillebox{17}&\braillebox{1345}&\braillebox{1245}&\braillebox{123}&\braillebox{24}&\braillebox{24}&\braillebox{3}&\braillebox{}&\braillebox{}&\braillebox{2457}&\braillebox{134}\\
\hline
\end{tabular}

Na začátku řádku jsem řešil problém se selektorem.  Musím poznamenat, že žák poznal, že selektor nefunguje, a čekal na mně. `i' četl nejdřív jako `e'.  Vím, že už dobře rozuměl selektoru, protože poznal, že jsou v textu dvě `i' a dvě mezery po tečce.  Zbytek řádku četl správně.

\paragraph{Řádek 13}
\begin{tabular}{|c|c|c|c|c|c|c|c|c|c|c|c|}
\hline
e&n&o&v&a&l& &s&e& &O&p\\
\braillebox{1578}&\braillebox{1345}&\braillebox{135}&\braillebox{1236}&\braillebox{1}&\braillebox{123}&\braillebox{}&\braillebox{234}&\braillebox{15}&\braillebox{}&\braillebox{1357}&\braillebox{1234}\\
\hline
\end{tabular}

`v' četl nejdřív jako `l', ale opravil se jedním dechem. Jinak četl správně.

\paragraph{Řádek 14}
\begin{tabular}{|c|c|c|c|c|c|c|c|c|c|c|c|}
\hline
t&o&f&o&n& &a& &p&ř&e&v\\
\braillebox{234578}&\braillebox{135}&\braillebox{124}&\braillebox{135}&\braillebox{1345}&\braillebox{}&\braillebox{1}&\braillebox{}&\braillebox{1234}&\braillebox{2456}&\braillebox{15}&\braillebox{1236}\\
\hline
\end{tabular}

Jednou jsem si myslel, že selektor nefunguje, protože nějak rychle cvakl zobrazovač. Ale když jsem ho vyzkoušel, zjistil jsem, že skutečně funguje správně.  Žák četl bez chyb.

\paragraph{Řádek 15}
\begin{tabular}{|c|c|c|c|c|c|c|c|c|c|c|c|}
\hline
á&d&ě&l& &s&v&ě&t&l&o& \\
\braillebox{1678}&\braillebox{145}&\braillebox{126}&\braillebox{123}&\braillebox{}&\braillebox{234}&\braillebox{1236}&\braillebox{126}&\braillebox{2345}&\braillebox{123}&\braillebox{135}&\braillebox{}\\
\hline
\end{tabular}

Četl bez chyb.

\paragraph{Řádek 16}
\begin{tabular}{|c|c|c|c|c|c|c|c|c|c|c|c|}
\hline
n&a& &z&v&u&k&.& & &K&d\\
\braillebox{134578}&\braillebox{1}&\braillebox{}&\braillebox{1356}&\braillebox{1236}&\braillebox{136}&\braillebox{13}&\braillebox{3}&\braillebox{}&\braillebox{}&\braillebox{137}&\braillebox{145}\\
\hline
\end{tabular}

Četl bez chyb a poznal, že `K' je velké.

\paragraph{Řádek 17}
\begin{tabular}{|c|c|c|c|c|c|c|c|c|c|c|c|}
\hline
y&ž& &č&t&e&n&á&ř& &p&o\\
\braillebox{1345678}&\braillebox{2346}&\braillebox{}&\braillebox{146}&\braillebox{2345}&\braillebox{15}&\braillebox{1345}&\braillebox{16}&\braillebox{2456}&\braillebox{}&\braillebox{1234}&\braillebox{135}\\
\hline
\end{tabular}

Četl správně až na `ř'. Navrhl jsem, že zkusíme nový způsob použití zobrazovače, a snažil jsem se žáka naučit, aby nechal ruku položenou na zobrazovači.  Během toho senzory přestaly reagovat a opět jsem musel resetovat stroj.

Začal číst znovu od začátku řádku. Audionahrávka je opět velmi nekvalitní, ale je slyšet, jak žák místo `y' hádá `ď'\braillebox{1456}, ale současně dává najevo, že si není jistý.  Ruku už nechal na zobrazovači.  Také nesprávně četl `č', ale není rozumět jak. Zbytek četl správně.

\paragraph{Řádek 18}
\begin{tabular}{|c|c|c|c|c|c|c|c|c|c|c|c|}
\hline
h&y&b&o&v&a&l& &s&p&e&c\\
\braillebox{12578}&\braillebox{13456}&\braillebox{12}&\braillebox{135}&\braillebox{1236}&\braillebox{1}&\braillebox{123}&\braillebox{}&\braillebox{234}&\braillebox{1234}&\braillebox{15}&\braillebox{14}\\
\hline
\end{tabular}

`h' četl správně na druhý pokus, prvnímu odhadu není rozumět.  `y' četl jako `d', ale ihned si uvědomil svoji chybu a na druhý pokus se opravil. Zbytek četl správně.

\paragraph{Řádek 19}
\begin{tabular}{|c|c|c|c|c|c|c|c|c|c|c|c|}
\hline
i&á&l&n&í&m& &p&e&r&e&m\\
\braillebox{2478}&\braillebox{16}&\braillebox{123}&\braillebox{1345}&\braillebox{34}&\braillebox{134}&\braillebox{}&\braillebox{1234}&\braillebox{15}&\braillebox{1235}&\braillebox{15}&\braillebox{134}\\
\hline
\end{tabular}

`i' četl nejdřív jako `s'.  `á' četl nejdřív jako `e'. Druhé `e' četl nejdřív jako `o'. `m' četl jako `p', ale opravil se jedním dechem.

Tím jsme ukončili setkání.  Zeptal jsem se ho: \uv{Myslíte, že byste se naučil to používat líp s praxí?} Odpověděl: \uv{Ne, to je těžký, já mám na to notebook, psací stroj.} Zeptal jsem se: \uv{A používáte braillský řádek?} On: \uv{Ne, hlasový vystup.}


\subsubsection{Žák 8}
Setkání 23.4.2013

Žák 8 je ve věku mezi 20 a 25 lety. Braillovo písmo čte od začátku školní docházky. Je úplně nevidomý. Je pravák.  Nepoužívá braillský řádek.

Četl tištěný text jednou(pravou) rukou.

% začátek 37.6 otočení 93.8 konec 180.7

Čas čtení tištěného textu: 143.1 vteřin

CPS: 2.84

Čas čtení tištěného textu po otočení listu: 87.1  vteřin

CPS: 3

Kdy jsem byl představený osmi žák hned chytil na to, že Timothy není české jméno a kdy jsem se zeptal osmi žákovi úvodní otázky, zkusil na mně mluvit anglicky, což dělal skoro bez přízvuku.

Když jsem ho ukazoval selektor, selektor začal poznat náhodné dotyky.  Po krátké době určení problém, jsem zjistil, že je tím, že účastník měl mokré ruce.

\paragraph{První řádek}

\begin{tabular}{|c|c|c|c|c|c|c|c|c|c|c|c|}
\hline
B&r&a&i&l&l&s&&&&&\\
\braillebox{1278}&\braillebox{1235}&\braillebox{1}&\braillebox{24}&\braillebox{123}&\braillebox{123}&\braillebox{234}&\braillebox{}&\braillebox{2358}&\braillebox{123}&\braillebox{}&\braillebox{}\\
\hline
\end{tabular}

Začali jsme číst první řádek tím, že se zeptal \uv{jak s tím na to}.  Začal jsem vysvětlení tím, že jsem ho ukázal ty díry na zobrazovače a pak kdy jeho pravá ruka byla na selektor zeptal jsem se které body jsou nahoře.  On říkal \uv{první} a odhadl `a', asi moje vysvětlení byla dost zmatený nepochopil jsem proč odhadl `ačko'.  Myslel, jsem si, že třeba myslí, že se zobrazí `á' a je zmatený z těch body 7 a 8.  Kdy měl prst na bod 8 jsem ho říkal, že to je pro velké písmena(což není pravda, bod 7 je pro velké písmena, bod 8 je pro čísla). Odhadl `v' a ještě `u'. Zřejmě nevěděl co to je osmibodové Braillovo písmo ale jsem to ještě nevěděl.  Začal jsem ho vykládat o selektoru.  Ale jak jsem ho vysvětlil, že jsou písmena `k' na selektor, hned odhadl, že zobrazené písmeno je `k'.

Jsem ještě vyřešil tomu, že měl mokré ruce a jsem malém leh tím, že selektor nefungoval.  Říkal jsem \uv{to mně dost překvapuje, aha!} když jsem viděl tu vodu.  On odpovídal \uv{Já nechci urazit, ale bych se nejdřív zeptal jestli už s tím umíte}.  Odpovídal jsem, že umím ale, že měl příliš mokré ruce na to a to chvíli nefungoval.

Zase jsem ho snažil vysvětlit zobrazovač.  Dal jsem jeho prst na každé díře a řekl jsem ho, které to je.  Teď jsem se dozvěděl, že snad v životě nesetkal s osmibodové Braillovo písmo protože se mně rovně zeptal na \uv{ty zbyli dva body}.  Jsem ho vysvětlil: \uv{Na počítače děláme velké písmeno tím, že dáváme nahoru asi sedmi bod a čísla tím že dáváme nahoru osmi bod}.

U toho jsem si rozhodl snížit citlivost senzorů, protože jenom tím jak jsem jeho ruku natřel papírovým kapesníkem nestačil.  Musel jsem restartovat celý softwarový sestavení.

Potom jsme se začaly znovu s poznání písmen.  Ještě na `B' jsem zase pojmenoval číslema body a dával jsem jeho prst na každý tyč nebo díru. Pojmenoval jsem a ukazoval jsem jenom první šest bodů abych ho zbytečně nezmátl.  Jasně musel cítit, že body jedna a dva jsou nahoře a žádné jiné ale stále odhadl nesprávně `včko'.  Vysvětlil jsem ho zase, že body sedm a osm nepoužíváme k poznání písmena a konečně on odhadl `bčko'.

Četl `r' nejdřív jako `h' ale odhadl správně potom co jsem ho říkal ať hledá dolu.

Říkal jsem ho, ať přesouvá na další písmeno a jsem musel zase utřít jeho ruce.  Byl tam dost vedro.  Natřel jsem selektor a on mě říkal: \uv{Bych měl radši braillský řádek na, kterém by to co bych napsal na počítači automatický zobrazoval.}

Říkal jsem ho jak má přesouvat na selektor ale četl `a' bez pomoci.

Odstranil ruku ze selektoru a jsem ho vysvětlil, že když nedotyká selektorem, tak se nezobrazí nic.  Dal ruku zpátky na selektor v přibližně správní místo a četl `i' jako `,' než jsem ho připomněl o další řádek bodů.

Zase odstranil ruku z selektoru ale vrátil ji zpět když jsem ho říkal ať čte další písmeno. Dal tu ruku na selektor o tři písmen dal než měl, ale četl správně `s'. Vrátil zpátky a četl `l' nejdřív jako `k'.

On začal hrát s selektorem a zobrazovačem.  Odstranil ruku z selektoru a pak poklepl na každý tyč který byl nahoru aby spadl.  Nechal jsem ho hrát 46 vteřin a pak on říkal tohle je `lko'(měl pravdu, tam se zkutečně `l' zobrazoval).  Pak jsem navrhoval, že čteme další řádek.

\paragraph{Druhý řádek}
\begin{tabular}{|c|c|c|c|c|c|c|c|c|c|c|c|}
\hline
k&ý& &ř&á&d&e&k&,& &k&t\\
\braillebox{1378}&\braillebox{12346}&\braillebox{}&\braillebox{2456}&\braillebox{16}&\braillebox{145}&\braillebox{15}&\braillebox{13}&\braillebox{2}&\braillebox{}&\braillebox{13}&\braillebox{2345}\\
\hline
\end{tabular}

Četl první písmeno jako `l', druhý odhad měl jako `v'.  Vysvětlil jsem mu, že body sedm a osm jsou tam jenom protože jsme u začátku řádek.  Odhadl `h' pak `u'.  Zase jsem dával jeho prst na každý bod a pojmenoval jsem je čísly. Pak jsem se zeptal, které body jsou nahoře.  On ukazoval na první bod a říkal \uv{To je ten sedmi} říkal jsem ho, že to není, že to je první.  Pak on říkal \uv{těžko říct, on je závorky}.  Říkal jsem ho \uv{Počítejte se mnou, jedna, dva, tři, čtyři, pět, šest, tak nahoře je bod jedna a bod tři, které písmeno to je} konečně správně říkal `kačko'.

Pokračovali jsme na další písmeno.  Nedržel ruku na selektor ale poklepal na senzor a pak prst odstranil.  Četl písmeno jako `l', pak `p' a konečně `ý'.

Odstranil zase ruku z selektoru, tak jsem si rozhodl zase snažit mu vysvětlit jak funguje.  Vedl jsem jeho prst nad `\uv{ty kačky}(ty senzory).

Vedl jsem jeho ruku na čtvrté písmeno, a četl to nejdřív jako `t' a potom, správně, jako `ř'.  Četl další písmeno `á' správně. Zase jsem musel utřít selektor a on utřel ruky na kalhotách a mně vykládal \uv{To je strašně přemodernizováné braillský řádek, vůbec se vtom nevyznám.} Vedl jsem jeho ruku na správné místo a jsme pokračovali.  Četl `d' nejdřív jako `č' než to četl správně.  On četl, tak že klepal na selektor a pak klepal každý bod na zobrazovač dole.  Četl `e' nejdřív jako `á', pak `o', pak `i'.  Konečně jsem ho jen říkal odpověď.  On říkal \uv{vůbec to není poznat} odpověděl jsem, že to jsou stejné písmena jenom větší, a on mě řek \uv{Já vím ale}.  Pořád klepal na ty senzory ale přibližně správně překračoval. Myslel, že chvili se zobrazoval `ě'.  Četl `k' správně.  Četl čárku správně.

Vysvětlil jsem mu, že už četl dvě slova.  Říkal mně, že to nepoznal.  Zeptal jsem se ho, jestli je unavený, ale říkal, že není.

Vedl jsem jeho ruku na předposlední znak, který četl správně jako `k'.  Pak začal zase ty senzory nefungovat a jsem je zase utřel.  On říkal anglicky \uv{I will get drunk}(budu opilý).  Vedl jsem jeho ruku zpátky na `t' který nejdřív četl jako `;'\braillebox{23} a pak správně.

\paragraph{Třetí řádek}
\begin{tabular}{|c|c|c|c|c|c|c|c|c|c|c|c|}
\hline
e&r&ý& &t&e&ď& &p&o&u&ž\\
\braillebox{1578}&\braillebox{1235}&\braillebox{12346}&\braillebox{}&\braillebox{2345}&\braillebox{15}&\braillebox{1456}&\braillebox{}&\braillebox{1234}&\braillebox{135}&\braillebox{136}&\braillebox{2346}\\
\hline
\end{tabular}

Na začátek třetí řádku odhadl nejdřív `k', pak `A', `t', `z'. Pak řekl, že neví. Odhadl `x' a jsem se zeptal, které body jsou nahoru. On říkal \uv{první, třetí} řekl jsem ho, že to je \uv{sedmi už}.  On říkal(správně) \uv{tuto logiku mizí}. Pak odhadl `ě' a jsem ho říkal \uv{ečko ale bez háčku}.  On mumlal \uv{tu logiku prostě mizí když}.  Další písmeno odhadl nejdřív jako `l' a pak správně jako `ý'. Četl `t' správně.  Pořád četl tím způsobem, že klepal krátce na selektor, a pak klepal na zobrazovače prstem. Je důležité poznamenat teď, že body na můj FCHAD nespadnou samy, musíte je pomoct. Říkal \uv{tohle nefunguje vůbec}. Vysvětlil jsem mu, že jenom body, které jsou pevné jsou významné. Četl `e' správně.  Pak četl `ď' jako `š'\braillebox{156}.  Připomněl jsem ho, že ještě ne zkusil hmatat na bodu čtyři.  Tento krát odhadl správně `ď'.  Četl tu mezeru jako čárka. Četl `p' nejdřív jako `t' pak `e', konečně jako `p'.  Četl `o' správně.  Přeskočil `u' ale četl `ř' správně.

\paragraph{Čtvrtý řádek}
\begin{tabular}{|c|c|c|c|c|c|c|c|c|c|c|c|}
\hline
í&v&á&t&e&,& &n&e&n&í& \\
\braillebox{3478}&\braillebox{1236}&\braillebox{16}&\braillebox{2345}&\braillebox{15}&\braillebox{2}&\braillebox{}&\braillebox{1345}&\braillebox{15}&\braillebox{2345}&\braillebox{34}&\braillebox{}\\
\hline
\end{tabular}
Četl, první písmeno nejdřív jako `ž', pak `i', a jsem ho řekl, že je to `í'.  Četl `v' a `á' správně přeskočil `t' ale četl `e' správně.  Řekl, že neví jaké to mělo byt slovo. Vysvětlil jsem mu, že je to slovo \uv{používáte} ale, že přeskočil písmena `u' a `t'.  Zeptal se, \uv{kde je učko} tak jsme se vrátili na předchozí řádku a jsem vedl jeho ruku k `o' pak `u' a `ž', písmena správně ztotožnil.

Pak jsme se začali číst od začátku řádku. Nejdřív, myslel, že `í' je `u' ale odhadl správně po druhý. Pak myslel, že se zobrazuje `,' ale ještě bylo nahoru `í'.  Zeptal se kolik je hodin a v stejném chvilce objevila v dveře sekretářka a zeptala se \uv{jak jste na to}. Říkal jsem, že už můžeme končit a žák 8 říkal, že by měl už jít.  Sekretářka ho řekla, že ještě má pět minut a jsme pokračovali.

Přeskočil ještě par písmen a jsme byli u `e' co ztotožnil správně. Myslel nejdřív je `n' je `g'. Přeskočil `í' a správně ztotožnil mezeru.  Neřekl jsem ho, že přeskočil, protože to by bylo zbytečné obtížení úkolu.

Dal jsem hu číst další řádek.

\paragraph{Pátý řádek}
\begin{tabular}{|c|c|c|c|c|c|c|c|c|c|c|c|}
\hline
p&r&v&n&í& &ř&á&d&e&k&,\\
\braillebox{123478}&\braillebox{1235}&\braillebox{1236}&\braillebox{1345}&\braillebox{34}&\braillebox{}&\braillebox{1235}&\braillebox{16}&\braillebox{145}&\braillebox{15}&\braillebox{13}&\braillebox{2}\\
\hline
\end{tabular}

Zeptal se \uv{jak ty řádky zadávají?}  Ukazoval jsem ho svůj notebook, a řekl jsem ho, že používám ty šípky abych přešel na další řádek.  Potom co jsem ho ukazoval ten počítač, nemohl znovu najít selektor, tak jsem musel jeho ruku tam dat.

Nejdřív myslel, že řádek začal `ý' ale pak správně řekl `p' když jsem ho připomněl o bodech sedm a osm.  Správně ztotožnil `r'.  Kdy přesunul ruku, zřejmě slyšel cvaknutí, protože zeptal se, jestli něco přeskočil.  Řekl jsem ho, že zkutečně přeskočil písmeno a vrátil se úspěšně na `v', myslel chvílí, že je `ě'\braillebox{126} a trval mu trochu, ale odhadl správně `v' podruhy.  Správně odhadl `n' a `í'.  Zeptal jsem ho, jestli pozná, které slovo to je, ale říkal, že ne.  Zeptal jsem se, jestli pamatuje první písmeno v řádku.  Řekl `y'.  Zase jsme se vrátili na začátek řádku a jsme zase uvažovali přes to, že jsou tam nahoru body sedm a osm.  Správně odhadl podruhy.  Zeptal se pak \uv{druhý byl co} a jsem ho říkal ať to najde sám.  Začal číst druhý písmeno jako `h'. Tím jsme ukončili, protože přišla sekretářka s dalším účastníkem.  Žák 8 říkal \uv{to je nad mou logiku} a rychle odešel.



\subsubsection{Žák 9}
Setkání 23.4.2013

Žák 9 je ve věku mezi 20 a 25 lety.  Jeho nejvyšší dosažené vzdělání je maturita.  Je úplně nevidomý a má problémy se sluchem.  Při našem setkání jsem na sobě měl mikrofon, který byl bezdrátově připojen k jeho sluchadlům.  Rozuměl mi bez znatelných potíží.

Četl tištěný text jednou (pravou) rukou.

%začatek 197.8 otočení 230.2 konec 279
Čas čtení tištěného textu: 81.2 vteřin

CPS: 5

Čas čtení tištěného textu po otočení listu: 48.8 vteřin

CPS: 5.4

Když jsem žákovi ukazoval zobrazovač, říkal jsem: \uv{Tady se zobrazí to písmeno. Tady je osm děr.} On vzal zobrazovač do obou rukou a začal dírky počítat. Nepočítal je ve standardním pořadí, jen je počítal.  Během celého setkání mluvil milým a překvapeným hlasem. Když si byl jistý, že je tam skutečně osm děr, vyzval jsem ho, aby položil levou ruku na zobrazovač.  Odpověděl se smíchem: \uv{Já jsem sice levák, ale čtu pravou}. Vysvětlil jsem mu, že současný stroj má jenom jedno nastavení.  Potom jsem mu ukázal selektor.  Řekl jsem, že zvuk, který vydává zobrazovač, je zvuk změny písmen, a že by měl taky cítit, jak se to písmeno změní.  On si začal se selektorem hrát, a tak jsem si myslel, že se samostatně pokusí o čtení.  Měl ruku správně položenou na zobrazovači a prst na prvním senzoru selektoru.  Občas říkal \uv{uf}.

Protože jsem se chtěl dozvědět, jestli bude číst sám, čekal jsem 146 vteřin.  Bohužel na audionahrávce slyším, jak na začátku této pauzy říkám \uv{um}, takže je docela možné, že žák čekal na mě.  Když jsem pochopil, že sám číst nebude, vrátil jsem se k výkladu o použití nástroje.

Vysvětlil jsem mu, že \uv{ty písmena `k'} jsou senzory.  Začal je počítat a když spočítal, že je jich 12, tak řekl \uv{aha}, jako že si je spojil s těmi dvanácti senzory zmíněnými v úvodním textu. Zase jsem ho nechal si je osahat a tentokrát jsem nedělal žádné matoucí \uv{um}.  Skutečně si s nástrojem hrál a já věřil, že se snaží číst. O 210 vteřin později však stále ještě nepokračoval.  Hmatal na zobrazovač, jako by se snažil z něj číst, ale nečetl. %881-671.9

\paragraph{První řádek}

\begin{tabular}{|c|c|c|c|c|c|c|c|c|c|c|c|}
\hline
B&r&a&i&l&l&s&&&&&\\
\braillebox{1278}&\braillebox{1235}&\braillebox{1}&\braillebox{24}&\braillebox{123}&\braillebox{123}&\braillebox{234}&\braillebox{}&\braillebox{2358}&\braillebox{123}&\braillebox{}&\braillebox{}\\
\hline
\end{tabular}


Zeptal jsem se ho, jaké je první písmeno, a on se zeptal \uv{Jak je to poznat?}.  Řekl jsem mu, ať počítá dírky.  Nejdřív mě nepochopil, zřejmě kvůli chybě mé češtiny.  Počítal, že jsou dva body nahoře, ale nepochopil ještě, o co jde.  Uchopil jsem jeho levou ruku a ukazoval mu \uv{To je jedna a dva, tak které písmeno to je?} Na to už bez přemýšlení odpověděl `b'.  

Bohužel žák 8 kymácel trupem tak, že jeho ruka na selektoru nebyla stabilní, a dále trpěl drobnými zrcadlovými pohyby, když se snažil číst na zobrazovači.  Myslím si, že už rozuměl, že by měl jít na druhé písmeno, i bez toho, že bych mu to řekl, ale nebyl toho fyzicky schopný.  Věděl, že jeho prsty se pohybují bez ohledu na jeho přání, a brzo ho to začalo frustrovat.  Držel jsem jeho prsty tak, aby se samovolně nepohybovaly, a vedl jsem je na selektoru.  Na `r' reagoval: \uv{první a pátý koukám}. Řekl jsem mu \uv{ještě}. Když objevil další body, řekl překvapeně: \uv{Druhý a třetí, aha, takže erko}.

`a' četl opět způsobem, že nejdřív jmenoval čísla bodů a písmeno až potom.  `l' už četl bez jmenování bodů. Další písmeno četl jako `p'. Když jsem ho řekl, že četl nesprávně, opravil se na `s': \uv{Aha, tohle zmizelo, tak pozor na to, co zmizí.} Body nespadnou, pokud na ně netlačíte, takže při změně z `l'\braillebox{123} na `s'\braillebox{234} se skutečně na chvíli zobrazilo `p'\braillebox{1234}.

\paragraph{Druhý řádek}
\begin{tabular}{|c|c|c|c|c|c|c|c|c|c|c|c|}
\hline
k&ý& &ř&á&d&e&k&,& &k&t\\
\braillebox{1378}&\braillebox{12346}&\braillebox{}&\braillebox{2456}&\braillebox{16}&\braillebox{145}&\braillebox{15}&\braillebox{13}&\braillebox{2}&\braillebox{}&\braillebox{13}&\braillebox{2345}\\
\hline
\end{tabular}

Přešli jsme na další řádek.  `k' četl bez problému. `ý' četl zase tak, že očísloval body.  Když jsme dorazili k mezeře, řekl \uv{nic} a rozesmál se. Písmena slova \uv{řádek} četl po jednotlivých písmenech, ale bez chyb a bez číslování bodů.  Četl tak dobře, že jsem na chvíli pustil jeho pravou ruku a nechal ho používat selektor samostatně.  Ruka se zase rytmicky pohybovala a on ji nedovedl ovládnout.  Znovu jsem ho tedy za ni uchopil, našli jsme místo, kde jsme skončili, a pokračovali jsme.

`k' četl nejdřív jako `a', ale sám se opravil, když jsem mu naznačil, že to není správně.

\paragraph{Třetí řádek}

\begin{tabular}{|c|c|c|c|c|c|c|c|c|c|c|c|}
\hline
e&r&ý& &t&e&ď& &p&o&u&ž\\
\braillebox{1578}&\braillebox{1235}&\braillebox{12346}&\braillebox{}&\braillebox{2345}&\braillebox{15}&\braillebox{1456}&\braillebox{}&\braillebox{1234}&\braillebox{135}&\braillebox{136}&\braillebox{2346}\\
\hline
\end{tabular}

Přešli jsme na třetí řádek.  Vždy, když se dostal k mezeře a nahmatal, že tam nic není zobrazeno, usmíval se.  Řekl \uv{mezera}, zřejmě rozradostněn tím, jak lehké je poznat mezery.  Četl správně až na `ž'\braillebox{2346}. Zaměnil ho za `z'\braillebox{1356}, které je jeho zrcadlovým obrazem. Řekl jsem: \uv{To je jako `z', ale obráceně.}  Nahlas vyjmenoval body: \uv{jedna tři pět šest, to je `z'.} čímž vyšlo najevo, že je čísluje obráceně. Upozornil jsem ho na to a on se polekal: \uv{Takže jsem to celou dobu říkal blbě!}  Uklidnil jsem ho, že do té doby četl správně, a on se krasně usmíval a říkal, že je úplně zmatený.

\paragraph{Čtvrtý řádek}

\begin{tabular}{|c|c|c|c|c|c|c|c|c|c|c|c|}
\hline
í&v&á&t&e&,& &n&e&n&í& \\
\braillebox{3478}&\braillebox{1236}&\braillebox{16}&\braillebox{2345}&\braillebox{15}&\braillebox{2}&\braillebox{}&\braillebox{1345}&\braillebox{15}&\braillebox{1345}&\braillebox{34}&\braillebox{}\\
\hline
\end{tabular}

Když jsme pokračovali na další řádek, `í'\braillebox{34} četl jako `á'\braillebox{16}.  Když jsem ho upozornil, že to není správně, tentokrát se úspěšně opravil sám.  Zase se u toho usmíval.  Následující `v' a `á' četl bez chyb.  `e' četl jako `i', ale tentokrát jsem udělal chybu já a neopravil ho. Čárku `,' četl správně.  Když jsme byli u `n'\braillebox{1345}, přemýšlel a čísloval body hodně dlouho (39 vteřin), než hádal `ž', které vzápětí sám odvolal. Čekal jsem dalších 18 vteřin a říkal mu: \uv{Je to `ž' vzhůru nohama.}  Nakonec jsem mu prozradil, že je to `n'.  Nevěřil mi a řekl: \uv{Mate to, že to je zrcadlově obrácený}. Konečně řekl \uv{Aha, už to vidím} a zase se krásně usmíval.  %1838.4

Během této diskuze o `n' jsme přeskočili dvě písmena.  Dostali jsme se k písmenu `í', což četl nejdřív jako `t', pak `á' a konečně `í'.

Zeptal jsem se ho, jestli není unavený, ale řekl, že je \uv{v pohodě}.

\paragraph{Pátý řádek}

\begin{tabular}{|c|c|c|c|c|c|c|c|c|c|c|c|}
\hline
p&r&v&n&í& &ř&á&d&e&k&,\\
\braillebox{123478}&\braillebox{1235}&\braillebox{1236}&\braillebox{1345}&\braillebox{34}&\braillebox{}&\braillebox{2456}&\braillebox{16}&\braillebox{145}&\braillebox{15}&\braillebox{13}&\braillebox{2}\\
\hline
\end{tabular}

Na dalším řádku četl písmena \uv{první ř} bez chyb. Dal velký důraz na `n'.  `á' opět četl nejdříve jako `í'. `d' a `e' četl správně, ale `k' četl nejdříve jako `a'. Čárku `,' četl správně.

\paragraph{Šestý řádek}
\begin{tabular}{|c|c|c|c|c|c|c|c|c|c|c|c|}
\hline
 &k&t&e&r&ý& &z&o&b&r&a\\
\braillebox{78}&\braillebox{13}&\braillebox{2345}&\braillebox{15}&\braillebox{1235}&\braillebox{12346}&\braillebox{}&\braillebox{1356}&\braillebox{135}&\braillebox{12}&\braillebox{1235}&\braillebox{1}\\
\hline
\end{tabular}

Šestý řádek četl správně až na `z', které nejdřív četl jako `n'.  `o' a `b' četl správně, ale `r' četl nejdřív jako `h'.

Je možné, že do této chvíle nepochopil selektor, protože jsem musel jeho pravou ruku vést. Když jsme byli u konce řádku, myslel, že se zobrazuje mezera.  Nic se nezobrazovalo, protože nebyl v kontaktu s žádným senzorem.

\paragraph{Sedmý řádek}
\begin{tabular}{|c|c|c|c|c|c|c|c|c|c|c|c|}
\hline
z&u&j&e& &j&e&n&o&m& &j\\
\braillebox{135678}&\braillebox{136}&\braillebox{245}&\braillebox{15}&\braillebox{}&\braillebox{245}&\braillebox{15}&\braillebox{1345}&\braillebox{135}&\braillebox{134}&\braillebox{}&\braillebox{245}\\
\hline
\end{tabular}

`z' četl jako `n', ale sám se hned opravil.  Vyzval jsem ho, ať zkusí posunovat pravou ruku na selektoru samostatně a přestal jsem ji držet.  Opět se pohybovala samovolně.  Zdál se mi tím být frustrovaný, tak jsem ji zase uchopil.   `u' četl nejdřív jako `a', ale když jsem mu řekl, že to není správně, opravil se.  `j'\braillebox{245} četl jako `h'\braillebox{125}.  Opět se opravil sám, když jsem mu řekl, že zaměnil zrcadlové obrazy. `e' a mezeru četl správně. `j' opět obrátil.  `e' četl správně, ale `n' četl jako `d'. Když jsem mu řekl, že to není správně, nejdřív si myslel, že si opět spletl zrcadlové obrazy. Vysvětlil jsem mu, že ne, a že je ještě nějaký bod dole.  Rychle poznal, že se zobrazuje `n'. `o' četl správně, ale `m' četl nejdřív jako `k'.  `j' četl jako čárku `,' Připomněl jsem mu, že je ještě další řádek bodů, než svůj odhad opravil. Jinak četl správně.

\paragraph{Osmý řádek}
\begin{tabular}{|c|c|c|c|c|c|c|c|c|c|c|c|}
\hline
e&d&n&o& &p&í&s&m&e&n&o\\
\braillebox{1578}&\braillebox{145}&\braillebox{1345}&\braillebox{135}&\braillebox{}&\braillebox{1234}&\braillebox{34}&\braillebox{234}&\braillebox{134}&\braillebox{15}&\braillebox{1345}&\braillebox{135}\\
\hline
\end{tabular}

Na osmém řádku textu četl správně až na `í', což zase obrátil jako `á'. `s' a `m' četl správně.  Myslel si, že už jsme na dalším písmenu, i když jsme nebyli.  Podruhé to samé `m'\braillebox{134} přečetl jako `c'\braillebox{14}.  `e' četl jako `o', ale sám objevil chybu.  `o' četl nejdřív jako `e'.

\paragraph{Devátý řádek}
\begin{tabular}{|c|c|c|c|c|c|c|c|c|c|c|c|}
\hline
.& & &P&r&v&n&í& &b&y&l\\
\braillebox{378}&\braillebox{}&\braillebox{}&\braillebox{12347}&\braillebox{1235}&\braillebox{1236}&\braillebox{1345}&\braillebox{34}&\braillebox{}&\braillebox{12}&\braillebox{13456}&\braillebox{123}\\
\hline
\end{tabular}

Devátý řádek začíná tečkou `.', ale byla zobrazena jako \braillebox{378}, protože jsme byli na začátku řádku.  Četl ji jako závorku, což je \braillebox{236}.  Připomněl jsem mu, že body sedm a osm jsou nahoře, protože se jedná o začátek řádku. Zeptal jsem se ho, které body jsou ještě nahoře.  Jeho druhý odhad, že to je znak pro velké písmeno\braillebox{6}, byl opět zrcadlově obrácený. Když jsem mu řekl, že velké písmeno to není, konečně uhodl správně.  Nebyl důvod, aby si myslel, že znak pro velké písmeno je bod šest a ne bod tři.  Tento znak se používá jenom v tištěném Braillovu písmu, kde nejsou žádné rámečky a mezi znaky tvořenými jediným bodem není znatelný rozdíl.  Zbytek řádku četl bez problému.

\paragraph{Desátý řádek}
\begin{tabular}{|c|c|c|c|c|c|c|c|c|c|c|c|}
\hline
 &v&y&n&a&l&e&z&e&n& &v\\
\braillebox{78}&\braillebox{1236}&\braillebox{13456}&\braillebox{1345}&\braillebox{1}&\braillebox{123}&\braillebox{15}&\braillebox{1356}&\braillebox{15}&\braillebox{1345}&\braillebox{}&\braillebox{1236}\\
\hline
\end{tabular}

Desátý řádek četl skoro bez chyb. Za jedinou jeho chybu jsem částečně mohl já, protože jsem vedl jeho ruce na selektoru. Táhl jsem jeho ruku příliš daleko od `n' na `l' a museli jsme se vrátit. Když `l' četl podruhé, myslel, že je to `b'. Opravil se dříve, než jsem stihl něco říct.

\paragraph{Jedenáctý řádek}
\begin{tabular}{|c|c|c|c|c|c|c|c|c|c|c|c|}
\hline
 &r&o&c&e& &1&9&1&3& &v\\
\braillebox{78}&\braillebox{1235}&\braillebox{135}&\braillebox{14}&\braillebox{15}&\braillebox{}&\braillebox{18}&\braillebox{248}&\braillebox{18}&\braillebox{148}&\braillebox{}&\braillebox{1236}\\
\hline
\end{tabular}

`r' zaměnil za `ř'.  `o' četl nejdřív jako `e'. Když dočetl \uv{roce}, zeptal jsem se ho, jestli pozná, které slovo četl.  Nevěděl.  Vrátili jsme se na začátek slova a on ho přečetl s tou chybou, že `e' četl nejdříve jako `a'. Vyslovil první polovinu slova, \uv{ro}, ale nepokračoval. Místo toho, abych ho s tím zbytečně otravoval, jsem mu prozradil, že to bylo slovo \uv{roce}.  Napověděl jsem mu, že teď zřejmě přijde číslo nebo datum.  Myslím, že jsem ho tím dost zmátl.  Nejdřív číslici 1\braillebox{18} četl jako `a'.  Řekl jsem mu, že to má být číslo a že bod osm je číselný bod. Stále nechápal.  Přemýšlel 4.5 vteřiny a pak řekl \uv{aha, jedna}. Devět četl nejdřív jako `i'.  Potom se zeptal, jestli to je také ještě číslo.  Správně doplnil, že to je devět. `1' a `3' četl správně. Doplnil jsem, že to je \uv{devatenáct set třináct}.  Mezeru a `v' četl správně.  Předtím, než uhádl `v', řekl \uv{předpokládám, že tohle už není číslo}. Nevím, jestli k tomuto závěru dospěl proto, že žádné číslo tvaru\braillebox{1236} není, nebo proto, že pochopil, že jsme dočetli datum.

\paragraph{Dvanáctý řádek}
\begin{tabular}{|c|c|c|c|c|c|c|c|c|c|c|c|}
\hline
 &A&n&g&l&i&i&.& & &J&m\\
\braillebox{78}&\braillebox{17}&\braillebox{1345}&\braillebox{1245}&\braillebox{123}&\braillebox{24}&\braillebox{24}&\braillebox{3}&\braillebox{}&\braillebox{}&\braillebox{2457}&\braillebox{134}\\
\hline
\end{tabular}

Zase jsem žáka nechal číst samostatně. Už zvládl ovládnout samovolný pohyb pravé ruky.  Vyřešil ho tak, že na selektor tlačil hodně silně.  Jestli nějaký pohyb byl, pohyboval se celý selektor. `n' četl nejdřív jako `d'. Měl velký problém s písmenem `g'\braillebox{1245}, které četl jako `x'\braillebox{1346}.  Musel jsem mu připomenout, aby posunul prst na selektoru.  `l' četl původně jako `b'.  `j' četl jako čárku `,' a `m' jako `c'.  Ruka se mu na selektoru pohybovala sama, ale snažil se to překonat a úspěšně četl dál.  U konce řádku si myslel, že je to mezera.

\paragraph{Řádek 13}
\begin{tabular}{|c|c|c|c|c|c|c|c|c|c|c|c|}
\hline
e&n&o&v&a&l& &s&e& &O&p\\
\braillebox{1578}&\braillebox{1345}&\braillebox{135}&\braillebox{1236}&\braillebox{1}&\braillebox{123}&\braillebox{}&\braillebox{234}&\braillebox{15}&\braillebox{}&\braillebox{1357}&\braillebox{1234}\\
\hline
\end{tabular}

První `o' četl nejdřív jako `e'. Druhé `O' četl nejdřív jako `k'. `p' četl nejdřív jako `l'.

Za úspěch považuji, že poslední dva řádky používal selektor samostatně. Když jsem se zeptal, jestli si myslí, že by se dokázal naučit samostatně číst, odpověděl \uv{no, snad ano}. Nezněl přesvědčeně. Když jsem se ho zeptal na jeho dojmy, řekl:\em \uv{No, spíš takhle jsem trochu zmaten z toho, že posouvám tou ... tabulkou, jestli to řeknu takhle, tou klávesnicí, jakoby tady tou pravou rukou, že jakoby spíš jak to funguje, když dám každé to písmeno, to znamená jeden ten jakoby posun.  Todle je pro mně zkouška, jak jsem občas myslel, že to písmeno je zrcadlově obráceně, i když jsem fakt občas měl pocit, nevím proč.}\em

