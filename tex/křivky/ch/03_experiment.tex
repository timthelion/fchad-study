\chapter{Výzkum}

\section{Popis studie}

\subsection{První Část: Čtení Tisku}

Účastnici nebo možno lepe žáci v teto studiu nejdřív poseděli u prázdného stolu.  Nevěděli co se stane kromě tomu, že budou vyzkoušet nový druh braillského řádku.  FCHAD byl na stole ale položený do zády. Tak aby nepřekážel ruk.  Hned od začátku studie účastnici měli úkol.  Měli číst text(v apendixu 3.1).  Četli na hlas. Text požádá souhlas o shromáždění dat taky vysvětli co to je FCHAD a jak se to používá. Text byl tisknutí na tuhém kusu braillského papíru velký 26.5 x 30.5cm.  Maximální délka řádku je 36 písmen.  Braillovo písmo je šestibodové.  Text byl tisknutí na obě straně, a na jedné straně papíru je sloupec děr aby papíry mohly byt vázané.

Text byl tisknutí v Knihovně a tiskárna pro nevidomé K.E.Macana, která je jediný veřejný tiskařský závod pro nevidomé v Praze.  Všechní účastnici studie již měli zkušenosti s materiály tisknutí u tiskárny a předpokládám, že tiskové Braillovo písmo bylo velmi standardní.

Když účastník dočte text musel čekat par vteřinu abych připravil FCHAD. Pokud ještě nedali souhlas k zveřejnění informaci jsem požádal o souhlas.  Žádný z mých účastníků odmítly, a nikdo který jsem pozval k studie odmítl.

\subsection{Otázky}

Po čtení a přípravě jsem se zeptal účastníků následující otázky:

Jak jste starý?

Jak dlouho čtete braillské písmo?

Jaký je Váš nejvyšší dosažený úroveň vzdělání?

Jaký je Váš úroveň zrakové postižený?

Když jsem se je zeptal na věk, jsem vždy objasnil, že nepotřebuji přesný věk.  Věky uvedené v teto studie jsou přesné jenom do pěti let.

Zajímavě, všechní účastnici kromě jeden odpověděli "úplně nevidomý" když jsem se je zeptal na jejich úroveň zrakové postižení.

WHO rozdělí zrakové postižený na 5 kategorii. Nejnižší dvě jsou:

\em "Praktická slepota
zraková ostrost s nejlepší možnou korekcí 1/60 (0,02), 1/50 až světlocit nebo omezení zorného pole do 5 stupňů kolem centrální fixace, i když centrální ostrost není postižena, kategorie zrakového postižení 4

Úplná slepota ztráta zraku zahrnující stavy od naprosté ztráty světlocitu až po zachování světlocitu s chybnou světelnou projekcí, kategorie zrakového postižení 5" \em \citep{sonsklasifikace}

To, že nikdo nehlásil "Praktická slepota" nebo "praktická nevidomost" muže znamenat, že oni otázku nepochopili a mysleli si, že nemusejí odpovědět přesně.  Anebo to muže znamenat, že zkutečně byly všechní úplně nevidomé.

\subsection{Čtení na FCHADu}

Po otázek jsem přemístil FCHAD aby byl před účastnici.  Musel jsem taky vyzkoušet jestli to funguje správně. To byl první zkušenost pro účastnici s FCHADem.  Slyšeli jak cvakne ty body na zobrazovače.  Potom co jsem zjistil, že software je spouštění správně jsem řekl účastníkům aby začali číst.  Jenom, že samozřejmě neuměli.  Vidomé lidí když setkají s novým nastroje činní. Oni koukají!  Nevidomé lidí nekoukají.  Sedí a nic nedělají.  Koukají do vzdálenosti bez pochybu.  Myslím si, že se boji hrabat na stroje, by ho mohli rozbit.  Není to nerozumná obava.

Musel jsem něco dělat aby svůj účastnici se snažili číst. Dal jsem jejích ruce na stroje tak jak by měly byt k správné čtení.

Vzal jsem je levou rukou, a jsem dal tu levou ruku na zobrazovač(viz Obrázek 1-A). Řekl jsem je \uv{tady se zobrazí to jedno písmeno.}  Pak jsem je vzal pravou rukou a dal jsem ji na selektor(vis Obrázek 1-B,C).

Teď je dobrý čas vysvětlit jak přesně vypadá a funguje FCHADy, které byli použití v studie.

\section{Popis FCHADu}

\subsection{První selektor}

První účastnici(2 a 3) používali jiný selektor než poslední.  To je kvůli technickým závadám v prvním selektoru.   Technologie selektorů je velmi jednoduchý.  Selektor má několik elektrod. Jedena katoda a vice anod. Když účastník tělem spojuje anod s katodou, elektrický proud jde přes tělo z anody k katodě. Podle toho, přes kterou anodou teče proud víme které písmeno je vybrané(selektováné).   První selektor měl 12 anod z napínáčků(vis Obrázek 1-C), a katoda v náramku(Obrázek 3).

Napínáčky jsou nepravidelně umístěné mezi 1.2 a 2 cm od sebe.  Celý selektor je 19cm od prvního senzoru k poslednímu.

Tento systém moc dobře nefungoval. Když jsem to zkusil doma fungoval to bezvadně ale když to pokusil nějaký chlupatější muž náramní katoda nedostala dobrý kontakt a stroj nevěděl kdy je dotyk a kdy ne.

\subsection{Druhý selektor}

Účastníci 4-8 měli lepší selektor(Vis obrázek 1-B). Nový selektor má dvě řádky elektrod. Horní řádek jsou anody a dolní řádek jsou katody.  Už proud nemusí projet tělem ale jenom kůži na prstu.  Tím způsobem, je daleko lepší spoj mezi anodou a katodou.  To ještě nebyl dostatečný.  Jsem musel namočit prsty účastníků vodou aby dostali spolehlivý spoj.

Další vylepšení je v tom, že senzory jsou v pravidelné umístění 0.5cm od sebe.  Jsou taky daleko blíž k sobě.

\subsection{Zobrazovač}
Další součást FCHADu je zobrazovač(vis Obrazy 1-A a 2).  Zobrazovač, který jsem použil v studie zobrazuje jedno osmibodové písmeno.  Zobrazovací technologie je zase velmi jednoduchá.  Jde to o cívka, která vybaví tyč aby pohyboval nahoru.  Taková cívka se říká solenoid.  Moje solenoidy vydají až 200 gramů síly\citep{multicomp}.  To je dost aby jejích pohyb bolelo.  Každý bod je jenom 0.45cm vysoký. Neboli to když nedržíte ruku na zobrazovače příliš silně.  Body nespadnou samy.  Čtenář musí položit ruku na zobrazovače aby správně fungoval.

Řádky bodů jsou 2cm od sebe a sloupci 3cm od sebe(pro účastnici 2 a 3, byly 5cm od sebe).  Celý zobrazovací prostor je 6 x 3cm.  Moje tři delší prsti jsou mezi 7 a 8.5cm dlouhé a proto dokážu číst písmena na zobrazovače prsty.  Některé účastnici zejména ženské měli kratší prsty a to nešlo.  Plastový báze Zobrazovače je 5.9cm široký(8cm u dřívější účastníci) 9.2cm dlouhý 6cm vysoký.

\subsection{Jak to jde dohromady}

FCHAD je přípojený na počítač a počítač má v pamětí tabulku 12ti písmen.  Písmena v tabulce vyplní speciální software, která se jmenuje \em čtečka obrazovky\em .  V studie používal jsem čtečku obrazovky Orca 3.6.3. Orca vyplní tabulku podle toho jaký text je blízko kursoru na obrazovce.

Selektorem čtenář vybírá, které z těch dvanácti písmen bude zobrazené na zobrazovače. Když pohybuje ruku na selektor čtenář slyší hlasitý cvaknutí jak solenoidy změní polohu.

Navrhoval jsem celou elektroniku stroje. Programoval jsem ovladač a firmware.  To mě dal možnost nahrávat fungování celý stroj.  Když jsem dělal výzkum, jsem mohl nahrávat lokace prstů uživatele, rychlost čtení, polohu ruky.  Napsal jsem další software, který pomáhá v shromáždění dat\footnote{\url{https://github.com/timthelion/anonGraph}}.  Mužete prohlidnout shromaždění data zde\footnote{\url{https://github.com/timthelion/fchad-study}}.

Přesnější detaily nastroje najdete v apendix A1.

\section{Moje žáci}

\subsection{Předběžná studie}

\subsubsection{Žáci 1 a 2}

První "žák" o čem budu psát jsem Já Váš badatel.  Já jsem se učil číst rukama před dvěma roky. Zajímal jsem o Braillovo písmo protože mívám migrény když čtu.  Nějakou dobu jsem si řekl: "naučím číst Braillovo písmo a už nebudu vůbec číst očima".  Bylo to spíš emoční reakce na frustrace než praktický plán.  Rychle jsem se dozvěděl, že Braillovo písmo vůbec nejde číst rychle a, že braillský řádek je daleko příliš drahý koupit jenom vztekem.  Můj zájem o Braillova písma se změnil na hněv o tom jak je všechno pro nevidomé předražený a snahu stvořit něco levnější.

Umím číst rukama ale nijak plynule.  Už čtu daleko lepe FCHADem než obyčejný tisknutým Braillským písmem.  Je lehčí číst FCHADem protože písmo je velké.

Když jsem se začal číst na FCHAD četl jsem velmi pomalu.  Ještě jsem neznal Braillova písma dokonalý a často jsem spletl f \braillebox{124}, d \braillebox{145}, h \braillebox{125}, a j \braillebox{245}. Teď je už zvládnu bez problému.  Dozvěděl jsem, že připojený mezi "psychologický rychlost čtení" a "skuteční rychlost čtení" není moc silné.  Čtení se zda rychlejší, tím miň dělám chyby a tím víc rozumím text ale to neznamená, že zkutečně čtu rychlejší.  Čtení zrychluje daleko pomalejší než psychologický rychlost čtení.  Často, jsem si myslel, že jsem dělal pokrok v svému schopnosti čtení ale když jsem koukal do nahrávky jsem dozvěděl, že rychlost či nezměnil, anebo jen mírně stoupal.

Pro mně, použity selektor je skoro bez myšlenek ale poznání znaků na zobrazovače stále někdy trvá.  Nastavěl jsem stroj tak, že se používá selektor pravou rukou a zobrazovač se čte levou. Dělal jsem to tak protože selektor je podobný jako počítačový myš, a myš se používá pravou rukou(pro praváci).  Myslím si, že jsem to nastavěl obraceně a byl by lepší používat zobrazovač dominantní rukou.  Nastavený byl stejný pro všechní účastnici studie.

Dával jsem "žáci 1 a 2" dohromady.  To je protože můj FCHAD vůbec nefungoval když jsem ho ukázal prvnímu účastníkovi. Nevím jakým kozlem to bylo ale když jsem držel Já katodu v ruce a používal stroj selektor fungoval bezvadně. Když on to vzal do ruky začal celý stroj zbláznit.


\subsubsection{Žák 3}

\uv{Žák} 3 je vysokoškolské vzdělaná žena ve věku mezi 45 a 50 let. Čte Braillovo písmo od začátku školní docházky.  Pracuje jako vysokoškolskou profesorkou němčiny.  Je úplně nevidomá.

Navštívil jsem ji během její konzultační hodiny.  Výzkum byl zpoždění tím, že jsme museli hledat prodlužovací kabel. Ona neměla zásuvku blízko stolu.  Ona se mě zdála čilá a zdravá.

Nehrabala s nástrojem když jsem to nastavil. Zase, FCHAD trpěl technické problémy. Senzory někdy \uv{cítily} dotyky, které v zkutečnosti nebyly.  Zajímavě i když FCHAD moc dobře nefungoval ona byla schopná na to číst.

Čas na čtení tisknutý text: 93.3 sekundy

Rychlost čtení tisku: 4.36CPS

Kdy se začala číst katoda nebyla v dobrém kontaktu s rukou, jsem to musel upravit.  Četla správně ale pomalu. Používala sevření prstu aby cítila každého tyče zvlášť.  Jelikož to funguje pro poznání písmena ale nejde tím způsobem číst rychle. Snažil jsem se ji naučit používat FCHAD správně.  Když jsem ji říkal, že by měla položit ruku rovně a jsem dal její ruku na zobrazovače správně říkala "Já nevím, jak ta ruka jen tak leží mně to nevyhovuje, Já se musím ty body zvyknout a osahat trochu jinak." %10:02.6

Položila ruku na selektor příliš daleko od sebe. Měla vybrat dotykem na napínači ale dala ruce na dráty, které vedou z nich.  Necítila správně kde jsou kraje mezi napínači a proto velmi těžko dozvěděla vzdálenost potřební posun mezi písmene a proto velmi těžko dozvěděla vzdálenost potřební posun mezi písmenem.  Nepochopila hned, že anody jsou dotykové a že silnější zmačknutí nemá na ně vliv.  To je ale pochopitelní, protože často nefungovaly. Snad by pochopila stroj lip, kdybych mluvil líp česky.  Opakovaně jsem dělal chyby, například jsem spletl slova \uv{písmo} a \uv{písmeno}.

Občas dělala chybu, že hmatala jenom na první sloupce bodů
ale ne na druhé.  Například, když bylo zobrazené \braillebox{24} myslela, že se zobrazuje jenom \braillebox{2}.

Protože anody jsou dotykové, stačí když okraj prstů je v kontaktu s anodou aby se zaregistroval dotyk.  Zobrazuje se písmeno od nejlevější anody, která je dotýkaná.  Ona měla velké potíže, tím že si myslela že posunula ruku na selektor, a že by mělo zobrazit další písmeno, ale protože její prst ještě byl v kontaktu s předcházející písmenem zobrazené písmeno zkutečně se nezměnilo.

První řádek textu, který účastnicí četli na FCHADU byl krátký, jenom sedm znaků.  To byl určitě chyba na moje straně.  Orca se zobrazuje znaků \braillebox{2358}\braillebox{123} u konce řádků.  To znamená, že účastnici dočetli "Braills" a pak najedeno byli konfrontovaný s tím, že jsou dvě znaky, které nepoznali.  Jsem jím říkal ať je nečtou a, že by měli pokračovat na další řádce.

FCHAD, který byl použitý v studie nemá žádné navigační tlačítka.  Aby účastníci pokračovali na další řádce museli jenom přesunou ruku k začátku řádku a pokračovat.  Používal jsem šípky na počítačové klávesnice přesunout na další řádku.

Další nezvyklost Orcy je, že pod čaruje začátky řádků. Například, kdy 'k' je u začátku řádku, je psáno jako \braillebox{1378} a ne \braillebox{13}.  Do jisté míru to účastnici zmátlo.

Další problém pro žáka 5 byl kdy četla 'ý' \braillebox{12346}. Nesevřela na bodu 6 a proto to původně četla jako 'p' \braillebox{1234}.  Kdy jsem ji říkal, že četla nesprávně a že ji chyby ještě jeden bod, ona to našla, ale zase nesprávně odhadla, že písmeno bude 'q' \braillebox{12345} ale rychle sama upravila.

Zajímavě, kdy dorazila na mezera \braillebox{} říkala \uv{tady nemám nic}. Nebylo mi jasný, že pochopila to jako mezera než jsem ji to říkal.

Ona věděla, že jsem Američan.  Kdy dorazila na \braillebox{1235} myslela, že to je Americký 'w' a ne český 'ř'.  Snad nepochopila, že text zobrazený na FCHAD je český.

Kdy dorazila na 'á'\braillebox{16} říkala nahlas \uv{to je první a šesti bod, to je a s čárkou} to je důležitý za dvoji důvodů.  Znamená, že ona nemohla přímo z tvarů znaků pochopit vyznám, musela explicitně o tom přemyslet.  Taky to znamená, že ji to pomohla přemyslet o čísla bodů, a že měla v explicitní paměti hned k dispozicí informace o tom, které body má 'á'.

Pokračovala tím, že volala nejdřív čísla bodů a potom, které písmeno to má byt.

Kdy dostala na ','\braillebox{2}, správně poznala, který bod je nahoru, ale nemohla říct, který znak to je i když čárka je úplně obyčejný a standardní znak v Braillovo písmo.  Když jsem ji říkal, že to je čárka hned pochopila.  Musel byt někdy v mozku tomu, že to má byt čárka, ale nějak ona to nespojila.

Kdy se vrátila na začátku selektoru potom co četla druhý řádek textu vrátila se jenom asi polovinu tak daleko jak musela.  Byl jsem překvapení z toho, protože selektor má jasně citelný okraj a jsem si myslel, že bude řídit podle citu a ne odhadem.  Je možný, že nechtěla dotykat na ty dotykové senzory kdy nečetla protože jinak nevidím důvod proč nepřejela prstem na začátku.

Na třetí řádce četla 'e' tím že pojmenovala na hlas čísla bodů, ale kdy dostal na 'r' stroj začal rychle vypnout a zapnout(protože její prst neměl dobrý kontakt s senzorem.  Říkala 'r' bez toho, že by nějak hledala body!  Tak myslím si, že ta vibrace ji pomohlo víc než zmátl.

Je to zajímavý z pedagogické hlediska, protože jsem sám lekl tím, že stroj nefunguje a jsem se snažil ji vysvětlit, že prst není správně položení na senzor(a proto to tak vibruje).  Měl jsem ji jen nechat číst, protože četla správně.

Podruhy kdy četla 'ý' dělala stejnou chybu jako poprvé.

Třetí 'e' v slovu \uv{teď} už četla bez toho, že by říkala nahlas čísla bodů, že by to znamenalo, že po třetí opakování už vidíme autonomizovace konkretní schopnosti?

Kdy četla 'ď'\braillebox{1456} říkala, že to používá jako lomítko.  Uznala, že v standardní Braillovo písmo je to 'ď'.

Kdy se vrátila na začátku selektoru u čtvrtý řádek říkala, že moc na to nezorientuje.  Zase posunula krátce ale jsem ji neopravil.

Zase zapomněla na druhý řádek bodů, a četla 'e'\braillebox{15} jako 'a'\braillebox{1}.  I když dříve neměla problém přečíst 'e' teď najednou neuměla.

Kdy pokračovala přesunula na selektor až moc(až na 'n') a jsem ji dal za úkol jít zpátky a hledat 't'.  Kdy četla písmena, které už četla dříve, četla je bez počítání bodů a rychle.

Kdy se vrátila na 'n'\braillebox{1345} nejdřív odhadla to jako 'd'\braillebox{145}. Četla 'e' správně, ale kdy dostala zase na další 'n' četla to jako 'm'\braillebox{134}.

Teď byla u předposledního písmena řádku 'í' a dělala chybu, že si myslela že už je konec řádku.  To je protože střed jejího prstu byl na poslední napínače ale okraj jejího prstu dotekl ještě jinde.

Je už obecně známý, že neopravené dotykové ovládání je nepřirozený pro lidí. Je normálně rozdíl mezi tím kde lidí si myslí že dotykají a kde kontakt mezí prstem a senzor zkutečně je.  Dotykové obrazovky používají speciální algoritmy odhadnout kde uživatel chtěl dotykat a psychologie dotyku je aktuální téma bádaní.  Algoritmy jsou na základě zrakové vnímání dotyků.  Dotyk na obrazovce je vždy po nějaké oblasti, ale počítač reaguje na dotyk jenom v jediném bodu.  Počítač musí odhadnout, který bod uživatel chtěl\citep{holz2011understanding}. Rozlišení selektor FCHADu je velmi málo na rozdílu od rozlišení dotykové obrazovky.  FCHAD neví, jestli je prst v silném kontaktu s senzorem, anebo je senzor dotykován jenom z okraji.  Dále, vnímání dotyků uživatele FCHADu je hmatní a ne zrakový.  Nemůžeme bohužel jen kopírovat stavějící algoritmy.



\subsubsection{Žák 4}

Čtvrtý žak je 20-25 let starý.  Jeho nejvíší uroveň dostažené vzdělávání je maturita.  Je můj spolužak tady na katedře angličtiny. Je uplně nevídomé a taky trpí problemy se sluchem.  Pracuje s braillským písmem od zakladní školy.

Měl velké potiže s přesunem na ty sensory.



\subsubsection{Žak 5}

Paty žak je 20-25 let starý.  Jeho nejvíší uroveň dostažený vzdělávání je maturita. Je uplně nevidomý.  Taký čte braillova písma od zakladní školy.

Kdy začal číst, zase selektor zle choval.  Musel jsem přenastavit prahy.  To mně trval asi deset minut.  To znamenalo, že potom co on četl o stroje, a potom co on teprve dal ruce na stroje, on seděl a čekal.  On měl zdaleko nejlepší vysledky ze všech učastníků.  Možní čekání mu pomohlo.

\subsection{Běžná studie}

\subsubsection{Žak 6}

\subsubsection{Žak 7}

\subsubsection{Žak 8}

\subsubsection{Žak 9}

\section{Shrnutí vysledků}
