\chapter{Výzkum}

\section{Popis studie}

\subsection{První Čast: Čtení Tisku}

Učastnici, nebo možno lepe žáci, v teto studiu nejdřiv poseděli u prazdného stolu.  Neveděli o čem to bude, kromě tomu, že budou vyzoušet nový druh braillský řádek.  FCHAD byl na stole, ale položený do zadu. Tak aby nepřekažel ruky.  Hněd od začatku studii, měli ukol.  Měli číst text v Appendix 3.1.  Četli na hlas a zvuk jsem nahraval.  Text požada souhlas o shromaždění dat, taky vysvětli co to je FCHAD a jak se to použiva. Text byl tísknutí na tuhém kusu braillského papíru velký 26.5x30.5cm.  Maximální delka řádku je 36 písmen.  Braill je šestibodové.  Text byl tisknutí obostraně, a na jedne straně papíru je sloupec dír aby mohl byt navazaný.

Text byl tisknutí v Knihovně a tiskárna pro nevidomé K.E.Macana, která je jediný veřejný tiskařský zavod pro nevidomé v Praze.  Proto předpokladam, že tiskové braill byl velmi standardní.

Když učastník dočte text musí čekat par vkteřinu jak připravím na počitač. Pokud ještě nedali souhlas k zveřejnění informaci jsem požadal o souhlas.  Žadný z mych učastníků odmitly, a nikdo, který jsem pozval k studie odmitl.

\subsection{Otazky}

 Potom, jsem se je zeptal nasledujicí otazek:

Jak jste starý?

Jak dlouho čtete braillské písmo?

Jaký je Váš nejviší dosažený úroveň vzdělání?

Jaký je Váš uroveň zrakové postižený?

Když jsem se je zeptal na věk, jsem vždy objasníl, že nepotřebuji přesný věk.  Jsem poznamenal věky přesné, jenom do pěti let.

Zajimavě, všechní učastnici kromě jeden odpověděli "uplně nevidomý" když jsem je zeptal na jejich uroveň nevidomosti.

WHO rozděli zrakové postižený na 5 kategorii. Nejnížší dvě jsou:

"Praktická slepota
zraková ostrost s nejlepší možnou korekcí 1/60 (0,02), 1/50 až světlocit nebo omezení zorného pole do 5 stupňů kolem centrální fixace, i když centrální ostrost není postižena, kategorie zrakového postižení 4

Úplná slepota
ztráta zraku zahrnující stavy od naprosté ztráty světlocitu až po zachování světlocitu s chybnou světelnou projekcí, kategorie zrakového postižení 5"\citep{sonsklasifikace}

To, že nikdo nehlasil "Praktická slepota" nebo "praktická nevidomost" muže znamenat, že oni otazku nepochopili, že maji odpovědět přesně.  Anebo to muže znamenat, že zkutečně byly všechní uplně nevidomé.

\subsection{Čtení na FCHADu}

Po otazek, jsem přemístil FCHAD aby seděl před učastici.  Musel jsem taky vyzkoušet jestli to funguje správně. To byl první zkušenost pro učastnici s FCHADem.  Slyšeli jak svakne ty body.  Potom, co jsem zjistil, že software je souštění správně jsem řekl učastníkům aby začali číst.  Jenom, že samolřejmě neuměli.  Vidomé lidí, když setkaji s novým nastroje činní. Oni koukaji!  Nevidomé lidí nekoukaji.  Seději, a nic nedělaji.  Koukaji do vzdalenosti, bez pochybu.  Myslím si, že se boji, že kdyby hrabali na stroje, by ho mohli rozbijet.  Není to nerozumní obavání.

Musel jsem něco dělat aby svůj učastnici se snažili. Aby se začali učit.  Jsem je nejdřiv dal ruce na stroje, tak jak by měly byt k spravné čtení.

Vzal jsem je levou rukou, a jsem dal tu levou ruku na zobrazovač(viz Obrazek 1-A). Řekl jsem je, "tady se zobrází to jedno písmeno."  Pak jsem je vzal pravou rukou, a dal jsem tu pravou ruku jejích na kursor selektor(vis Obrazek 1-B,C).

Teď je dobrý čas vysvětlit jak přesně vypada a funguje FCHADy, které byli použití v studie.

\section{Popis FCHADu}

\subsection{První selektor}

První učastnici(1-3) použivali jiný kursor selektor než poslední.  To je kvůli technické zavadům v prvním selektoru.   Technologie selektorů je velmí jednoduchý.  Selektor musí mít dva elektrody.  Když učastník dotyka elektrodem, elektrický proud jde přes tělo z jeden elektrodem k drůhem.  Je jedena katoda, a vice anod.  Podle toho přes, kterou anodou teče proud víme, které písmeno je vybrané(selektované).   První selektor měl 12 anod z napínáčků(vis Obrazek 1-C), a katoda v náramku(Obrazek 3).

Tento system moc dobře nefungoval. Když jsem to zkusil doma, fungoval to bezvadně, ale když to pokusil nějaký chlupatější muž náramní katoda nedostala dobrá kontakt a stroj nevěděl kdy je dotyk a kdy ne.

\subsection{Druhý selektor}

Učastníci 4-6 měli lepší

\section{Moje žaci}

\subsection{Žaci 1 a 2}

\subsection{Žak 3}

\subsection{Žak 4}

\subsection{Žak 5}

\subsection{Žak 6}

\section{Shrnutí vysledků}
