\subsubsection{Žák 3}

Setkání 6.3.2013

\uv{Žák} 3 je vysokoškolské vzdělaná žena ve věku mezi 45 a 50 let. Čte Braillovo písmo od začátku školní docházky.  Pracuje jako vysokoškolskou profesorkou němčiny.  Je úplně nevidomá.

Navštívil jsem ji během její konzultační hodiny což byly odpoledne hned po obědě.  Výzkum byl zpoždění tím, že jsme museli hledat prodlužovací kabel. Ona neměla zásuvku blízko pracovního stolu.  Ona se mě zdála čilá a zdravá.

Nehrabala s nástrojem když jsem to nastavil. Během naše setkání FCHAD trpěl technické problémy. Senzory někdy \uv{cítily} dotyky, které v zkutečnosti nebyly.  Zajímavě i když FCHAD moc dobře nefungoval ona byla schopná na to číst.

Začali jsme s tiskovém návodem:

Četla tiskový text obě ruce.

Čas na čtení celý tisknutý text: 93.3 vteřin (časy zaměřené ručně z nahrávek)

Počet znaků: 407

Rychlost čtení tisku: 4.36 CPS (počet znaků za sekundu)

Čas na čtení tisknutý text po otočený stranu: 51.8 vteřin

Počet znaků: 265

Rychlost čtení: 5.11 CPS

Ona jistě nebyla ve formě pro čtení tiskové Braillova písma.  Není divů kdy má doma braillský řádek.  V tiskovém Braillovu písmu řetězci velkých písmen jsou naznačen speciálním znakem: \braillebox{23} tak, že \uv{FCHAD} je psáno jako \braillebox{23}\braillebox{124}\braillebox{14}\braillebox{1}\braillebox{145}. Žák 3 snad nevnímala znak na řetězec velkých písmen.  Začala číst řetězec po jednotlivých písmen a pak říkala \uv{to budou asi čísla} a četla je jako čísla.  Dále, kdy dostala na konec stránky se chovala nějak překvapená.

Kdy se začala číst katoda nebyla v dobrém kontaktu s rukou.  Musel jsem to upravit.  Četla správně ale pomalu. Používala sevření prstu aby cítila každého tyče zvlášť.  Jelikož to funguje pro poznání písmena nejde tím způsobem číst rychle. Snažil jsem ji naučit používat zobrazovač správně.  Když jsem ji říkal, že by měla položit ruku rovně a dal jsem její ruku na zobrazovače správně říkala \uv{Já nevím, jak ta ruka jen tak leží mně to nevyhovuje, Já se musím ty body zvyknout a osahat trochu jinak.} %10:02.6

Položila ruku na selektor příliš daleko od sebe. Měla mít ruku napínači ale dala ruce na dráty, které vedou z nich.  Necítila správně kde jsou kraje mezi napínači a proto velmi těžko dozvěděla vzdálenost potřební posun mezi písmenem.  Nepochopila hned, že anody jsou dotykové a, že silnější zmačknutí nemá na ně vliv.  To je ale pochopitelní protože často nefungovaly. Snad by pochopila stroj lip kdybych mluvil líp česky.  Opakovaně jsem dělal chyby. Například jsem spletl slova \uv{písmo} a \uv{písmeno}.

Občas dělala chybu, že hmatala jenom na první sloupce bodů kdy četla písmena ze zobrazovače. Například, když bylo zobrazené \braillebox{24} myslela, že se zobrazuje jenom \braillebox{2}.

Protože anody jsou dotykové stačí když okraj prstů je v kontaktu s anodou aby se zaregistroval dotyk.  Zobrazuje se písmeno od nejlevější anody, která je dotýkaná.  Ona měla velké potíže, tím že si myslela že posunula ruku na selektor, a že by mělo zobrazit další písmeno ale protože její prst ještě byl v kontaktu s předcházející písmenem zobrazené písmeno zkutečně se nezměnilo.

\paragraph{ První řádek:}

\begin{tabular}{|c|c|c|c|c|c|c|c|c|c|c|c|}
\hline
B&r&a&i&l&l&s&&&&&\\
\braillebox{1278}&\braillebox{1235}&\braillebox{1}&\braillebox{24}&\braillebox{123}&\braillebox{123}&\braillebox{234}&\braillebox{}&\braillebox{2358}&\braillebox{123}&\braillebox{}&\braillebox{}\\
\hline
\end{tabular}

První řádek textu, který účastnicí četli na FCHADU byl kratší než selektor.  To byl určitě chyba na moje straně.  Orca se zobrazuje znaků \braillebox{2358}\braillebox{123} u konce řádků.  To znamená, že účastnici dočetli \uv{Braills} a pak najedeno byli konfrontovaný s tím, že jsou dvě znaky, které nepoznali.  Jsem jím říkal ať je nečtou a, že by měli pokračovat na další řádce.

FCHAD, který byl použitý v studie nemá žádné navigační tlačítka.  Aby účastníci pokračovali na další řádce museli jenom přesunout ruku k začátku řádku a pokračovat. Ovládal jsem změna řádku často bez jejích vědomi.

\paragraph{Druhý řádek:}

\begin{tabular}{|c|c|c|c|c|c|c|c|c|c|c|c|}
\hline
k&ý& &ř&á&d&e&k&,& &k&t\\
\braillebox{1378}&\braillebox{12346}&\braillebox{}&\braillebox{2456}&\braillebox{16}&\braillebox{145}&\braillebox{15}&\braillebox{13}&\braillebox{2}&\braillebox{}&\braillebox{13}&\braillebox{2345}\\
\hline
\end{tabular}


Další nezvyklost Orcy je, že pod čaruje začátky řádků. Například kdy 'k' je u začátku řádku, je psáno jako \braillebox{1378} a ne \braillebox{13}.  Do jisté míru to účastnici zmátlo.

Další problém pro žáka 5 byl kdy četla 'ý' \braillebox{12346}. Nesevřela na bodu 6 a proto to původně četla jako 'p' \braillebox{1234}.  Kdy jsem ji říkal, že četla nesprávně a že ji chyby ještě jeden bod, ona to našla, ale zase nesprávně odhadla, že písmeno bude 'q' \braillebox{12345} ale rychle sama upravila.

Zajímavě, kdy dorazila na mezera \braillebox{} říkala \uv{tady nemám nic}. Nebylo mi jasný, že pochopila to jako mezera než jsem ji to říkal.

Ona věděla, že jsem Američan.  Kdy dorazila na \braillebox{1235} myslela, že to je Americký 'w' a ne český 'ř'.  Snad nepochopila, že text zobrazený na FCHAD je český.

Kdy dorazila na 'á'\braillebox{16} říkala nahlas \uv{to je první a šesti bod, to je a s čárkou}.  Znamená to, že ona nemohla přímo z tvarů znaků pochopit vyznám. Musela explicitně o tom přemyslet.  Taky to znamená, že ji to pomohla přemyslet o čísla bodů, a že měla v explicitní paměti hned k dispozicí informace o tom, které body má 'á'.

Pokračovala tím, že volala nejdřív čísla bodů a potom, které písmeno to má byt.

Kdy dostala na ','\braillebox{2}, správně poznala, který bod je nahoru, ale nemohla říct, který znak to je i když čárka je úplně obyčejný a standardní znak v Braillovu písmu.  Když jsem ji říkal, že to je čárka hned pochopila.  Musel byt někdy v mozku tomu, že to má byt čárka, ale nějak ona to nespojila.

\paragraph{Třetí řádek}

\begin{tabular}{|c|c|c|c|c|c|c|c|c|c|c|c|}
\hline
e&r&ý& &t&e&ď& &p&o&u&ž\\
\braillebox{1578}&\braillebox{1235}&\braillebox{12346}&\braillebox{}&\braillebox{2345}&\braillebox{15}&\braillebox{1456}&\braillebox{}&\braillebox{1234}&\braillebox{135}&\braillebox{136}&\braillebox{2346}\\
\hline
\end{tabular}

Kdy se vrátila na začátek selektoru potom co četla druhý řádek textu vrátila se jenom asi polovinu tak daleko jak musela.  Byl jsem překvapení z toho protože selektor má jasně citelný okraj a jsem si myslel, že bude řídit podle citu a ne odhadem.  Je možný, že nechtěla dotykat na dotykové senzory kdy nečetla. Jinak nevidím důvod proč nepřejela prstem na začátku.

Na třetí řádce četla 'e', tím že pojmenovala na hlas čísla bodů ale kdy dostal na 'r' stroj začal rychle vypnout a zapnout(protože její prst neměl dobrý kontakt s senzorem).  Říkala 'r' bez toho, že by nějak hledala body!  Tak myslím si, že vibrace ji pomohlo víc než zmátl.

Je to zajímavý z pedagogické hlediska, protože jsem sám lekl tím, že stroj nefunguje a jsem se snažil ji vysvětlit, že prst není správně položení na senzor(a proto to tak vibruje).  Měl jsem ji jen nechat číst, protože četla správně.

Podruhy kdy četla 'ý' dělala stejnou chybu jako poprvé.

Třetí 'e' v slovu \uv{teď} už četla bez toho, že by říkala nahlas čísla bodů.  Znamená to, že třetím opakování už vidíme autonomizovace konkretní schopnosti?

Kdy četla 'ď'\braillebox{1456} říkala, že to používá jako lomítko(normálně \braillebox{12456}).  Uznala, že v standardní Braillovo písmo zobrazuje se 'ď'.

\paragraph{Čtvrtý řádek}

\begin{tabular}{|c|c|c|c|c|c|c|c|c|c|c|c|}
\hline
í&v&á&t&e&,& &n&e&n&í& \\
\braillebox{3478}&\braillebox{1236}&\braillebox{16}&\braillebox{2345}&\braillebox{15}&\braillebox{2}&\braillebox{}&\braillebox{1345}&\braillebox{15}&\braillebox{2345}&\braillebox{34}&\braillebox{}\\
\hline
\end{tabular}

Kdy se vrátila na začátku selektoru u čtvrtý řádek říkala, že moc na to nezorientuje.  Zase posunula krátce ale jsem ji neopravil.

I když dřív četla 'e' bez problému, zase zapomněla sevřít na druhém řádku bodů a četla 'e'\braillebox{15} jako 'a'\braillebox{1}.

Kdy pokračovala přesunula na selektor až moc(až na 'n') a jsem ji dal za úkol jít zpátky a hledat 't'.  Kdy četla písmena, které už četla dříve, četla je bez počítání bodů a rychle.

Kdy se vrátila na 'n'\braillebox{1345} nejdřív odhadla to jako 'd'\braillebox{145}. Četla 'e' správně, ale kdy dostala zase na další 'n' četla to jako 'm'\braillebox{134}.

Teď byla u předposledního písmena řádku 'í' a dělala chybu, že si myslela že už je konec řádku.  To je protože střed jejího prstu byl na poslední napínače ale okraj jejího prstu dotekl ještě jinde.

Je už obecně známý, že neopravené dotykové ovládání je nepřirozený pro lidí. Je normálně rozdíl mezi tím kde lidí si myslí že dotykají a kde kontakt mezí prstem a senzor zkutečně je.  Dotykové obrazovky používají speciální algoritmy odhadnout kde uživatel chtěl dotykat a psychologie dotyku je aktuální téma bádaní.

I když algoritmy jsou dostatečně rozvinuté, že používáte smartphony bez větší úsilí většina algoritmy jsou na základě zrakové vnímání dotyků.  Dotyk na obrazovce je vždy po nějaké oblasti ale počítač reaguje na dotyk jenom v jediném bodu.  Počítač musí odhadnout, který bod uživatel chtěl dotykat\citep{holz2011understanding}. Rozlišení selektor FCHADu je velmi málo na rozdílu od rozlišení dotykové obrazovky a psychologie dotyku u nevidomých je samozřejmě na základě hmatu a ne zraku.  FCHAD neví jestli je prst v plném kontaktu s senzorem anebo je senzor dotykován jenom z okraji.  Dále, vnímání dotyků uživatele FCHADu je hmatní a ne zrakový.  Nemůžeme bohužel jen kopírovat stavějící algoritmy.

\paragraph{paty řádek}

\begin{tabular}{|c|c|c|c|c|c|c|c|c|c|c|c|}
\hline
p&r&v&n&í& &ř&á&d&e&k&,\\
\braillebox{123478}&\braillebox{1235}&\braillebox{1236}&\braillebox{1345}&\braillebox{34}&\braillebox{}&\braillebox{1235}&\braillebox{16}&\braillebox{145}&\braillebox{15}&\braillebox{13}&\braillebox{2}\\
\hline
\end{tabular}

Na paty řádku myslela, že 'v'\braillebox{1236} je 'r'\braillebox{1235} protože si myslela, že paty bod je nahoru a ne šesti.

Stále neuměla přesunout ruku na selektor a stále ji přesunula příliš daleko do práva jak četla, tak že přeskočila několik písmem na raz.

Během čtení pátého řádku přestal stroj fungovat a musel jsem restartovat ovládací software.  Třetí účastník byla extrémně trpělivá a bych ji chtěl poděkovat za to!

Kdy jsem stroj upravil začala číst paty řádek zase od začátku. Opakování část řádku četla rychle a pokračovala v rychle čtení až dorazila na slovo \uv{řádek}.  Přečetla začátek slovo správně, ale četla 'e'\braillebox{15} jako 'á'\braillebox{16}.

Po paty řádku dokončili jsme setkání. Můj účastník ještě chtěla sdílet své dojmy.  O FCHADu říkala:

%330
\em \uv{Já si myslím, že ... Taklhe Vám řeknu, že jistě to víte sám, že ten řádek běžní je určitě komfortnější ale určitě než mít žádný řádek je lepší tenhle. Já si myslím, že na ten zobrazování těch písmen bych se zvykla myslím, že asi by to bylo dobré, kdyby každý uživatel by mohl trochu vybrat velikost, a například vy tady to máte rozděleno tou látkou a třeba osobně bych chtěla mít ty dva sloupce bodů naopak blíž. Protože Já mám tu ruku menší a užší. Teda jako za svou osobu, hlavní problém je v tom posunu jo, protože tam nějak ty aktivní body mě se zatím velice špatně hledají za svojí osobu jako veliko překážkou toho užívání bych především bych viděla v hledání těch písmena} \em


