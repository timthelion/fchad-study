\subsubsection{Žák 8}
Setkání 23.4.2013

Žák 8 je ve věku mezi 20 a 25 lety. Braillovo písmo čte od začátku školní docházky. Je úplně nevidomý. Je pravák.  Nepoužívá braillský řádek.

Četl tištěný text jednou(pravou) rukou.

% začátek 37.6 otočení 93.8 konec 180.7

Čas čtení tištěného textu: 143.1 vteřin

CPS: 2.84

Čas čtení tištěného textu po otočení listu: 87.1  vteřin

CPS: 3

Kdy jsem byl představený osmi žák hned chytil na to, že Timothy není české jméno a kdy jsem se zeptal osmi žákovi úvodní otázky, zkusil na mně mluvit anglicky, což dělal skoro bez přízvuku.

Když jsem ho ukazoval selektor, selektor začal poznat náhodné dotyky.  Po krátké době určení problém, jsem zjistil, že je tím, že účastník měl mokré ruce.

\paragraph{První řádek}

\begin{tabular}{|c|c|c|c|c|c|c|c|c|c|c|c|}
\hline
B&r&a&i&l&l&s&&&&&\\
\braillebox{1278}&\braillebox{1235}&\braillebox{1}&\braillebox{24}&\braillebox{123}&\braillebox{123}&\braillebox{234}&\braillebox{}&\braillebox{2358}&\braillebox{123}&\braillebox{}&\braillebox{}\\
\hline
\end{tabular}

Začali jsme číst první řádek tím, že se zeptal \uv{jak s tím na to}.  Začal jsem vysvětlení tím, že jsem ho ukázal ty díry na zobrazovače a pak kdy jeho pravá ruka byla na selektor zeptal jsem se které body jsou nahoře.  On říkal \uv{první} a odhadl `a', asi moje vysvětlení byla dost zmatený nepochopil jsem proč odhadl `ačko'.  Myslel, jsem si, že třeba myslí, že se zobrazí `á' a je zmatený z těch body 7 a 8.  Kdy měl prst na bod 8 jsem ho říkal, že to je pro velké písmena(což není pravda, bod 7 je pro velké písmena, bod 8 je pro čísla). Odhadl `v' a ještě `u'. Zřejmě nevěděl co to je osmibodové Braillovo písmo ale jsem to ještě nevěděl.  Začal jsem ho vykládat o selektoru.  Ale jak jsem ho vysvětlil, že jsou písmena `k' na selektor, hned odhadl, že zobrazené písmeno je `k'.

Jsem ještě vyřešil tomu, že měl mokré ruce a jsem malém leh tím, že selektor nefungoval.  Říkal jsem \uv{to mně dost překvapuje, aha!} když jsem viděl tu vodu.  On odpovídal \uv{Já nechci urazit, ale bych se nejdřív zeptal jestli už s tím umíte}.  Odpovídal jsem, že umím ale, že měl příliš mokré ruce na to a to chvíli nefungoval.

Zase jsem ho snažil vysvětlit zobrazovač.  Dal jsem jeho prst na každé díře a řekl jsem ho, které to je.  Teď jsem se dozvěděl, že snad v životě nesetkal s osmibodové Braillovo písmo protože se mně rovně zeptal na \uv{ty zbyli dva body}.  Jsem ho vysvětlil: \uv{Na počítače děláme velké písmeno tím, že dáváme nahoru asi sedmi bod a čísla tím že dáváme nahoru osmi bod}.

U toho jsem si rozhodl snížit citlivost senzorů, protože jenom tím jak jsem jeho ruku natřel papírovým kapesníkem nestačil.  Musel jsem restartovat celý softwarový sestavení.

Potom jsme se začaly znovu s poznání písmen.  Ještě na `B' jsem zase pojmenoval číslema body a dával jsem jeho prst na každý tyč nebo díru. Pojmenoval jsem a ukazoval jsem jenom první šest bodů abych ho zbytečně nezmátl.  Jasně musel cítit, že body jedna a dva jsou nahoře a žádné jiné ale stále odhadl nesprávně `včko'.  Vysvětlil jsem ho zase, že body sedm a osm nepoužíváme k poznání písmena a konečně on odhadl `bčko'.

Četl `r' nejdřív jako `h' ale odhadl správně potom co jsem ho říkal ať hledá dolu.

Říkal jsem ho, ať přesouvá na další písmeno a jsem musel zase utřít jeho ruce.  Byl tam dost vedro.  Natřel jsem selektor a on mě říkal: \uv{Bych měl radši braillský řádek na, kterém by to co bych napsal na počítači automatický zobrazoval.}

Říkal jsem ho jak má přesouvat na selektor ale četl `a' bez pomoci.

Odstranil ruku ze selektoru a jsem ho vysvětlil, že když nedotyká selektorem, tak se nezobrazí nic.  Dal ruku zpátky na selektor v přibližně správní místo a četl `i' jako `,' než jsem ho připomněl o další řádek bodů.

Zase odstranil ruku z selektoru ale vrátil ji zpět když jsem ho říkal ať čte další písmeno. Dal tu ruku na selektor o tři písmen dal než měl, ale četl správně `s'. Vrátil zpátky a četl `l' nejdřív jako `k'.

On začal hrát s selektorem a zobrazovačem.  Odstranil ruku z selektoru a pak poklepl na každý tyč který byl nahoru aby spadl.  Nechal jsem ho hrát 46 vteřin a pak on říkal tohle je `lko'(měl pravdu, tam se zkutečně `l' zobrazoval).  Pak jsem navrhoval, že čteme další řádek.

\paragraph{Druhý řádek}
\begin{tabular}{|c|c|c|c|c|c|c|c|c|c|c|c|}
\hline
k&ý& &ř&á&d&e&k&,& &k&t\\
\braillebox{1378}&\braillebox{12346}&\braillebox{}&\braillebox{2456}&\braillebox{16}&\braillebox{145}&\braillebox{15}&\braillebox{13}&\braillebox{2}&\braillebox{}&\braillebox{13}&\braillebox{2345}\\
\hline
\end{tabular}

Četl první písmeno jako `l', druhý odhad měl jako `v'.  Vysvětlil jsem mu, že body sedm a osm jsou tam jenom protože jsme u začátku řádek.  Odhadl `h' pak `u'.  Zase jsem dával jeho prst na každý bod a pojmenoval jsem je čísly. Pak jsem se zeptal, které body jsou nahoře.  On ukazoval na první bod a říkal \uv{To je ten sedmi} říkal jsem ho, že to není, že to je první.  Pak on říkal \uv{těžko říct, on je závorky}.  Říkal jsem ho \uv{Počítejte se mnou, jedna, dva, tři, čtyři, pět, šest, tak nahoře je bod jedna a bod tři, které písmeno to je} konečně správně říkal `kačko'.

Pokračovali jsme na další písmeno.  Nedržel ruku na selektor ale poklepal na senzor a pak prst odstranil.  Četl písmeno jako `l', pak `p' a konečně `ý'.

Odstranil zase ruku z selektoru, tak jsem si rozhodl zase snažit mu vysvětlit jak funguje.  Vedl jsem jeho prst nad `\uv{ty kačky}(ty senzory).

Vedl jsem jeho ruku na čtvrté písmeno, a četl to nejdřív jako `t' a potom, správně, jako `ř'.  Četl další písmeno `á' správně. Zase jsem musel utřít selektor a on utřel ruky na kalhotách a mně vykládal \uv{To je strašně přemodernizováné braillský řádek, vůbec se vtom nevyznám.} Vedl jsem jeho ruku na správné místo a jsme pokračovali.  Četl `d' nejdřív jako `č' než to četl správně.  On četl, tak že klepal na selektor a pak klepal každý bod na zobrazovač dole.  Četl `e' nejdřív jako `á', pak `o', pak `i'.  Konečně jsem ho jen říkal odpověď.  On říkal \uv{vůbec to není poznat} odpověděl jsem, že to jsou stejné písmena jenom větší, a on mě řek \uv{Já vím ale}.  Pořád klepal na ty senzory ale přibližně správně překračoval. Myslel, že chvili se zobrazoval `ě'.  Četl `k' správně.  Četl čárku správně.

Vysvětlil jsem mu, že už četl dvě slova.  Říkal mně, že to nepoznal.  Zeptal jsem se ho, jestli je unavený, ale říkal, že není.

Vedl jsem jeho ruku na předposlední znak, který četl správně jako `k'.  Pak začal zase ty senzory nefungovat a jsem je zase utřel.  On říkal anglicky \uv{I will get drunk}(budu opilý).  Vedl jsem jeho ruku zpátky na `t' který nejdřív četl jako `;'\braillebox{23} a pak správně.

\paragraph{Třetí řádek}
\begin{tabular}{|c|c|c|c|c|c|c|c|c|c|c|c|}
\hline
e&r&ý& &t&e&ď& &p&o&u&ž\\
\braillebox{1578}&\braillebox{1235}&\braillebox{12346}&\braillebox{}&\braillebox{2345}&\braillebox{15}&\braillebox{1456}&\braillebox{}&\braillebox{1234}&\braillebox{135}&\braillebox{136}&\braillebox{2346}\\
\hline
\end{tabular}

Na začátek třetí řádku odhadl nejdřív `k', pak `A', `t', `z'. Pak řekl, že neví. Odhadl `x' a jsem se zeptal, které body jsou nahoru. On říkal \uv{první, třetí} řekl jsem ho, že to je \uv{sedmi už}.  On říkal(správně) \uv{tuto logiku mizí}. Pak odhadl `ě' a jsem ho říkal \uv{ečko ale bez háčku}.  On mumlal \uv{tu logiku prostě mizí když}.  Další písmeno odhadl nejdřív jako `l' a pak správně jako `ý'. Četl `t' správně.  Pořád četl tím způsobem, že klepal krátce na selektor, a pak klepal na zobrazovače prstem. Je důležité poznamenat teď, že body na můj FCHAD nespadnou samy, musíte je pomoct. Říkal \uv{tohle nefunguje vůbec}. Vysvětlil jsem mu, že jenom body, které jsou pevné jsou významné. Četl `e' správně.  Pak četl `ď' jako `š'\braillebox{156}.  Připomněl jsem ho, že ještě ne zkusil hmatat na bodu čtyři.  Tento krát odhadl správně `ď'.  Četl tu mezeru jako čárka. Četl `p' nejdřív jako `t' pak `e', konečně jako `p'.  Četl `o' správně.  Přeskočil `u' ale četl `ř' správně.

\paragraph{Čtvrtý řádek}
\begin{tabular}{|c|c|c|c|c|c|c|c|c|c|c|c|}
\hline
í&v&á&t&e&,& &n&e&n&í& \\
\braillebox{3478}&\braillebox{1236}&\braillebox{16}&\braillebox{2345}&\braillebox{15}&\braillebox{2}&\braillebox{}&\braillebox{1345}&\braillebox{15}&\braillebox{2345}&\braillebox{34}&\braillebox{}\\
\hline
\end{tabular}
Četl, první písmeno nejdřív jako `ž', pak `i', a jsem ho řekl, že je to `í'.  Četl `v' a `á' správně přeskočil `t' ale četl `e' správně.  Řekl, že neví jaké to mělo byt slovo. Vysvětlil jsem mu, že je to slovo \uv{používáte} ale, že přeskočil písmena `u' a `t'.  Zeptal se, \uv{kde je učko} tak jsme se vrátili na předchozí řádku a jsem vedl jeho ruku k `o' pak `u' a `ž', písmena správně ztotožnil.

Pak jsme se začali číst od začátku řádku. Nejdřív, myslel, že `í' je `u' ale odhadl správně po druhý. Pak myslel, že se zobrazuje `,' ale ještě bylo nahoru `í'.  Zeptal se kolik je hodin a v stejném chvilce objevila v dveře sekretářka a zeptala se \uv{jak jste na to}. Říkal jsem, že už můžeme končit a žák 8 říkal, že by měl už jít.  Sekretářka ho řekla, že ještě má pět minut a jsme pokračovali.

Přeskočil ještě par písmen a jsme byli u `e' co ztotožnil správně. Myslel nejdřív je `n' je `g'. Přeskočil `í' a správně ztotožnil mezeru.  Neřekl jsem ho, že přeskočil, protože to by bylo zbytečné obtížení úkolu.

Dal jsem hu číst další řádek.

\paragraph{Pátý řádek}
\begin{tabular}{|c|c|c|c|c|c|c|c|c|c|c|c|}
\hline
p&r&v&n&í& &ř&á&d&e&k&,\\
\braillebox{123478}&\braillebox{1235}&\braillebox{1236}&\braillebox{1345}&\braillebox{34}&\braillebox{}&\braillebox{1235}&\braillebox{16}&\braillebox{145}&\braillebox{15}&\braillebox{13}&\braillebox{2}\\
\hline
\end{tabular}

Zeptal se \uv{jak ty řádky zadávají?}  Ukazoval jsem ho svůj notebook, a řekl jsem ho, že používám ty šípky abych přešel na další řádek.  Potom co jsem ho ukazoval ten počítač, nemohl znovu najít selektor, tak jsem musel jeho ruku tam dat.

Nejdřív myslel, že řádek začal `ý' ale pak správně řekl `p' když jsem ho připomněl o bodech sedm a osm.  Správně ztotožnil `r'.  Kdy přesunul ruku, zřejmě slyšel cvaknutí, protože zeptal se, jestli něco přeskočil.  Řekl jsem ho, že zkutečně přeskočil písmeno a vrátil se úspěšně na `v', myslel chvílí, že je `ě'\braillebox{126} a trval mu trochu, ale odhadl správně `v' podruhy.  Správně odhadl `n' a `í'.  Zeptal jsem ho, jestli pozná, které slovo to je, ale říkal, že ne.  Zeptal jsem se, jestli pamatuje první písmeno v řádku.  Řekl `y'.  Zase jsme se vrátili na začátek řádku a jsme zase uvažovali přes to, že jsou tam nahoru body sedm a osm.  Správně odhadl podruhy.  Zeptal se pak \uv{druhý byl co} a jsem ho říkal ať to najde sám.  Začal číst druhý písmeno jako `h'. Tím jsme ukončili, protože přišla sekretářka s dalším účastníkem.  Žák 8 říkal \uv{to je nad mou logiku} a rychle odešel.


