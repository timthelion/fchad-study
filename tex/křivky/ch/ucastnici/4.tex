\subsubsection{Žák 4}

Setkání 8.3.2013

Čtvrtý žák je 20-25 let starý.  Jeho nejvyšší úroveň vzdělávání je maturita.  Je můj spolužák tady na katedře angličtiny. Je úplně nevidomé a taky trpí problémy se sluchem.  Pracuje s Braillovým písmem od základní školy. Používá braillský řádek doma na počítače.

Měl velké potíže s přesunem na selektor.

Když jsem ho dával číst úvodní papír on se mně zeptal jestli to má číst nahlas.  Říkal jsem ne, protože první část papíru se zkutečně nahlas číst nemusí.  To byla chyba.  Čekal jsem, že až dorazí na větu \uv{Následující text čtěte prosím nahlas a bez přestání až do konce.} začíná číst nahlas, ale četl dal po tichu.  Z toho důvodu mám jeho rychlost čtení tiskové Braillovo písmo jenom z druhý straně papíru.  Je to obecně zajímavý, že moje účastnicí nereagovali na psáné příkazy.  Žák 4 dal souhlas bez ústního zeptání ale některé z nich nepochopili z věty \uv{Než budeme pokračovat, potřebuji Váš souhlas se shromážděním a zveřejněním těchto dat.}, že mají nějak mě aktivně říct že s tím souhlasili.

Četl tištěný text oběma rukama.

Čas čtení tištěného textu:43 vteřin %169.2-126.1

CPS: 9.46*

Čas čtení tištěného textu po otočení listu:25.9 vteřin %169.2-143.3

CPS: 10.23**

*) přečetl to nahlas ale potom co to četl potichu

**) Přečetl to jenom nahlas.

Potom co jsem se zeptal otázky, přemístil jsem FCHAD před účastníkem a říkal jsem mu ať začíná prozkoumat stroj.  On seděl a nic nedělal.  Třeba jsem mu měl nechat díl. Kdy jsem tam seděl s něj měl jsem pocit, že sedí strašně dlouho ale na zvukové nahrávce jsem čekal jenom deset vteřin.

Protože žák 3 měla tolik problému s selektorem rozhodl jsem se vysvětlit čtvrtému žákovi selektor líp.  Dal jsem jeho ruku na selektor a táhl jsem jeho prsty nad senzory. Jak jeho prsty dotekly se senzory zobrazovač vydal zvuk cvaknutí jak se změnilo písmeno.  Říkal jsem ho, že ten zvuk je zvuk změna písmen.  Jenom až potom co jsem mu vysvětlil selektor jsem dal jeho levou ruku na zobrazovače.

Věnoval jsem větší čas vysvětlení fungování nastroje než u žáka 3. Kdy jsem ho ukazoval zobrazovače zobrazilo se písmeno `o' a jsem ho říkal, které písmeno to je.

Ještě během ukázku přestala stroj reagovat.  Naše setkání bylo usekaná mnoho krát technickými potíží.  Protože měl chlupatou kůži, katoda neměla dobrá kontakt.  Proto, body strašně zatřásly.

\paragraph{První řádek}

\begin{tabular}{|c|c|c|c|c|c|c|c|c|c|c|c|}
\hline
B&r&a&i&l&l&s&&&&&\\
\braillebox{1278}&\braillebox{1235}&\braillebox{1}&\braillebox{24}&\braillebox{123}&\braillebox{123}&\braillebox{234}&\braillebox{}&\braillebox{2358}&\braillebox{123}&\braillebox{}&\braillebox{}\\
\hline
\end{tabular}

Konečně jsme se začali číst.  Žák 4 nepoužíval způsob pojmenování bodů kdy četl.  Odhadl nejdřív, že první písmeno je `l'\braillebox{123} a kdy jsem ho říkal že to není opravil sebe, že je to `b'.

Kdy pokračoval v čtení stále katoda neměla dobrý spoj.  Protože jsem věděl, že budou problémy s katodou jsem ještě měl další katoda(ve formě měděnou samolepicí paskou) nalepený na zobrazovače. Snažil jsem se mu vysvětlit, že bude fungovat lip kdy je v kontaktu s druhou katodou.

Přečetl `r' správně a zase začal zbláznit stroj.  Vysvětlil jsem mu ten problém s kontaktem s katodou a poprosil jsem mu aby držel katodu v ruce.  Pak četl další dvě písmena správně a zase nefungoval stroj.

Po par pokusu resetnutí stroje konečně už to fungoval až ke konci setkání.  Četl od začátku prvního řádku a četl první tři písmena správně. Přeskočil jedno písmeno na selektor a jsem mu říkal ať se vrátí zpět.  Četl `i'\braillebox{24} jako `c'\braillebox{14}. Když jsem ho říkal, že četl nesprávně používal stejný způsob uvažování, které používala žák 3 celou dobu, říkal na hlas, které body jsou nahoru \uv{to je bod jedna čtyři} říkal. \uv{Není to bod jedna} jsem mu odpověděl. Potom už se upravil správně sám.

Na rozdíl od žáka 3 pochopil hned, že čte české slova.  Kdy viděl, že další písmeno `l' říkal \uv{Brail}. Četl ještě `l' a `s' správně a doplnil \uv{Braills}.

Jsem ho říkal ať se vrátí na začátek textu radši než, že bych mu vysvětlil jaké znaky používá Orca označit konec řádku, v retrospekce myslím si že to byla chyba.

\paragraph{Druhý řádek}

\begin{tabular}{|c|c|c|c|c|c|c|c|c|c|c|c|}
\hline
k&ý& &ř&á&d&e&k&,& &k&t\\
\braillebox{1378}&\braillebox{12346}&\braillebox{}&\braillebox{2456}&\braillebox{16}&\braillebox{145}&\braillebox{15}&\braillebox{13}&\braillebox{2}&\braillebox{}&\braillebox{13}&\braillebox{2345}\\
\hline
\end{tabular}

Kdy začal číst druhý řádek dostal na `s' a `ý' a doplnil, že dočetl \uv{Braillský} pak \uv{četl} dal. Domyslel, že se zobrazí písmeno je `ř'.  Ještě ale nepřesunul ruku na selektor.  Říkal jsem mu, že stále ze zobrazuje `ý'.  Přesunul ruku na selektor až k `ř' a říkal, \uv{aha, tak tohle je opravdu `ř'}.  Přečetl slovo \uv{řádek} skoro plynule, kromě tomu, že měl problém s přesunem na selektor.

\paragraph{Třetí řádek}

\begin{tabular}{|c|c|c|c|c|c|c|c|c|c|c|c|}
\hline
e&r&ý& &t&e&ď& &p&o&u&ž\\
\braillebox{1578}&\braillebox{1235}&\braillebox{12346}&\braillebox{}&\braillebox{2345}&\braillebox{15}&\braillebox{1456}&\braillebox{}&\braillebox{1234}&\braillebox{135}&\braillebox{136}&\braillebox{2346}\\
\hline
\end{tabular}

Vrátil se k začátku selektoru bez pomoci, ale první písmeno `e' četl jako `á'\braillebox{16}, pojmenoval čísla bodů jako jedna a šest a jsem ho říkal, že šesti bod není nahoru.  Hmatal prstem na díry na zobrazovače a počítal, který bod zkutečně nahoru je.  Pak se upravil sám.

Kdy četl další slovo \uv{teď}, četl `t' správně ale zase původně četl `e' jako `á'.  Měl taky problém s písmenem `ď'\braillebox{1456}, který původně četl jako `č'\braillebox{146}.  Neříkal nahlas ani \uv{který} ani \uv{teď} po jejich čtení.  Četl `p' správně, měl trochu problém s posunem na selektor ale sám našel `o'.  Myslel si, že je u konce řádku.  Říkal jsem mu, že není. Přečetl `u' a pak už opravdu věřil, že je u konce řádku. Říkal jsem mu, že mu zbývá ještě jedno písmeno.  Snažil se pokračovat. Písmeno na zobrazovače začal vibrovat chaotickým způsobem. Říkal jsem mu, že musím zkusit jestli stroj teď funguje správně. Zkoušel jsem stroj a fungoval. Říkal jsem mu ať pokračuje. Četl `ž' jako `t'. Říkal jsem, že to není správní. Opravil se sám.

Dokončili jsme u toho setkání.

Kdy jsme dokončili setkání ještě jsme mluví o FCHADech.  Nabídl další spolupráce a říkal, že \uv{kdyby to bylo opravdu levná alternativa, tak by to bylo opravdu super}. Říkal mně, že \uv{by to bylo lepší kdyby ty body byly blíž k sobě}.


