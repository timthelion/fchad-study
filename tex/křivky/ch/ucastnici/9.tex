\subsubsection{Žák 9}
Setkání 23.4.2013

Žák 9 je 20-25 let starý.  Nejvyšší dosažení úroveň vzdělávání je maturita.  Je úplně nevidomý a má problémy se sluchem.  Při naše setkání jsem nosil na sebe mikrofon, který byl bezdrátově přípojení na jeho sluchadla.  Rozuměl mě bez znatelný potíže.

Četl tiskový text jednou(pravou) rukou.

%začatek 197.8 otočení 230.2 konec 279
Čas na čtení celého textu: 81.2 vteřin

CPS: 5

Čas na čtení druhé strany tiskového textu: 48.8 vteřin

CPS: 5.4

Kdy jsem ho ukazoval zobrazovač, říkal jsem mu \uv{tady se zobrazí to písmeno. Tady jsou osm děr.} On vzal zobrazovače obě ruce a začal je počítat, ale nepočítal je v standardní pořadí číslování jen je počítal.  Mluvil milým a překvapeným hlasem po celé setkání. Kdy byl jistý, že jsou tam zkutečně osm děr, říkal jsem ho, že by měl položit levou ruku na zobrazovače.  On mě odpovídal smíchem, že \uv{Já jsem sice levák ale čtu pravou}. Vysvětlil jsem mu, že současní stroj má jenom jeden nastavení.  Potom jsem ho ukazoval selektor.  Říkal jsem mu, že zvuk, který vydá zobrazovač je zvuk změna písmen a že by měl taky cítit jak se to písmeno změní.  On začal \uv{hrát} se selektorem a jsem si myslel, že bude samostatně pokusit o čtení.  Měl ruku správně položenou na zobrazovače a správně byl u první písmena na selektor.  Občas říkal \uv{uf}.

Protože jsem chtěl dozvědět, jestli bude číst sám, čekal jsem 146 vteřin.  Bohužel v analýze audio-nahrávku vidím, že jsem říkal \uv{um} u začátku toho času, tak je docela možný, že on sám čekal na mně.  Kdy mně bylo jasno, že sám číst nebude, jsem se začal vykládat o použitý nastroje dal.

Vysvětlil jsem mu, že \uv{ty písmena `k'} jsou senzory.  On začal je počítat a když počítal, že jsou jích 12 tak říkal \uv{aha} jak že už spojil je s těmi dvanácti senzory vysvětlené v úvodním textu. Zase jsem čekal ať hraje, a tento krát jsem neříkal žádný zmítající \uv{um}.  Zkutečně hrál s nástrojem a věřil jsem, že se snaží čist. 210 vteřin později ještě nepokračoval.  Stále hmatal na zobrazovače jako by se snažil z něj číst ale nečetl. %881-671.9

\paragraph{První řádek}

\begin{tabular}{|c|c|c|c|c|c|c|c|c|c|c|c|}
\hline
B&r&a&i&l&l&s&&&&&\\
\braillebox{1278}&\braillebox{1235}&\braillebox{1}&\braillebox{24}&\braillebox{123}&\braillebox{123}&\braillebox{234}&\braillebox{}&\braillebox{2358}&\braillebox{123}&\braillebox{}&\braillebox{}\\
\hline
\end{tabular}


Zeptal jsem se mu co je první písmeno a on se mě zeptal \uv{jak to poznat?}.  Říkal jsem mu, ať počítá díry.  Nejdřív se mně nepochopil, zřejmě kvůli chybě mé češtiny.  Počítal, že jsou dva body nahoru, ale nepochopil ještě o co jde.  Vzal jsem jeho levou ruku a ukazoval jsem ho \uv{to je jedna a dva tak které písmeno to je?} a u toho už říkal bez myšlení `b'.  Protože jsme byli u začátek řádku ještě byly nahoře body 7 a 8.  Bohužel žák 8 se kymácel trupem tak, že jeho ruce na selektor nebyl stabilní a dále trpěl malé zrcadloví pohyby kdy se snažil číst na zobrazovače.  Myslím si, že rozuměl už, že by měl jít na druhé písmeno i bez toho, že bych mu řekl, ale nebyl fyzické schopní.  Věděl, že jeho prsty pohybují bez jeho přání a začal brzo byt frustrovaní tím.  Držel jsem jeho prstí tak aby nepohybovaly a vedl jsem jeho prsti na selektor.  Přečetl `r' \uv{první a pátý koukám} jsem mu říkal \uv{ještě} a on překvapeně, když objevoval ještě \uv{druhý a třetí aha, takže rko}.

Dal četl `a' stále způsobem, že nejdřív říkal čísla bodů a písmeno až potom.  `l' už četl bez pojmenování bodů. Četl další písmeno jako `p' a pak opravil, že je `s' když jsem ho říkal že četl nesprávně \uv{aha,tohle zmizelo, tak dávat pozor na to co se zmizí} on myslel tím, že nejdřív četl `l'\braillebox{123} a pak `s'\braillebox{234} protože body nespadnou než na něj tlačíte, chvílí zkutečně se zobrazilo `p'\braillebox{1234}.

\paragraph{Druhý řádek}
\begin{tabular}{|c|c|c|c|c|c|c|c|c|c|c|c|}
\hline
k&ý& &ř&á&d&e&k&,& &k&t\\
\braillebox{1378}&\braillebox{12346}&\braillebox{}&\braillebox{2456}&\braillebox{16}&\braillebox{145}&\braillebox{15}&\braillebox{13}&\braillebox{2}&\braillebox{}&\braillebox{13}&\braillebox{2345}\\
\hline
\end{tabular}

Přišli jsme na další řádek a četl `k' bez problému, ale četl `ý' zase tím, že pojmenoval body.  Kdy jsme dorazili na mezeru, on říkal \uv{nic} a začal se smát.   Četl slovo \uv{řádek} po jednotlivých písmenech ale bez chyb a bez pojmenování čísla bodů.  Četl tak dobře, že jsem chvíli pustil jeho pravou ruku ať používá selektor samostatně ale zase začala pochybovat rytmický bez jeho ovládání.  Našli jsme místo kde jsme skončili a pokračovali jsme.

Četl `k' nejdřív jako `a' ale sám se opravil když jsem ho říkal, že to není správný.

\paragraph{Třetí řádek}

\begin{tabular}{|c|c|c|c|c|c|c|c|c|c|c|c|}
\hline
e&r&ý& &t&e&ď& &p&o&u&ž\\
\braillebox{1578}&\braillebox{1235}&\braillebox{12346}&\braillebox{}&\braillebox{2345}&\braillebox{15}&\braillebox{1456}&\braillebox{}&\braillebox{1234}&\braillebox{135}&\braillebox{136}&\braillebox{2346}\\
\hline
\end{tabular}

Přešli jsme na třetí řádek.  Vždy, kdy dostal na mezera, hmatal jak tam nic nahoru není a usmíval se a řekl \uv{mezera} s radosti tím jak je lehký poznat mezery.  Četl správně až na `ž', ale četl `ž'\braillebox{2346} jako `z'\braillebox{1356}. Říkal jsem \uv{To je jako `z' ale obraceně} a on četl(obraceně) \uv{jedna tři pět šest, to je `z'} jsem ho ukazoval, že pojmenoval čísla ty body obraceně a on odpovídal rozčileně \uv{Tak, že jsem to celou dobu říkal blbě!}  Uklidnil jsem ho, že četl správně do té doby a on se krasně usmíval a říkal, že je úplně zmatený.

\paragraph{Čtvrtý řádek}

\begin{tabular}{|c|c|c|c|c|c|c|c|c|c|c|c|}
\hline
í&v&á&t&e&,& &n&e&n&í& \\
\braillebox{3478}&\braillebox{1236}&\braillebox{16}&\braillebox{2345}&\braillebox{15}&\braillebox{2}&\braillebox{}&\braillebox{1345}&\braillebox{15}&\braillebox{2345}&\braillebox{34}&\braillebox{}\\
\hline
\end{tabular}

Kdy jsme pokračovali na další řádek, četl `í'\braillebox{34} jako `á'\braillebox{16} ale tento krát upravil se sám když jsem ho říkal, že to není správní.  Zase u toho usmíval.  Pak četl `v' a `á' bez chybně.  Četl `e' jako `i' ale tento krát jsem Já dělal chybu a neopravil jsem ho. Četl `,' správně a když jsme byli na `n'\braillebox{1345} přemyslel a pojmenoval body hodně dlouho(39 vteřin) než odhadl `ž', který hned sám odvolal. Čekal jsem další 18 vteřin a říkal jsem ho \uv{je to `ž' z horu nohama}(I Já jsem byl zřejmě trochu zmatený, protože `ž' z horu nohama je `ň').  Konečně jsem ho jen tak říkal, že je to `n' ale mně nevěřil a říkal \uv{mate to, že to je zrcadlově obracený} ale konečně říkal \uv{aha, už to vidím} a zase se krasně usmíval.  %1838.4

Nějak během toho, jsme přeskočili dvě písmena a jsme byly u `í', což četl nejdřív jako `t', pak `á' a konečně `í'.

Zeptal jsem ho jestli není unavený, ale říkal že je \uv{v pohodě}.

\paragraph{Pátý řádek}

\begin{tabular}{|c|c|c|c|c|c|c|c|c|c|c|c|}
\hline
p&r&v&n&í& &ř&á&d&e&k&,\\
\braillebox{123478}&\braillebox{1235}&\braillebox{1236}&\braillebox{1345}&\braillebox{34}&\braillebox{}&\braillebox{1235}&\braillebox{16}&\braillebox{145}&\braillebox{15}&\braillebox{13}&\braillebox{2}\\
\hline
\end{tabular}

Na další řádek četl písmena \uv{první ř} bez chyb a dal velký důraz na `n'.  Zase četl `á' nejdřív jako `í'. Četl `d' a `e' správně ale `k' četl nejdřív jako `a'. Četl `,' správně.

\paragraph{Šesti řádek}
\begin{tabular}{|c|c|c|c|c|c|c|c|c|c|c|c|}
\hline
 &k&t&e&r&ý& &z&o&b&r&a\\
\braillebox{78}&\braillebox{13}&\braillebox{2345}&\braillebox{15}&\braillebox{1235}&\braillebox{12346}&\braillebox{}&\braillebox{1356}&\braillebox{135}&\braillebox{12}&\braillebox{1235}&\braillebox{1}\\
\hline
\end{tabular}

Šesti řádek četl správně až na `z' který nejdřív četl jako `n'.  Četl `o' a `b' správně ale četl `r' nejdřív jako `h'.

Je možní, že protože jsem musel vest jeho pravou ruku ještě nerozuměl selektor. Kdy jsme byli u konce řádku myslel, že se zobrazuje mezera.  Nic se nezobrazoval, ale to byl protože nebyl v kontaktu s žádným senzorem.

\paragraph{Sedmi řádek}
\begin{tabular}{|c|c|c|c|c|c|c|c|c|c|c|c|}
\hline
z&u&j&e& &j&e&n&o&m& &j\\
\braillebox{135678}&\braillebox{136}&\braillebox{245}&\braillebox{15}&\braillebox{}&\braillebox{245}&\braillebox{15}&\braillebox{1345}&\braillebox{135}&\braillebox{134}&\braillebox{}&\braillebox{245}\\
\hline
\end{tabular}

Četl `z' jako `n' ale sám se upravil.  Říkal jsem ho ať zkusí posunut pravou ruku na selektor samostatně, a přestal jsem držet jeho ruku ale jak jsem nedržel jeho ruka tak pohybovala sama a on si povzdechl frustraci, tak jsem zase tu ruku držel.   Četl `u' nejdřív jako `a' ale upravil se když jsem ho říkal, že to není správný.  Četl `j'\braillebox{245} jako `h'\braillebox{125} ale upravil se sám když jsem ho říkal, že je obracený. Četl `e' a mezera správně ale zase obrátil `j'.  Četl `e` správně ale `n' jako `d'. Když jsem ho říkal, že to není správný, nejdřív si myslel, že četl obraceně. Říkal jsem ho, že neobracel to a, že ještě nějaký bod dolu a rychle už našel, že je to `n'. Četl `o' správně ale četl `m' nejdřív jako `k'.  Četl `j' jako `,' a jsem ho připomněl, že je ještě další řádek bodu než svůj odhad opravil. Jinak četl správně.

\paragraph{Osmi řádek}
\begin{tabular}{|c|c|c|c|c|c|c|c|c|c|c|c|}
\hline
e&d&n&o& &p&í&s&m&e&n&o\\
\braillebox{1578}&\braillebox{145}&\braillebox{1345}&\braillebox{135}&\braillebox{}&\braillebox{1234}&\braillebox{34}&\braillebox{234}&\braillebox{134}&\braillebox{15}&\braillebox{1345}&\braillebox{135}\\
\hline
\end{tabular}

Na osmi řádek textu četl správně až na `í' což zase obrátil jako `á'. Četl `s' a `m' správně ale pak si myslel, že už jsme na další písmeno, když jsme v zkutečnosti nebyli.  Četl `m'\braillebox{134} podruhý jako `c'\braillebox{14}.  Četl `e' jako `o' ale sám si objevil tu chybu.  A četl `o' nejdřív jako `e'.

\paragraph{Devátý řádek}
\begin{tabular}{|c|c|c|c|c|c|c|c|c|c|c|c|}
\hline
.& & &P&r&v&n&í& &b&y&l\\
\braillebox{378}&\braillebox{}&\braillebox{}&\braillebox{12347}&\braillebox{1235}&\braillebox{1236}&\braillebox{1345}&\braillebox{34}&\braillebox{}&\braillebox{12}&\braillebox{13456}&\braillebox{123}\\
\hline
\end{tabular}

Devátý řádek začíná s `.' ale bylo zobrazený jako \braillebox{378} protože jsme byli na začátek řádku.  On to četl jako závorka což je \braillebox{236}.  Připomněl jsem ho, že body sedm a osm jsou nahoru protože je to začátek řádku a zeptal jsem se, které body jsou ještě nahoru.  Jeho první odhad byl obracený, že to je znak pro velké písmeno\braillebox{6} a konečně odhadl správně když jsem ho říkal že to není.  Není důvod proč by věděl, že znak pro velké písmeno je bod šest a ne bod 3, protože ten znak je použitý jenom v tiskovém Braillovu písmu a bez rámečku není podstatní rozdíl.  Četl zbytek řádku bez problému.

\paragraph{Desátý řádek}
\begin{tabular}{|c|c|c|c|c|c|c|c|c|c|c|c|}
\hline
 &v&y&n&a&l&e&z&e&n& &v\\
\braillebox{78}&\braillebox{1236}&\braillebox{13456}&\braillebox{1345}&\braillebox{1}&\braillebox{123}&\braillebox{15}&\braillebox{1356}&\braillebox{15}&\braillebox{1345}&\braillebox{}&\braillebox{1236}\\
\hline
\end{tabular}

Četl desátý řádek skoro bez chyb, jedině byl částečně moje vína(protože jsem vedl jeho ruce na selektor). Já jsem táhl jeho ruku příliš daleko od `n' na `l' a museli jsme se vrátit. Kdy četl `l' podruhy četl to jako `b' ale opravil se hned než jsem stihl něco říct.

\paragraph{Jedenáctý řádek}
\begin{tabular}{|c|c|c|c|c|c|c|c|c|c|c|c|}
\hline
 &r&o&c&e& &1&9&1&3& &v\\
\braillebox{78}&\braillebox{1235}&\braillebox{135}&\braillebox{14}&\braillebox{15}&\braillebox{}&\braillebox{18}&\braillebox{248}&\braillebox{18}&\braillebox{148}&\braillebox{}&\braillebox{1236}\\
\hline
\end{tabular}

Jedenácti řádek obracel `r' jako `ř'.  Četl `o' nejdřív jako `e'. Kdy dočetl \uv{roce} zeptal jsem ho jestli pozná, které slovo to bylo ale nevěděl.  I potom co jsme se vrátili a on to četl znovu(s chybou, že nejdřív četl `e' jako `a') začal vykládat slovo jako \uv{ro} ale nepokračoval. Místo tomu, že bych ho zbytečně otrávil, jsem ho říkal, že to bylo slovo \uv{roce} a jsem ho říkal, že teď zřejmě bude číslo nebo datum.  Myslím, že jsem ho zmátl dost, protože nejdřív četl 1\braillebox{18} jako `a'.  Když jsem ho říkal, že to má byt číslo a, že bod osm je číselní bod stále to nepochopil.  Přemyslel o tom 4.5 vteřin a pak, řekl \uv{aha, jedna}. Devět četl nejdřív jako `i' a pak se zeptal jestli to je ještě číslo a správně doplnil, že to je devět. Četl `1' a `3' správně a jsem doplnil, že to je \uv{devatenátset třináct}.  Četl mezera a `v' správně.  Předtím, že řekl `v' říkal \uv{předpokládám, že tohle už není číslo}. Nevím jestli je to protože žádný číslo je v tvaru\braillebox{1236} anebo protože pochopil, že jsme dočetli datum.

\paragraph{Dvanáctý řádek}
\begin{tabular}{|c|c|c|c|c|c|c|c|c|c|c|c|}
\hline
 &A&n&g&l&i&i&.& & &J&m\\
\braillebox{78}&\braillebox{17}&\braillebox{1345}&\braillebox{1245}&\braillebox{123}&\braillebox{24}&\braillebox{24}&\braillebox{3}&\braillebox{}&\braillebox{}&\braillebox{2457}&\braillebox{134}\\
\hline
\end{tabular}

Dvanáctý řádek.  Zase jsem ho nechal číst samostatně. Už zvládl nepohybovat pravou ruku.  Přizpůsoboval tak, že tlačil hodně silně prstem, tak že jestli nějaký pohyb byl, pohyboval celý selektor. Četl `n' nejdřív jako `d' a měl velký problém s `g'\braillebox{1245}, četl jako `x'\braillebox{1346}.  Musel jsem mu říct ať pohybuje pravou ruku na selektor.  Četl `l' původně jako `b'.  Četl `j' jako `,' a `m' jako `c'.  Ještě mu pohyboval ruku na selektor ale snažil se, úspěšně to překonat.  U konce řádku myslel, že je mezera.

\paragraph{Řádek 13}
\begin{tabular}{|c|c|c|c|c|c|c|c|c|c|c|c|}
\hline
e&n&o&v&a&l& &s&e& &O&p\\
\braillebox{1578}&\braillebox{1345}&\braillebox{135}&\braillebox{1236}&\braillebox{1}&\braillebox{123}&\braillebox{}&\braillebox{234}&\braillebox{15}&\braillebox{}&\braillebox{1357}&\braillebox{1234}\\
\hline
\end{tabular}

První `o' četl nejdřív jako `e' a druhý `O' četl nejdřív jako `k', a četl `p' nejdřív jako `l'.

Hlavní je, že poslední dvě řádky používal selektor samostatně.

Když jsem se zeptal, jestli si mysli jestli by se dokázal naučit samostatně číst odpovídal \uv{no, snad ano} ale nijak nepřesvědčeně. Ještě když jsem se zeptal na jeho dojmy z toho, říkal:\em \uv{No, spíš tahle jsem trochu zmaten z toho, že posouvám tou ... tabulkou jestli řeknu takhle tou klávesnicí jako by tady tou pravou rukou že jako by spíš jak to funguje když dam každé to písmeno to znamená jeden ten jakoby posun.  Todle pro mně zkouška jak jsem občas myslel, že to je to písmeno zrcadlově obraceně, i když jsem fakt občas měl jsem pocit, nevím proč.}\em
