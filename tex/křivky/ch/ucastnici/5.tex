\subsubsection{Žák 5}
Setkání 12:00 20.3.2013

Pátý žák je ve věku mezi 20 a 25 lety.  Jeho nejvyšší dosažené vzdělání je maturita. Je úplně nevidomý.  Také čte Braillovo písmo od základní školy.  Používá braillský řádek.

Než začal číst tištěný text, zeptal se, jestli má číst z obou stran.  Během čtení  otočil stranu velmi rychle, takže celou dobu četl plynule bez jakéhokoliv přestání.  Jedno slovo přečetl nesprávně: \uv{nejlevnější} četl jako \uv{nejlepší}.

Četl tištěný text oběma rukama.

% začatek 176.8 otočení 192.9 konec 212.6

Čas čtení tištěného textu:35.8 vteřin

CPS: 11.37

Čas čtení tištěného textu po otočení listu:19.7 vteřin

CPS: 13.45

Nejdříve jsem mu ukázal selektor.  Popsal jsem ho jako \uv{dvanáct kovových bodů, které vypadají jako písmo[sic] `k'.}.  Pak jsem mu ukázal zobrazovač.  Řekl jsem, že \uv{to má osm dír[sic] a že z každé z nich skáče kovový tyč.}

\paragraph{První Řádek}
\begin{tabular}{|c|c|c|c|c|c|c|c|c|c|c|c|}
\hline
B&r&a&i&l&l&s&&&&&\\
\braillebox{1278}&\braillebox{1235}&\braillebox{1}&\braillebox{24}&\braillebox{123}&\braillebox{123}&\braillebox{234}&\braillebox{}&\braillebox{2358}&\braillebox{123}&\braillebox{}&\braillebox{}\\
\hline
\end{tabular}

Přečetl první písmeno správně, ale potom začaly tyčky v zobrazovači náhodně vyskakovat. Usoudil jsem, že senzory jsou přecitlivělé, a snížil jsem práh citlivosti tak, aby fungovaly. Protože jsem to dělal poprvé, trvalo mi to asi 10 minut, během kterých žák seděl a nic nedělal.

Vedl jsem jeho ruku nad prvními třemi senzory (abych ověřil, že fungují) a řekl jsem na hlas `b' `r' `a', takže druhá dvě písmena žák sám nečetl.  Potřeboval jsem namočit jeho prst, aby senzory fungovaly. Tak jsem se ho zeptal, kde najdu vodu, a on mi řekl o umyvadle.  Namočil jsem jeho ukazovák na pravé ruce a začali jsme číst.

Začátek řádku četl správně. Jedinou jeho chybou bylo, že četl `s' jako `i'. Opravil se poté, co jsem ho upozornil, že to není správně.

\paragraph{Druhý Řádek}
\begin{tabular}{|c|c|c|c|c|c|c|c|c|c|c|c|}
\hline
k&ý& &ř&á&d&e&k&,& &k&t\\
\braillebox{1378}&\braillebox{12346}&\braillebox{}&\braillebox{2456}&\braillebox{16}&\braillebox{145}&\braillebox{15}&\braillebox{13}&\braillebox{2}&\braillebox{}&\braillebox{13}&\braillebox{2345}\\
\hline
\end{tabular}

Četl správně až na `,'.  Choval se, jako že nerozumí.  Zeptal jsem se, který bod je nahoře.  On správně řekl, že druhý.  Vysvětlil jsem mu tedy, že druhý bod je čárka.  Když dorazil ke `k', myslel, že už je u konce řádku.

\paragraph{Třetí Řádek}
\begin{tabular}{|c|c|c|c|c|c|c|c|c|c|c|c|}
\hline
e&r&ý& &t&e&ď& &p&o&u&ž\\
\braillebox{1578}&\braillebox{1235}&\braillebox{12346}&\braillebox{}&\braillebox{2345}&\braillebox{15}&\braillebox{1456}&\braillebox{}&\braillebox{1234}&\braillebox{135}&\braillebox{136}&\braillebox{2346}\\
\hline
\end{tabular}

Četl správně až na druhé `e'.  Kvůli nekvalitní nahrávce nevím, co odhadl.  Zřejmě ho zmátlo, že body nespadly samy. Zeptal se na to a já odpověděl \uv{ano, nespadnou, když na ně netlačíte}. Zřejmě už sledoval slova, protože poté, co přečetl \uv{teď} po jednotlivých písmenech, řekl slovo \uv{teď} jako celek.

\paragraph{Čtvrtý Řádek}
\begin{tabular}{|c|c|c|c|c|c|c|c|c|c|c|c|}
\hline
í&v&á&t&e&,& &n&e&n&í& \\
\braillebox{3478}&\braillebox{1236}&\braillebox{16}&\braillebox{2345}&\braillebox{15}&\braillebox{2}&\braillebox{}&\braillebox{1345}&\braillebox{15}&\braillebox{1345}&\braillebox{34}&\braillebox{}\\
\hline
\end{tabular}

`t' četl nejdříve jako `j'.  První `n' četl nejdříve jako `d'.  Přeskočil druhé `e' a byl na dalším `n'.  Řekl jsem mu, ať se vrátí zpátky, a on to udělal a přečetl zbytek řádku správně.

\paragraph{Pátý Řádek}
\begin{tabular}{|c|c|c|c|c|c|c|c|c|c|c|c|}
\hline
p&r&v&n&í& &ř&á&d&e&k&,\\
\braillebox{123478}&\braillebox{1235}&\braillebox{1236}&\braillebox{1345}&\braillebox{34}&\braillebox{}&\braillebox{1235}&\braillebox{16}&\braillebox{145}&\braillebox{15}&\braillebox{13}&\braillebox{2}\\
\hline
\end{tabular}

Četl všechno správně.

\paragraph{Šestý Řádek}
\begin{tabular}{|c|c|c|c|c|c|c|c|c|c|c|c|}
\hline
 &k&t&e&r&ý& &z&o&b&r&a\\
\braillebox{78}&\braillebox{13}&\braillebox{2345}&\braillebox{15}&\braillebox{1235}&\braillebox{12346}&\braillebox{}&\braillebox{1356}&\braillebox{135}&\braillebox{12}&\braillebox{1235}&\braillebox{1}\\
\hline
\end{tabular}

U začátku řádku byl nějak překvapený.  Vysvětlil jsem mu, že body 7 a 8 \uv{znamenají} začátek řádku.  Jinak četl správně.

\paragraph{Řádek 7}
\begin{tabular}{|c|c|c|c|c|c|c|c|c|c|c|c|}
\hline
z&u&j&e& &j&e&n&o&m& &j\\
\braillebox{135678}&\braillebox{136}&\braillebox{245}&\braillebox{15}&\braillebox{}&\braillebox{245}&\braillebox{15}&\braillebox{1345}&\braillebox{135}&\braillebox{134}&\braillebox{}&\braillebox{245}\\
\hline
\end{tabular}

Četl všechno správně.

\paragraph{Řádek 8}
\begin{tabular}{|c|c|c|c|c|c|c|c|c|c|c|c|}
\hline
e&d&n&o& &p&í&s&m&e&n&o\\
\braillebox{1578}&\braillebox{145}&\braillebox{1345}&\braillebox{135}&\braillebox{}&\braillebox{1234}&\braillebox{34}&\braillebox{234}&\braillebox{134}&\braillebox{15}&\braillebox{1345}&\braillebox{135}\\
\hline
\end{tabular}

Četl všechno správně a já přestal potvrzovat, že čte správně. Zdálo se mi, že poté zrychlil.

\paragraph{Řádek 9}
\begin{tabular}{|c|c|c|c|c|c|c|c|c|c|c|c|}
\hline
.& & &P&r&v&n&í& &b&y&l\\
\braillebox{378}&\braillebox{}&\braillebox{}&\braillebox{12347}&\braillebox{1235}&\braillebox{1236}&\braillebox{1345}&\braillebox{34}&\braillebox{}&\braillebox{12}&\braillebox{13456}&\braillebox{123}\\
\hline
\end{tabular}

Četl všechno správně. Zeptal se, jestli má říct, že `P' je velké.

\paragraph{Řádek 10}
\begin{tabular}{|c|c|c|c|c|c|c|c|c|c|c|c|}
\hline
 &v&y&n&a&l&e&z&e&n& &v\\
\braillebox{78}&\braillebox{1236}&\braillebox{13456}&\braillebox{1345}&\braillebox{1}&\braillebox{123}&\braillebox{15}&\braillebox{1356}&\braillebox{15}&\braillebox{1345}&\braillebox{}&\braillebox{1236}\\
\hline
\end{tabular}

Přečetl všechno správně.

\paragraph{Řádek 11}
\begin{tabular}{|c|c|c|c|c|c|c|c|c|c|c|c|}
\hline
 &r&o&c&e& &1&9&1&3& &v\\
\braillebox{78}&\braillebox{1235}&\braillebox{135}&\braillebox{14}&\braillebox{15}&\braillebox{}&\braillebox{18}&\braillebox{248}&\braillebox{18}&\braillebox{148}&\braillebox{}&\braillebox{1236}\\
\hline
\end{tabular}

Když dorazil k datu, řekl: \uv{Předpokládám, že to asi bude číslo    jedna devět   to znamená devatenáct set třináct}. Četl všechno správně.

\paragraph{Řádek 12}
\begin{tabular}{|c|c|c|c|c|c|c|c|c|c|c|c|}
\hline
 &A&n&g&l&i&i&.& & &J&m\\
\braillebox{78}&\braillebox{17}&\braillebox{1345}&\braillebox{1245}&\braillebox{123}&\braillebox{24}&\braillebox{24}&\braillebox{3}&\braillebox{}&\braillebox{}&\braillebox{2457}&\braillebox{134}\\
\hline
\end{tabular}

`n' četl nejdřív jako `d' a tečku `.' nejdřív jako mezeru.  Jinak četl správně.

\paragraph{Řádek 13}
\begin{tabular}{|c|c|c|c|c|c|c|c|c|c|c|c|}
\hline
e&n&o&v&a&l& &s&e& &O&p\\
\braillebox{1578}&\braillebox{1345}&\braillebox{135}&\braillebox{1236}&\braillebox{1}&\braillebox{123}&\braillebox{}&\braillebox{234}&\braillebox{15}&\braillebox{}&\braillebox{1357}&\braillebox{1234}\\
\hline
\end{tabular}

Četl všechno správně kromě velkého `O', které původně četl jako `n'.

\paragraph{Řádek 14}
\begin{tabular}{|c|c|c|c|c|c|c|c|c|c|c|c|}
\hline
t&o&f&o&n& &a& &p&ř&e&v\\
\braillebox{234578}&\braillebox{135}&\braillebox{124}&\braillebox{135}&\braillebox{1345}&\braillebox{}&\braillebox{1}&\braillebox{}&\braillebox{1234}&\braillebox{2456}&\braillebox{15}&\braillebox{1236}\\
\hline
\end{tabular}

Četl všechno správně.

\paragraph{Řádek 15}
\begin{tabular}{|c|c|c|c|c|c|c|c|c|c|c|c|}
\hline
á&d&ě&l& &s&v&ě&t&l&o& \\
\braillebox{1678}&\braillebox{145}&\braillebox{126}&\braillebox{123}&\braillebox{}&\braillebox{234}&\braillebox{1236}&\braillebox{126}&\braillebox{2345}&\braillebox{123}&\braillebox{135}&\braillebox{}\\
\hline
\end{tabular}

Četl všechno správně.

\paragraph{Řádek 16}
\begin{tabular}{|c|c|c|c|c|c|c|c|c|c|c|c|}
\hline
n&a& &z&v&u&k&.& & &K&d\\
\braillebox{134578}&\braillebox{1}&\braillebox{}&\braillebox{1356}&\braillebox{1236}&\braillebox{136}&\braillebox{13}&\braillebox{3}&\braillebox{}&\braillebox{}&\braillebox{137}&\braillebox{145}\\
\hline
\end{tabular}

`z' četl nejdřív jako `e'. Jinak četl správně.

\paragraph{Řádek 17}
\begin{tabular}{|c|c|c|c|c|c|c|c|c|c|c|c|}
\hline
y&ž& &č&t&e&n&á&ř& &p&o\\
\braillebox{1345678}&\braillebox{2346}&\braillebox{}&\braillebox{146}&\braillebox{2345}&\braillebox{15}&\braillebox{1345}&\braillebox{16}&\braillebox{2456}&\braillebox{}&\braillebox{1234}&\braillebox{135}\\
\hline
\end{tabular}

Četl všechno správně.

Ukončil jsem setkání a on řekl:
\uv{To se hrozně špatně poznává, když člověk je zvyklý číst Braillovo písmo ... Tím, že ten taktilní bod má určité zvyklosti, pro mě je to těžší poznat}

Tim: \uv{Vy jste moc nepoložil tu ruku rovně na ten článek, musím říct, že já čtu docela rychlejší, než jste četl vy teď na ten stroj, i když určitě čtu rukama pomalejší než vy}

Žák : \uv{Já si myslím, že někomu kdo už má zkušenosti s normálním Braillovým písmem od začátku je to trošku problém}

Tim: \uv{Poznal jste, že jsou tam slova?}

Žák: \uv{Ano}
Tim: \uv{Rozuměl jste?}
Žák: \uv{Ano}

Pozoroval jsem, že pro něj bylo těžké zvyknout si na malý selektor.  Byl zvyklý používat dlouhý braillský řádek. Když se vrátil na začátek selektoru poté, co dočetl řádek textu, často pohyboval svou rukou příliš vlevo.
