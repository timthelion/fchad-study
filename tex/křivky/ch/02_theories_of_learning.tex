\chapter{Teorie}

\section{Pedagogický a psychologický výzkum}

\subsection{Role pedagoga v drilování}
Někdy jste zkusil trejnovat v matice sam doma?  S papírovou knihou těžko jde.  Jestli nepochopite zadaný ukol, vůbec nemate možnost pokračovat.  Pokud rozumíte ale nejste jistý, může Vám pomoc ty "odpovědí v zadu."  Mnoho učebnic nemaji takové odpovědí.  Myšlenka je to, že meně motivované žáci by jen koukali v zadě, a nepočitali.  Nemůžete dat odpovědi v zaděkdyž chcete aby meně motivované žáci něco ziskali z domaci ukolu.

Je jistý, že mrtvý papír nemůže byt dokonalým prostředkem driilování.  Lepší variant najdeme v počitače.  Počitač muže nam řict jestli jste počital správně.  Dale nam dava emoční podporu.  Za každý pět správných odpovědí, hraje se zvuk a animace. Počitač nemůže Vám pomahat v případě, že nevíte jak se začinat.

V tom případě, je nutný mít učitel.

Hlavní ukoly učitela v drílování je potvrzení odpovědí, vysvětlení nejasností a upřímné lhání.  To je, "ano Tome, máš to správně" nebo "ne, to nebyl uplně přesný". "No, musíte nejdřív..." anebo "Jak jste to vyřešil u čisla 5?" a konečně "Dělaš to moc dobře, jseš hodně chytrý kluk."

\subsection{Vliv pedagoga v drilování}
Každý pedagog je jiný, a to bude mít vliv na vysledky žáků.  Pedagogické vlastnosti jsou, emoční citlivost, schopnost rychle potvrdít odpovědí, jistotu v povtrzení či vysvětlení, správnost potvrzení nebo vysvětlení.

\subsection{Podstata prvních dojmů a neznámost materiálů}
Pedagog je vymečně podstatní když jde o neznámy material.  Mnoho krát, pokročilý žák se uči líp sam než s pomoci.  Jim to upevňuje samostatnosti a schopnosti rozumět složité materiály.  Žák, který předmět ještě nerozumí je naopak. Sam vůbec neumí dělat pokroky.  Když neumí ani začit.

Je obecně známý fakt, že první dojmy jsou hodně důležity.  Často jen toho jak pedagog prezentuje nový obor, předmět, anebo koncept ovlivný ochota a schopnost žáka to rozumět.

Obory a koncepty jsou často kulturně těžké rozumět.  Tím, že v kultuře, v halách v škole, se řika, že geometrie je těžka, znamena, že žáci ji rozumět nebudou.  Kulturně předurčená složitost muže mít vliv i na moje experiment. Ačkoliv FCHAD je kulturně neznamý nastroj, "nová technologie" je.  Člověk, který má rad novou technologie snad bude mít lehčí čas se učit než člověk, který ji nesnese.

\section{Teorie a modely učení}

\subsection{Rozdělení učení podle Swifta}

\section{Role curve fitting v studii křivek učení}
