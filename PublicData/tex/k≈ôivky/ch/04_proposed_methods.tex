
\chapter{Proposed Method}

\emph{He et al.} \cite{he2004spectral} showed that the different layers of the network
protocols inprint distinct patterns in a frequency spectrum. 
Further work of \emph{Chen} and \emph{Hwang} \cite{chen2007spectral}
used the features derived %TODO use better word
from frequency spectrum to classify malicious and normal traffic.
They noticed that transport protocols (transmission control protocol --
TCP and user datagram protocol - UDP) 
have distinct frequency spectrum and pover spectral density. 
They exploited this property to identify low-rate attacks on TCP protocol.
%an reduction of service (RoS) attacks.
%TODO more writing about features used
\emph{Wright et al.} \cite{wright2006inferring} researched methods based on 
hidden Markov models to classify different application protocols embeded in 
encrypted application layer.
They developed classification method, able to classify different 
application protocols multiplexed in single encrypted packet flow.
\emph{Dusi et al.} \cite{dusi2009tunnel} brought an statistical approach 
to detect an tunnel inside application layer.
In the paper they described different tunneling techniques and designed 
statistical pattern recognition classifier to identify them.

Our goal is to involve statistical analysis of the frequency components of a 
time-domain signal (spectral analysis) to detection of the tunneled 
connection in application layer.
Our idea is that legitimate connection, that is not missusing application protocol, 
imprints specific patterns in the 
power spectral density that are distinct from spectral density 
of the other unwanted protocols tunneled through application layer. 

In experiments we focused on the application protocols that are most 
spread in computer networks and that are thus most widely miss-used. 
In particular we will analyze protocol HTTP and its encrypted version -- HTTPS. 
Most of the network gateways allow only usage of network protocols HTTP or HTTPS. 
As protocol HTTPS uses encryption, the gateway is unable to detect miss-use by 
analyzing content and thus it is unable to enforce policy restrictions.

Even though it is not possible to analyze payload of particular packets in 
ecrypted connection, it is possible to observe the time of the packet transit, 
its size, direction, source and destination endpoint, etc. 
This data is denoted as packet traces and it is extracted from unencrypted 
part of the packet. 
The goal is to develop feature creation and pattern recognition method for 
the network packet traces involving spectral analysis.
The method is suppoded to infer different application protocols of 
the network traffic only by observation of the packet traces.

\section{Data Collection}

The input data for our method consits of timestamped packet traces. 
Following information has been used from the captured packet data:
\begin{itemize}
	\item packet \emph{timestamp}, i.e. the time of the pass trought the capturing gateway,
	\item identification of source and destination endpoint 
	(i.e. \emph{source address } and \emph{port},
	\emph{destination address}  and \emph{port}
	\footnote{in literature 4-tuple of the  \emph{source address} and 
	\emph{port} and  \emph{destination address} and \emph{port} is 
	denoted as \emph{quad} %TODO cite
	}) and tranmission \emph{protocol} -- we use this infomation (5-tuple) 
	as a key to identify packet \emph{flow},
	\item \emph{size} of packet`s payload,
	\item packet \emph{direction} \footnote{we embed \emph{direction information} 
	into the \emph{size} parameter using negative size  
	if packet travels from destination to source and otherwise positive},
\end{itemize}

\subsection{Training Dataset}

%TODO move to section 5
A \texttt{tcpdump} software \cite{jacobson2009tcpdump}  has been used to 
capture packet data in experimental testbed. 
Testbed consist of three or more peers \emph{A}, \emph{B}, \emph{C}, \emph{D}  etc. 
Peer \emph{A} (a client) is connected with each other using secure tunnel 
to the destination TCP port 443 (destination is on the peers \emph{B},  \emph{C}, ...).
Peer \emph{B} serves virtual private network service and provides network address
translation from the tunnel to the internet.
Peers \emph{C}, \emph{D}, etc. are able to serve single service on the end of tunnel 
(such as HTTP server, telnet, file tranfer service, VOIP, ...).
The traffic tunneled trough peer \emph{B} consist varying number of connections 
and different protocols,
while the others are tunneling single application protocol.

The peer \emph{A} generates traffic to other peers simultaneously, 
mimicking typical behavior of the client of particular application protocol 
(i.e. web browser).
%In addition connections to HTTPS servers are performed from peer \emph{A} directly.
Packet capturing is set up on peer \emph{A} intercepting and storing every packet.
Captured packet flows are then labeled using name of known the application service. 

\subsection{Evaluation Dataset%
\footnote{Evaluation dataset is used during 
assesment of the method on real data.
As we have some old packet trace data, we want to somehow compare it with training samples.
My idea is to prove that exctracted features are epoch invariant by means
statistical goodnes-of-fit test.
On the other hand older protocols can behave in diferent way and thus the 
normal models derived from current data would be
underestimated during assesment proces.
}%
}
The data used for evaluation of proposed method, consists of public packet
traces obtained at....\emph{(to be discussed)}
mixed with simulated and anotated packet traces captured in our testbed.
Before mixing, statistical test has been performed.
The test procedure first extracts and tansforms features
(denoted here as $\vec{f}$) by using our proposed method 
and then compares distribution of features with respect to 
the label -- $Pr\left[ \vec{f} | label \right] $.

Labeling of the obtained dataset is obtained by inferring application 
protocol using destination port.
Since the public dataset can contain malicious connections, 
the result of this statistical test will underestimate the real fitness.
%TODO why?

\section{Feature Creation and Pattern Definition}
%\subsection{Feature exctraction}
For a specified flow $f$ the packet arrival process $x_f\left[t\right]$ 
(or simply \emph{packet process})  is defined as a count of packet arrivals 
at given timespan $I = \left\langle \frac{t}{s}, \frac{t+1}{s} \right)$:
\begin{equation}\label{packetprocess}
	 \forall t \in \mathcal{N} : x_f\left[t\right] = \left| 
	\left\lbrace p : f = flow(p) \wedge timestamp(p) \in I \right\rbrace \right|\, .
\end{equation}
where $s$ is the sample rate, function $flow(p)$  yields the \emph{flow} 5-tuple 
and function $timestamp(p)$  yields the \emph{timestamp} of given packet $p$. 
\footnote{
This is very important formula, we need to focus on it; according 
to \emph{Dusi et al.} \cite{dusi2009tunnel}
zero-length packets are ulikely to be induced by application, 
so we can exclude them here; in addition
they extract incomming and outgoing  stream separately -- 
I think it is good idea to work with in- and out- streams separately 
and compute cross-correlation function instead of auto-correlation function.
The idea of usage I/O cross-correlation function $R_{io}\left(\tau\right)$ 
instead of auto-correlation is that $R_{io}\left(\tau\right)$ (at the specific 
time-lag $\tau$) would enforce the typical request-response round-trip. 
Questionable is, how to physically interpret the resulting spectral components and if the Wiener-Kitchin theorem is applicable.
}

According to a Wiener–Khinchin %TODO reference
theorem the power spectral density $S_{xx}(f)$ is obtained by application of discrete-time 
Fourier transform $\mathcal{F}_{\cdot}(f)$ on autocorrelation function 
of the packet process $R_{xx}\left[\tau\right]$:

\begin{equation}\label{eq:corr}
R_{xx}\left[\tau\right] = E[x\left[t\right]x\left[t+\tau\right]]\, , 
\end{equation}

\begin{equation}\label{eq:psd}
\begin{split}
S_{xx}(f) = \mathcal{F}_{R_{xx}}\left(f\right) = \sum_{\tau=-\infty}^{\infty} 
\left( R_{xx}\left[\tau\right] \exp\left( -\imath 2\pi f\tau \right)\right) \\ 
\forall f \in \left\langle -\frac{s}{2},\frac{s}{2} \right\rangle\, , 
\end{split}
\end{equation}

where $\tau$ is the time-lag, $E\left[\cdot\right]$ is expected value of a random variable and 
$f$ is the frequency. The autocorrelation function is capable of enforcing periodicity.
Equations (\ref{eq:corr}) and (\ref{eq:psd}) hold under assumption that packet process is
\emph{wide-sense stationary random proces}.

As the last assumption seems to be false for infinite time span in network  traffic, 
we involved an \emph{windowing function}  $w(n)$ and 
\emph{discrete Fourier tranform} instead of discrete-time 
Fourier transform. Rectangular windowing is defined as follows:
\begin{equation}
w(n) = \left\lbrace \begin{array}{l} 
1, \mbox{ if } n\in \left\langle 0, M \right) \\ 
0, \mbox{ otherwise} \end{array}\right. \,.
\end{equation}
Windowing function is nonzero inside specified interval $\left\langle 0, M \right)$ 
otherwise it is zero. 
Parameter  $M$ is length of sub-sequence selected from packet arrival proces. 
We iteratively apply windowing function to the whole sequence generating 
non-overlapping adjancent sequence of windows. 
We identify the particular window in this sequence with upper index -- 
e.g. $S_{xx}^i$ is a power spectral density function of an $i$-th window.
Thus, we rewrite equations (\ref{eq:corr}) and (\ref{eq:psd}) for non-overlapping 
windows as follows:
\begin{equation}\label{eq:corr2}
R_{xx}^i\left[m\right] = \frac{1}{M} \sum_{t=0}^{M} x\left[t+iM\right]x\left[t+m+iM\right] \, , 
\end{equation}
\begin{equation}\label{eq:psd2}
\begin{split}
S_{xx}^i(k) = \mathcal{F}_{R_{xx}^i}\left(k\right) = \sum_{m=0}^{M-1}
\left( R_{xx}^i \left[m\right] w(m) \exp\left( -\imath 2\pi m\frac{k}{M} \right)\right)\\
\forall k \in \left\{ 0,1,2,...,M-1 \right\}\, . 
\end{split}
\end{equation}
Note that the windowing function is inherent in equations (\ref{eq:corr2}) and (\ref{eq:psd2})
by using limitted ranges of sumation, and domain of definition of the power spectral density.

By involving windowing function we introduced
\emph{spectral leakge}. Use of diferent windowing function, e.g. \emph{Hann} 
(see equation \ref{eq:hann}),
%or \emph{Hamming} (\ref{eq:hamm}) 
could be appropiate in decreasing leakage.
\begin{equation}\label{eq:hann}
w(n) = \left\lbrace \begin{array}{l} 
0.5\left(1 - \cos \left ( \frac{2 \pi n}{N-1} \right) \right), \mbox{ if } n\in \left\langle 
0, M \right) \\ 0, \mbox{ otherwise} \end{array}\right. \,.
\end{equation}
%\begin{equation}\label{eq:hamm}
%w(n) = \left\lbrace \begin{array}{l} 
%0.54 - 0.46\cos \left ( \frac{2\pi n}{N-1} \right), \mbox{ if } n\in \left\langle 0, M \right) \\
% 0, \mbox{ otherwise} \end{array}\right. \,.
%\end{equation}
Furthermore if the parameter $M$ is too high the packet proces is unlikely to be stationary, 
on the other hand selecting too low value causes that the spectrum is sensitive
to transient phenomena on the network. %TODO why? 

The sampling rate $s$ must be selected according to the Nyquist theorem. 
Too low value entails aliasing%
\footnote{The aliasing is caused by folding of the frequencies above Nyquist frequency
$\frac{s}{2}$ symmetrically below this frequency. Thus this two frequencies are undistinguishable.
To properly reconstruct the signal that contain no frequency higher than $f_{max}$ 
the sample rate is bounded by $s > 2f_{max}$.}%
, while too high value incurs data storage and processing overhead. 

%\subsection{Pattern definition}
By the application of the Fourier transform original features has been mapped into new space.
We denote features in new space as \emph{spectral components} (or \emph{frequency components}).
Temporal context of the features has been altered. 
In new feature space the temporal context is determined by sequence
of the detection windows of size $M$. 
At the level of the detection window a temporal notion is decomposed 
into tuple of temporal functions parametrised by \emph{spectral components}.
Although temporal aspect is still present it has not been used in further analysis.
Anomalies in new feature space are thus regarded to as \emph{point anomalies}. %TODO ?
%The rationale is that the new features decomppsed desired temporal aspects 
%of the original sequential data into declarative form.

There are few parameters affecting quality of the  features: %TODO what?
the smaple rate $s$ and the window length $M$. 
In addition, \emph{spectral components} can be subject to further feature extraction 
or construction, comprising combination of extisting features or discarding irrelevant, 
redundant  or noisy%
\footnote{In case of low signal-to-noise ratio, a feature is typically not usefull 
for discriminative outcome.}%
ones based on the domain knowledge (e.g. sum of a lower resp. a upper half of the 
spectral components resulting in two features a low- and a high-frequency power densities;
or retaining specific spectral features known for its relation to 
periodic stochatic processes to be in research interest).

This aspects and process of seeking of proper parameters are subject of further 
research and they are discussed in \emph{chapter \ref{sec:experiments}}.

\section{Pattern Recognition}

\section{Assessment and Interpretation of Results}
